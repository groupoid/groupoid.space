\documentclass{article}
\usepackage{microtype}
\usepackage{titlesec}
\usepackage[absolute]{textpos}
\usepackage{amsmath}
\usepackage{cite}
\usepackage[strict=true,style=english]{csquotes}
\setquotestyle{english}
\usepackage{amssymb}
\usepackage{fontspec}
\usepackage{amsthm}
\usepackage{url}
\usepackage{tikz}
\usepackage{tikz-cd}
\usetikzlibrary{matrix}
\usepackage[utf8]{inputenc}
\usepackage[english,russian]{babel}
\usetikzlibrary{babel}
\usepackage{listings}
\usepackage[tableposition=top]{caption}
\theoremstyle{definition}
\newtheorem{theorem}{Theorem}
\newtheorem{definition}{Definition}
\newtheorem{exercise}{Exercise}
\newtheorem{example}{Example}
\captionsetup[table]{labelformat=empty}
\lstMakeShortInline[mathescape=true,columns=fixed]|
\def\mapright#1{\xrightarrow{{#1}}}
\def\mapleft#1{\xleftarrow{{#1}}}
\def\mapup#1{\Big\uparrow\rlap{\raise2pt{\scriptstyle{#1}}}}
\def\mapupl#1{\Big\uparrow\llap{\raise2pt{\scriptstyle{#1}}}}
\def\mapdown#1{\Big\downarrow\rlap{\raise2pt{\scriptstyle{#1}}}}
\def\mapdownl#1{\Big\downarrow\llap{\raise2pt{\scriptstyle{#1}}}}
\def\mapdiagl#1{\vcenter{\searrow}\rlap{\raise2pt{\scriptstyle{#1}}}}
\def\mapdiagr#1{\vcenter{\swarrow}\rlap{\raise2pt{\scriptstyle{#1}}}}
\lstset{basicstyle=\small,inputencoding=utf8}
\setlength{\TPHorizModule}{4mm}
\setlength{\TPVertModule}{\TPHorizModule}
\textblockorigin{4mm}{4mm}
\setlength{\parindent}{0pt}
\titleformat{\section}[block]{\Large\bfseries\filcenter}{}{1em}{}

\newfontfamily{\cyrillicfont}{Times New Roman}
\setmainfont{Times New Roman}

\begin{textblock}{3}(2,3)
\small \text{УДК 510.2:510.6}
\end{textblock}

\selectlanguage{ukranian}


\begin{document}

\renewcommand{\abstractname}{Анотація}

\title{\small М.Е. Сохацький^{*}}
\author{Випуск 1: Вбудовування теорії типів Мартіна-Льофа}
\date{ \small Національний технічний університет України \\
      «Київський політехнічний інститут імені Ігоря Сікорського».\\
       \small $^*$Кореспондент: maxim@synrc.com}

\maketitle

\begin{abstract}

Ця стаття демонструє формальне вбудовування теорії типів Мартіна-Льофа
в виконуючу кубічну типову систему з повним набором правил виводу.
Це стало можливим завдяки кубічній теорії типів та типовому кубічному верифікатору
{\bf cubicaltt}\footnote{http://github.com/mortberg/cubicaltt} в 2017 році.
Був пройдений довгий шлях від чистих типових систем AUTOMATH де Брейна до
гомотопічних типових верифікаторів. Ця стаття стосується тільки формального ядра
теорії типів Мартіна-Льофа: $\Pi$ и $\Sigma$ типів (які відповідають
квантору загальності $\forall$ та квантору існування $\exists$ у класичній логіці)
та типу-рівності.

Кожна мовна імплементація повинна бути протестована. Один з можливих сценаріїв
тестування типових верифікаторів це пряме вбудовування в модель теорії типів
виконуючого верифікатора. Так як всі типи в теорії формулюються за допомогою п'яти
прарвил: формації, інтро, елімінації, обчисленя, рівності), ми зконструювали
номінальні типи-синоніми для виконуючого верифікатора та довели, що це є реалізацією MLTT.
Це може розглядатися як універсальний тест для імплементації типового верифікатора,
позаяк компенсаця інтро правила та правила елімінатора пов'язані в правилі
обчислення та рівності (бета та ета редукціях). Таким чином, доводжучи реалізацію MLTT,
ми доводимо властивості самого виконуючого верифікатора.

Більш формально, кубічне MLTT вбудовування конструктивно виражає
J елімінатор типу-рівності та його рівняння — правило обчислення,
що було неможливо до кубічної інтерпретації. Також цей випуск
відкриває серію статей присвячених формалізації основ математики в кубічній теорії типів,
MLTT моделюванню та кубічнії верифікації. Так як не всі можуть бути знайомі з теорією типів,
це випуск також містить їх інтерпретації з точки зору різних розділів математики.

Додамо, що це тільки вхід в техніку прямого вбудовування і після MLTT моделювання,
ми можем піднятися вище — до вбудовування в систему індуктивних типів, і далі,
до вбудовування CW-комплексів як зклейок вищих індуктивних типів.

\end{abstract}
\end{document}