\documentclass{article}
\usepackage{microtype}
\usepackage{titlesec}
\usepackage[absolute]{textpos}
\usepackage{amsmath}
\usepackage{cite}
\usepackage[strict=true,style=english]{csquotes}
\setquotestyle{english}
\usepackage{amssymb}
\usepackage{fontspec}
\usepackage{amsthm}
\usepackage{url}
\usepackage{tikz}
\usepackage{tikz-cd}
\usetikzlibrary{matrix}
\usepackage[utf8]{inputenc}
\usepackage[english,russian]{babel}
\usetikzlibrary{babel}
\usepackage{listings}
\usepackage[tableposition=top]{caption}
\theoremstyle{definition}
\newtheorem{theorem}{Theorem}
\newtheorem{definition}{Definition}
\newtheorem{exercise}{Exercise}
\newtheorem{example}{Example}
\captionsetup[table]{labelformat=empty}
\lstMakeShortInline[mathescape=true,columns=fixed]|
\def\mapright#1{\xrightarrow{{#1}}}
\def\mapleft#1{\xleftarrow{{#1}}}
\def\mapup#1{\Big\uparrow\rlap{\raise2pt{\scriptstyle{#1}}}}
\def\mapupl#1{\Big\uparrow\llap{\raise2pt{\scriptstyle{#1}}}}
\def\mapdown#1{\Big\downarrow\rlap{\raise2pt{\scriptstyle{#1}}}}
\def\mapdownl#1{\Big\downarrow\llap{\raise2pt{\scriptstyle{#1}}}}
\def\mapdiagl#1{\vcenter{\searrow}\rlap{\raise2pt{\scriptstyle{#1}}}}
\def\mapdiagr#1{\vcenter{\swarrow}\rlap{\raise2pt{\scriptstyle{#1}}}}
\lstset{basicstyle=\small,inputencoding=utf8}
\setlength{\TPHorizModule}{4mm}
\setlength{\TPVertModule}{\TPHorizModule}
\textblockorigin{4mm}{4mm}
\setlength{\parindent}{0pt}
\titleformat{\section}[block]{\Large\bfseries\filcenter}{}{1em}{}

\newfontfamily{\cyrillicfont}{Times New Roman}
\setmainfont{Times New Roman}

\begin{textblock}{3}(2,3)
\small \text{УДК 510.2:510.6}
\end{textblock}


\begin{document}


\title{\small М.Э. Сохацкий^{*}}
\author{Выпуск 1: Встраивание теории типов Мартина-Лёфа}
\date{ \small Национальный технический университет Украины \\
      «Киевский политехнический институт имени Игоря Сикорского».\\
       \small $^*$Корреспондент: maxim@synrc.com}

\maketitle

\begin{abstract}

Эта статья демонстрирует формальное встраивание теории типов Мартина-Лёфа
в исполняющую кубическую типовую систему с полным набором правил вывода.
Это стало возможным недавно благодаря кубической теории типов и типовому кубическому
верификатору {\bf cubicaltt}\footnote{http://github.com/mortberg/cubicaltt} в 2017 году.
Был пройдет длинный путь от чистых типовых систем AUTOMATH авторства де Брейна к
гомотопическим типовым верификаторарм. Эта статья касается только формального ядра
теории типов Мартина-Лёфа: $\Pi$ и $\Sigma$ типов (которые соотвествуют
квантороу всеобщности $\forall$ и квантору существования $\exists$ для математических рассуждений)
и типа равенства.

Каждая языковая имплементация должна быть протестирована. Один из возможных
сценариев тестирования типовых верификаторов это прямое встраивание в модель
теории типов исполняющего верификатора. Так как все типы в теории формулируются
с помощью пяти правил: формации, интро, элиминатора, уравнение вычисления, уравнение равества),
мы сконструировали номинальные типы-синоними для исполняющего верификатора и доказали,
что это является реализацией MLTT. Это может рассматриваться как универсальный
тест для имплементации типового верификатора, так как компенсация интор правила и правила элиминатора
которое спрятано в бета и эта редукциях. Таким образом, доказывая реализацию MLTT,
мы докажем свойства самого исполняющего верификатора.

Более формально, кубическое MLTT встраиваение доказывает J элиминатор
типа-равенства и его уравнение вычисления, что не было возможно до
геометрической кубической интерпрертации. Также этот выпуск открывает
серию статей посвящённых формализации оснований математики в кубической теории типов,
MLTT моделированию и кубической верификации. Так как многие могут быть незнакомы
с системами типов этот выпуск также содержит их интерпретацию с точки
зрения разных разделов математики.

Отметим, что это только вход в технику прямого встраивания и после MLTT
моделирования мы можем выйти выше — во встраивание в индуктивную систему типов,
и далее, до встраивание CW-комплексов как склейки высших индуктивных типов.

\end{abstract}
\end{document}