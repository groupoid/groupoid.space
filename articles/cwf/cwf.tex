% copyright (c) 2018 Groupoid Infinity

\documentclass{article}
\usepackage[english,ukrainian]{babel}
\usepackage{listings}
\usepackage{xcolor}
%\usepackage{amscd}
%\usepackage{amsmath}
\usepackage{amssymb}
\usepackage{amsthm}
\usepackage{url}
\usepackage[utf8]{inputenc}

\theoremstyle{definition}
\newtheorem{definition}{Визначення}[section]
\newtheorem{theorem}{Теорема}[section]

\newcommand*{\thead}[1]{\multicolumn{1}{c}{\bfseries #1}}
\lstset{basicstyle=\footnotesize,inputencoding=utf8,extendedchars=true,
 literate={
    {→}{{$\rightarrow$}}1
    {Π}{{$\Pi$}}1
    {Σ}{{$\Sigma$}}1
    {Г}{{$\Gamma$}}1
    {Θ}{{$\Theta$}}1
    {∆}{{$\Delta$}}1
    {Σ}{{$\Sigma$}}1
    {Ф}{{$\Phi$}}1
    {σ}{{$\sigma$}}1
    {δ}{{$\delta$}}1
    {ν}{{$\nu$}}1
    {η}{{$\eta$}}1
    {₁}{{$_1$}}1
    {•}{{$\bullet$}}1
    {є}{{$\varepsilon$}}1
    {ᵀ}{{$\textsuperscript{T}$}}1
    {ᵗ}{{$\textsuperscript{t}$}}1
    {◊}{{$\lozenge$}}1
    {ᵀ}{{$^T$}}1
    {ᵗ}{{$^t$}}1
    {<i>}{{$\langle i\rangle$}}1
    {<_>}{{$\langle\_\rangle$}}1
  }
}

\begin{document}

\title{Випуск I: Категорії з сім'ями}
\author{Максим Сохацький $^1$}
\date{ $^1$ Національний технічний університет України \\
       \small Київський політехнічний інститут імені Ігоря Сікорського \\
       \today }

\maketitle

\begin{abstract}

Категорійна семантика залежної теорії типів. \\
\\
{\bf Ключові слова}: теорія категорій, категорії з сім'ями, залежна теорія типів
\end{abstract}
\tableofcontents

\newpage
\section{Категорії з сім'ями}

Тут подано короткий неформальний опис категорійної семантики залежної теорії типів, запропонований Пітером Диб’єром.
Категоріальна абстрактна машина Диб'єра на Haskell описана тут\footnote{\url{https://www.cse.chalmers.se/~peterd/papers/Ise2008.pdf}}.

\subsection{Основні визначення}

\begin{definition}[Fam]
Категорія $Fam$ --- це категорія сімей множин, де об’єкти є залежними функціональними просторами $(x:A)\rightarrow B(x)$, а морфізми з доменом $\Pi(A,B)$ і кодоменом $\Pi(A',B')$ --- це пари функцій $\langle f:A\rightarrow A', g(x:A):B(x)\rightarrow B'(f(x)) \rangle$.
\end{definition}

\begin{definition}[$\Pi$-похідність]
Для контексту $\Gamma$ і типу $A$ позначимо $\Gamma\vdash A = (\gamma:\Gamma)\rightarrow A(\gamma)$.
\end{definition}

\begin{definition}[$\Sigma$-охоплення]
Для контексту $\Gamma$ і типу $A$ маємо $\Gamma;A = (\gamma:\Gamma)*A(\gamma)$. Охоплення не є асоціативним:
\[
    \Gamma;A;B \neq \Gamma;B;A
\]
\end{definition}

\begin{definition}[Контекст]
Категорія контекстів $C$ --- це категорія, де об’єкти є контекстами, а морфізми --- підстановками. Термінальний об’єкт $\Gamma=0$ у $C$ називається порожнім контекстом. Операція охоплення контексту $\Gamma;A = (x:\Gamma)*A(x)$ має елімінатори: $p:\Gamma;A\vdash\Gamma$, $q:\Gamma;A\vdash A(p)$, що задовольняють універсальну властивість: для будь-якого $\Delta:ob(C)$, морфізму $\gamma:\Delta\rightarrow\Gamma$ і терму $a:\Delta\rightarrow A$ існує єдиний морфізм $\theta=\langle\gamma,a\rangle:\Delta\rightarrow\Gamma;A$, такий що $p\circ\theta=\gamma$ і $q(\theta)=a$. Твердження: підстановка є асоціативною:
\[
    \gamma(\gamma(\Gamma,x,a),y,b) = \gamma(\gamma(\Gamma,y,b),x,a)
\]
\end{definition}

\begin{definition}[CwF-об’єкт]
CwF-об’єкт --- це пара $\Sigma(C, C\rightarrow Fam)$, де $C$ --- категорія контекстів з об’єктами-контекстами та морфізмами-підстановками, а $T:C\rightarrow Fam$ --- функтор, який відображає контекст $\Gamma$ у $C$ на сім’ю множин термів $\Gamma\vdash A$, а підстановку $\gamma:\Delta\rightarrow\Gamma$ --- на пару функцій, що виконують підстановку $\gamma$ у термах і типах відповідно.
\end{definition}

\begin{definition}[CwF-морфізм]
Нехай $(C,T):ob(C)$, де $T:C\rightarrow Fam$. CwF-морфізм $m: (C,T)\rightarrow(C',T')$ --- це пара $\langle F:C\rightarrow C', \sigma:T\rightarrow T'(F) \rangle$, де $F$ --- функтор, а $\sigma$ --- натуральна трансформація.
\end{definition}

\begin{definition}[Категорія типів]
Для CwF з об’єктами $(C,T)$ і морфізмами $(C,T)\rightarrow(C',T')$, для заданого контексту $\Gamma \in Ob(C)$ можна побудувати категорію $Type(\Gamma)$ --- категорію типів у контексті $\Gamma$, де об’єкти --- множина типів у контексті, а морфізми --- функції $f:\Gamma;A\rightarrow B(p)$.
\end{definition}

\subsection{Семантика залежної теорії типів}

\begin{definition}[Терми та типи]
У CwF для контексту $\Gamma$ терми $\Gamma\vdash a:A$ є елементами множини $A(\gamma)$, де $\gamma:\Gamma$. Типи $\Gamma\vdash A$ є об’єктами в $Type(\Gamma)$, а підстановка $\gamma:\Delta\rightarrow\Gamma$ діє на типи та терми через функтор $T$.
\end{definition}

\begin{theorem}[Композиція підстановок]
Підстановки в категорії контекстів $C$ є асоціативними та мають одиницю (ідентичну підстановку). Формально, для $\gamma:\Delta\rightarrow\Gamma$, $\delta:\Theta\rightarrow\Delta$ і $\epsilon:\Gamma\rightarrow\Lambda$ виконується:
\[
    (\gamma \circ \delta) \circ \epsilon = \gamma \circ (\delta \circ \epsilon), \quad id_{\Gamma} \circ \gamma = \gamma, \quad \gamma \circ id_{\Delta} = \gamma.
\]
\end{theorem}

\begin{proof}
Асоціативність випливає з універсальної властивості охоплення контексту (Визначення 1.4). Для будь-яких $\gamma,\delta,\epsilon$ композиція морфізмів у $C$ відповідає послідовному застосуванню підстановок, що зберігає структуру контекстів. Ідентична підстановка $id_{\Gamma}$ діє як нейтральний елемент, оскільки $p \circ id_{\Gamma} = id_{\Gamma}$ і $q(id_{\Gamma}) = q$.
\end{proof}

\begin{definition}[Залежні типи]
Залежний тип у контексті $\Gamma$ --- це відображення $\Gamma \rightarrow Fam$, де для кожного $\gamma:\Gamma$ задається множина $A(\gamma)$. У категорії $Type(\Gamma)$ залежні типи є об’єктами, а морфізми між $A$ і $B$ --- це функції $f: \Gamma;A \rightarrow B(p)$, що зберігають структуру підстановок.
\end{definition}

\begin{theorem}[Універсальна властивість залежних типів]
Для будь-якого контексту $\Gamma$, типу $A$ і терму $a:\Gamma\vdash A$ існує унікальний морфізм $\theta:\Gamma \rightarrow \Gamma;A$, який задовольняє $p \circ \theta = id_{\Gamma}$ і $q(\theta) = a$. Це забезпечує коректність залежної типізації в CwF.
\end{theorem}

\begin{proof}
За Визначенням 1.4, універсальна властивість охоплення контексту гарантує існування $\theta = \langle id_{\Gamma}, a \rangle$. Унікальність випливає з того, що будь-який інший морфізм $\theta'$ з тими ж властивостями ($p \circ \theta' = id_{\Gamma}$, $q(\theta') = a$) збігається з $\theta$ через єдиність композиції в $C$.
\end{proof}

\newpage
\subsection{Формалізація в Anders}

Для формалізації CwF у Agda чи Lean необхідно визначити категорію $C$ як запис із полями
для об’єктів, морфізмів, композиції та ідентичності, а також функтор $T:C \rightarrow Fam$.
Нижче наведено псевдокод для Anders\footnote{\url{https://anders.groupoid.space/lib/mathematics/categories/meta/kraus.anders}}:

\begin{lstlisting}

def algebra : U₁ := Σ
    -- a semicategory of contexts and substitutions:
    (Con: U)
    (Sub: Con → Con → U)
    (◊: Π (Г Θ ∆ : Con), Sub Θ ∆ → Sub Г Θ → Sub Г ∆)
    (◊-assoc: Π (Г Θ ∆ Ф : Con) (σ: Sub Г Θ) (δ: Sub Θ ∆)
        (ν: Sub ∆ Ф), PathP (<_>Sub Г Ф) (◊ Г ∆ Ф ν (◊ Г Θ ∆ δ σ))
                                         (◊ Г Θ Ф (◊ Θ ∆ Ф ν δ) σ))
    -- identity morphisms as identity substitutions:
    (id: Π (Г : Con), Sub Г Г)
    (id-left: Π (Θ ∆ : Con) (δ : Sub Θ ∆),
              Path (Sub Θ ∆) δ (◊ Θ ∆ ∆ (id ∆) δ))
    (id-right: Π (Θ ∆ : Con) (δ : Sub Θ ∆),
               Path (Sub Θ ∆) δ (◊ Θ Θ ∆ δ (id Θ)))
    -- a terminal oject as empty context:
    (•: Con)
    (є: Π (Г : Con), Sub Г •)
    (•-η: Π (Г: Con) (δ: Sub Г •), Path (Sub Г •) (є Г) δ)
    (Ty: Con → U)
    (_|_|ᵀ: Π (Г ∆ : Con), Ty ∆ → Sub Г ∆ → Ty Г)
    (|id|ᵀ: Π (∆ : Con) (A : Ty ∆), Path (Ty ∆) (_|_|ᵀ ∆ ∆ A (id ∆)) A)
    (|◊|ᵀ: Π (Г ∆ Ф: Con) (A : Ty Ф) (σ : Sub Г ∆) (δ : Sub ∆ Ф),
        PathP (<_>Ty Г) (_|_|ᵀ Г Ф A (◊ Г ∆ Ф δ σ))
                        (_|_|ᵀ Г ∆ (_|_|ᵀ ∆ Ф A δ) σ))
    -- a (covariant) presheaf on the category of elements as terms:
    (Tm: Π (Г : Con), Ty Г → U)
    (_|_|ᵗ: Π (Г ∆ : Con) (A : Ty ∆) (B : Tm ∆ A)
              (σ: Sub Г ∆), Tm Г (_|_|ᵀ Г ∆ A σ))
    (|id|ᵗ: Π (∆ : Con) (A : Ty ∆) (t: Tm ∆ A),
            PathP (<i> Tm ∆ (|id|ᵀ ∆ A @ i))
                  (_|_|ᵗ ∆ ∆ A t (id ∆)) t)
    (|◊|ᵗ: Π (Г ∆ Ф: Con) (A : Ty Ф) (t: Tm Ф A)
             (σ : Sub Г ∆) (δ : Sub ∆ Ф),
             PathP (<i> Tm Г (|◊|ᵀ Г ∆ Ф A σ δ @ i))
                   (_|_|ᵗ Г Ф A t (◊ Г ∆ Ф δ σ))
          (_|_|ᵗ Г ∆ (_|_|ᵀ ∆ Ф A δ) (_|_|ᵗ ∆ Ф A t δ) σ))
\end{lstlisting}

Ця структура дозволяє реалізувати Визначення 1.1–1.11, а Теореми 1.10 і 1.12 доводяться
через перевірку асоціативності та універсальних властивостей.

\newpage
\subsection{Висновки}

Категорії з сім’ями (CwF) є потужним інструментом для моделювання залежної теорії типів.
Вони забезпечують чітку семантику для контекстів, підстановок і залежних типів,
що полегшує аналіз і формалізацію.

\bibliographystyle{plain}
\bibliography{cwf}

\end{document}
