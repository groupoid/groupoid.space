% copyright (c) 2018 Groupoid Infinity
\documentclass{article}
\usepackage[english]{babel}
\usepackage{hyphenat}
\usepackage{listings}
\usepackage{amsmath}
\usepackage{amssymb}
\usepackage{amsthm}
\usepackage{mathtools}
\usepackage{url}
\usepackage{tikz-cd}
\usepackage[utf8]{inputenc}

\theoremstyle{definition}
\newtheorem{theorem}{Theorem}
\newtheorem{definition}{Definition}
\newtheorem{example}{Example}
\newcommand*{\incmap}{\hookrightarrow}
\newcommand*{\thead}[1]{\multicolumn{1}{c}{\bfseries #1}}
\lstset{basicstyle=\small,inputencoding=utf8}

\begin{document}

\title{Issue IV: Higher Inductive Types}
\author{Maxim Sokhatsky $^1$}
\date{ $^1$ National Technical University of Ukraine \\
       \small Igor Sikorsky Kyiv Polytechnical Institute \\
       \today }
\maketitle

\begin{abstract}
CW-complexes are key in homotopy type theory (HoTT) and are encoded
in cubical type checkers as higher inductive types (HITs). Like
recursive trees for (co)inductive types, HITs represent CW-complexes.
An HIT defines a CW-complex using cubical composition as an initial
algebra element in a cubical model. We explore HIT motivation, their
topological role, and implementation in Agda Cubical, focusing on infinity constructors. \\
{\bf Keywords}: Cellular Topology, Cubical Type Theory, HITs
\end{abstract}

\tableofcontents

\section{CW-Complexes}
CW-complexes are spaces built by attaching cells of increasing
dimension. In HoTT, they are encoded as HITs, with cells as
constructors for points and paths.\begin{definition} (Cell Attachment). Attaching an \( n \)-cell
to a space \( X \) along \( f : S^{n-1} \to X \) is a pushout:
\[
\begin{tikzcd}
S^{n-1} \arrow[r, "f"] \arrow[d, "\iota"] & X \arrow[d, "j"] \\
D^n \arrow[r, "g"] & X \cup_f D^n
\end{tikzcd}
\]
Here, \( \iota : S^{n-1} \incmap D^n \) is the boundary
inclusion, and \( X \cup_f D^n \) is the pushout, gluing an
\( n \)-cell to \( X \) via \( f \). The result depends on the
homotopy class of \( f \).
\end{definition}\begin{definition} (CW-Complex). A CW-complex is a space \( X \)
built inductively by attaching cells, with a skeletal
filtration:
\begin{itemize}
    \item The \((-1)\)-skeleton is \( X_{-1} = \varnothing \).
    \item For \( n \geq 0 \), the \( n \)-skeleton \( X_n \) is
      obtained by attaching \( n \)-cells to \( X_{n-1} \). For
      indices \( J_n \) and maps \( \{ f_j : S^{n-1} \to X_{n-1}
      \}_{j \in J_n} \), \( X_n \) is the pushout:
    \[
    \begin{tikzcd}
    \coprod_{j \in J_n} S^{n-1} \arrow[r, "\coprod f_j"] \arrow[d, "\coprod \iota_j"] & X_{n-1} \arrow[d, "i_n"] \\
    \coprod_{j \in J_n} D^n \arrow[r, "\coprod g_j"] & X_n
    \end{tikzcd}
    \]
    where \( \coprod_{j \in J_n} S^{n-1} \), \( \coprod_{j \in
    J_n} D^n \) are disjoint unions, and \( i_n : X_{n-1}
    \incmap X_n \) is the inclusion.
    \item \( X \) is the colimit of:
    \[
    \varnothing = X_{-1} \incmap X_0 \incmap X_1 \incmap \dots
    \incmap X,
    \]
    with \( X_n \) the \( n \)-skeleton, and \( X = \text{colim}_{n
    \to \infty} X_n \). The sequence is the skeletal filtration.
\end{itemize}
In HoTT, CW-complexes are HITs, with constructors for cells and
path constructors for gluing.
\end{definition}\begin{example} (Sphere as a CW-Complex). The \( n \)-sphere
\( S^n \) is a CW-complex with one 0-cell and one \( n \)-cell:
\begin{itemize}
    \item \( X_0 = \{ \text{base} \} \), a point.
    \item \( X_k = X_0 \) for \( 0 < k < n \), no cells added.
    \item \( X_n \): Attach an \( n \)-cell to \( X_{n-1} =
      \{ \text{base} \} \) along \( f : S^{n-1} \to
      \{ \text{base} \} \):
    The \( \text{cell} \) constructor glues boundaries to
    \( \text{base} \), yielding \( S^n \).
\end{itemize}
\end{example}

\subsection{Motivation for Higher Inductive Types}
HITs in HoTT enable direct encoding of topological spaces like
CW-complexes. In homotopy theory, spaces are built by gluing
cells via attaching maps. HoTT views types as spaces, elements
as points, and equalities as paths, making HITs a natural fit.
Standard inductive types cannot capture higher homotopies, but
HITs allow constructors for points and paths.For example, the circle \( S^1 \) (Definition 2) has a base
point and a loop, encoding its fundamental group
\( \mathbb{Z} \). HITs avoid set-level quotients, preserving
HoTT’s synthetic nature. In cubical type theory, paths are
intervals (e.g., \( <i> \)) with computational content, unlike
propositional equalities, enabling efficient type checking in
tools like Agda Cubical.

\subsection{HITs with Infinity Constructors}
Some HITs require infinite constructors for spaces like Eilenberg-MacLane spaces or the infinite sphere $S^\infty$.
\begin{lstlisting}
data SInf
   = base
   | loopn (n: Nat) <i> [ (i = 0) -> base, (i = 1) -> base ]
\end{lstlisting}
Challenges include type checking, computation, and expressivity.

Agda Cubical uses cubical primitives to handle HITs, supporting infinite constructors via indexed HITs.
\begin{lstlisting}
data InfHIT (A: Type) : Type where
point : InfHIT A
pathn : (n: Nat) -> PathP (\lambda i -> InfHIT A) point point
\end{lstlisting}

\newpage
\section{Higher Inductive Types}
CW-complexes are central in HoTT and appear in cubical type checkers as HITs. Unlike inductive types (recursive trees), HITs encode CW-complexes, capturing points (0-cells) and higher paths (n-cells). Defining an HIT specifies a CW-complex via cubical composition, an initial algebra in a cubical model.

\subsection{Suspension}
The suspension $\Sigma A$ of a type $A$ is a higher inductive type (HIT) that constructs a new type by adding two points, called poles, and paths connecting each point of $A$ to these poles. It is a fundamental construction in homotopy type theory, often used to shift homotopy groups, e.g., producing $S^{n+1}$ from $S^n$.

\begin{definition} (Formation)
For a type $A : \mathcal{U}$, the suspension $\Sigma A : \mathcal{U}$.
\end{definition}

\begin{definition} (Introduction)
The suspension is generated by the following higher inductive composition structure:
\[
\begin{cases}
\text{north} : \Sigma A \\
\text{south} : \Sigma A \\
\text{merid} : (a : A) \to \text{Path}_{\Sigma A} (\text{north}, \text{south})
\end{cases}
\]
\begin{lstlisting}
data Suspension (A: U)
   = north
   | south
   | merid (a: A) <i> [ (i=0) -> north, (i=1) -> south ]
\end{lstlisting}
\end{definition}

\begin{theorem} (Elimination)
For a type $B : \mathcal{U}$, points $n, s : B$, and a family of paths $m : (a : A) \to \text{Path}_B (n, s)$, there exists a map $\text{rec}_{\Sigma A} : \Sigma A \to B$ such that:
\[
\begin{cases}
\text{rec}_{\Sigma A}(\text{north}) = n \\
\text{rec}_{\Sigma A}(\text{south}) = s \\
\text{rec}_{\Sigma A}(\text{merid } a) = m(a)
\end{cases}
\]
\begin{lstlisting}
SuspensionRec (A B: U) (n s: B) (m: (a: A) -> Path B n s)
  : Suspension A -> B
  = split
    north -> n
    south -> s
    merid a @ i -> m a @ i
\end{lstlisting}
\end{theorem}

\begin{theorem} (Computation)
For $x : \Sigma A$,
\[
\begin{cases}
\text{rec}_{\Sigma A}(\text{north}) \equiv n \\
\text{rec}_{\Sigma A}(\text{south}) \equiv s \\
\text{rec}_{\Sigma A}(\text{merid } a \, @ \, i) \equiv m(a) \, @ \, i
\end{cases}
\]
\begin{lstlisting}
SuspensionBeta (A B: U) (n s: B) (m: (a: A) -> Path B n s)
    (x: Suspension A)
  : Path B (SuspensionRec A B n s m x)
    (split north -> n; south -> s; merid a @ i -> m a @ i) x
  = idp B (SuspensionRec A B n s m x)
\end{lstlisting}
\end{theorem}

\begin{theorem} (Uniqueness)
Any two maps $h_1, h_2 : \Sigma A \to B$ are homotopic if they agree on $\text{north}$, $\text{south}$, and $\text{merid}$, i.e., if $h_1(\text{north}) = h_2(\text{north})$, $h_1(\text{south}) = h_2(\text{south})$, and $h_1(\text{merid } a) = h_2(\text{merid } a)$ for all $a : A$.
\end{theorem}

\begin{example} (Suspension of $S^0$)
The suspension of the 0-sphere $S^0$ (two points) yields the 1-sphere $S^1$. If $A = S^0 = \{ \text{base}_0, \text{base}_1 \}$, then $\Sigma S^0$ has two points $\text{north}$, $\text{south}$, and two paths $\text{merid}(\text{base}_0)$, $\text{merid}(\text{base}_1)$, resembling the loop structure of $S^1$.
\end{example}


\subsection{Pushout}
The pushout is a higher inductive type (HIT) that constructs a type by gluing two types $A$ and $B$ along a common type $C$ via maps $f : C \to A$ and $g : C \to B$. It is a fundamental construction in homotopy type theory, used to model cell attachments and cofibrant objects, generalizing the topological notion of pushouts.

\begin{definition} (Formation)
For types $A, B, C : \mathcal{U}$ and maps $f : C \to A$, $g : C \to B$, the pushout $\text{pushout } A B C f g : \mathcal{U}$.
\end{definition}

\begin{definition} (Introduction)
The pushout is generated by the following higher inductive composition structure:
\[
\begin{cases}
\text{po1} : A \to \text{pushout } A B C f g \\
\text{po2} : B \to \text{pushout } A B C f g \\
\text{po3} : (c : C) \to \text{Path}_{\text{pushout } A B C f g} (\text{po1 } (f c), \text{po2 } (g c))
\end{cases}
\]
\begin{lstlisting}
data pushout (A B C: U) (f: C -> A) (g: C -> B)
   = po1 (_: A)
   | po2 (_: B)
   | po3 (c: C) <i> [ (i = 0) -> po1 (f c) , (i = 1) -> po2 (g c) ]
\end{lstlisting}
\end{definition}

\begin{theorem} (Elimination)
For a type $D : \mathcal{U}$, maps $h_A : A \to D$, $h_B : B \to D$, and a family of paths $h_C : (c : C) \to \text{Path}_D (h_A (f c), h_B (g c))$, there exists a map $\text{pushoutRec} : \text{pushout } A B C f g \to D$ such that:
\[
\begin{cases}
\text{pushoutRec}(\text{po1 } a) = h_A(a) \\
\text{pushoutRec}(\text{po2 } b) = h_B(b) \\
\text{pushoutRec}(\text{po3 } c \, @ \, i) = h_C(c) \, @ \, i
\end{cases}
\]
\begin{lstlisting}
pushoutRec (A B C D: U) (f: C -> A) (g: C -> B) 
           (hA: A -> D) (hB: B -> D) (hC: (c: C) -> Path D (hA (f c)) (hB (g c)))
   : pushout A B C f g -> D
   = split
     po1 a -> hA a
     po2 b -> hB b
     po3 c @ i -> hC c @ i
\end{lstlisting}
\end{theorem}

\begin{theorem} (Computation)
For $x : \text{pushout } A B C f g$,
\[
\begin{cases}
\text{pushoutRec}(\text{po1 } a) \equiv h_A(a) \\
\text{pushoutRec}(\text{po2 } b) \equiv h_B(b) \\
\text{pushoutRec}(\text{po3 } c \, @ \, i) \equiv h_C(c) \, @ \, i
\end{cases}
\]
\end{theorem}

\begin{theorem} (Uniqueness)
Any two maps $h_1, h_2 : \text{pushout } A B C f g \to D$ are homotopic if they agree on $\text{po1}$, $\text{po2}$, and $\text{po3}$, i.e., if $h_1(\text{po1 } a) = h_2(\text{po1 } a)$ for all $a : A$, $h_1(\text{po2 } b) = h_2(\text{po2 } b)$ for all $b : B$, and $h_1(\text{po3 } c) = h_2(\text{po3 } c)$ for all $c : C$.
\end{theorem}

\begin{example} (Cell Attachment)
The pushout models attaching an $n$-cell to a space $X$. Given $f : S^{n-1} \to X$ and the inclusion $g : S^{n-1} \to D^n$, the pushout $\text{pushout } X D^n S^{n-1} f g$ is the space $X \cup_f D^n$, gluing the $n$-disk to $X$ along $f$.
\[
\begin{tikzcd}
S^{n-1} \arrow[r, "f"] \arrow[d, "g"] & X \arrow[d] \\
D^n \arrow[r] & X \cup_f D^n
\end{tikzcd}
\]
\end{example}

\newpage
\subsection{Spheres}
Spheres are HITs with higher-dimensional paths, representing fundamental topological spaces like the circle ($S^1$) or 2-sphere ($S^2$).

\begin{definition} (Spheres and Disks)
\[
S^1 = \begin{cases}
\quad \text{base} \\
\quad \text{loop } <i> \begin{cases} (i = 0) \mapsto \text{base} \\ (i = 1) \mapsto \text{base} \end{cases} \\
\end{cases}
\]

\[
S^2 = \begin{cases}
\quad \text{point} \\
\quad \text{surf} <i \, j> \begin{cases}
(i = 0) \mapsto \text{point} \\
(i = 1) \mapsto \text{point} \\
(j = 0) \mapsto \text{point} \\
(j = 1) \mapsto \text{point}
\end{cases}
\end{cases}
\]
\begin{lstlisting}
data S1
   = base
   | loop <i> [ (i = 0) -> base, (i = 1) -> base ]
data S2
   = point
   | surf <i j> [ (i = 0) -> point, (i = 1) -> point,
                  (j = 0) -> point, (j = 1) -> point ]
\end{lstlisting}
\end{definition}

\begin{example} (Sphere as a CW-Complex)
The $n$-sphere $S^n$ is a CW-complex with one 0-cell and one $n$-cell:
\[
\begin{cases}
X_0 = \{ \text{base} \}, \text{a point} \\
X_k = X_0 \text{ for } 0 < k < n, \text{no cells added} \\
X_n: \text{Attach an } n\text{-cell to } X_{n-1} = \{ \text{base} \} \text{ along } f : S^{n-1} \to \{ \text{base} \}
\end{cases}
\]
\begin{lstlisting}
data Sn (n: Nat)
   = base
   | cell <i1 ... in> [ (i1 = 0) -> base, (i1 = 1) -> base,
                        ...,
                        (in = 0) -> base, (in = 1) -> base ]
\end{lstlisting}
The $\text{cell}$ constructor glues boundaries to $\text{base}$, yielding $S^n$.
\end{example}

\newpage
\subsection{Hub and Spoke}
The Hub and Spoke construction defines $n$-truncations, ensuring a type has no non-trivial homotopy groups above dimension $n$. It models a type as a CW-complex with a hub (central point) and spokes (paths to points).

\begin{definition} (Hub and Spokes)
For types $S, A : \mathcal{U}$, the Hub and Spokes type $\text{hubSpokes } S A : \mathcal{U}$.
\[
\begin{cases}
\text{base} : A \to \text{hubSpokes } S A \\
\text{hub} : (S \to \text{hubSpokes } S A) \to \text{hubSpokes } S A \\
\text{spoke} : (f : S \to \text{hubSpokes } S A) \to (s : S) \to \text{Path}_{\text{hubSpokes } S A} (\text{hub } f, f s) \\
\text{hubEq} : (x, y : A) \to (p : S \to \text{Path}_A(x, y)) \to \text{Path}_{\text{hubSpokes } S A} (\text{base } x, \text{base } y) \\
\text{spokeEq} : (x, y : A) \to (p : S \to \text{Path}_A(x, y)) \to (s : S) \to \text{Path}_{\text{hubSpokes } S A} (\text{hubEq } x y p, \text{base } (p s))
\end{cases}
\]
\begin{lstlisting}
data hubSpokes (S A: U)
   = base (x: A)
   | hub (f: S -> hubSpokes S A)
   | spoke (f: S -> hubSpokes S A) (s:S)
     <i> [ (i=0) -> hub f , (i=1) -> f s ]
   | hubEq (x y: A) (p: S -> Path A x y)
     <i> [ (i=0) -> base x , (i=1) -> base y ]
   | spokeEq (x y: A) (p: S -> Path A x y) (s: S)
     <i> [ (i=0) -> hubEq x y p , (i=1) -> base (p s) ]
\end{lstlisting}
\end{definition}

\begin{theorem} (Elimination $\text{hubSpokes}$)
For a type $B : \mathcal{U}$, a map $g : A \to B$, a point $h : (S \to \text{hubSpokes } S A) \to B$, and path maps ensuring coherence, there exists $\text{rec}_{\text{hubSpokes}} : \text{hubSpokes } S A \to B$, such that $\text{rec}_{\text{hubSpokes}}(\text{base } x) = g(x)$ and $\text{rec}_{\text{hubSpokes}}(\text{hub } f) = h(f)$.
\end{theorem}

\newpage
\subsection{Set-Truncations}
Set truncation (0-truncation), denoted $\| A \|_0$, ensures a type is a set, with homotopy groups vanishing above dimension 0.

\begin{definition} (Set Truncation)
For $A : \mathcal{U}$, $\| A \|_0 : \mathcal{U}$.
\[
\begin{cases}
\text{inc} : A \to \| A \|_0 \\
\text{squash} : (a, b : \| A \|_0) \to (p, q : \text{Path}_{\| A \|_0}(a, b)) \to \text{Path}_{\text{Path}_{\| A \|_0}(a, b)}(p, q)
\end{cases}
\]
\begin{lstlisting}
data setTrunc (A: U)
   = inc (a: A)
   | squash (a b: setTrunc A) (p q: Path (setTrunc A) a b)
     <i j> [ (i = 0) -> p @ j, (i = 1) -> q @ j,
             (j = 0) -> a,     (j = 1) -> b ]
\end{lstlisting}
\end{definition}

\begin{theorem} (Elimination $\| A \|_0$)
For a set $B : \mathcal{U}$ (i.e., $\text{isSet}(B)$), a map $f : A \to B$, there exists $\text{setTruncRec} : \| A \|_0 \to B$, such that $\text{setTruncRec}(\text{inc}(a)) = f(a)$.
\end{theorem}

\subsection{Groupoid-Truncations}
Groupoid truncation (1-truncation), denoted $\| A \|_1$, ensures a type is a 1-groupoid, with homotopy groups vanishing above dimension 1.

\begin{definition} (Groupoid Truncation)
For $A : \mathcal{U}$, $\| A \|_1 : \mathcal{U}$.
\[
\begin{cases}
\text{inc} : A \to \| A \|_1 \\
\text{squash} : (a, b : \| A \|_1) \to (p, q : \text{Path}_{\| A \|_1}(a, b)) \to (r, s : \text{Path}_{\text{Path}_{\| A \|_1}(a, b)}(p, q)) \to \text{Path}_{\text{Path}_{\text{Path}_{\| A \|_1}(a, b)}(p, q)}(r, s)
\end{cases}
\]
\begin{lstlisting}
data grpdTrunc (A: U)
   = inc (a: A)
   | squash (a b: grpdTrunc A)
            (p q: Path (grpdTrunc A) a b)
            (r s: Path (Path (grpdTrunc A) a b) p q)
     <i j k> [ (i = 0) -> r @ j @ k, (i = 1) -> s @ j @ k,
               (j = 0) -> p @ k,     (j = 1) -> q @ k,
               (k = 0) -> a,         (k = 1) -> b ]
\end{lstlisting}
\end{definition}

\begin{theorem} (Elimination $\| A \|_1$)
For a 1-groupoid $B : \mathcal{U}$ (i.e., $\text{isGroupoid}(B)$), a map $f : A \to B$, there exists $\text{grpdTruncRec} : \| A \|_1 \to B$, such that $\text{grpdTruncRec}(\text{inc}(a)) = f(a)$.
\end{theorem}

\newpage

\subsection{Set-Quotients}
Set quotients construct a type $A$ quotiented by a relation $R : A \to A \to \mathcal{U}$, ensuring the result is a set.

\begin{definition} (Set Quotient)
For a type $A : \mathcal{U}$ and a relation $R : A \to A \to \mathcal{U}$, the set quotient $\text{setQuot } A R : \mathcal{U}$.
\[
\begin{cases}
\text{quotient} : A \to \text{setQuot } A R \\
\text{identification} : (a, b : A) \to R a b \to \text{Path}_{\text{setQuot } A R} (\text{quotient } a, \text{quotient } b) \\
\text{trunc} : (a, b : \text{setQuot } A R) \to (p, q : \text{Path}_{\text{setQuot } A R}(a, b)) \to \text{Path}_{\text{Path}_{\text{setQuot } A R}(a, b)}(p, q)
\end{cases}
\]
\begin{lstlisting}
data setQuot (A: U) (R: A -> A -> U)
   = quotient (a: A)
   | identification (a b: A) (r: R a b)
     <i> [ (i=0) -> quotient a, (i=1) -> quotient b ]
   | trunc (a b : setQuot A R) (p q : Path (setQuot A R) a b)
     <i j> [ (i = 0) -> p @ j , (i = 1) -> q @ j ,
             (j = 0) -> a ,     (j = 1) -> b ]
\end{lstlisting}
\end{definition}

\begin{theorem} (Elimination $\text{setQuot}$)
For a type family $B : \text{setQuot } A R \to \mathcal{U}$ with $\text{isSet}(B x)$, and maps $f : (x : A) \to B(\text{quotient } x)$, $g : (a, b : A) \to (r : R a b) \to \text{PathP} (\langle i \rangle B(\text{identification } a b r @ i)) (f a) (f b)$, there exists $\text{setQuotElim} : \Pi_{x:\text{setQuot } A R} B(x)$, such that $\text{setQuotElim}(\text{quotient } a) = f a$.
\end{theorem}

\subsection{Groupoid-Quotients}
Groupoid quotients extend set quotients to produce a 1-groupoid, incorporating higher path constructors.

\begin{definition} (Groupoid Quotient)
For a type $A : \mathcal{U}$ and a relation $R : A \to A \to \mathcal{U}$, the groupoid quotient $\text{grpdQuot } A R : \mathcal{U}$ includes constructors for points, paths, and higher paths ensuring 1-groupoid structure. (Note: Full definition requires additional structure, partially omitted for brevity.)
\end{definition}

\newpage

\subsection{Wedge Sum}
The wedge sum of two pointed types $A$ and $B$, denoted $A \vee B$, is a higher inductive type (HIT) that represents the union of $A$ and $B$ with their base points identified. Topologically, it corresponds to $A \times \{ y_0 \} \cup \{ x_0 \} \times B$, where $x_0$ and $y_0$ are the base points of $A$ and $B$, respectively.

\begin{definition} (Formation)
For pointed types $A, B : \text{pointed}$, the wedge sum $\text{Wedge } A B : \mathcal{U}$.
\end{definition}

\begin{definition} (Introduction)
The wedge sum is generated by the following higher inductive composition structure:
\[
\begin{cases}
\text{winl} : A.1 \to \text{Wedge } A B \\
\text{winr} : B.1 \to \text{Wedge } A B \\
\text{wglue} : \text{Path}_{\text{Wedge } A B} (\text{winl } A.2, \text{winr } B.2)
\end{cases}
\]
\begin{lstlisting}
data Wedge (A : pointed) (B : pointed)
   = winl (a : A.1)
   | winr (b : B.1)
   | wglue <x> [ (x = 0) -> winl A.2 , (x = 1) -> winr B.2 ]
\end{lstlisting}
\end{definition}

\begin{theorem} (Elimination)
For a type $C : \mathcal{U}$, maps $f : A.1 \to C$, $g : B.1 \to C$, and a path $p : \text{Path}_C (f (A.2), g (B.2))$, there exists a map $\text{WedgeRec} : \text{Wedge } A B \to C$ such that:
\[
\begin{cases}
\text{WedgeRec}(\text{winl } a) = f(a) \\
\text{WedgeRec}(\text{winr } b) = g(b) \\
\text{WedgeRec}(\text{wglue } @ \, x) = p \, @ \, x
\end{cases}
\]
\begin{lstlisting}
WedgeRec (A B : pointed) (C : U) (f : A.1 -> C) (g : B.1 -> C)
         (p : Path C (f A.2) (g B.2))
   : Wedge A B -> C
   = split
     winl a -> f a
     winr b -> g b
     wglue @ x -> p @ x
\end{lstlisting}
\end{theorem}

\begin{theorem} (Computation)
For $z : \text{Wedge } A B$,
\[
\begin{cases}
\text{WedgeRec}(\text{winl } a) \equiv f(a) \\
\text{WedgeRec}(\text{winr } b) \equiv g(b) \\
\text{WedgeRec}(\text{wglue } @ \, x) \equiv p \, @ \, x
\end{cases}
\]
\end{theorem}

\begin{theorem} (Uniqueness)
Any two maps $h_1, h_2 : \text{Wedge } A B \to C$ are homotopic if they agree on $\text{winl}$, $\text{winr}$, and $\text{wglue}$, i.e., if $h_1(\text{winl } a) = h_2(\text{winl } a)$ for all $a : A.1$, $h_1(\text{winr } b) = h_2(\text{winr } b)$ for all $b : B.1$, and $h_1(\text{wglue}) = h_2(\text{wglue})$.
\end{theorem}

\newpage
\subsection{Smash Product}
The smash product of two pointed types $A$ and $B$, denoted $A \wedge B$, is a higher inductive type that quotients the product $A \times B$ by the wedge sum $A \vee B$. It represents the space $A \times B / (A \times \{ y_0 \} \cup \{ x_0 \} \times B)$, collapsing the wedge to a single point.

\begin{definition} (Formation)
For pointed types $A, B : \text{pointed}$, the smash product $\text{Smash } A B : \mathcal{U}$.
\end{definition}

\begin{definition} (Introduction)
The smash product is generated by the following higher inductive composition structure:
\[
\begin{cases}
\text{spair} : A.1 \to B.1 \to \text{Smash } A B \\
\text{smash} : (a : A.1) \to (b : B.1) \to \text{Path}_{\text{Smash } A B} (\text{spair } a \, B.2, \text{spair } A.2 \, b) \\
\text{smashpt} : \text{Path}_{\text{Smash } A B} (\text{smash } A.2 \, B.2, \text{spair } A.2 \, B.2)
\end{cases}
\]
\begin{lstlisting}
data Smash (A : pointed) (B : pointed)
   = spair (a : A.1) (b : B.1)
   | smash (a : A.1) (b : B.1) <x> [(x=0) -> spair a B.2, (x=1) -> spair A.2 b]
   | smashpt <x y> [(x=0) -> smash A.2 B.2 @ y,
                    (x=1) -> spair A.2 B.2,
                    (y=0) -> spair A.2 B.2,
                    (y=1) -> spair A.2 B.2]
\end{lstlisting}
\end{definition}

\begin{theorem} (Elimination)
For a type $C : \mathcal{U}$, a map $f : A.1 \to B.1 \to C$, paths $g : (a : A.1) \to (b : B.1) \to \text{Path}_C (f a \, B.2, f A.2 \, b)$, and a 2-path $h : \text{Path}_{\text{Path}_{\text{Smash } A B} (f A.2 \, B.2, f A.2 \, B.2)} (g A.2 \, B.2, \text{idp } (f A.2 \, B.2))$, there exists a map $\text{SmashRec} : \text{Smash } A B \to C$ such that:
\[
\begin{cases}
\text{SmashRec}(\text{spair } a \, b) = f(a, b) \\
\text{SmashRec}(\text{smash } a \, b \, @ \, x) = g(a, b) \, @ \, x \\
\text{SmashRec}(\text{smashpt } @ \, x \, @ \, y) = h \, @ \, x \, @ \, y
\end{cases}
\]
\end{theorem}

\begin{theorem} (Computation)
For $z : \text{Smash } A B$,
\[
\begin{cases}
\text{SmashRec}(\text{spair } a \, b) \equiv f(a, b) \\
\text{SmashRec}(\text{smash } a \, b \, @ \, x) \equiv g(a, b) \, @ \, x \\
\text{SmashRec}(\text{smashpt } @ \, x \, @ \, y) \equiv h \, @ \, x \, @ \, y
\end{cases}
\]
\end{theorem}

\begin{theorem} (Uniqueness)
Any two maps $h_1, h_2 : \text{Smash } A B \to C$ are homotopic if they agree on $\text{spair}$, $\text{smash}$, and $\text{smashpt}$.
\end{theorem}

\begin{example} (Smash of Spheres)
The smash product $S^1 \wedge S^1$ is homotopy equivalent to $S^2$, as it quotients the torus $S^1 \times S^1$ by the wedge $S^1 \vee S^1$, collapsing the base points and their fibers.
\end{example}

\newpage
\subsection{Join}
The join of two types $A$ and $B$, denoted $A * B$, is a higher inductive type that constructs a type by connecting every point of $A$ to every point of $B$ with a path. Topologically, it corresponds to the join of spaces, forming a space that interpolates between $A$ and $B$.

\begin{definition} (Formation)
For types $A, B : \mathcal{U}$, the join $\text{Join } A B : \mathcal{U}$.
\end{definition}

\begin{definition} (Introduction)
The join is generated by the following higher inductive composition structure:
\[
\begin{cases}
\text{joinl} : A \to \text{Join } A B \\
\text{joinr} : B \to \text{Join } A B \\
\text{join} : (a : A) \to (b : B) \to \text{Path}_{\text{Join } A B} (\text{joinl } a, \text{joinr } b)
\end{cases}
\]
\begin{lstlisting}
data Join (A : U) (B : U)
   = joinl (a : A)
   | joinr (b : B)
   | join (a:A) (b:B) <i> [(i=0) -> joinl a, (i=1) -> joinr b]
\end{lstlisting}
\end{definition}

\begin{theorem} (Elimination)
For a type $C : \mathcal{U}$, maps $f : A \to C$, $g : B \to C$, and a family of paths $h : (a : A) \to (b : B) \to \text{Path}_C (f a, g b)$, there exists a map $\text{JoinRec} : \text{Join } A B \to C$ such that:
\[
\begin{cases}
\text{JoinRec}(\text{joinl } a) = f(a) \\
\text{JoinRec}(\text{joinr } b) = g(b) \\
\text{JoinRec}(\text{join } a \, b \, @ \, i) = h(a, b) \, @ \, i
\end{cases}
\]
\begin{lstlisting}
JoinRec (A B C : U) (f : A -> C) (g : B -> C)
        (h : (a : A) -> (b : B) -> Path C (f a) (g b))
   : Join A B -> C
   = split
     joinl a -> f a
     joinr b -> g b
     join a b @ i -> h a b @ i
\end{lstlisting}
\end{theorem}

\begin{theorem} (Computation)
For $z : \text{Join } A B$,
\[
\begin{cases}
\text{JoinRec}(\text{joinl } a) \equiv f(a) \\
\text{JoinRec}(\text{joinr } b) \equiv g(b) \\
\text{JoinRec}(\text{join } a \, b \, @ \, i) \equiv h(a, b) \, @ \, i
\end{cases}
\]
\end{theorem}

\begin{theorem} (Uniqueness)
Any two maps $h_1, h_2 : \text{Join } A B \to C$ are homotopic if they agree on $\text{joinl}$, $\text{joinr}$, and $\text{join}$.
\end{theorem}

\begin{example} (Join of Spheres)
The join $S^0 * S^0$ is homotopy equivalent to $S^1$, as it connects two points (from each $S^0$) with paths, forming a circle-like structure.
\end{example}


\newpage

\subsection{Colimits}
Colimits construct the limit of a sequence of types connected by maps, such as propositional truncations.

\begin{definition} (Colimit)
For a sequence of types $A : \text{nat} \to \mathcal{U}$ and maps $f : (n : \text{nat}) \to A n \to A (\text{succ } n)$, the colimit type $\text{colimit } A f : \mathcal{U}$.
\[
\begin{cases}
\text{ix} : (n : \text{nat}) \to A n \to \text{colimit } A f \\
\text{gx} : (n : \text{nat}) \to (a : A n) \to \text{Path}_{\text{colimit } A f} (\text{ix} (\text{succ } n) (f n a), \text{ix} n a)
\end{cases}
\]
\begin{lstlisting}
data colimit (A : nat -> U)
             (f : (n : nat) -> A n -> A (succ n))
   = ix (n : nat) (x: A n)
   | gx (n : nat) (a: A n) <i> [ (i=0) -> ix (succ n) (f n a)
                               , (i=1) -> ix n a]
\end{lstlisting}
\end{definition}

\begin{theorem} (Elimination $\text{colimit}$)
For a type $P : \text{colimit } A f \to \mathcal{U}$, with $p : (n : \text{nat}) \to (x : A n) \to P(\text{ix}(n, x))$ and $q : (n : \text{nat}) \to (a : A n) \to \text{PathP} (\langle i \rangle P(\text{gx}(n, a) @ i)) (p (\text{succ } n) (f n a)) (p n a)$, there exists $i : \Pi_{x:\text{colimit } A f} P(x)$, such that $i(\text{ix}(n, x)) = p n x$.
\end{theorem}

\newpage

\subsection{Equalizer}
The equalizer of two maps $f, g : A \to B$ is a higher inductive type (HIT) that constructs a type consisting of elements in $B$ where $f$ and $g$ agree, along with paths enforcing this equality. It is a fundamental construction in homotopy type theory, capturing the subspace of $B$ where $f(a) = g(a)$ for $a : A$.

\begin{definition} (Formation)
For types $A, B : \mathcal{U}$ and maps $f, g : A \to B$, the equalizer $\text{coeq } A B f g : \mathcal{U}$.
\end{definition}

\begin{definition} (Introduction)
The equalizer is generated by the following higher inductive composition structure:
\[
\begin{cases}
\text{inC} : B \to \text{coeq } A B f g \\
\text{glueC} : (a : A) \to \text{Path}_{\text{coeq } A B f g} (\text{inC } (f a), \text{inC } (g a))
\end{cases}
\]
\begin{lstlisting}
data coeq (A B: U) (f g: A -> B)
   = inC (_: B)
   | glueC (a: A) <i> [(i=0) -> inC (f a), (i=1) -> inC (g a) ]
\end{lstlisting}
\end{definition}

\begin{theorem} (Elimination)
For a type $C : \mathcal{U}$, a map $h : B \to C$, and a family of paths $y : (x : A) \to \text{Path}_C (h (f x), h (g x))$, there exists a map $\text{coequRec} : \text{coeq } A B f g \to C$ such that:
\[
\begin{cases}
\text{coequRec}(\text{inC } x) = h(x) \\
\text{coequRec}(\text{glueC } x \, @ \, i) = y(x) \, @ \, i
\end{cases}
\]
\begin{lstlisting}
coequRec (A B C : U) (f g : A -> B) (h: B -> C) (y: (x : A) -> Path C (h (f x)) (h (g x)))
   : (z : coeq A B f g) -> C
   = split
     inC x -> h x
     glueC x @ i -> y x @ i
\end{lstlisting}
\end{theorem}

\begin{theorem} (Computation)
For $z : \text{coeq } A B f g$,
\[
\begin{cases}
\text{coequRec}(\text{inC } x) \equiv h(x) \\
\text{coequRec}(\text{glueC } x \, @ \, i) \equiv y(x) \, @ \, i
\end{cases}
\]
\end{theorem}

\begin{theorem} (Uniqueness)
Any two maps $h_1, h_2 : \text{coeq } A B f g \to C$ are homotopic if they agree on $\text{inC}$ and $\text{glueC}$, i.e., if $h_1(\text{inC } x) = h_2(\text{inC } x)$ for all $x : B$ and $h_1(\text{glueC } a) = h_2(\text{glueC } a)$ for all $a : A$.
\end{theorem}

\begin{example} (Equalizer as Subspace)
The equalizer $\text{coeq } A B f g$ represents the subspace of $B$ where $f(a) = g(a)$. For example, if $A = B = \mathbb{R}$ and $f(x) = x^2$, $g(x) = x$, the equalizer captures points where $x^2 = x$, i.e., $\{0, 1\}$.
\end{example}

\subsection{Path-Equalizer}
The path-equalizer is a higher inductive type that generalizes the equalizer to handle pairs of paths in $B$. Given a map $p : A \to (b_1, b_2 : B) \times (\text{Path}_B (b_1, b_2)) \times (\text{Path}_B (b_1, b_2))$, it constructs a type where elements of $A$ induce pairs of paths between points in $B$, with paths connecting the endpoints of these paths.

\begin{definition} (Formation)
For types $A, B : \mathcal{U}$ and a map $p : A \to (b_1, b_2 : B) \times (\text{Path}_B (b_1, b_2)) \times (\text{Path}_B (b_1, b_2))$, the path-equalizer $\text{coeqP } A B p : \mathcal{U}$.
\end{definition}

\begin{definition} (Introduction)
The path-equalizer is generated by the following higher inductive composition structure:
\[
\begin{cases}
\text{inP} : B \to \text{coeqP } A B p \\
\text{glueP} : (a : A) \to \text{Path}_{\text{coeqP } A B p} (\text{inP } (((p \, a).2.2.1) \, @ \, 0), \text{inP } (((p \, a).2.2.2) \, @ \, 1))
\end{cases}
\]
\begin{lstlisting}
data coeqP (A B: U) (p : A -> (b1 b2: B) * (_: Path B b1 b2) * (Path B b1 b2))
   = inP (b: B)
   | glueP (a:A) <i> [(i=0) -> inP (((p a).2.2.1) @ 0), (i=1) -> inP (((p a).2.2.2) @ 1) ]
\end{lstlisting}
\end{definition}

\begin{theorem} (Elimination)
For a type $C : \mathcal{U}$, a map $h : B \to C$, and a family of paths $y : (a : A) \to \text{Path}_C (h (((p \, a).2.2.1) \, @ \, 0), h (((p \, a).2.2.2) \, @ \, 1))$, there exists a map $\text{coequPRec} : \text{coeqP } A B p \to C$ such that:
\[
\begin{cases}
\text{coequPRec}(\text{inP } b) = h(b) \\
\text{coequPRec}(\text{glueP } a \, @ \, i) = y(a) \, @ \, i
\end{cases}
\]
\begin{lstlisting}
coequPRec (A B C : U) (p : A -> (b1 b2: B) * (_: Path B b1 b2) * (Path B b1 b2))
          (h: B -> C) (y: (a : A) -> Path C (h (((p a).2.2.1) @ 0)) (h (((p a).2.2.2) @ 1)))
   : (z : coeqP A B p) -> C
   = split
     inP b -> h b
     glueP a @ i -> y a @ i
\end{lstlisting}
\end{theorem}

\begin{theorem} (Computation)
For $z : \text{coeqP } A B p$,
\[
\begin{cases}
\text{coequPRec}(\text{inP } b) \equiv h(b) \\
\text{coequPRec}(\text{glueP } a \, @ \, i) \equiv y(a) \, @ \, i
\end{cases}
\]
\end{theorem}

\begin{theorem} (Uniqueness)
Any two maps $h_1, h_2 : \text{coeqP } A B p \to C$ are homotopic if they agree on $\text{inP}$ and $\text{glueP}$, i.e., if $h_1(\text{inP } b) = h_2(\text{inP } b)$ for all $b : B$ and $h_1(\text{glueP } a) = h_2(\text{glueP } a)$ for all $a : A$.
\end{theorem}

\begin{example} (Path-Equalizer for Homotopy)
The path-equalizer can model spaces where elements of $A$ specify pairs of paths between points in $B$. For instance, if $p(a)$ provides two paths from $b_1$ to $b_2$ in $B$, $\text{coeqP}$ constructs a type connecting the starting and ending points of these paths, useful in studying homotopy classes.
\end{example}

\newpage

\subsection{K(G,n)}
Eilenberg-MacLane spaces $K(G,n)$ have a single non-trivial homotopy group $\pi_n(K(G,n)) = G$. They are defined using truncations and suspensions.

\begin{definition} (K(G,n))
For an abelian group $G : \text{abgroup}$, the type $KGn G : \text{nat} \to \mathcal{U}$.
\[
\begin{cases}
n = 0: KGn G 0 = \text{discreteTopology}(G) \\
n \geq 1: KGn G (\text{succ } n) = \text{nTrunc}(\text{suspension}(K1' (G.1, G.2.1)) n) (\text{succ } n)
\end{cases}
\]
\begin{lstlisting}
KGn (G: abgroup)
  : nat -> U
  = split
    zero -> discreteTopology G
    succ n -> nTrunc (suspension (K1' (G.1,G.2.1)) n) (succ n)
\end{lstlisting}
\end{definition}

\begin{theorem} (Elimination $KGn$)
For $n \geq 1$, a type $B : \mathcal{U}$ with $\text{isNGroupoid}(B, \text{succ } n)$, and a map $f : \text{suspension}(K1' G) \to B$, there exists $\text{rec}_{KGn} : KGn G (\text{succ } n) \to B$, defined via $\text{nTruncRec}$.
\end{theorem}

\newpage

\subsection{Localization}
Localization constructs an $F$-local type from a type $X$, with respect to a family of maps $F_A : S(a) \to T(a)$.

\begin{definition} (Localization Modality)
For a family of maps $F_A : S(a) \to T(a)$, the $F$-localization $L^{AST}_F(X) : \mathcal{U}$.
\[
\begin{cases}
\text{center} : X \to L_{F_A}(X) \\
\text{ext} : (a : A) \to (S(a) \to L_{F_A}(X)) \to T(a) \to L_{F_A}(X) \\
\text{isExt} : (a : A) \to (f : S(a) \to L_{F_A}(X)) \to (s : S(a)) \to \text{Path}_{L_{F_A}(X)} (\text{ext } a f (F a s), f s) \\
\text{extEq} : (a : A) \to (g, h : T(a) \to L_{F_A}(X)) \to (p : (s : S(a)) \to \text{Path}_{L_{F_A}(X)} (g (F a s), h (F a s))) \to (t : T(a)) \to \text{Path}_{L_{F_A}(X)} (g t, h t) \\
\text{isExtEq} : (a : A) \to (g, h : T(a) \to L_{F_A}(X)) \to (p : (s : S(a)) \to \text{Path}_{L_{F_A}(X)} (g (F a s), h (F a s))) \to (s : S(a)) \to \text{Path}_{L_{F_A}(X)} (\text{extEq } a g h p (F a s), p s)
\end{cases}
\]
\begin{lstlisting}
data Localize (A X: U) (S T: A -> U) (F : (x:A) -> S x -> T x)
   = center (x: X)
   | ext (a: A) (f: S a -> Localize A X S T F) (t: T a)
   | isExt (a: A) (f: S a -> Localize A X S T F) (s: S a) <i>
     [ (i=0) -> ext a f (F a s) , (i=1) -> f s ]
   | extEq (a: A) (g h: T a -> Localize A X S T F)
     (p: (s : S a) -> Path (Localize A X S T F) (g (F a s)) (h (F a s)))
     (t : T a) <i> [ (i=0) -> g t , (i=1) -> h t ]
   | isExtEq (a: A) (g h : T a -> Localize A X S T F)
     (p: (s : S a) -> Path (T a -> Localize A X S T F) (g (F a s)) (h (F a s)))
     (s : S a) <i> [ (i=0) -> extEq a g h p (F a s) , (i=1) -> p s ]
\end{lstlisting}
\end{definition}

\begin{theorem} (Localization Induction)
For any $P : \Pi_{X:U} L_{F_A}(X) \to U$ with $\{n, r, s\}$ satisfying coherence conditions, there exists $i : \Pi_{x:L_{F_A}(X)} P(x)$ such that $i \cdot \text{center}_X = n$.
\end{theorem}


\section{Conclusion}
HITs encode CW-complexes in HoTT, bridging topology and type theory. They capture cell attachments, with examples like spheres, tori, and truncations. Infinity constructors extend HITs to infinite spaces, handled by Agda Cubical’s primitives and indexed HITs.

\bibliographystyle{plain}
\begin{thebibliography}{9}
\bibitem{univalent}
The Univalent Foundations Program, \emph{Homotopy Type Theory: Univalent Foundations of Mathematics}, IAS, 2013.
\bibitem{cubicaltt}
C. Cohen, T. Coquand, S. Huber, A. Mörtberg, \emph{Cubical Type Theory}, Journal of Automated Reasoning, 2018.
\bibitem{agda-cubical}
A. Mörtberg et al., \emph{Agda Cubical Library}, \url{https://github.com/agda/cubical}, 2023.
\bibitem{hott-book}
M. Shulman, \emph{Higher Inductive Types in HoTT}, \url{https://arxiv.org/abs/1705.07088}, 2017.
\bibitem{localization}
J. D. Christensen, M. Opie, E. Rijke, L. Scoccola, \emph{Localization in Homotopy Type Theory}, \url{https://arxiv.org/pdf/1807.04155.pdf}, 2018.
\bibitem{modalities}
E. Rijke, M. Shulman, B. Spitters, \emph{Modalities in Homotopy Type Theory}, \url{https://arxiv.org/pdf/1706.07526v6.pdf}, 2017.
\bibitem{synthetic-spectra}
M. Riley, E. Finster, D. R. Licata, \emph{Synthetic Spectra via a Monadic and Comonadic Modality}, \url{https://arxiv.org/pdf/2102.04099.pdf}, 2021.
\end{thebibliography}

\end{document}
