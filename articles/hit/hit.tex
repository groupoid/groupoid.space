% copyright (c) 2018 Groupoid Infinity

\documentclass{article}
\usepackage{listings}
\usepackage{amsmath}
\usepackage{amssymb}
\usepackage{amsthm}
\usepackage{url}
\usepackage{tikz-cd}
\usepackage[utf8]{inputenc}

\theoremstyle{definition}
\newtheorem{definition}{Definition}
\newtheorem{theorem}{Theorem}
\newcommand*{\incmap}{\hookrightarrow}
\newcommand*{\thead}[1]{\multicolumn{1}{c}{\bfseries #1}}
\lstset{basicstyle=\small,inputencoding=utf8}

\begin{document}

\title{Issue IV: Higher Inductive Types}
\author{Maxim Sokhatsky $^1$}
\date{ $^1$ National Technical University of Ukraine \\
       \small Igor Sikorsky Kyiv Polytechnical Institute \\
       \today }

\maketitle

\begin{abstract}
CW-complexes are fundamental objects in homotopy type theory
and even included inside cubical type checker in a form of
higher inductive types (HIT). Just like regular (co)-inductive
types could be described as recursive terminating (well-founded)
or non-terminating trees, higher inductive types could be described as CW-complexes.
Defining HIT means to define some CW-complex
directly using cubical homogeneous composition structure as an
element of initial algebra inside cubical model.
\\
\\
{\bf Keywords}: Collular Piecewise Topology, Cubical Type Theory, Higher Inductive Types
\end{abstract}
\tableofcontents

\newpage
\section{Higher Inductive Types}
CW-complexes are fundamental objects in homotopy type theory
and even included inside cubical type checker in a form of
higher (co)-inductive types (HITs).
Just like regular (co)-inductive types could be described as recursive
terminating (well-founded) or non-terminating trees,
higher inductive types could be described as CW-complexes.
Defining HIT means to define some CW-complex
directly using cubical homogeneous composition structure as an
element of initial algebra inside cubical model.

\begin{definition} (Pushout). One of the notable examples is pushout as it's used
to define the cell attachment formally, as others cofibrant objects.
\begin{lstlisting}
data pushout (A B C: U) (f: C -> A) (g: C -> B)
   = po1 (_: A)
   | po2 (_: B)
   | po3 (c: C) <i> [ (i = 0) -> po1 (f c) ,
                      (i = 1) -> po2 (g c) ]
\end{lstlisting}
\end{definition}

\begin{definition} (Shperes and Disks).
Here are some example of using dimensions to construct spherical shapes.
\begin{lstlisting}
data S1
   = base
   | loop <i> [ (i = 0) -> base,
                (i = 1) -> base ]
\end{lstlisting}
\begin{lstlisting}
data S2
   = point
   | surf <i j> [ (i = 0) -> point, (i = 1) -> point,
                  (j = 0) -> point, (j = 1) -> point ]
                  (j = 0) -> point, (j = 1) -> point ]
\end{lstlisting}
\end{definition}

\section{CW-Complexes}

The definition of homotopy groups, a special role is played
by the inclusions $S^{n-1} \incmap D^n$. We study spaces
obtained iterated attachments of $D^n$ along $S^{n-1}$.

\begin{definition} (Attachment).
Attaching n-cell to a space $X$
along a map $f : S^{n-1} \rightarrow X$ means taking a pushout figure.
$$
\begin{tikzcd}
 & S^{n-1} \arrow[r, "k"] \arrow[d, ""]
 & X \arrow[d, ""] \\
 & D^n \arrow[r, "g"] & \cup_f D^n
\end{tikzcd}
$$
where the notation $X \cup_f D^n$ means result depends
on homotopy class of $f$.
\end{definition}

\begin{definition} (CW-Complex).
Inductively. The only CW-complex of dimention $-1$ is $\varnothing$.
A CW-complex of dimension $\leqslant n$ on $X$ is a
space $X$ obtained by attaching a collection of n-cells
to a CW-complex of dimension $n-1$.

A CW-complex is a space $X$ which is the $colimit(X_i)$ of a
sequence $X_{-1} = \varnothing \incmap X_0 \incmap X_1 \incmap X_2 \incmap ... X$ of
CW-complexes $X_i$ of dimension $\leqslant n$, with $X_{i+1}$
obtained from $X_i$ by i-cell attachments.
Thus if X is a CW-complex, it comes with a filtration
$$
    \varnothing \incmap X_0 \incmap X_1 \incmap X_2 \incmap ... X
$$
where $X_i$ is a CW-complex of dimension $\leqslant i$ called
the i-skeleton, and hence the filtration is called the skeletal
filtration.
\end{definition}

\bibliographystyle{plain}
\bibliography{cwf}

\end{document}

