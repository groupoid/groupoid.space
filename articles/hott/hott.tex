% copyright (c) 2018 Groupoid Infinity

\documentclass{article}
\usepackage{listings}
\usepackage{amsmath}
\usepackage{amssymb}
\usepackage{float}
\usepackage{amsthm}
\usepackage{url}
\usepackage{tikz-cd}
\usepackage[utf8]{inputenc}

\theoremstyle{definition}
\newtheorem{definition}{Definition}
\newtheorem{theorem}{Theorem}
\newcommand*{\incmap}{\hookrightarrow}
\newcommand*{\thead}[1]{\multicolumn{1}{c}{\bfseries #1}}
\lstset{basicstyle=\small,inputencoding=utf8}

\begin{document}

\title{Issue III: Homotopy Type Theory}
\author{Maxim Sokhatsky $^1$}
\date{ $^1$ National Technical University of Ukraine \\
       \small Igor Sikorsky Kyiv Polytechnical Institute \\
       \today }

\maketitle

\begin{abstract}
Here is presented destinctive points of Homotopy Type Theory
as an extension of Martin-Löf Type Theory but without higher inductive types
which will be given in the next issue. The fibrational (geometric) interpretation
of equivalence type is introduced with following univalence releation
between equivalence and Path equality. Groupoid (categorical) interpretation
is presented as categories of spaces and paths between them as invertible morphisms.
At last constructive proof $\Omega(S^1)=\mathbb{Z}$ is given through helix.
\\
\\
{\bf Keywords}: Homotopy Type Theory
\end{abstract}
\tableofcontents

\newpage
\section{Homotypy Type Theory}

Homotypy Type Theory takes its origins in 1996 from groupoid interpretation by
Hofmann and Streicher's, and later (in 10 years) was formalized by Awodey,
Warren and Voevodsky. Voevodsky constrtucted Kan simplicial sets interpretation
of type theory and discovered the property of this model, that was named univalence.
This property allows to identify isomorphic structures in terms of type theory.

Homotopy type theory to classical homotopy theory is like Euclidian
syntethic geometry (points, lines, axioms and deduction rules) to
analytical geometry with cartesian coordinates on $\mathbb{R}^n$ (geometric and algebraic)
\footnote{We will denote geometric, type theoretical and homotopy constants bold font ${\bf R}$ while
analitical will be denoted with double lined letters $\mathbb{R}$.}

In the same way as inductive types extends MLTT for inductive programming,
the higher inductive types (HIT) extend homotopy type theory for geometry programming.
You can directly encode CW-complexes by using HIT. The definition of HIT syntax will
be given in the next {\bf Issue IV: Higher Inductive Types}.

\subsection{Homotopies}
The first higher equality we meet in homotopy theory is a notion of homotopy,
where we compare two functions or two path spaces (which is sort of dependent families).
The homotopy interval $\mathrm{I}=[0,1]$ is the perfect foundation for definition of homotopy.

\begin{definition} (Interval). Compact interval.
\begin{lstlisting}
data I = i0
       | i1
       | seg <i> [(i=0) -> i0,
                  (i=1) -> i1]
\end{lstlisting}
\end{definition}

You can think of ${\bf I}$ as isomorphism of equality type,
disregarding carriers on the edges. By mapping $i0,i1:{\bf I}$ to $x,y:A$ one can
obtain identity or equality type from classic type theory.

\begin{definition} (Interval Split).
The convertion function from $\mathrm{I}$ to a type of comparison
is a direct eliminator of interval. The interval is also known as one of
primitive higher inductive types which will be given in the next
{\bf Issue IV: Higher Inductive Types}.
\begin{lstlisting}
pathToHtpy (A: U) (x y: A) (p: Path A x y): I -> A
   = split { i0 -> x; i1 -> y; seg @ i -> p @ i }
\end{lstlisting}
\end{definition}

\begin{definition} (Homotopy). The homotopy between two function $f,g: X \rightarrow Y$
is a continuous map of cylinder $H : X \times {\bf I} \rightarrow Y$ such that
$$
\begin{cases}
H(x,0)=f(x), \\
H(x,1)=g(x).
\end{cases}
$$
\begin{lstlisting}
homotopy (X Y: U) (f g: X -> Y)
         (p: (x: X) -> Path Y (f x) (g x))
         (x: X): I -> Y = pathToHtpy Y (f x) (g x) (p x)
\end{lstlisting}
\end{definition}

\subsection{Groupoid Interpretation}
The first text about groupoid interpretation of type theory can be found in Francois Lamarche:
A proposal about Foundations\footnote{\url{http://www.cse.chalmers.se/~coquand/Proposal.pdf}}.
Then Martin Hofmann and Thomas Streicher wrote the initial
document on groupoid interpretation of type
theory\footnote{Martin Hofmann and Thomas Streicher. The Groupoid Interpretation of Type Theory. 1996.}.

\begin{table}[H]
\begin{center}
\begin{tabular}{lccc}
\hline
{\bf Equality} & {\bf Homotopy} & {\bf $\infty$-Groupoid} \\
\hline
reflexivity  & constant path & identity morphism \\
symmetry     & inversion of path & inverse morphism \\
transitivity & concatenation of paths & composition of mopphisms \\
\hline
\end{tabular}
\end{center}
\end{table}

There is a deep connection between higher-dimentinal groupoids in category theory and
spaces in homotopy theory, equipped with some topology. The category or groupoid could
be built where the objects are particular spaces or types, and morphisms are path types
between these types, composition operation is a path concatenation. We can write this
groupoid here recalling that it should be category with inverted morphisms.

\begin{lstlisting}
cat: U = (A: U) * (A -> A -> U)
groupoid: U = (X: cat) * isCatGroupoid X
PathCat (X: U): cat = (X,\(x y:X)->Path X x y)
\end{lstlisting}

\begin{lstlisting}
isCatGroupoid (C: cat): U
  = (id: (x: C.1) -> C.2 x x)
  * (c: (x y z:C.1) -> C.2 x y -> C.2 y z -> C.2 x z)
  * (inv: (x y: C.1) -> C.2 x y -> C.2 y x)
  * (inv_left:  (x y: C.1) (p: C.2 x y) ->
    Path (C.2 x x) (c x y x p (inv x y p)) (id x))
  * (inv_right: (x y: C.1) (p: C.2 x y) ->
    Path (C.2 y y) (c y x y (inv x y p) p) (id y))
  * (left: (x y: C.1) (f: C.2 x y) ->
    Path (C.2 x y) (c x x y (id x) f) f)
  * (right: (x y: C.1) (f: C.2 x y) ->
    Path (C.2 x y) (c x y y f (id y)) f)
  * ((x y z w:C.1)(f:C.2 x y)(g:C.2 y z)(h:C.2 z w)->
    Path (C.2 x w) (c x z w (c x y z f g) h)
                   (c x y w f (c y z w g h)))
\end{lstlisting}

\newpage
\begin{lstlisting}
PathGrpd (X: U)
  : groupoid
  = ((Ob,Hom),id,c,sym X,compPathInv X,compInvPath X,L,R,Q) where
    Ob: U = X
    Hom (A B: Ob): U = Path X A B
    id (A: Ob): Path X A A = refl X A
    c (A B C: Ob) (f: Hom A B) (g: Hom B C): Hom A C
      = comp (<i> Path X A (g@i)) f []
\end{lstlisting}
From here should be clear what it meant to be groupoid interpretation
of path type in type theory. In the same way we can construct categories of $\prod$ and $\sum$ types.
In {\bf Issue VIII: Topos Theory} such categories will be given.

\subsection{Functional Extensionality}

\begin{definition} (funExt-Formation)
\begin{lstlisting}
funext_form (A B: U) (f g: A -> B): U
  = Path (A -> B) f g
\end{lstlisting}
\end{definition}

\begin{definition} (funExt-Introduction)
\begin{lstlisting}
funext (A B: U) (f g: A -> B) (p: (x:A) -> Path B (f x) (g x))
  : funext_form A B f g
  = <i> \(a: A) -> p a @ i
\end{lstlisting}
\end{definition}

\begin{definition} (funExt-Elimination)
\begin{lstlisting}
happly (A B: U) (f g: A -> B) (p: funext_form A B f g) (x: A)
  : Path B (f x) (g x)
  = cong (A -> B) B (\(h: A -> B) -> apply A B h x) f g p
\end{lstlisting}
\end{definition}

\begin{definition} (funExt-Computation)
\begin{lstlisting}
funext_Beta (A B: U) (f g: A -> B) (p: (x:A) -> Path B (f x) (g x))
  : (x:A) -> Path B (f x) (g x)
  = \(x:A) -> happly A B f g (funext A B f g p) x
\end{lstlisting}
\end{definition}

\begin{definition} (funExt-Uniqueness)
\begin{lstlisting}
funext_Eta (A B: U) (f g: A -> B) (p: Path (A -> B) f g)
  : Path (Path (A -> B) f g) (funext A B f g (happly A B f g p)) p
  = refl (Path (A -> B) f g) p
\end{lstlisting}
\end{definition}

\subsection{Pullbacks}

\subsection{Fibrations}

\begin{definition} (Fibration-1) Dependent fiber bundle derived from Path contractability.
\begin{lstlisting}
isFBundle1 (B: U) (p: B -> U) (F: U): U
  = (_: (b: B) -> isContr (Path U (p b) F))
  * ((x: Sigma B p) -> B)
\end{lstlisting}
\end{definition}

\begin{definition} (Fibration-2). Dependent fiber bundle derived from surjective function.
\begin{lstlisting}
isFBundle2 (B: U) (p: B -> U) (F: U): U
  = (V: U)
  * (v: surjective V B)
  * ((x: V) -> Path U (p (v.1 x)) F)
\end{lstlisting}
\end{definition}

\begin{definition} (Fibration-3). Non-dependent fiber bundle derived from fiber truncation.
\begin{lstlisting}
im1 (A B: U) (f: A -> B): U = (b: B) * pTrunc ((a:A) * Path B (f a) b)
BAut (F: U): U = im1 unit U (\(x: unit) -> F)
unitIm1 (A B: U) (f: A -> B): im1 A B f -> B = \(x: im1 A B f) -> x.1
unitBAut (F: U): BAut F -> U = unitIm1 unit U (\(x: unit) -> F)

isFBundle3 (E B: U) (p: E -> B) (F: U): U
  = (X: B -> BAut F)
  * (classify B (BAut F) (\(b: B) -> fiber E B p b) (unitBAut F) X) where
  classify (A' A: U) (E': A' -> U) (E: A -> U) (f: A' -> A): U
    = (x: A') -> Path U (E'(x)) (E(f(x)))
\end{lstlisting}
\end{definition}

\begin{definition} (Fibration-4). Non-dependen fiber bundle derived as pullback square.
\begin{lstlisting}
isFBundle4 (E B: U) (p: E -> B) (F: U): U
  = (V: U)
  * (v: surjective V B)
  * (v': prod V F -> E)
  * pullbackSq (prod V F) E V B p v.1 v' (\(x: prod V F) -> x.1)
\end{lstlisting}
\end{definition}

\subsection{Equivalence}

\begin{definition} (Equivalence).
\begin{lstlisting}
fiber (A B: U) (f: A -> B) (y: B): U = (x: A) * Path B y (f x)
isSingleton (X:U): U = (c:X) * ((x:X) -> Path X c x)
isEquiv (A B: U) (f: A -> B): U = (y: B) -> isContr (fiber A B f y)
equiv (A B: U): U = (f: A -> B) * isEquiv A B f
\end{lstlisting}
\end{definition}

\begin{definition} (Surjective).
\begin{lstlisting}
isSurjective (A B: U) (f: A -> B): U
  = (b: B) * pTrunc (fiber A B f b)

surjective (A B: U): U
  = (f: A -> B)
  * isSurjective A B f
\end{lstlisting}
\end{definition}

\begin{definition} (Injective).
\begin{lstlisting}
isInjective' (A B: U) (f: A -> B): U
  = (b: B) -> isProp (fiber A B f b)

injective (A B: U): U
  = (f: A -> B)
  * isInjective A B f
\end{lstlisting}
\end{definition}

\begin{definition} (Embedding).
\begin{lstlisting}
isEmbedding (A B: U) (f: A -> B) : U
  = (x y: A) -> isEquiv (Path A x y) (Path B (f x) (f y)) (cong A B f x y)

embedding (A B: U): U
  = (f: A -> B)
  * isEmbedding A B f
\end{lstlisting}
\end{definition}

\begin{definition} (Half-adjoint Equivalence).
\begin{lstlisting}
isHae (A B: U) (f: A -> B): U
  = (g: B -> A)
  * (eta_: Path (id A) (o A B A g f) (idfun A))
  * (eps_: Path (id B) (o B A B f g) (idfun B))
  * ((x: A) -> Path B (f ((eta_ @ 0) x)) ((eps_ @ 0) (f x)))

hae (A B: U): U
  = (f: A -> B)
  * isHae A B f
\end{lstlisting}
\end{definition}

\subsection{Isomorphism}

\begin{definition} (iso-Formation)
\begin{lstlisting}
iso_Form (A B: U): U = isIso A B -> Path U A B
\end{lstlisting}
\end{definition}

\begin{definition} (iso-Introduction)
\begin{lstlisting}
iso_Intro (A B: U): iso_Form A B
\end{lstlisting}
\end{definition}

\begin{definition} (iso-Elimination)
\begin{lstlisting}
iso_Elim (A B: U): Path U A B -> isIso A B
\end{lstlisting}
\end{definition}

\begin{definition} (iso-Computation)
\begin{lstlisting}
iso_Comp (A B : U) (p : Path U A B)
  : Path (Path U A B) (iso_Intro A B (iso_Elim A B p)) p
\end{lstlisting}
\end{definition}

\begin{definition} (iso-Uniqueness)
\begin{lstlisting}
iso_Uniq (A B : U) (p: isIso A B)
  : Path (isIso A B) (iso_Elim A B (iso_Intro A B p)) p
\end{lstlisting}
\end{definition}

\subsection{Univalence}

\begin{definition} (uni-Formation)
\begin{lstlisting}
univ_Formation (A B: U): U = equiv A B -> Path U A B
\end{lstlisting}
\end{definition}

\begin{definition} (uni-Introduction)
\begin{lstlisting}
equivToPath (A B: U): univ_Formation A B
  = \(p: equiv A B) -> <i> Glue B [(i=0) -> (A,p),
    (i=1) -> (B, subst U (equiv B) B B (<_>B) (idEquiv B)) ]
\end{lstlisting}
\end{definition}

\begin{definition} (uni-Elimination)
\begin{lstlisting}
pathToEquiv (A B: U) (p: Path U A B) : equiv A B
  = subst U (equiv A) A B p (idEquiv A)
\end{lstlisting}
\end{definition}

\begin{definition} (uni-Computation)
\begin{lstlisting}
eqToEq (A B : U) (p : Path U A B)
  : Path (Path U A B) (equivToPath A B (pathToEquiv A B p)) p
  = <j i> let Ai: U = p@i in Glue B
    [ (i=0) -> (A,pathToEquiv A B p),
      (i=1) -> (B,pathToEquiv B B (<k> B)),
      (j=1) -> (p@i,pathToEquiv Ai B (<k> p @ (i \/ k))) ]

\end{lstlisting}
\end{definition}

\begin{definition} (uni-Uniqueness)
\begin{lstlisting}
transPathFun (A B : U) (w: equiv A B)
  : Path (A -> B) w.1 (pathToEquiv A B (equivToPath A B w)).1
\end{lstlisting}
\end{definition}

\subsection{Loop Spaces}

\begin{definition} (Pointed Space). A pointed type $(A,a)$ is a type $A:U$
together with a point $a:A$, called its basepoint.
\begin{lstlisting}
pointed: U = (A: U) * A
point (A: pointed): A.1 = A.2
space (A: pointed): U = A.1
\end{lstlisting}
\end{definition}

\begin{definition} (Loop Space).
$$\Omega(A,a) =_{def} ((a =_A a), refl_A(a)).$$
\begin{lstlisting}
omega1 (A: pointed) : pointed
  = (Path (space A) (point A) (point A), refl A.1 (point A))
\end{lstlisting}
\end{definition}

\begin{definition} (n-Loop Space).
$$
\begin{cases}
\Omega^0(A, a) =_{def} (A, a)\\
\Omega^{n+1}(A,a) =_{def} \Omega^{n}(\Omega(A,a))\\
\end{cases}
$$
\begin{lstlisting}
omega : nat -> pointed -> pointed = split
  zero -> idfun pointed
  succ n -> \(A: pointed) -> omega n (omega1 A)
\end{lstlisting}
\end{definition}

\subsection{Homotopy Groups}

\begin{definition} (n-th Homotopy Group of m-Sphere).
$$\pi_{n}S^{m} = ||\Omega^{n}(S^{m})||_0.$$
\begin{lstlisting}
piS (n: nat): (m: nat) -> U = split
   zero   -> sTrunc (space (omega n (bool,false)))
   succ x -> sTrunc (space (omega n (Sn (succ x),north)))
\end{lstlisting}
\end{definition}

\begin{theorem} ($\Omega(S^1)=\mathbb{Z}$).
\begin{lstlisting}
data S1 = base
        | loop <i> [ (i=0) -> base ,
                     (i=1) -> base ]

loopS1 : U = Path S1 base base

encode (x:S1) (p:Path S1 base x)
  : helix x
  = subst S1 helix base x p zeroZ

decode : (x:S1) -> helix x -> Path S1 base x = split
  base -> loopIt
  loop @ i -> rem @ i where
    p : Path U (Z -> loopS1) (Z -> loopS1)
      = <j> helix (loop1@j) -> Path S1 base (loop1@j)
    rem : PathP p loopIt loopIt
      = corFib1 S1 helix (\(x:S1)->Path S1 base x) base
        loopIt loopIt loop1 (\(n:Z) ->
        comp (<i> Path loopS1 (oneTurn (loopIt n))
             (loopIt (testIsoPath Z Z sucZ predZ
                      sucpredZ predsucZ n @ i)))
             (<i>(lem1It n)@-i) [])

loopS1eqZ : Path U Z loopS1
  = isoPath Z loopS1 (decode base) (encode base)
    sectionZ retractZ
\end{lstlisting}
\end{theorem}

\bibliographystyle{plain}
\bibliography{hott}

\end{document}

