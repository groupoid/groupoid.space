% copyright (c) 2018 Synrc Research Center

\documentclass{article}
\usepackage{listings}
\usepackage{amsmath}
\usepackage{amssymb}
\usepackage{amsthm}
\usepackage{url}
\usepackage{tikz-cd}
\usepackage[utf8]{inputenc}
\lstset{basicstyle=\small,inputencoding=utf8}

\begin{document}

\title{Issue VII: Inductive Types}
\author{Maksym Sokhatskyi $^1$}
\date{ $^1$ National Technical University of Ukraine \\
       \small Igor Sikorsky Kyiv Polytechnical Institute \\
       \today }

\maketitle

\begin{abstract}

Impredicative Encoding of Inductive Types in HoTT.
\\
\\
{\bf Keywords}: Formal Methods, Type Theory, Programming Languages,
          Theoretical Computer Science, Applied Mathematics,
          Cubical Type Theory, Martin-Löf Type Theory
\end{abstract}

\tableofcontents
\newpage

\section{Inductive Types}

\subsection{Empty}

\subsection{Unit}

\subsection{Bool}

\subsection{Maybe}

\subsection{Either}

\subsection{Nat}

\subsection{List}

\section{Inductive Encodings}

\subsection{Church Encoding}

You know Church encoding which also has its dependent alanolgue in CoC, however
in Coq it is imposible to detive Inductive Principle as type system lacks fixpoint
and functional extensionality. The example of working compiler of PTS languages are Om and Morte.
Assume we have Church encoded NAT:

\begin{lstlisting}[mathescape=true]
nat = (X:U) -> (X -> X) -> X -> X
\end{lstlisting}

where first parameter $(X -> X)$ is a $succ$, the second parameter $X$ is $zero$,
and the result of encoding is landed in X. Even if we encode the parameter

\begin{lstlisting}[mathescape=true]
list (A: U) = (X:U) -> X -> (A -> X) -> X
\end{lstlisting}

and paremeter A let's say live in 42 universe and X live in 2 universe, then by
the signature of encoding the term will be landed in X, thus 2 universe. In other words
such dependency is called impredicative displaying that landed term is not a predicate over parameters.
This means that Church encoding is incompatible with predicative type checkers with predicative
of predicative-cumulative hierarchies.

\subsection{Impredicative Encoding}

In HoTT n-types is encoded as n-groupoids, thus we need to add a predicate in which n-type
we would like to land the encoding:

\begin{lstlisting}[mathescape=true]
NAT (A: U) = (X:U) -> isSet X -> X -> (A -> X) -> X
\end{lstlisting}

Here we added isSet predicate. With this motto we can implement propositional
truncation by landing term in isProp or even HIT by langing in isGroupoid:

\begin{lstlisting}[mathescape=true]
TRUN (A:U) type = (X: U) -> isProp X -> (A -> X) -> X
S1 = (X:U) -> isGroupoid X -> ((x:X) -> Path X x x) -> X
MONOPLE (A:U) = (X:U) -> isSet X -> (A -> X) -> X
NAT = (X:U) -> isSet X -> X -> (A -> X) -> X
\end{lstlisting}

The main publication on this topic could be found at \cite{Awodey17} and \cite{Speight17}.

\subsection{The Unit Example}

Here we have the implementation of Unit impredicative encoding in HoTT.

\begin{lstlisting}[mathescape=true]
upPath     (X Y:U)(f:X->Y)(a:X->X): X -> Y = o X X Y f a
downPath   (X Y:U)(f:X->Y)(b:Y->Y): X -> Y = o X Y Y b f
naturality (X Y:U)(f:X->Y)(a:X->X)(b:Y->Y): U
  = Path (X->Y)(upPath X Y f a)(downPath X Y f b)

unitEnc': U = (X: U) -> isSet X -> X -> X
isUnitEnc (one: unitEnc'): U
  = (X Y:U)(x:isSet X)(y:isSet Y)(f:X->Y) ->
    naturality X Y f (one X x)(one Y y)

unitEnc: U = (x: unitEnc') * isUnitEnc x
unitEncStar: unitEnc = (\(X:U)(_:isSet X) ->
  idfun X,\(X Y: U)(_:isSet X)(_:isSet Y)->refl(X->Y))
unitEncRec  (C: U) (s: isSet C) (c: C): unitEnc -> C
  = \(z: unitEnc) -> z.1 C s c
unitEncBeta (C: U) (s: isSet C) (c: C)
  : Path C (unitEncRec C s c unitEncStar) c = refl C c
unitEncEta (z: unitEnc): Path unitEnc unitEncStar z = undefined
unitEncInd (P: unitEnc -> U) (a: unitEnc): P unitEncStar -> P a
  = subst unitEnc P unitEncStar a (unitEncEta a)
unitEncCondition (n: unitEnc'): isProp (isUnitEnc n)
  =  \ (f g: isUnitEnc n) ->
     <h> \ (x y: U) -> \ (X: isSet x) -> \ (Y: isSet y)
  -> \ (F: x -> y) -> <i> \ (R: x) -> Y (F (n x X R)) (n y Y (F R))
       (<j> f x y X Y F @ j R) (<j> g x y X Y F @ j R) @ h @ i
\end{lstlisting}

\bibliographystyle{plain}
\bibliography{inductive}

\end{document}
