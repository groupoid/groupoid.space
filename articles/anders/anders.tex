\documentclass{article}
\bibliographystyle{plainurl}
\usepackage{newunicodechar}
\usepackage{verbatim}
\usepackage[english,ukrainian]{babel}
\usepackage{listings}
\usepackage[utf8]{inputenc}
\usepackage{hyphenat}
\usepackage{hyperref}
\usepackage{adjustbox}
\usepackage{xcolor}
\usepackage{microtype}
\usepackage{amscd}
\usepackage{amsmath}
\usepackage{amssymb}
\usepackage{amsthm}
\usepackage{tikz}
\usepackage{tikz-cd}
\usepackage{url}

\theoremstyle{definition}
\newtheorem{definition}{Definition}
\newtheorem{theorem}{Theorem}
\newcommand*{\incmap}{\hookrightarrow}
\newcommand*{\thead}[1]{\multicolumn{1}{c}{\bfseries #1}}
\lstset{basicstyle=\small,inputencoding=utf8}

\addto\captionsukrainian{\renewcommand{\contentsname}{Зміст}}
\addto\captionsukrainian{\def\refname{Список використаних джерел}}
\addto\captionsukrainian{\renewcommand{\abstractname}{Анотація}}

\newcommand{\tabstyle}[0]{\scriptsize\ttfamily\fontseries{l}\selectfont}

\begin{document}

\author { М.Е. Сохацький $^1$ }
\title { Issue XIX: Modal Homotopy Type System }
\date{ \small $^1$ Національний технічний університет України \\
       ім. Ігоря Сікорського \\
       26 листопада 2021 }
\maketitle

\begin{abstract}
Here is presented a reincarnation of \textbf{cubicaltt} called \textbf{anders}.
\end{abstract}

\newpage
\tableofcontents
\newpage

\section{Introduction}

\textbf{Anders} is a Modal HoTT proof assistant based on: classical MLTT-80 \cite{MLTT80}
with 0, 1, 2, W types; CCHM \cite{CCHM} in CHM \cite{CHM} flavour as cubical type system with
hcomp/transp operations; HTS \cite{HTS} strict equality on pretypes;
infinitisemal \cite{deRham} modality primitives for differential geometry purposes.
We tend not to touch general recursive higher inductive schemes,
instead we will try to express as much HIT as possible through Suspensions, Truncations,
Quotients primitives built into type checker core.
Anders also aims to support simplicial types Simplex along with Hopf Fibrations
built into core for sphere homotopy groups processing. This modification is called \textbf{Dan}.
Full stack of Groupoid Infinity languages is given at AXIO/1\footnote{\url{https://axio.groupoid.space}} homepage.

The HTS language proposed by Voevodsky exposes two different presheaf models of type theory:
the inner one is homotopy type system presheaf that models HoTT and the outer one is
traditional Martin-Löf type system presheaf that models set theory with UIP.
The motivation behind this doubling is to have an ability to express semisemplicial types.
Theoretical work on merging inner and outer languages was continued in 2LTT \cite{2LTT}.

$\mathbf{Installation}$. While we are on our road to Lean-like tactic language, currently we are at the stage of
regular cubical HTS type checker with CHM-style primitives. You may try it from Github
sources: groupoid/anders\footnote{\url{https://github.com/groupoid/anders/}} or install
through OPAM package manager. Main commands are \textbf{check} (to check a program)
and \textbf{repl} (to enter the proof shell).

\begin{table}[ht]
\centering
\begin{tabular}{l}
\$ opam install anders
\end{tabular}
\end{table}

Anders is fast, idiomatic and educational (think of optimized Mini-TT). We carefully draw the favourite Lean-compatible
syntax to fit 200 LOC in Menhir. The CHM kernel is 1K LOC. Whole Anders compiles under 1
second and checks all the base library under 1/3 of a second [i5-12400]. Anders proof assistant
as Homotopy Type System comes with its own Homotopy Library\footnote{\url{https://anders.groupoid.space/lib/}}.

\section{Syntax}

The syntax resembles original syntax of the reference CCHM type checker cubicaltt,
is slightly compatible with Lean syntax and contains the full set of Cubical Agda \cite{CubicalAgda}
primitives (except generic higher inductive schemes).

Here is given the mathematical pseudo-code notation of the language
expressions that come immediately after parsing. The core syntax definition of HTS language
corresponds to the type defined in OCaml module:

\begin{table}[ht]
\centering
\begin{tabular}{rl}
 cosmos :=&$\textbf{U}_j \ |\ \textbf{V}_k$ \\
    var :=&$\textbf{var}\ name\ |\ \textbf{hole}$ \\
     pi :=&$\Pi\ name\ E\ E\ |\ \lambda\ name\ E\ E\ |\ E\ E$ \\
  sigma :=&$\Sigma\ name\ E\ E\ |\ (E,E)\ |\ E.1\ |\ E.2$ \\
      0 :=&$\textbf{0}\ |\ \textbf{ind$_0$}\ \textrm{E}\ \textrm{E}\ \textrm{E}$ \\
      1 :=&$\textbf{1}\ |\ \star\ |\ \textbf{ind$_1$}\ \textrm{E}\ \textrm{E}\ \textrm{E}$ \\
      2 :=&$\textbf{2}\ |\ 0_2\ |\ 1_2\ |\ \textbf{ind$_2$}\ \textrm{E}\ \textrm{E}\ \textrm{E}$ \\
      W :=&$\textbf{W}\ \textrm{ident}\ \textrm{E}\ \textrm{E}\ |\ \textbf{sup}\ \textrm{E}\ \textrm{E}\ |\ \textbf{ind$_W$}\ \textrm{E}\ \textrm{E}$ \\
     id :=&$\textbf{Id}\ E\ |\ \textbf{ref}\ E\ |\ \textbf{id$_J$}\ E$ \\
   path :=&$\textbf{Path}\ E\ |\ E^i\ |\ E\ @\ E$ \\
      I :=&$\textbf{I}\ |\ 0\ |\ 1\ |\ E\ \bigvee\ E\ |\ E\ \bigwedge\ E\ |\ \neg E$ \\
   part :=&$\textbf{Partial}\ E\ E\ |\ \textbf{[}\ (E=I) \rightarrow E, ...\ \textbf{]}$ \\
    sub :=&$\textbf{inc}\ E\ |\ \textbf{ouc}\ E\ |\ E\ \textbf{[}\ I\ \mapsto\ E\ \textbf{]}$ \\
    kan :=&$\textbf{transp}\ E\ E\ |\ \textbf{hcomp}\ E$ \\
   glue :=&$\textbf{Glue}\ E\ |\ \textbf{glue}\ E\ |\ \textbf{unglue}\ E\ E$ \\
     Im :=&$\textbf{Im}\ \mathrm{E}\ |\ \textbf{Inf}\ \mathrm{E}\ |\ \textbf{Join}\ \mathrm{E}\ |\ \textbf{ind$_{Im}$}\ \mathrm{E}\ \mathrm{E}$ \\
\end{tabular}
\end{table}

Further Menhir BNF notation will be used to describe the top-level language E parser.

\begin{table}[ht]
\centering
\begin{tabular}{rl}
      E :=&$\textrm{cosmos}\ |\ \textrm{var}\ |\ \textrm{MLTT}\ |\ \textrm{CCHM}\ |\ \textrm{Im}$ \\
   CCHM :=&$\textrm{path}\ |\ \textrm{I}\ |\ \textrm{part}\ |\ \textrm{sub}\ |\ \textrm{kan}\ |\ \textrm{glue}$ \\
   MLTT :=&$\textrm{pi}\ |\ \textrm{sigma}\ |\ \textrm{id}$ \\
\end{tabular}
\end{table}

\textbf{Keywords}. The words of a top-level language, file or repl, consist of keywords or identifiers.
The keywords are following: \texttt{module}, \texttt{where}, \texttt{import}, \texttt{option}, \texttt{def}, \texttt{axiom},
\texttt{postulate}, \texttt{theorem}, \texttt{(}, \texttt{)}, \texttt{[}, \texttt{]}, \texttt{<}, \texttt{>},
\texttt{/}, \texttt{.1}, \texttt{.2}, \texttt{$\Pi$}, \texttt{$\Sigma$}, \texttt{,}, \texttt{$\lambda$},
\texttt{V}, \texttt{$\bigvee$}, \texttt{$\bigwedge$}, \texttt{-}, \texttt{+}, \texttt{@}, \texttt{PathP},
\texttt{transp}, \texttt{hcomp}, \texttt{zero}, \texttt{one}, \texttt{Partial}, \texttt{inc},
\texttt{$\times$}, \texttt{$\rightarrow$}, \texttt{:}, \texttt{:=}, \texttt{$\mapsto$}, \texttt{U},
\texttt{ouc}, \texttt{interval}, \texttt{inductive}, \texttt{Glue}, \texttt{glue}, \texttt{unglue}.

\textbf{Indentifiers}. Identifiers support UTF-8. Indentifiers couldn't
start with $\texttt{:}$, $\texttt{-}$, $\rightarrow$. Sample identifiers:
$\neg\texttt{-of-}\vee$, $\texttt{1}$$\rightarrow$$\texttt{1}$, $\texttt{is-?}$,
$\texttt{=}$, $\texttt{\$$\sim$]!}$$\texttt{005x}$, $\infty$, $\texttt{x}$$\rightarrow$$\texttt{Nat}$.

\textbf{Modules}. Modules represent files with declarations. More accurate, BNF notation of module consists of imports, options and declarations.
\begin{table}[ht]
\tabstyle
\begin{tabular}{l}
\textbf{menhir} \\
\ \ \ start <Module.file> file \\
\ \ \ start <Module.command> repl \\
\ \ \ repl : COLON IDENT exp1 EOF | COLON IDENT EOF | exp0 EOF | EOF \\
\ \ \ file : MODULE IDENT WHERE line* EOF \\
\ \ \ path : IDENT \\
\ \ \ line : IMPORT path+ | OPTION IDENT IDENT | declarations \\
\end{tabular}
\end{table}

\textbf{Imports}. The import construction supports file folder structure (without file extensions)
by using reserved symbol / for hierarchy walking.

\textbf{Options}. Each option holds bool value. Language supports following options:
1) girard (enables U : U);
2) pre-eval (normalization cache);
3) impredicative (infinite hierarchy with impredicativity rule);
In Anders you can enable or disable language core types, adjust syntaxes or
tune inner variables of the type checker.

\textbf{Declarations}. Language supports following top level declarations:
1) axiom (non-computable declaration that breakes normalization);
2) postulate (alternative or inverted axiom that can preserve consistency);
3) definition (almost any explicit term or type in type theory);
4) lemma (helper in big game);
5) theorem (something valuable or complex enough).

\begin{table}[ht!]
\tabstyle
\begin{tabular}{l}
\textbf{axiom} isProp (A : U) : U \\
\textbf{def} isSet (A : U) : U := \textbf{$\Pi$} (a b : A) (x y : Path A a b), Path (Path A a b) x y
\end{tabular}
\end{table}

Sample declarations. For example, signature isProp (A : U) of type U could be
defined as normalization-blocking axiom without proof-term or by providing proof-term as definition.

In this example (A : U), (a b : A) and (x y : Path A a b) are called telescopes.
Each telescope consists of a series of lenses or empty. Each lense provides a
set of variables of the same type. Telescope defines parameters of a declaration.
Types in a telescope, type of a declaration and a proof-terms are a language expressions exp1.

\begin{table}[ht]
\tabstyle
\begin{tabular}{l}
\textbf{menhir} \\
\ \ \ ident : IRREF | IDENT \\
\ \ \ lense : LPARENS ident+ COLON exp1 RPARENS \\
\ \ \ telescope : lense telescope \\
\ \ \ params : telescope | [] \\
\ \ \ declarations : \\
\ \ \ \ \ \ | DEF IDENT params DEFEQ exp1 \\
\ \ \ \ \ \ | DEF IDENT params COLON exp1 DEFEQ exp1 \\
\ \ \ \ \ \ | AXIOM IDENT params COLON exp1
\end{tabular}
\end{table}

\textbf{Expressions}. All atomic language expressions are grouped by four categories:
exp0 (pair constructions), exp1 (non neutral constructions), exp2 (path and pi applcations),
exp3 (neutral constructions).

\begin{table}[ht!]
\tabstyle
\begin{tabular}{ll}
\textbf{menhir} \\
\multicolumn{2}{l}{\ \ \ face : LPARENS IDENT IDENT IDENT RPARENS } \\
\multicolumn{2}{l}{\ \ \ part : face+ ARROW exp1 } \\
\multicolumn{2}{l}{\ \ \ exp0 : exp1 COMMA exp0 | exp1 } \\
\multicolumn{2}{l}{\ \ \ exp1 : LSQ separated(COMMA, part) RSQ } \\
\ \ \ \ \ \ | LAM telescope COMMA exp1   & | PI telescope COMMA exp1 \\
\ \ \ \ \ \ | SIGMA telescope COMMA exp1 & | LSQ IRREF ARROW exp1 RSQ \\
\ \ \ \ \ \ | LT ident+ GT exp1          & | exp2 ARROW exp1 \\
\ \ \ \ \ \ | exp2 PROD exp1             & | exp2 \\
\end{tabular}
\end{table}

The LR parsers demand to define exp1 as expressions that cannot be used (without a parens enclosure)
as a right part of left-associative application for both Path and Pi lambdas.
Universe indicies $U_j$ (inner fibrant), $V_k$ (outer pretypes) and S (outer strict omega)
are using unicode subscript letters that are already processed in lexer.

\begin{table}[ht!]
\tabstyle
\begin{tabular}{llll}
\textbf{menhir} \\
\multicolumn{3}{l}{\ \ \ exp2 : exp2 exp3 | exp2 APPFORMULA exp3 | exp3 } \\
\multicolumn{4}{l}{\ \ \ exp3 : LPARENS exp0 RPARENS LSQ exp0 MAP exp0 RSQ } \\
\ \ \ \ \ \   | HOLE              & | PRE          & | KAN          & | IDJ exp3 \\
\ \ \ \ \ \   | exp3 FST          & | exp3 SND     & | NEGATE exp3  & | INC exp3 \\
\ \ \ \ \ \   | exp3 AND exp3     & | exp3 OR exp3 & | ID exp3      & | REF exp3\\
\ \ \ \ \ \   | OUC exp3          & | PATHP exp3   & | PARTIAL exp3 & | IDENT \\
\multicolumn{3}{l}{\ \ \ \ \ \    | IDENT LSQ exp0 MAP exp0 RSQ }   & | HCOMP exp3 \\
\multicolumn{3}{l}{\ \ \ \ \ \    | LPARENS exp0 RPARENS }          & | TRANSP exp3 exp3 \\
\end{tabular}
\end{table}


\newpage
\section{Semantics}
The idea is to have a unified layered type checker, so you can disbale/enable any MLTT-style inference,
assign types to universes and enable/disable hierachies. This will be done by providing linking API for
pluggable presheaf modules. We selected 5 levels of type checker awareness from universes and pure type
systems up to synthetic language of homotopy type theory. Each layer corresponds to its presheaves with
separate configuration for universe hierarchies.
\begin{table}[ht]
\tabstyle
\begin{tabular}{rl}
     \textbf{def} & lang :\textbf{U} := \textbf{inductive} \\
               \{ & UNI: cosmos $\rightarrow$ lang \\
               |  & PI: pure lang $\rightarrow$ lang \\
               |  & SIGMA: total lang $\rightarrow$ lang \\
               |  & ID: strict lang $\rightarrow lang$ \\
               |  & PATH: homotopy lang $\rightarrow$ lang \\
               |  & GLUE: glue lang $\rightarrow$ lang \\
               |  & INDUCTIVE: w012 lang $\rightarrow$ lang \\
               \} & \\
\end{tabular}
\end{table}
We want to mention here with homage to its authors all categorical models of dependent type theory:
Comprehension Categories (Grothendieck, Jacobs), LCCC (Seely), D-Categories and CwA (Cartmell),
CwF (Dybjer), C-Systems (Voevodsky), Natural Models (Awodey). While we can build some transports
between them, we leave this excercise for our mathematical components library.
We will use here the Coquand's notation for Presheaf Type Theories in terms of restriction maps.

\subsection{Universe Hierarchies}

Language supports Agda-style hierarchy of universes: prop, fibrant (U), interval pretypes (V) and
strict omega with explicit level manipulation. All universes are bounded with preorder
\begin{equation}
Fibrant_j \prec Pretypes_k
\end{equation}
in which $j,k$ are bounded with equation:
\begin{equation}
j < k.
\end{equation}
Large elimination to upper universes is prohibited. This is extendable to Agda model:

\begin{table}[ht]
\tabstyle
\begin{tabular}{rl}
     \textbf{def} & cosmos : \textbf{U} := \textbf{inductive} \\
               \{ & fibrant: nat \\
               |  & pretypes: nat \\
               \} & \\
\end{tabular}
\end{table}

The \textbf{anders} model contains only fibrant $U_j$ and pretypes $V_k$ universe hierarchies.

\newpage

\subsection{Dependent Types}

\begin{definition}[Type]
A type is interpreted as a presheaf $A$, a family of sets $A_I$ with restriction maps
$u \mapsto u\ f, A_I \rightarrow A_J$ for $f: J\rightarrow I$. A dependent type
B on A is interpreted by a presheaf on category of elements of $A$: the objects
are pairs $(I,u)$ with $u : A_I$ and morphisms $f: (J,v) \rightarrow (I,u)$ are
maps $f : J \rightarrow$ such that $v = u\ f$. A dependent type B is thus given
by a family of sets $B(I,u)$ and restriction maps $B(I,u) \rightarrow B(J,u\ f)$.
\end{definition}


We think of $A$ as a type and $B$ as a family of presheves $B(x)$ varying $x:A$.
The operation $\Pi(x:A)B(x)$ generalizes the semantics of
implication in a Kripke model.

\begin{definition}[Pi]
An element $w:[\Pi(x:A)B(x)](I)$ is a family of functions $w_f : \Pi(u:A(J))B(J,u)$
for $f : J \rightarrow I$ such that $(w_f u)g=w_{f\ g}(u\ g)$ when $u:A(J)$ and $g:K\rightarrow J$.
\end{definition}
\begin{table}[ht]
\tabstyle
\begin{tabular}{rl}
  \textbf{def} & pure (lang : U) : \textbf{U} := \textbf{inductive} \\
            \{ & pi: name $\rightarrow$ nat $\rightarrow$ lang $\rightarrow$ lang $\rightarrow$ pure lang \\
             | & lambda: name $\rightarrow$ nat $\rightarrow$ lang $\rightarrow$ lang \\
             | & app: lang $\rightarrow$ lang \\
            \} & \\
\end{tabular}
\end{table}

\begin{definition}[Sigma]
The set $\Sigma(x:A)B(x)$ is the set of pairs $(u,v)$ when $u:A(I),v:B(I,u)$ and restriction map $(u,v)\ f=(u\ f,v\ f)$.
\end{definition}
\begin{table}[ht]
\tabstyle
\begin{tabular}{rl}
  \textbf{def} & total (lang : U) : \textbf{U} := \textbf{inductive} \\
            \{ & sigma: name $\rightarrow$ lang $\rightarrow$ total lang \\
             | & pair: lang $\rightarrow$ lang \\
             | & fst: lang \\
             | & snd: lang \\
            \} & \\
\end{tabular}
\end{table}

The presheaf with only Pi and Sigma is called \textbf{MLTT-72} \cite{MLTT72}. Its internalization in \textbf{anders} is as follows:

\begin{table}[ht!]
\tabstyle
\begin{tabular}{rl}
\textbf{def} & MLTT-72 (A : U) (B : A $\rightarrow$ U) : U := $\Sigma$ \\
             & ($\Pi$-form$_1$ : U) \\
             & ($\Pi$-ctor$_1$ : Pi A B $\rightarrow$ Pi A B) \\
             & ($\Pi$-elim$_1$ : Pi A B $\rightarrow$ Pi A B) \\
             & ($\Pi$-comp$_1$ : (a : A) (f : Pi A B), $\Pi$-elim$_1$ ($\Pi$-ctor$_1$ f) a = f a) \\
             & ($\Pi$-comp$_2$ : (a : A) (f : Pi A B), f = $\lambda$ (x : A), f x) \\
             & ($\Sigma$-form$_1$ : U) \\
             & ($\Sigma$-ctor$_1$ : $\Pi$ (a : A) (b : B a) , Sigma A B) \\
             & ($\Sigma$-elim$_1$ : $\Pi$ (p : Sigma A B), A) \\
             & ($\Sigma$-elim$_2$ : $\Pi$ (p : Sigma A B), B (pr$_1$ A B p)) \\
             & ($\Sigma$-comp$_1$ : $\Pi$ (a : A) (b: B a), a = $\Sigma$-elim$_1$ ($\Sigma$-ctor$_1$ a b)) \\
             & ($\Sigma$-comp$_2$ : $\Pi$ (a : A) (b: B a), b = $\Sigma$-elim$_2$ (a, b)) \\
             & ($\Sigma$-comp$_3$ : $\Pi$ (p : Sigma A B), p = (pr$_1$ A B p, pr$_2$ A B p)), \textbf{1}
\end{tabular}
\end{table}

\newpage
\subsection{Path Equality}

The fundamental development of equality inside MLTT provers led us to the
notion of $\infty$-groupoid as spaces. In this way Path identity type appeared
in the core of type checker along with De Morgan algebra on built-in interval type.

\begin{table}[ht]
\tabstyle
\begin{tabular}{rl}
     \textbf{def} & homotopy (lang : U) : \textbf{U} := \textbf{inductive} \\
               \{ & PathP: lang $\rightarrow$ lang $\rightarrow$ lang \\
  |& plam: name $\rightarrow$ lang $\rightarrow$ lang \\
  |& papp: lang $\rightarrow$ lang \\
  |& I \\
  |& zero \\
  |& one \\
  |& meet: lang $\rightarrow$ lang \\
  |& join: lang $\rightarrow$ lang \\
  |& neg: lang \\
  |& system: lang \\
  |& Partial: lang \\
  |& transp: lang $\rightarrow$ lang \\
  |& hcomp: lang \\
  |& Sub: lang \\
  |& inc: lang \\
  |& ouc: lang \\
 \}& \\
\end{tabular}
\end{table}

\begin{definition}[Cubical Presheaf $\mathbb{I}$]
The identity types modeled with another presheaf, the presheaf on Lawvere
category of distributive lattices (theory of De Morgan algebras) denoted
with $\Box$ — $\textbf{I} : \Box^{op} \rightarrow \textrm{Set}$.
\end{definition}

\begin{definition}[Properties of $\textbf{I}$] The presheaf $\textbf{I}$:
i) has to distinct global elements $0$ and $1$ (B$_1$);
ii) $\textbf{I}$(I) has a decidable equality for each $I$ (B$_2$);
iii) $\textbf{I}$ is tiny so the path functor $X \mapsto X^\textbf{I}$ has right adjoint (B$_3$).;
iv) $\textbf{I}$ has meet and join (connections).
\end{definition}

$\textbf{Interval\ Pretypes}$. While having pretypes universe V with interval and
associated De Morgan algebra ($\wedge$, $\vee$, -, 0, 1, I) is enough to
perform DNF normalization and proving some basic statements about path, including:
contractability of singletons, homotopy transport, congruence, functional
extensionality; it is not enough for proving $\beta$ rule for Path type or path composition.
\\
\\
\indent $\textbf{Generalized\ Transport}$. Generalized transport transp adresses
first problem of deriving the computational $\beta$ rule for Path types:

\begin{table}[ht]
\tabstyle
\begin{tabular}{rl}
\textbf{theorem} & Path$_\beta$ (A : U) (a : A) (C : D A) (d: C a a (refl A a)) \\
               : & Equ (C a a (refl A a)) d (J A a C d a (refl A a)) \\
              := & $\lambda$ (A : U), $\lambda$ (a : A), \\
                 & $\lambda$ (C : $\Pi$ (x : A), $\Pi$ (y : A), PathP (<\_> A) x y $\rightarrow$ U), \\
                 & $\lambda$ (d : C a a (<\_> a)), <j> transp (<\_> C a a (<\_> a)) -j d \\
\end{tabular}
\end{table}

Transport is defined on fibrant types (only) and type checker should cover all the cases
Note that transp$^i$ (Path$^j$ A v w) $\varphi$ u$_0$ case is relying on comp
operation which depends on hcomp primitive. Here is given the first part of Simon Huber equations \cite{Huber} for \textbf{transp}:

\begin{table}[ht]
\tabstyle
\begin{tabular}{l}
transp$^{i}$ N $\varphi$ u$_0$ = u$_0$ \\
transp$^{i}$ U $\varphi$ A = A \\
transp$^{i}$ ($\Pi$ (x : A), B) $\varphi$ u$_0$ v = transp$^i$ B(x/w) $\varphi$ (u$_0$ w(i/0)) \\
transp$^{i}$ ($\Sigma$ (x : A), B) $\varphi$ u$_0$ = (transp$^i$ A $\varphi$ (u$_0$.1),transp$^i$ B(x/v) $\varphi$ (u$_0$.2)) \\
transp$^{i}$ (Path$^j$ v w) $\varphi$ u$_0$ = <j> comp$^i$ A [$\phi$ u$_0$ j, (j=0) $\mapsto$ v, (j=1) $\mapsto$ w] (u$_0$ j) \\
transp$^{i}$ (Glue [$\varphi$ $\mapsto$ (T,w)] A) $\psi$ u$_0$ = glue [$\phi$(i/1) $\mapsto$ t'$_1$] a'$_1$ : B(i/1) \\
\end{tabular}
\end{table}

\newpage
$\textbf{Partial\ Elements}$. In order to explicitly define hcomp we need to specify
n-cubes where some faces are missing. Partial primitives isOne, 1=1 and UIP on pretypes
are derivable in Anders due to landing strict equality Id in V universe. The idea is
that (Partial A r) is the type of cubes in A that are only defined when IsOne r holds.
(Partial A r) is a special version of the function space IsOne r → A with a more
extensional equality: two of its elements are considered judgmentally equal if
they represent the same subcube of A. They are equal whenever they reduce to
equal terms for all the possible assignment of variables that make r equal to 1.

\begin{table}[ht]
\tabstyle
\begin{tabular}{l}
\textbf{def} Partial' (A : U) (i : I) := Partial A i \\
\textbf{def} isOne : I -> V := Id I 1 \\
\textbf{def} 1=>1 : isOne 1 := ref 1 \\
\textbf{def} UIP (A : V) (a b : A) (p q : Id A a b) : Id (Id A a b) p q := ref p \\
\end{tabular}
\end{table}

$\textbf{Cubical\ Subtypes}$. For (A : U) (i : I) (Partial A i) we can define
subtype A [ i $\mapsto$ u ]. A term of this type is a term of type A that is
definitionally equal to u when (IsOne i) is satisfied. We have forth and back
fusion rules ouc (inc v) = v and inc (outc v) = v. Moreover, ouc v will reduce to u 1=1 when i=1.

\begin{table}[ht]
\tabstyle
\begin{tabular}{l}
\textbf{def} sub' (A : U) (i : I) (u : Partial A i) : V := A [i $\mapsto$ u ] \\
\textbf{def} inc' (A : U) (i : I) (a : A) : A [i $\mapsto$ [(i = 1) $\rightarrow$ a]] := inc A i a \\
\textbf{def} ouc' (A : U) (i : I) (u : Partial A i) (a : A [i $\mapsto$ u]) : A := ouc a \\
\end{tabular}
\end{table}

$\textbf{Homogeneous\ Composition}$. hcomp is the answer to second problem: with hcomp and transp
one can express path composition, groupoid, category of groupoids (groupoid interpretation and
internalization in type theory). One of the main roles of homogeneous composition is to be a
carrier in [higher] inductive type constructors for calculating of homotopy colimits and direct
encoding of CW-complexes. Here is given the second part of Simon Huber equations \cite{Huber} for $\textbf{hcomp}$:

\begin{table}[ht]
\tabstyle
\begin{tabular}{l}
hcomp$^i$ N [$\phi$ $\mapsto$ 0] 0 = 0 \\
hcomp$^i$ N [$\phi$ $\mapsto$ S u] (S u$_0$) = S (hcomp$^i$ N [$\phi$ $\mapsto$ u] u$_0$) \\
hcomp$^i$ U [$\phi$ $\mapsto$ E] A = Glue [$\phi$ $\mapsto$ (E(i/1), equiv$^i$ E(i/1-i))] A \\
hcomp$^i$ ($\Pi$ (x : A), B) [$\phi$ $\mapsto$ u] u$_0$ v = hcomp$^i$ B(x/v) [$\phi$ $\mapsto$ u v] (u$_0$ v) \\
hcomp$^i$ ($\Sigma$ (x : A), B) [$\phi$ $\mapsto$ u] u$_0$ = (v(i/1), comp$^i$ B(x/v) [$\phi$ $\mapsto$ u.2] u$_0$.2) \\
hcomp$^i$ (Path$^j$ A v w) [$\phi$ $\mapsto$ u] u$_0$ = <j> hcomp$^i$ A[$\phi$ $\mapsto$ u j, (j=0) $\mapsto$ v, (j=1) $\mapsto$ w] (u$_0$ j) \\
hcomp$^i$ (Glue [$\phi$ $\mapsto$ (T,w)] A) [$\psi$ $\mapsto$ u] u$_0$ = glue [$\phi$ $\mapsto$ u(i/1)] (unglue u(i/1)) \\
\end{tabular}
\end{table}

\subsection{Strict Equality}

To avoid conflicts with path equalities which live in fibrant universes strict
equalities live in pretypes universes.

\begin{table}[ht!]
\tabstyle
\begin{tabular}{rl}
     \textbf{def} & strict (lang : U) : \textbf{U} := \textbf{inductive} \\\
 \{ & Id: name $\rightarrow$ lang \\
  |& ref: lang $\rightarrow$ lang \\
  |& idJ: lang $\rightarrow$ lang $\rightarrow$ lang \\
                         \} & \\
\end{tabular}
\end{table}

We use strict equality in HTS for pretypes and partial elements which live in V.
The presheaf configuration with Pi, Sigma and Id is called \textbf{MLTT-75} \cite{MLTT75}. The presheaf
configuration with Pi, Sigma, Id and Path is called \textbf{HTS} (Homotopy Type System).

\newpage
\subsection{Glue Types}

The main purpose of Glue types is to construct a cube where some faces have
been replaced by equivalent types. This is analogous to how hcomp lets us
replace some faces of a cube by composing it with other cubes, but for Glue
types you can compose with equivalences instead of paths. This implies the
univalence principle and it is what lets us transport along paths built
out of equivalences.

\begin{table}[ht]
\tabstyle
\begin{tabular}{rl}
     \textbf{def} & glue (lang : U) : \textbf{U} := \textbf{inductive} \\
 \{ & Glue: lang $\rightarrow$ lang $\rightarrow$ lang \\
  |& glue: lang $\rightarrow$ lang \\
  |& unglue: lang $\rightarrow$ lang \\
                         \} & \\
\end{tabular}
\end{table}

Basic Fibrational HoTT core by Pelayo, Warren, and Voevodsky (2012).

\begin{table}[ht]
\tabstyle
\begin{tabular}{l}
\textbf{def} fiber (A B : U) (f: A → B) (y : B): U := $\Sigma$ (x : A), Path B y (f x) \\
\textbf{def} isEquiv (A B : U) (f : A → B) : U := $\Pi$ (y : B), isContr (fiber A B f y) \\
\textbf{def} equiv (A B : U) : U := $\Sigma$ (f : A → B), isEquiv A B f \\
\textbf{def} contrSingl (A : U) (a b : A) (p : Path A a b) \\
\ \ \ \ : Path ($\Sigma$ (x : A), Path A a x) (a,<i>a) (b,p) := <i> (p @ i, <j> p @ i $\vee$ j) \\
\textbf{def} idIsEquiv (A : U) : isEquiv A A (id A) := \\
\ \ \ \ $\lambda$ (a : A), ((a,<i>a), $\lambda$ (z : fiber A A (id A) a), contrSingl A a z.1 z.2) \\
\textbf{def} idEquiv (A : U) : equiv A A := (id A, isContrSingl A) \\
\end{tabular}
\end{table}

The notion of Univalence was discovered by Vladimir Voevodsky
as forth and back transport between fibrational equivalence
as contractability of fibers and homotopical multi-dimentional
heterogeneous path equality. The Equiv → Path type is called Univalence type,
where univalence intro is obtained by Glue type and elim (Path → Equiv) is
obtained by sigma transport from constant map.

\begin{table}[ht]
\tabstyle
\begin{tabular}{l}
\textbf{def} univ-formation (A B : U) := equiv A B → PathP (<i> U) A B \\
\textbf{def} univ-intro (A B : U) : univ-formation A B := $\lambda$ (e : equiv A B), \\
\ \ \ \ <i> Glue B ($\partial$ i) [(i = 0) → (A, e), (i = 1) → (B, idEquiv B)] \\
\textbf{def} univ-elim (A B : U) (p : PathP (<i> U) A B) \\
\ \ : equiv A B := transp (<i> equiv A (p @ i)) 0 (idEquiv A) \\
\textbf{def} univ-computation (A B : U) (p : PathP (<i> U) A B) \\
\ \ : PathP (<i> PathP (<i> U) A B) (univ-intro A B (univ-elim A B p)) p \\
\ \ := <j i> Glue B (j $\vee$ $\partial$ i) \\
\ \ \ \ \ [(i = 0) → (A, univ-elim A B p), (i = 1) → (B, idEquiv B), \\
\ \ \ \ \ \ (j = 1) → (p @ i, univ-elim (p @ i) B (<k> p @ (i $\vee$ k)))] \\
\end{tabular}
\end{table}

Similar to Fibrational Equivalence the notion of Retract/Section based Isomorphism could be introduced
as forth-back transport between isomorphism and path equality. This notion is somehow cannonical to all
cubical systems and is called Unimorphism here.

\begin{table}[ht!]
\tabstyle
\begin{tabular}{l}
\textbf{def} iso-Form (A B: U) : U$_1$ := iso A B -> PathP (<i>U) A B \\
\textbf{def} iso-Intro (A B: U) : iso-Form A B := \\
\ \ \ $\lambda$ (x : iso A B), isoPath A B x.f x.g x.s x.t \\
\textbf{def} iso-Elim (A B : U) : PathP (<i> U) A B -> iso A B \\
\ := $\lambda$ (p : PathP (<i> U) A B), \\
\ \ \ ( coerce A B p, coerce B A (<i> p @ -i), \\
\ \ \ \ \ trans$^{-1}$-trans A B p, $\lambda$ (a : A), <k> trans-trans$^{-1}$ A B p a @-k, $\star$) \\
\end{tabular}
\end{table}

Orton-Pitts basis for univalence computability (2017):

\begin{table}[ht!]
\tabstyle
\begin{tabular}{l}
\textbf{def} ua (A B : U) (p : equiv A B) : PathP (<i> U) A B := univ-intro A B p \\
\textbf{def} ua$-\beta$ (A B : U) (e : equiv A B) : Path (A → B) (trans A B (ua A B e)) e.1 \\
 := <i> $\lambda$ (x : A), (idfun=idfun'' B @ -i) \\
\ \ \ \ ( (idfun=idfun'' B @ -i) ((idfun=idfun' B @ -i) (e.1 x)) ) \\
\end{tabular}
\end{table}

\newpage
\subsection{de Rham Stack}

Stack de Rham or Infinitezemal Shape Modality is a basic primitive for proving theorems
from synthetic differential geometry. This type-theoretical framework was developed
for the first time by Felix Cherubini under the guidance of Urs Schreiber. The Anders
prover implements the computational semantics of the de Rham stack.

\begin{table}[ht]
\tabstyle
\begin{tabular}{l}
\textbf{def} $\iota$ (A : U) (a : A) : $\Im$ A := $\Im-$unit a \\
\textbf{def} $\mu$ (A : U) (a : $\Im$ ($\Im$ A)) := $\Im-$join a \\
\\
\textbf{def} is-coreduced (A : U) : U := isEquiv A ($\Im$ A) ($\iota$ A) \\
\textbf{def} $\Im$-coreduced (A : U) : is-coreduced ($\Im$ A) := isoToEquiv \\
\ \ \ \ ($\Im$ A) ($\Im$ ($\Im$ A)) ($\iota$ ($\Im$ A)) ($\mu$ A) ($\lambda$ (x : $\Im$ ($\Im$ A)), <i>x) ($\lambda$ (y : $\Im$ A),<i>y) \\
\textbf{def} ind$-\Im\beta$ (A : U) (B : $\Im$ A → U) (f : $\Pi$ (a : A), $\Im$ (B ($\iota$ A a))) (a : A) \\
\ \ : Path ($\Im$ (B ($\iota$ A a))) (ind$-\Im$ A B f ($\iota$ A a)) (f a) := <i> f a \\
\textbf{def} ind$-\Im-$const (A B : U) (b : $\Im$ B) (x : $\Im$ A) \\
\ \ : Path ($\Im$ B) (ind$-\Im$ A ($\lambda$ (i : $\Im$ A), B) ($\lambda$ (i : A), b) x) b := <i> b \\
\end{tabular}
\end{table}

Coreduced induction and its $\beta-$quation.

\begin{table}[ht]
\tabstyle
\begin{tabular}{l}
\textbf{def} $\Im-$ind (A : U) (B : $\Im$ A → U) (c : $\Pi$ (a : $\Im$ A), \\
\ \ \ \ is-coreduced (B a)) (f : $\Pi$ (a : A), B ($\iota$ A a)) (a : $\Im$ A) : B a \\
\ \ \ \ := (c a (ind$-\Im$ A B ($\lambda$ (x : A), $\iota$ (B ($\iota$ A x)) (f x)) a)).1.1 \\
\textbf{def} $\Im-$ind$\beta$ (A : U) (B : $\Im$ A → U) (c : $\Pi$ (a : $\Im$ A), \\
\ \ \ \ is-coreduced (B a)) (f : $\Pi$ (a : A), B ($\iota$ A a)) (a : A) \\
\ \ : Path (B ($\iota$ A a)) (f a) (($\Im-$ind A B c f) ($\iota$ A a)) \\
\ := <i> sec-equiv (B ($\iota$ A a)) ($\Im$ (B ($\iota$ A a))) ($\iota$ (B ($\iota$ A a)), c ($\iota$ A a)) (f a) @-i \\
\end{tabular}
\end{table}

Geometric Modal HoTT Framework: Infinitesimal Proximity, Formal Disk, Formal Disk Bundle, Differential.

\begin{table}[ht]
\tabstyle
\begin{tabular}{l}
\textbf{def} $\sim$ (X : U) (a x' : X) : U := Path ($\Im$ X) ($\iota$ X a) ($\iota$ X x') \\
\textbf{def} $\mathbb{D}$ (X : U) (a : X) : U := $\Sigma$ (x' : X), $\sim$ X a x' \\
\textbf{def} inf-prox-ap (X Y : U) (f : X → Y) (x x' : X) (p : $\sim$ X x x') \\
\ \ : $\sim$ Y (f x) (f x') := <i> $\Im-$app X Y f (p @ i) \\
\textbf{def} T$^\infty$ (A : U) : U := $\Sigma$ (a : A), $\mathbb{D}$ A a \\
\textbf{def} inf-prox-ap (X Y : U) (f : X → Y) (x x' : X) (p : $\sim$ X x x') \\
\ \ : $\sim$ Y (f x) (f x') := <i> $\Im-$app X Y f (p @ i) \\
\textbf{def} d (X Y : U) (f : X → Y) (x : X) ($\varepsilon$ : $\mathbb{D}$ X x) \\
\ \ : $\mathbb{D}$ Y (f x) := (f $\varepsilon$.1, inf-prox-ap X Y f x $\varepsilon$.1 $\varepsilon$.2) \\
\textbf{def} T$^\infty$-map (X Y : U) (f : X → Y) ($\tau$ : T$^\infty$ X) : T$^\infty$ Y := (f $\tau$.1, d X Y f $\tau$.1 $\tau$.2) \\
\end{tabular}
\end{table}

\newpage
\subsection{Inductive Types}

Anders currently don’t support Lean-compatible generic inductive schemes
definition. So instead of generic inductive schemes Anders supports well-founded trees (W-types).
Basic data types like List, Nat, Fin, Vec are implemented as W-types in base library.

\begin{itemize}
\item W, 0, 1, 2 basis of MLTT-80 (Martin-L\"{o}f)
\item General Schemes of Inductive Types (Paulin-Mohring)
\end{itemize}

\subsection{Higher Inductive Types}

As for higher inductive types Anders has Three-HIT foundation (Coequalizer, Path Coequalizer and Colimit)
to express other HITs. Also there are other foundations to consider motivated by typical tasks in homotopy (type) theory:

\begin{itemize}
\item Coequalizer, Path Coequalizer and Colimit (van der Weide)
\item Suspension, Truncation, Quotient (Groupoid Infinity)
\item General Schemes of Higher Inductive Types (Cubical Agda)
\end{itemize}

\subsection{Simplicial Types}

Modification of Anders with Simplicial types and Hopf Fibrations built intro the core of type checker
is called \textbf{Dan} with following recursive syntax (having $f$ as Simplecies and $coh$ as Path-coherence functions):

\begin{table}[ht!]
\tabstyle
\begin{tabular}{l}
   \textbf{simplex} n [v$_0$ .. v$_n$] \{ f$_0$, f$_1$, ..., f$_n$ | coh i$_1$ i$_2$ ... i$_n$ \} : Simplex
\end{tabular}
\end{table}

and instantiation example:

\begin{table}[ht!]
\tabstyle
\begin{tabular}{l}

\textbf{def} s$_\infty$ : Simplicial \\
\ := $\Pi$ (v e : Simplex), \\
\ \ \ \ $\delta_{10}$ = v, $\delta_{11}$ = v, s$_0$ < v, \\
\ \ \ \ $\delta_{20}$ = e $\circ$ e, s$_{10}$ < $\delta_{20}$ \\
\ \ \ \ $\vdash$ $\infty$ (v, e, $\delta_{20}$ | $\delta_{10}$ $\delta_{11}$, s$_0$, $\delta_{20}$, s$_{10}$)
\end{tabular}
\end{table}

\newpage
\section{Properties}

Soundness and completeness link syntax to semantics.
Canonicity, normalization, and totality ensure computational adequacy.
Consistency and decidability guarantee logical and practical usability.
Conservativity and initiality support extensibility and universality.

\subsection{Soundness and Completeness}

Soundness is proven via cubical sets \cite{CCHM, Awodey12, Coquand18}.

\subsection{Canonicity, Normalization and Totality}

Canonicity and normalization hold constructively \cite{Huber17, Streicher91}.

\subsection{Consistency and Decidability}

Consistency follows from the model \cite{Bezem14}.
Decidability is achieved for type checking \cite{Coquand18}.

\subsection{Conservativity and Initiality}

Conservativity and initiality is discussed bu Shulman\cite{Hofmann97, Shulman15}.
Initiality is implicit in the syntactic construction \cite{Awodey12}.

\section{Conclusion}

This paper presents Anders, a proof assistant that reimplements
cubicaltt within a Modal Homotopy Type System framework, based
on MLTT-80 and CCHM/CHM. It integrates HTS strict equality,
infinitesimal modalities, and primitives like suspensions or quotients,
with the extension adding simplicial types and Hopf fibrations.
Anders offers an efficient, idiomatic system — compiling in under
one second — using a syntax of Lean and semantics of cubicaltt and Cubical Agda.
As a practical refinement of cubicaltt, Anders serves as an accessible
tool for homotopy type theory, with potential for incremental
enhancements like a tactic language.

\newpage
\bibliography{anders}

\end{document}
