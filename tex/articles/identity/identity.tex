% copyright (c) 2015 Synrc Research Center

\documentclass[11pt,oneside]{article}

\usepackage[osf]{mathpazo}
\usepackage[mathcal,mathbf]{euler}
\usepackage{amsmath,amssymb,amsthm}
\usepackage[utf8]{inputenc}
\usepackage{graphicx,sidecap,tikz}
\usepackage{siunitx}
\usepackage{fullwidth}
\usepackage{fontspec}
\usepackage{hyphenat}
\usepackage{ifthen}
\usepackage[usenames,dvipsnames]{color}
\usepackage[english,russian]{babel}
\usepackage{bussproofs}
\usepackage{tabstackengine}
\usepackage{graphicx}
\usepackage{cite}
\usepackage{hyperref}
\usepackage{listingsutf8}
\usepackage{moreverb}
\usepackage{listings}
\usepackage{caption}
\usepackage{amssymb}
\usepackage{mathtools}

% To get lining figures in tables set by siunitx, which apparently uses the
% \mathrm font instead of \mathnormal
\SetMathAlphabet{\mathrm}{normal}{U}{eur}{m}{n}

% =========================
% = Setting up the layout =
% =========================

% With a 9pt body font we want a little extra line spacing (I mean *leading*)
\setSingleSpace{1.2}
\SingleSpacing

% Okay, holy crap. Calculating the correct type block height by hand is quite
% challenging (partially because not all lines are \baselineskip high;
% apparently the first line is \topskip high?), and \checkandfixthelayout will
% in the end actually *change* it so that the type block is always an integer
% multiple of lines. The easiest thing is to set the approximate desired type
% block height, the width, the left or right margin, the bottom margin, and
% the headdrop, and then let memoir take care of everything else. Changing the
% algorithm used to check the layout helps as well.
\setstocksize{9in}{6in}
%\stockimperialvo

\settrimmedsize{\stockheight}{\stockwidth}{*}
\settrims{0pt}{0pt}

\settypeblocksize{46\baselineskip}{4in}{*}
\setlrmargins{*}{0.5in}{*}
\setulmargins{*}{0.5in}{*}

\setheadfoot{\baselineskip}{\baselineskip} % headheight and footskip
\setheaderspaces{0.5in}{*}{*} % headdrop, headsep, and ratio

\checkandfixthelayout[lines]

% Set up custom headers and footers
\makepagestyle{stylish}
\copypagestyle{stylish}{headings}
\makerunningwidth{stylish}{5in}
\makeheadposition{stylish}{flushleft}{flushright}{}{}
\pagestyle{stylish}

% ============================
% = Table of contents tweaks =
% ============================
\renewcommand*{\contentsname}{Table of Contents}
\setsecnumdepth{subsubsection}
\settocdepth{subsection}

% ============
% = Chapters =
% ============
\newcommand{\swelledrule}{
    \tikz \filldraw[scale=.015,very thin]
    (0,0) -- (100,1) -- (200,1) -- (300,0) --
    (200,-1) -- (100,-1) -- cycle;}
\makeatletter
\makechapterstyle{grady}{
    \setlength{\beforechapskip}{0pt}
    \renewcommand*{\chapnamefont}{\large\centering\scshape}
%    \renewcommand*{\chapnumfont}{\large}
%    \renewcommand*{\printchapternum}{
%        \chapnumfont \ifanappendix \thechapter \else \numtoName{\c@chapter}\fi}
    \renewcommand*{\printchapternonum}{
        \vphantom{\printchaptername}
        \vphantom{\chapnumfont 1}
        \afterchapternum
        \vskip -\onelineskip}
    \renewcommand*{\chaptitlefont}{\Huge\itshape}
    \renewcommand*{\printchaptertitle}[1]{
        \centering\chaptitlefont ##1\par\swelledrule}}
\makeatother
\chapterstyle{grady}

% See below, after introduction, for \clearforchapter

% Prevent page numbers from appearing on chapter pages
\aliaspagestyle{chapter}{empty}

% ===================
% = Marginal labels =
% ===================
\strictpagecheck % Turn on robust page checking
\captiondelim{} % Don't print a colon after "Figure #.#"

\makeatletter
\renewcommand{\fnum@figure}{
    \checkoddpage
    \ifoddpage
        \makebox[0pt][l]{\hspace{-1in}{\scshape\figurename~\thefigure}}
    \else
        \makebox[0pt][r]{{\scshape\figurename~\thefigure}\hspace*{-5in}}
    \fi
    }

\renewcommand{\fnum@table}{
    \checkoddpage
    \ifoddpage
        \makebox[0pt][l]{\hspace{-1in}{\scshape\tablename~\thetable}}
    \else
        \makebox[0pt][r]{{\scshape\tablename~\thetable}\hspace*{-5in}}
    \fi
    }

\let\mytagform@=\tagform@
\def\tagform@#1{
\checkoddpage
    \ifoddpage
    \makebox[1sp][l]{\hspace{-5in}\maketag@@@{(\ignorespaces#1 \unskip \@@italiccorr)}}
    \else
    \makebox[1sp][r]{\maketag@@@{(\ignorespaces#1
                                  \unskip \@@italiccorr)}\hspace*{-1in}}
    \fi
    }
\renewcommand{\eqref}[1]{\textup{\mytagform@{\ref{#1}}}}
\makeatother

\usetikzlibrary{arrows,positioning,decorations.pathmorphing,trees}

\newcommand{\titleINF}{

\frontmatter
\thispagestyle{empty}

\mbox{}\vspace{3in}
\noindent
\begin{flushright}
{\HUGE Дисертація}\\
\vspace{0.5cm}
{\LARGE \bf Система верифікації програмного забезпечення}\\
\vspace{0.5em}
{\bf Автореферат на здобуття ступені доктора філософії}
\end{flushright}

\vspace{7cm}
\hfill{\Large\scshape{}Максим Сохацький, Павло Маслянко}

\cleartorecto\tableofcontents*

\mainmatter

}

\lefthyphenmin=1
\hyphenpenalty=100
\tolerance=3000

\fontencoding{T1}
\newfontfamily{\cyrillicfont}{Geometria}
\setmainfont{Geometria}

\addto\captionsrussian{\renewcommand{\contentsname}{Зміст}}
\addto\captionsrussian{\renewcommand{\bibname}{Бібліографія}}

\lstset{morekeywords={record,data,inductive,extend,enum,Type,Path,unit,Unit,Nat,List,let,in,eq,type,sh,sub,star,var,dep,norm,fun,app,lambda,arrow,pi,case,receive,spawn,send}}

\lstset{
    backgroundcolor=\color{white},
    keywordstyle=\color{blue},
    inputencoding=utf8,
    basicstyle=\bf\ttfamily\footnotesize,
    xleftmargin=0cm,
    columns=fixed}

\begin{document}

\thispagestyle{empty}
\begin{center}

\begingroup
\parbox[t][][l]{0.30\textwidth}{ \includegraphics[scale=0.3]{img/grp} }
\parbox[t][][r]{0.60\textwidth}{ \flushright \textsc{{\Large {\bf {Groupoid Infinity}}}}\\
                                             \textsc{Kostiantynivska 20/14, Kyiv, Ukraine 04071}\\  }\endgroup

\vspace{6cm}   {\Large \bf Identity Type Encoding\\}\par
\vspace{0.3cm} {\Large Maxim Sokhatsky\par}
\vspace{6cm}   {\Large Groupoid Infinity\par}

\end{center}

\newpage
\vspace{2cm}
\tableofcontents
\newpage

\section{Abstract}

In this article Gropoid Infinit will show how to encode classical
Identity Types used in Type Theory using EXE language. We provide brief explanation
of Identity Type properties and our motivation why use Setoid Types instead
of classical Identity Types from Type Theory.

\begin{lstlisting}[mathescape=true]
record Id (A: Type): A $\rightarrow$ A $\rightarrow$ Type :=
       (refl (a: A): Id a a)
\end{lstlisting}

\section{Identity Type}

\subsection{Typing Rules}

\begin{equation}
\tag{$Id$-formation}
\dfrac
  {a:A\ \ \ \ b:A\ \ \ \ A:Type}
  {Id(A,a,b) : Type}
\end{equation}

\begin{equation}
\tag{$Id$-intro}
\dfrac
  {a:A}
  {refl(A,a) : Id(A,a,a) }
\end{equation}

\begin{equation}
\tag{$J$-elimination}
\dfrac
  {p:Id(a,b)\ \ \ \ x,y:A\ \ \ \ u:Id(x,y) \vdash E:Type\ \ \ \ x:A \vdash d: E\ [x/y,\ refl(x)/u]}
  {J(a,b,p,(x,y,u)\ d) : E\ [a/x,\ b/y,\ p/u]}
\end{equation}

\begin{equation}
\tag{$Id$-computation}
\dfrac
  {a,x,y:A,\ \ \ \ u:Id(x,y) \vdash E:Type\ \ \ \ x:A \vdash d:E\ [x/y,\ refl(x)/u]}
  {J(a,a,refl(a),(x,y,u)\ d) = d\ [a/x] : E\ [a/y,\ refl(a)/u]}
\end{equation}

\begin{lstlisting}[mathescape=true]
record Id (A: Type): Type :=
       (Id: A $\rightarrow$ A $\rightarrow$ Type)
       (refl (a: A): Id a a)
       (Predicate := $\forall$ (x,y: A) $\rightarrow$ Id x y $\rightarrow$ Type)
       (Forall (C: Predicate) := $\forall$ (x,y: A) $\rightarrow$ $\forall$ (p: Id x y) $\rightarrow$ C x y p)
       ($\Delta$ (C: Predicate) := $\forall$ (x: A) $\rightarrow$ C x x (refl x))
       (axiom-J (C: Predicate): $\Delta$ C $\rightarrow$ Forall C)
       (computation-rule (C: Predicate) (t: $\Delta$ C):
                         $\forall$ (x: A) $\rightarrow$ (J C t x x (refl x)) ==> (t x))
\end{lstlisting}

\newpage
\subsection{Symmetry and Transitivity}

\begin{lstlisting}[mathescape=true]
record Properties (A: Type): Type :=
       (Trans (a,b,c: A) : Id a b $\rightarrow$ Id b c $\rightarrow$ Id a c :=
                           Id.axiom-J ($\lambda$ a b p1 $\rightarrow$ Id b c $\rightarrow$ Id a c) ($\lambda$ x p2 $\rightarrow$ p2) a ab)
       (Sym (a,b: A)     : Id a b $\rightarrow$ Id b a :=
                           Id.axiom-J ($\lambda$ a b p $\rightarrow$ Id b a) Id.refl a b)
\end{lstlisting}

\subsection{Substitution in Predicates}

\begin{lstlisting}[mathescape=true]
record Subst (A: Type): Type :=
       (intro (P (a: A): Type) (a1, a2: A) : Id a1 a2 $\rightarrow$ P a1 $\rightarrow$ P a2 :=
        Id.axiom-J ($\lambda$ a1 a2 p12 $\rightarrow$ P a1 $\rightarrow$ P a2) ($\lambda$ a0 p0 $\rightarrow$ p0) a1 a2)
\end{lstlisting}

\subsection{Uniqueness of Identity Proofs}

\begin{lstlisting}[mathescape=true]
record UIP (A: Type): Type :=
       (intro (A: Type) (a,b: A) (x,y: Id a b) : Id (Id A a b) x y)
\end{lstlisting}

\subsection{Axiom K}

\begin{lstlisting}[mathescape=true]
record K (A: Type): Type :=
       (Predicate$_K$ := $\forall$ (a: A) $\rightarrow$ Id a a $\rightarrow$ Type)
       (Forall$_K$ (C: Predicate$_K$) := $\forall$ (a: A) $\rightarrow$ $\forall$ (p: Id a a) $\rightarrow$ C a p)
       ($\Delta_K$ (C: Predicate$_K$) := $\forall$ (a: A) $\rightarrow$ C a (Id.refl a))
       (axiom-K (C: Predicate): $\Delta_K$ C $\rightarrow$ Forall$_K$ C)
\end{lstlisting}

\subsection{Congruence}

\begin{lstlisting}[mathescape=true]
define Respect$_1$ (A,B: Type) (C: A $\rightarrow$ Type) (D: B $\rightarrow$ Type) (R: A $\rightarrow$ B $\rightarrow$ Prop)
       (Ro: $\forall$ (x: A) (y: B) $\rightarrow$ C x $\rightarrow$ D y $\rightarrow$ Prop) :
            ($\forall$ (x: A) $\rightarrow$ C x) $\rightarrow$ ($\forall$ (x: B) $\rightarrow$ D x) $\rightarrow$ Prop :=
                       $\lambda$ (f g: Type $\rightarrow$ Type) $\rightarrow$ ($\forall$ (x y) $\rightarrow$ R x y) $\rightarrow$ Ro x y (f x) (g y)
\end{lstlisting}

\begin{lstlisting}[mathescape=true]
record Respect$_2$ (A: Type): Type :=
       (intro (A,B: Type) (f: A $\rightarrow$ B) (a,b: A) : Id A a b $\rightarrow$ Id B (f a) (f b) :=
              Id.axiom-J ($\lambda$ a b p12 $\rightarrow$ Id B (f a) (f b)) ($\lambda$ x $\rightarrow$ refl B (f x)) a b)
\end{lstlisting}

\subsection{Setoid}

\begin{lstlisting}[mathescape=true]
record Setoid: Type :=
       (Carrier: Type)
       (Equ: Carrier $\rightarrow$ Carrier $\rightarrow$ Prop)
       (Refl:  $\forall$ (e$_0$: Carrier) $\rightarrow$ Equ e$_0$ e$_0$)
       (Trans: $\forall$ (e$_1$,e$_2$,e$_3$: Carrier) $\rightarrow$ Equ e$_1$ e$_2$ $\rightarrow$ Equ e$_2$ e$_3$ $\rightarrow$ Equ e$_1$ e$_3$)
       (Sym:   $\forall$ (e$_1$ e$_2$: Carrier) $\rightarrow$ Equ e$_1$ e$_2$ $\rightarrow$ Equ e$_2$ e$_1$)
\end{lstlisting}

\subsection{Conclusion}
As you can see EXE language has enough expressive power to be used for
drawing MLTT axioms in computer science articles and papers.

\newpage
\begin{thebibliography}{9}

\bibitem{bis} E.Bishop \textit{Foundations of Constructive Analysis} 1967
\bibitem{str} T.Streicher, M.Hofmann  \textit{The groupoid interpretation of type theory} 1996
\bibitem{bar} G.Barthe, V.Capretta \textit{Setoids in type theory} 2003
\bibitem{soz} M.Sozeau, N.Tabareau \textit{Internalizing Intensional Type Theory} 2013
\bibitem{voe} V.Voevodsky \textit{Identity types in the C-systems defined by a universe category} 2015
\bibitem{con} D.Selsam and L.de Moura \textit{Congruence Closure in Intensional Type Theory} 2016
\end{thebibliography}

\end{document}
