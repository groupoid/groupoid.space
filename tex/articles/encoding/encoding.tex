% copyright (c) 2015 Synrc Research Center

\documentclass[11pt,oneside]{article}

\usepackage[osf]{mathpazo}
\usepackage[mathcal,mathbf]{euler}
\usepackage{amsmath,amssymb,amsthm}
\usepackage[utf8]{inputenc}
\usepackage{graphicx,sidecap,tikz}
\usepackage{siunitx}
\usepackage{fullwidth}
\usepackage{fontspec}
\usepackage{hyphenat}
\usepackage{ifthen}
\usepackage[usenames,dvipsnames]{color}
\usepackage[english,russian]{babel}
\usepackage{bussproofs}
\usepackage{tabstackengine}
\usepackage{graphicx}
\usepackage{cite}
\usepackage{hyperref}
\usepackage{listingsutf8}
\usepackage{moreverb}
\usepackage{listings}
\usepackage{caption}
\usepackage{amssymb}
\usepackage{mathtools}

% To get lining figures in tables set by siunitx, which apparently uses the
% \mathrm font instead of \mathnormal
\SetMathAlphabet{\mathrm}{normal}{U}{eur}{m}{n}

% =========================
% = Setting up the layout =
% =========================

% With a 9pt body font we want a little extra line spacing (I mean *leading*)
\setSingleSpace{1.2}
\SingleSpacing

% Okay, holy crap. Calculating the correct type block height by hand is quite
% challenging (partially because not all lines are \baselineskip high;
% apparently the first line is \topskip high?), and \checkandfixthelayout will
% in the end actually *change* it so that the type block is always an integer
% multiple of lines. The easiest thing is to set the approximate desired type
% block height, the width, the left or right margin, the bottom margin, and
% the headdrop, and then let memoir take care of everything else. Changing the
% algorithm used to check the layout helps as well.
\setstocksize{9in}{6in}
%\stockimperialvo

\settrimmedsize{\stockheight}{\stockwidth}{*}
\settrims{0pt}{0pt}

\settypeblocksize{46\baselineskip}{4in}{*}
\setlrmargins{*}{0.5in}{*}
\setulmargins{*}{0.5in}{*}

\setheadfoot{\baselineskip}{\baselineskip} % headheight and footskip
\setheaderspaces{0.5in}{*}{*} % headdrop, headsep, and ratio

\checkandfixthelayout[lines]

% Set up custom headers and footers
\makepagestyle{stylish}
\copypagestyle{stylish}{headings}
\makerunningwidth{stylish}{5in}
\makeheadposition{stylish}{flushleft}{flushright}{}{}
\pagestyle{stylish}

% ============================
% = Table of contents tweaks =
% ============================
\renewcommand*{\contentsname}{Table of Contents}
\setsecnumdepth{subsubsection}
\settocdepth{subsection}

% ============
% = Chapters =
% ============
\newcommand{\swelledrule}{
    \tikz \filldraw[scale=.015,very thin]
    (0,0) -- (100,1) -- (200,1) -- (300,0) --
    (200,-1) -- (100,-1) -- cycle;}
\makeatletter
\makechapterstyle{grady}{
    \setlength{\beforechapskip}{0pt}
    \renewcommand*{\chapnamefont}{\large\centering\scshape}
%    \renewcommand*{\chapnumfont}{\large}
%    \renewcommand*{\printchapternum}{
%        \chapnumfont \ifanappendix \thechapter \else \numtoName{\c@chapter}\fi}
    \renewcommand*{\printchapternonum}{
        \vphantom{\printchaptername}
        \vphantom{\chapnumfont 1}
        \afterchapternum
        \vskip -\onelineskip}
    \renewcommand*{\chaptitlefont}{\Huge\itshape}
    \renewcommand*{\printchaptertitle}[1]{
        \centering\chaptitlefont ##1\par\swelledrule}}
\makeatother
\chapterstyle{grady}

% See below, after introduction, for \clearforchapter

% Prevent page numbers from appearing on chapter pages
\aliaspagestyle{chapter}{empty}

% ===================
% = Marginal labels =
% ===================
\strictpagecheck % Turn on robust page checking
\captiondelim{} % Don't print a colon after "Figure #.#"

\makeatletter
\renewcommand{\fnum@figure}{
    \checkoddpage
    \ifoddpage
        \makebox[0pt][l]{\hspace{-1in}{\scshape\figurename~\thefigure}}
    \else
        \makebox[0pt][r]{{\scshape\figurename~\thefigure}\hspace*{-5in}}
    \fi
    }

\renewcommand{\fnum@table}{
    \checkoddpage
    \ifoddpage
        \makebox[0pt][l]{\hspace{-1in}{\scshape\tablename~\thetable}}
    \else
        \makebox[0pt][r]{{\scshape\tablename~\thetable}\hspace*{-5in}}
    \fi
    }

\let\mytagform@=\tagform@
\def\tagform@#1{
\checkoddpage
    \ifoddpage
    \makebox[1sp][l]{\hspace{-5in}\maketag@@@{(\ignorespaces#1 \unskip \@@italiccorr)}}
    \else
    \makebox[1sp][r]{\maketag@@@{(\ignorespaces#1
                                  \unskip \@@italiccorr)}\hspace*{-1in}}
    \fi
    }
\renewcommand{\eqref}[1]{\textup{\mytagform@{\ref{#1}}}}
\makeatother

\usetikzlibrary{arrows,positioning,decorations.pathmorphing,trees}

\newcommand{\titleINF}{

\frontmatter
\thispagestyle{empty}

\mbox{}\vspace{3in}
\noindent
\begin{flushright}
{\HUGE Дисертація}\\
\vspace{0.5cm}
{\LARGE \bf Система верифікації програмного забезпечення}\\
\vspace{0.5em}
{\bf Автореферат на здобуття ступені доктора філософії}
\end{flushright}

\vspace{7cm}
\hfill{\Large\scshape{}Максим Сохацький, Павло Маслянко}

\cleartorecto\tableofcontents*

\mainmatter

}

\lefthyphenmin=1
\hyphenpenalty=100
\tolerance=3000

\fontencoding{T1}
\newfontfamily{\cyrillicfont}{Geometria}
\setmainfont{Geometria}

\addto\captionsrussian{\renewcommand{\contentsname}{Зміст}}
\addto\captionsrussian{\renewcommand{\bibname}{Бібліографія}}

\lstset{morekeywords={record,data,inductive,extend,enum,Type,Path,unit,Unit,Nat,List,let,in,eq,type,sh,sub,star,var,dep,norm,fun,app,lambda,arrow,pi,case,receive,spawn,send}}

\lstset{
    backgroundcolor=\color{white},
    keywordstyle=\color{blue},
    inputencoding=utf8,
    basicstyle=\bf\ttfamily\footnotesize,
    xleftmargin=0cm,
    columns=fixed}

\begin{document}

\thispagestyle{empty}
\begin{center}

\begingroup
\parbox[t][][l]{0.30\textwidth}{ \includegraphics[scale=0.3]{img/grp} }
\parbox[t][][r]{0.60\textwidth}{ \flushright \textsc{{\Large {\bf {Groupoid Infinity}}}}\\
                                             \textsc{Kostiantynivska 20/14, Kyiv, Ukraine 04071}\\  }\endgroup

\vspace{6cm}   {\Large \bf Categorical Encoding of Inductive Types\\}\par
\vspace{0.3cm} {\Large Maxim Sokhatsky\par}
\vspace{6cm}   {\Large Groupoid Infinity\par}
\vspace{0.3cm} {\Large Kamianets-Podilsky 2016}

\end{center}

\newpage
\vspace{2cm}
\tableofcontents
\newpage

\section{Abstract}

\paragraph{}
{\bf From PTS to HTS}. We want to have flexible detachable
layers on top of PTS core. Then Sigma for proving.
Then well-founded trees or polynomial functors as known as data and record.
Then higher path types, interval arithmetic, glue and comp for HIT.
Each layers is driven by differenth math, the common in only the method -- category theory.

\paragraph{}
{\bf Extensible Language Design}. Encoding of inductive types is based on categorical
semantic of compilation to PTS. All other syntax constructions are inductive
definitions, plugged into the stream parser. AST of the PTS language is also
defined in terms of inductive constructions and thus allowed in the macros.
The language of polynomial functors (data and record) and core language of
the process calculus (spawn, receive and send) are just macrosystem over Om language,
its syntax extensions.

\paragraph{}
{\bf Changable Encodings}. In pure CoC we have only arrows, so all inductive type encodings would
be Church-encoding variations. Most extended nowadays is Church-Boehm-Berrarducci encoding,
which dedicated to inductive types. Another well known are Scott (lazyness),
Parigot (lazyness and constant-time iterators) and CPS (continuations) encodings.
However most of them require variations of Fixpoint types.

\paragraph{}
{\bf Proved Categorical Semantic}. There was modeled a math model (using
higher-order categorical logic) of encoding, which calculates (co)limits in a
cathegory of (co)algebras built with given set of (de)constructors.
We call such encoding in honour of Lambek lemma that leeds us to the
equality of (co)initial object and (co)limit in the categories of (co)algebras.
Such encoding works with dependent types and its consistency is proved in Lean model.


\newpage
\section{Type Theory}

\subsection{Intuitionistic Type Theory}

\paragraph{}
Type theories are formal calculi that can be seen as both mathematical
formal logics, set theories, programming languages and formalized
programming logics used to prove program properties and derive a correct program from a specification \cite{nordstrom}.

\paragraph{}
We will follow the tradition in the literature on type systems \cite{henk1} and formulate our calculus with judgment rules in a well-known
notation of Natural Deduction. There are two types of judgement rules: {\bf a} is a object of type {\bf A};
{\bf a} and {\bf b} are definitionaly equal objects of type {\bf A}.
Rules for each type can be in turn classified into {\bf formations}, {\bf introductions}, {\bf eliminators} and {\bf computational rules}.

\paragraph{}
Formation rules are typing rules,
introductions (type constructors) and eliminators are adjunctions that correspond to axioms,
and computational rules correspond to operational semantics.
From categorical point of view {\bf typing rules} correspond to {\bf functors},
{\bf axioms} to {\bf morphisms} and {\bf operational semantics} to {\bf equalities on morphisms}.

\paragraph{}
Our variant of intuitionistic type system has an {\bf infinite} hierarchy of universes,
the hierarchy is {\bf non-cumulative}, the universe of propositions is {\bf impredicative} while universes of values, types and higher types are {\bf predicative}. The next paragraphs briefly
describe the reasons for such design choices.

\paragraph{}
While there are variants of MLTT with finite number of universes (see
for example the lambda cube and PTS formulations of Calculus of
Constructions in \cite{henk1}), most modern provers have an infinite
hierarchy of universes as it's {\bf more expressive}.

\paragraph{}
The first universe, the universe of propositions, has impredicative quantification
while all other universes are predicative. This setup
is known to simplify the type system while keeping it free of
paradoxes and thus {\bf sound as a formal logic}.

\paragraph{}
The first few universes have analogs in logic and axiomatic set theories. The bottom is contractible space.
Universe 0 can be thought of as the universe of propostions. Universe 1 contains elements.
Universe 1 contains sets. Universe 2 contains classes.  The hierarchy can be cumulative or non-cumulative.
In cumulative setting lower universes are included in higher ones --- e.g. all elements
are sets. In non-cumulative setting the universes don't intersect,
so the case is analogous to axiomatic set theories {\bf with urelements}.

\newpage
\subsection{Universes}

\begin{equation}
\tag{sorts}
\dfrac
  {i : Nat}
  {Type_i}
\end{equation}

\begin{equation}
\tag{axioms}
\dfrac
  {i: Nat}
  {Type_i : Type_{i+1}}
\end{equation}

\begin{equation}
\tag{rules}
\dfrac
  {i : Nat,\ \ \ \ j : Nat}
  {Type_i \rightarrow Type_{j} : Type_{max(i,j)}}
\end{equation}

  \subsection{Dependent Types}

\begin{equation}
\tag{subst}
\dfrac
  {\pi_1 : A\ \ \ \ u:A \vdash \pi_2 : B}
  {[\pi_1/u]\ \pi_2 : B}
\end{equation}

\begin{equation}
\tag{$\Pi$-formation}
\dfrac
  {x:A \vdash B : Type}
  {\Pi\ (x:A) \rightarrow B : Type}
\end{equation}

\begin{equation}
\tag{$\lambda$-intro}
\dfrac
  {x:A \vdash b : B}
  {\lambda\ (x:A) \rightarrow b : \Pi\ (x: A) \rightarrow B }
\end{equation}

\begin{equation}
\tag{$App$-elimination}
\dfrac
  {f: (\Pi\ (x:A) \rightarrow B)\ \ \ a: A}
  {f\ a : B\ [a/x]}
\end{equation}

\begin{equation}
\tag{$\beta$-computation}
\dfrac
  {x:A \vdash b: B\ \ \ a:A}
  {(\lambda\ (x:A) \rightarrow b)\ a = b\ [a/x] : B\ [a/x]}
\end{equation}

\newpage

\begin{lstlisting}[mathescape=true]
record $\Pi$ (A: Type): Type :=
       ($\Pi$: (A $\rightarrow$ Type) $\rightarrow$ Type)
       (fun: (B: A $\rightarrow$ Type) $\rightarrow$ $\forall$ (a: A) $\rightarrow$ B a $\rightarrow$ $\Pi$ B)
       (app: (B: A $\rightarrow$ Type) $\rightarrow$ $\Pi$ B $\rightarrow$ $\forall$ (a: A) $\rightarrow$ B a)
       (app-fun (B: A $\rightarrow$ Type) (f: $\forall$ (a: A) $\rightarrow$ B a): $\forall$ (a: A) $\rightarrow$ app (fun f) a ==> f a)
       (fun-app (B: A $\rightarrow$ Type) (p: $\Pi$ B): fun ($\lambda$(a: A) $\rightarrow$ app p a) ==> p)
\end{lstlisting}

  \subsection{Identity Types}

\begin{equation}
\tag{$Id$-formation}
\dfrac
  {a:A\ \ \ \ b:A\ \ \ \ A:Type}
  {Id(A,a,b) : Type}
\end{equation}

\begin{equation}
\tag{$Id$-intro}
\dfrac
  {a:A}
  {refl(A,a) : Id(A,a,a) }
\end{equation}

\begin{equation}
\tag{$J$-elimination}
\dfrac
  {p:Id(a,b)\ \ \ \ x,y:A\ \ \ \ u:Id(x,y) \vdash E:Type\ \ \ \ x:A \vdash d: E\ [x/y,\ refl(x)/u]}
  {J(a,b,p,(x,y,u)\ d) : E\ [a/x,\ b/y,\ p/u]}
\end{equation}


\begin{equation}
\tag{$Id$-computation}
\dfrac
  {a,x,y:A,\ \ \ \ u:Id(x,y) \vdash E:Type\ \ \ \ x:A \vdash d:E\ [x/y,\ refl(x)/u]}
  {J(a,a,refl(a),(x,y,u)\ d) = d\ [a/x] : E\ [a/y,\ refl(a)/u]}
\end{equation}

\begin{lstlisting}[mathescape=true]
record Id (A: Type): Type :=
       (Id: A $\rightarrow$ A $\rightarrow$ Type)
       (refl (a: A): Id a a)
       (Predicate := $\forall$ (x,y: A) $\rightarrow$ Id x y $\rightarrow$ Type)
       (Forall (C: Predicate) := $\forall$ (x,y: A) $\rightarrow$ $\forall$ (p: Id x y) $\rightarrow$ C x y p)
       ($\Delta$ (C: Predicate) := $\forall$ (x: A) $\rightarrow$ C x x (refl x))
       (axiom-J (C: Predicate): $\Delta$ C $\rightarrow$ Forall C)
       (computation-rule (C: Predicate) (t: $\Delta$ C):
                         $\forall$ (x: A) $\rightarrow$ (J C t x x (refl x)) ==> (t x))
\end{lstlisting}

\subsection{Inductive Types}

\begin{equation}
\tag{$W$-formation}
\dfrac
  {A:Type\ \ \ \ x:A\ \ \ \ B(x):Type}
  {W (x:A) \rightarrow B(x) : Type}
\end{equation}

\begin{equation}
\tag{$W$-intro}
\dfrac
  {a:A\ \ \ \ t: B(a) \rightarrow W}
  {sup(a,t) : W}
\end{equation}

\begin{equation}
\tag{$W$-elimination}
\dfrac
  {w: W \vdash C(w) : Type\ \ \ \ x:A,\ u:B(x) \rightarrow W,\ \ \ \ v:\Pi (y:B(x)) \rightarrow C(u(y)) \vdash c(x,u,v):C(sup(x,u))}
  {w:W \vdash wrec(w,c):C(w)}
\end{equation}

\begin{equation}
\tag{$W$-computation}
\dfrac
  {w: W \vdash C(w) : Type\ \ \ \ x:A,\ u:B(x) \rightarrow W,\ \ \ \ v:\Pi (y:B(x)) \rightarrow C(u(y)) \vdash c(x,u,v):C(sup(x,u))}
  {x:A,\ \ \ \ u:B(x) \rightarrow W \vdash wrec(sup(x,u),c)\ =\ c(x,u,\lambda (y:B(x)) \rightarrow wrec(u(y),c)):C(sup(x,u))}
\end{equation}

\newpage
\section{Category Theory}
\vspace{0.3cm}

   \subsection{Programs and Functions}
   Category theory is widely used as an instrument for mathematicians for software analisys.
   Category theory could be treated as an abstract algebra of functions. Let's define an Category
   formally: {\bf Category} consists of: morphisms (arrows) and objects (domains and codomains of arrows)
   along with associative operation of composition and unit morphism that exists for all objects in category.

%   \paragraph{}
%   The formation axoims of objects and arrows are not given here and autopostulating yet. Formation axoims
%   will be introduced during exponential definition. Objects $A$ and $B$ of an arrow $f: A \rightarrow B$
%   are called {\bf domain} and {\bf codomain} respectively.

%   \paragraph{}
%   Intro axioms -- associativity of composition and left/right unit arrow compisitions show that
%   categories are actually typed monoids, which consist of morphisms and operation of composition.
%   There are many languages to show the semantic of categories such as commutative diagrams and string diagrams
%   however here we define here in proof-theoretic manner:

\begingroup
\parbox[t][][l]{0.70\textwidth}{
\flushleft
\begin{prooftree}
\RightLabel{($Comp$)}
\AxiomC{$f: A \rightarrow B$ }
\AxiomC{$g: B \rightarrow C$ }
\BinaryInfC{$g \circ f : A \rightarrow C $}
\end{prooftree}

\begin{prooftree}
\RightLabel{($Assoc$)}
\AxiomC{$f : B \rightarrow A$ }
\AxiomC{$g : C \rightarrow B$ }
\AxiomC{$h : D \rightarrow C$ }
\TrinaryInfC{$(f \circ g) \circ h = f \circ (g \circ h) : D \rightarrow A $}
\end{prooftree}

}
\hspace{0.1cm}
\parbox[t][][r]{0.40\textwidth}{

\begin{prooftree}
\RightLabel{($Id$)}
\AxiomC{$$ }
\UnaryInfC{$id_A : A \rightarrow A $}
\end{prooftree}

\begin{prooftree}
\RightLabel{($Id_L$)}
\AxiomC{$f: A \rightarrow B$ }
\UnaryInfC{$f \circ id_A = f : A \rightarrow B$}
\end{prooftree}

\begin{prooftree}
\RightLabel{($Id_R$)}
\AxiomC{$f: A \rightarrow B$ }
\UnaryInfC{$id_B \circ f = f : A \rightarrow B $}
\end{prooftree}
}
\endgroup

\paragraph{}

\begin{lstlisting}[mathescape=true]
record Setoid: Type :=
       (Carrier: Type)
       (Equ: Carrier $\rightarrow$ Carrier $\rightarrow$ Prop)
       (Refl:  $\forall$ (e$_0$: Carrier) $\rightarrow$ Equ e$_0$ e$_0$)
       (Trans: $\forall$ (e$_1$,e$_2$,e$_3$: Carrier) $\rightarrow$ Equ e$_1$ e$_2$ $\rightarrow$ Equ e$_2$ e$_3$ $\rightarrow$ Equ e$_1$ e$_3$)
       (Sym:   $\forall$ (e$_1$ e$_2$: Carrier) $\rightarrow$ Equ e$_1$ e$_2$ $\rightarrow$ Equ e$_2$ e$_1$)
\end{lstlisting}

\begin{lstlisting}[mathescape=true]
record Cat: Type :=
       (Ob: Type)
       (Hom:    $\forall$ (dom,cod: Ob) $\rightarrow$ Setoid)
       (Id:     $\forall$ (x: Ob) $\rightarrow$ Hom x x)
       (Comp:   $\forall$ (x,y,z: Ob) $\rightarrow$ Hom x y $\rightarrow$ Hom y z $\rightarrow$ Hom x z)
       (Dom$_{1\circ}$:    $\forall$ (x,y: Ob) (f: Hom x y) $\rightarrow$ (Hom.Equ x y (Comp x x y id f) f))
       (Cod$_{1\circ}$:    $\forall$ (x,y: Ob) (f: Hom x y) $\rightarrow$ (Hom.Equ x y (Comp x y y f id) f))
       (Subst$_{\circ}$: $\forall$ (x,y,z: Ob)
            (f$_1$, f$_2$: Hom x y) (Hom.Equ x y f$_1$ f$_2$)
            (g$_1$, g$_2$: Hom y z) (Hom.Equ y z g$_1$ g$_2$) $\rightarrow$
                (Hom.Equ x z (Comp x y z f$_1$ g$_1$) (Comp x y z f$_2$ g$_2$)) )
       (Assoc$_{\circ}$: $\forall$ (x,y,z,w: Ob) (f: Hom x y) (g: Hom y z) (h: Hom z w)
            $\rightarrow$ (Hom.Equ x w (Comp x y w f (Comp y z w g h))
                    (Comp x z w (Comp x y z f g) h)))
\end{lstlisting}

\newpage
\subsection{Algebraic Types and Cartesian Categories}

After composition operation of construction of new objects with morphisms we introduce
operation of construction cartesian product of two objects $A$ and $B$ of a given
category along with morphism product $<f,g>$ with a common domain, that is needed
for full definition of cartesian product of $A \times B$.

\paragraph{}
This is an internal language of cartesian category, in which for all two selected objects there is an object
of cartesian product (sum) of two objects along with its $\bot$ terminal (or $\top$ coterminal) type.
Exe languages is always equiped with product and sum types.

\paragraph{}
Product has two eliminators $\pi$ with an common domain, which are also called projections of an product.
The sum has eliminators $i$ with an common codomain.
Note that eliminators $\pi$ and $i$ are isomorphic, that is $\pi \circ \sigma = \sigma \circ \pi = id$.

\begingroup
\parbox[t][][l]{0.40\textwidth}{

\begin{prooftree}
\AxiomC{$x: A \times B$ }
\UnaryInfC{$\pi_1\ : A \times B \rightarrow A$;
           $\pi_2\ : A \times B \rightarrow B$}
\end{prooftree}

\begin{prooftree}
\AxiomC{$a:A$ }
\AxiomC{$b:B$ }
\BinaryInfC{$(a,b) : A \times B$ }
\end{prooftree}

\begin{prooftree}
\AxiomC{}
\UnaryInfC{$\top$ }
\end{prooftree}

}
\hspace{0.1cm}
\parbox[t][][r]{0.60\textwidth}{

\begin{prooftree}
\AxiomC{}
\UnaryInfC{$\bot$ }
\end{prooftree}

\begin{prooftree}
\AxiomC{$a:A$ }
\AxiomC{$b:B$ }
\BinaryInfC{$a\ |\ b : A \otimes B$}
\end{prooftree}

\begin{prooftree}
\AxiomC{$x: A \otimes B$ }
\UnaryInfC{$\i_1: A \rightarrow A \otimes B$;
           $\i_2: B \rightarrow A \otimes B$}
\end{prooftree}

}
\endgroup

   \paragraph{}

   The $\bot$ type in Haskell is used as {\bf undefined} type (empty sum component presented in all types), that
   is why Hask category is not based on cartesian closed but CPO\cite{cpo}. The $\bot$ type has no values.
   The $\top$ type is known as unit type or zero tuple $()$ often
   used as an default argument for function with zero arguments.
   Also we include here an axiom of morphism product which is given during full definition
   of product using commutative diagram. This axiom is needed for applicative
   programming in categorical abstract machine. Also consider co-version of this
   axiom for $[f,g]: B+C \rightarrow A$ morphism sums.

\begin{prooftree}
\AxiomC{$f:A \rightarrow B$ }
\AxiomC{$g:A \rightarrow C$ }
\AxiomC{$B \times C$ }
\TrinaryInfC{$\langle f,g \rangle : A \rightarrow B \times C$ }
\end{prooftree}

\begin{center}
%$(f \circ g) \circ h = f \circ (g \circ h)$\\
%$f \circ id = f$\\
%$id \circ f = f$\\
$\pi_1 \circ \langle f, g \rangle = f$\\
$\pi_2 \circ \langle f, g \rangle = g$\\
$\langle f \circ \pi_1, f \circ \pi_2 \rangle = f$\\
$\langle f, g \rangle \circ h = \langle f \circ h, g \circ h \rangle$\\
$\langle \pi_1, \pi_2 \rangle = id$\\
\end{center}

\newpage
   \subsection{Exponential, $\lambda$-calculus and Cartesian Closed Categories}
   Being an internal language of cartesian closed category, lambda calculus except variables and constants
   provides two operations of abstraction and applications which defines complete evaluation language
   with higher order functions, recursion and corecursion, etc.

   \paragraph{}
   To explain functions from the categorical point of vew we need to define categorica exponential
   $f: A^B$, which are analogue to functions $f: A \rightarrow B$.
   As we already defined the products and terminals we could define an exponentials with three
   axioms of function construction, one eliminator of application with apply a function to its argument
   and axiom of currying the function of two arguments to function of one argument.

\begingroup
\parbox[t][][l]{0.40\textwidth}{

\begin{prooftree}
\AxiomC{$x:A \vdash M : B$}
\UnaryInfC{$\lambda\ x\ .\ M : A \rightarrow B$}
\end{prooftree}

\begin{prooftree}
\AxiomC{$f:A \rightarrow B$ }
\AxiomC{$a:A$ }
\BinaryInfC{$apply\ f\ a\ : (A \rightarrow B) \times A \rightarrow B$}
\end{prooftree}

\begin{prooftree}
\AxiomC{$f: A \times B \rightarrow C$ }
\UnaryInfC{$curry\ f : A \rightarrow (B \rightarrow C)$}
\end{prooftree}

}
\hspace{0.1cm}
\parbox[t][][r]{0.60\textwidth}{

\begin{center}
$apply \circ \langle (curry\ f) \circ \pi_1 , \pi_2 \rangle = f$\\
$curry\ apply \circ \langle g \circ \pi_1, \pi_2 \rangle) = g$\\
$apply \circ \langle curry\ f, g \rangle = f \circ \langle id , g\rangle$\\
$(curry\ f) \circ g = curry\ (f \circ \langle g \circ \pi_1,\pi_2\rangle)$\\
$curry\ apply = id$\\
\end{center}


}
\endgroup

\subsection*{$\lambda$-language as CAM machine}

\begin{center}
Objects : $\top\ |\ \rightarrow\ |\ \times$\\
Morphisms : $id\ |\ f \circ g\ |\ \langle f, g \rangle\ |\ apply\ |\ \lambda\ |\ curry$
\end{center}

\newpage
  \subsection{Functors, $\Lambda$-calculus}

  Functor comes as a notion of morphisms in categories whose objects are categories.
  Functors preserve compositions of arrows and identities, otherwise it would
  be impossible to deal with categories. One level up is notion of morphism between categories whose
  objects are Functors, such morphisms are called natural transformations. Here we need
  only functor definition which is needed as general type declarations.

\begin{prooftree}
\AxiomC{$f\ :\ A \rightarrow B$}
\UnaryInfC{$F\ f\ :\ (A \rightarrow B) \rightarrow (F\ a \rightarrow F\ b)$}
\end{prooftree}

\begin{prooftree}
\AxiomC{$id_A\ :A \rightarrow A$}
\UnaryInfC{$F\ id_A\ =\ id_{F A}\ :\ F\ A \rightarrow F\ A$}
\end{prooftree}

\begin{prooftree}
\AxiomC{$f\ :B \rightarrow C,\ g : A \rightarrow B$}
\UnaryInfC{$F\ f \circ F\ g\ =\ F (f \circ g)\ :\ F\ A \rightarrow F\ C$}
\end{prooftree}

   We start thinking of functors on dealing with typed theories, because functors usually
   could be seen as higher order type con

\vspace{1cm}
\begin{lstlisting}[mathescape=true]
   record Functor ($\varphi$: Type $\rightarrow$ Type): Type :=
          (fmap: $\forall$ ($\alpha$,$\beta$: Type) $\rightarrow$ ($\alpha$ $\rightarrow$ $\beta$) $\rightarrow$ $\varphi$ $\beta$ $\rightarrow$ $\varphi$ $\beta$;
          (id$_\rightarrow$: $\forall$ ($\alpha$: Type) (x: $\varphi$ $\alpha$) $\rightarrow$ fmap id x = x)
          (comp$_\rightarrow$: $\forall$ ($\alpha$,$\beta$,$\gamma$: Type) (f: $\beta$ $\rightarrow$ $\gamma$) (g: $\alpha$ $\rightarrow$ $\beta$) (x: $\varphi$ $\alpha$)
                   $\rightarrow$ fmap (f $\circ$ g) x = (fmap f $\circ$ fmap g) x)
\end{lstlisting}





\newpage
   \subsection{Algebras}

   F-Algebras gives us a categorical understanding recursive types.
   Let $F : C \rightarrow C$ be an endofunctor on category $C$.
   An F-algebra is a pair $(C, \phi)$, where C is an object and $\phi\ : F\ C \rightarrow C$
   an arrow in the category C. The object C is the carrier and the functor
   F is the signature of the algebra. Reversing arrows gives us F-Coalgebra.

\vspace{1cm}

\begin{center}
\begin{tabular}{lcl}
\begin{tikzcd}
  F\ C \arrow{d}[left]{F\ f} \arrow{r}{\varphi} & C \arrow{d}{f} \\
  F\ D \arrow{r}{\psi} & D \end{tikzcd} & & \begin{tikzcd}
  C \arrow{d}[left]{f} \arrow{r}{\varphi} & F\ C \arrow{d}{F\ f} \\
  D \arrow{r}{\psi} & F\ D \end{tikzcd} \\
  \ & \  &\  \\
  $f \circ \varphi = \psi \circ F\ f$ & & $\psi \circ f =  F\ f \circ \varphi$ \\
\end{tabular}
\end{center}

  \subsection{Initial Algebras}

  A F-algebra $(\mu F, in)$ is the initial F-algebra if for any F-algebra $(C, \varphi)$
  there exists a unique arrow $\llparenthesis \varphi \rrparenthesis : \mu F \rightarrow C$ where $f = \llparenthesis \varphi \rrparenthesis$
  and is called catamorphism. Similar a F-coalgebra $(\nu F, out)$ is the terminal
  F-coalgebra if for any F-coalgebra $(C, \varphi)$ there exists unique arrow
  $\llbracket \varphi \rrbracket : C \rightarrow \nu F$ where $f =
  \llbracket \varphi \rrbracket$

\begin{center}
\begin{tabular}{lcl}
\begin{tikzcd}
  F\ \mu F \arrow{d}[left]{F\ \llparenthesis \varphi \rrparenthesis} \arrow{r}{in} & \mu F \arrow{d}{\llparenthesis \varphi \rrparenthesis} \\
  F C \arrow{r}{\varphi} & C \end{tikzcd} & & \begin{tikzcd}
  C \arrow{d}[left]{ \llbracket \varphi \rrbracket} \arrow{r}{\phi} & F\ C\arrow{d}{F\ \llbracket \varphi \rrbracket} \\
  \nu F \arrow{r}{out} & F \nu F\end{tikzcd} \\
  \ & \  &\  \\
  $f \circ in = \varphi \circ F\ f \equiv f = \llparenthesis \varphi \rrparenthesis$& &
  $out \circ f = F\ f \circ \varphi \equiv f = \llbracket \varphi \rrbracket$ \\
\end{tabular}
\end{center}

   \subsection{Recursive Types}

  As was shown by Wadler\cite{recursive} we could deal with recusrive equations having
  three axioms: one $fix: (A \rightarrow A) \rightarrow A$ fixedpoint axiom,
  and axioms $in: F\ T \rightarrow T$ and $out: T \rightarrow F\ T$ of recursion direction. We need to
  define fixed point as axiom because we can't define recursive axioms. This axioms
  also needs functor axiom defined earlier.


\begingroup
\parbox[t][][l]{0.40\textwidth}{

\begin{prooftree}
\AxiomC{$M : F\ (\mu\ F)$}
\UnaryInfC{$in_{\mu F}\ M : \mu\ F$}
\end{prooftree}

\begin{prooftree}
\AxiomC{$M : \mu F$}
\UnaryInfC{$out_{\mu F}\ M : F\ (\mu\ F)$}
\end{prooftree}

}
\hspace{0.1cm}
\parbox[t][][r]{0.60\textwidth}{


\begin{prooftree}
\AxiomC{$M : A \rightarrow A$}
\UnaryInfC{$fix\ M : A$}
\end{prooftree}

}
\endgroup

%  \subsection{Peano Numbers}
%  Pointer Unary System is a category C with terminal object
%  and a carrier X having morphism $[zero: 1_C \rightarrow X,succ: X \rightarrow X]$.
%  The initial object of C is called Natural Number Object and models Peano axiom set.


\newpage
\section{Categorical Encoding}

\subsection{List Sample}
  Lambek encoding is a categorical proof-representation of higher inductive types encoding.
  Let's start with simple Curch/Boem/Berrarducci encoding for List as one of the basic inductive types:

\begin{center}
  Natural Numbers: $\mu\ X \rightarrow 1 + X$ \\
  List A: $\mu\ X \rightarrow 1 + A \times X$ \\
  Lambda calculus: $\mu\ X \rightarrow 1 + X \times X + X$ \\
  Stream: $\nu\ X \rightarrow A \times X$ \\
  Potentialy Infinite List A: $\nu\ X \rightarrow 1 + A \times X$ \\
  Finite Tree: $\mu\ X \rightarrow \mu\ Y \rightarrow 1 + X \times Y = \mu\ X = List\ X$ \\
\end{center}

\subsubsection*{Initial Object}

\begin{lstlisting}[mathescape=true]
    data List: (A: Type) $\rightarrow$ Type :=
         (nil: List A)
         (cons: A $\rightarrow$ List A $\rightarrow$ List A)

    $F_A$ = 1 + A $\times$ X
\end{lstlisting}

\subsubsection*{Construct corresponding F-Algebra}

\begin{lstlisting}[mathescape=true]
  record listAlg (A: Type) : Type :=
         (X: Type)
         (nil: Unit $\rightarrow$ X)
         (cons: A $\rightarrow$ X $\rightarrow$ X)

\end{lstlisting}

\subsubsection*{Introduce List Morphisms}

\begin{lstlisting}[mathescape=true]
  record listMor   (A: Type) (x1,x2: ListAlg A) : Type :=
         (map:     x1.X $\rightarrow$ x2.X)
         (mapNil:  Path x2.X (map (x1.nil unit)) (x2.nil unit))
         (mapCons: $\forall$ (a: A) $\rightarrow$ $\forall$ (x: x1) $\rightarrow$
                   Path x2.X (map (x1.cons a x)) (x2.cons a (map x)))
\end{lstlisting}

\subsubsection*{Introduce connected points of List type}

\begin{lstlisting}[mathescape=true]
  record listPoint (A: Type) : Type :=
         (point: $\forall$ (x: ListAlg A) $\rightarrow$ x.X)
         (map: $\forall$ (x1,x2: listAlg A) $\rightarrow$ $\forall$ (m: ListMor A x1 x2) $\rightarrow$
               Path x2.X (m.map (point x1)) (point x2))

\end{lstlisting}

\subsubsection*{Theorems Section}

\begin{center}
%\begin{tabular}{l}
$$
\begin{tikzcd}
  lim\ U \arrow{d}[left]{\pi_i} \\
  X_i
\end{tikzcd}
\Longrightarrow
\begin{tikzcd}
  F\ lim\ U \arrow{d}[left]{F\ \pi_i} \\
  F\ X_i
\end{tikzcd}
\Longrightarrow
lim\left\{
\begin{tikzcd}
  F\ lim\ U \arrow{d}[left]{F\ \pi_i} \\
  F\ X_i
\end{tikzcd}
\right\}
$$
%\end{tabular}
\end{center}

\subsection{Basic Ornaments}

Our encoding allows you to precise control the type of encoded parameter.
There is only three cases and three equations: 1) for unit; 2) particular
functorial type over a parameter type and 3) recursive embedding such as
in Cons constructor.

q — is a limit in Dialg P category. The constructor body is calculated
with q applied to forgetful functor U.

$$
\begin{cases}
q_{P,D,G} : End\ P\ (G'(-),G'(-)) \rightarrow P\ (Lim\ G',Lim\ G') \\
P : Set^{op} \times Set \rightarrow Set \\
U : Dialg\ P \rightarrow Set \\
G : D \rightarrow Dialg\ P \\
G' = UG : D \rightarrow Set \\
U (Lim\ G) = Lim\ G'
\end{cases}
$$

\subsubsection{Unit}

Like for Bool or Nil constructors encoding.

$$
\begin{cases}
P_0(A,B) =  B \\
q_0\ e : Lim\ G' \\
q_0\ e = e
\end{cases}
$$

\subsubsection{Fixed Type}

Like for Cons first parameter.

$$
\begin{cases}
P_1(A,B) = X \rightarrow P(A,B) \\
q_1\ e : X \rightarrow P ( Lim\ G',  Lim\ G' ) \\
q_1\ e\ x\ A = e\ A\ x
\end{cases}
$$

\subsubsection{Recursive Parameters}

Like for Cons second parameter. This case is a key in encoding recursive
   data types such as {\bf Lists} and recursive record types such as {\bf Streams}.

$$
\begin{cases}
P_2(A,B) = A \rightarrow P(A,B) \\
q_2\ e : Lim\ G' \rightarrow P ( Lim\ G',  Lim\ G' ) \\
q_2\ e\ I\ A = e\ A\ (I\ A)
\end{cases}
$$

\newpage
  \paragraph{Data Type, Polymorphic Functions and Theorems}
  The data type of lists over a given set A can be represented as the initial algebra
  $(\mu L_A, in)$ of the functor $L_A(X) = 1 + (A \times X)$. Denote $\mu L_A = List(A)$.
  The constructor functions $nil: 1 \rightarrow List(A)$ and
  $cons: A \times List(A) \rightarrow List(A)$ are defined by
  $nil = in \circ inl$ and $cons = in \circ inr$, so $in = [nil,cons]$.
  Given any two functions $c: 1 \rightarrow C$ and $h: A \times C \rightarrow C$,
  the catamorphism $f = \llparenthesis [c,h] \rrparenthesis : List(A) \rightarrow C$
  is the unique solution of the equation system:

$$
\begin{cases}
  f \circ nil  = c \\
  f \circ cons = h \circ (id \times f)
\end{cases}
$$

  where $f = foldr(c,h)$. Having this the initial algebra is presented with functor
  $\mu (1 + A \times X)$ and morphisms sum $[1 \rightarrow List(A), A \times List(A) \rightarrow List(A)]$
  as catamorphism. Using this encoding the base library of List will have following form:

$$
\begin{cases}
 foldr = \llparenthesis [ f \circ nil , h] \rrparenthesis, f \circ cons = h \circ (id \times f)\\
 len = \llparenthesis [ zero, \lambda\ a\ n \rightarrow succ\ n ] \rrparenthesis \\
 (++) = \lambda\ xs\ ys \rightarrow \llparenthesis [ \lambda (x) \rightarrow ys, cons ] \rrparenthesis (xs) \\
 map = \lambda\ f \rightarrow \llparenthesis [ nil, cons \circ (f \times id)] \rrparenthesis
\end{cases}
$$

  \paragraph{Lists in Exe language}

  We encode List as usually we do in $\Pi,\Sigma$-provers.

\begin{lstlisting}[mathescape=true]
             data list: (A: *) $\rightarrow$ * :=
                  (nil: list A)
                  (cons: A $\rightarrow$ list A $\rightarrow$ list A)
\end{lstlisting}

$$
\begin{cases}
list = \lambda\ ctor \rightarrow \lambda\ cons \rightarrow \lambda\ nil \rightarrow ctor\\
cons = \lambda\ x\ \rightarrow \lambda\ xs \rightarrow \lambda\ list \rightarrow \lambda\ cons \rightarrow\ \lambda\ nil \rightarrow cons\ x\ (xs\ list\ cons\ nil)\\
nil = \lambda\ list \rightarrow \lambda\ cons \rightarrow \lambda\ nil \rightarrow nil\\
\end{cases}
$$

\begin{lstlisting}[mathescape=true]
           record lists: (A B: *) :=
                  (len: list A $\rightarrow$ integer)
                  ((++): list A $\rightarrow$ list A $\rightarrow$ list A)
                  (map: (A $\rightarrow$ B) $\rightarrow$ (list A $\rightarrow$ list B))
                  (filter: (A $\rightarrow$ bool) $\rightarrow$ (list A $\rightarrow$ list A))
\end{lstlisting}
$$
\begin{cases}
len = foldr\ (\lambda\ x\ n \rightarrow succ\ n)\ 0\\
(++) = \lambda\ ys \rightarrow foldr\ cons\ ys\\
map = \lambda\ f \rightarrow foldr\ (\lambda x\ xs \rightarrow cons\ (f\ x)\ xs)\ nil\\
filter = \lambda\ p \rightarrow foldr\ (\lambda x\ xs \rightarrow if\ p\ x\ then\ cons\ x\ xs\ else\ xs)\ nil\\
foldl = \lambda\ f\ v\ xs = foldr\ (\lambda\ xg\rightarrow\ (\lambda \rightarrow g\ (f\ a\ x)))\ id\ xs\ v\\
\end{cases}
$$

%\subsection{Identity Functor Encoding}

%As known as predicative encoding.\\
%\\

%\begin{array}{ccccccc}
%      A &           & \mapright{f} &           & B & \mapright{f} & Set \\[3pt]
%        & \mapdiagl{a}  & \varphi      & \mapdiagr{b}  &   & \searrow     & \mapdown{\mathbf{1}} \\
%        &           & Set          &           &   &              & Set
%\end{array}


%$\varphi : a \rightarrow b$

%\subsection{Identity Functor Limit Encoding}

%As known as impredicative encoding.

%\begin{array}{ccccccc}
%      $A &           & \mapright{f} &           & B & \mapright{f} & Set \\[3pt]
%        & \mapdiagl{a}  & \varphi      & \mapdiagr{b}  &   & \searrow     & \mapdown{\mathbf{1}} \\
%        &           & Set          &           &   &              & Set
%\end{array}

%$lim\ \varphi : lim\ a \rightarrow lim\ b$

\newpage
\subsection{Recursor and Induction}
\subsubsection{Recursor}

There is a belief that recursor (non-dependent eliminator) in Type Theory
is a weaker property than induction principle (dependent eliminator). At the same time
from category theory we know that Universal Property defines the object uniquely.
In the case of initial object in the category of algebras, the initiality could be defined by recursor.
That means that all properties of algebra follow from its initiality,
as a case it is possible to get the recursor from induction. There is a sensitive moment here,
all categorical constructions are being formulated with defined equality on morphisms,
in type theory the equality is built-in type that could have extended properties. Simplify we could say
that we can get recursor from induction without equality, and with proper equality we could get induction from recursor.

\subsubsection{Fibrations}

Mechanism of getting induction principle from equality is based on the presentation
of dependent types through fibrations. Hereby dependent type $(D: B \rightarrow Type)$ is defined as $(p: Sigma\ B\ P \rightarrow B)$
which projects dependent pair to the first field. In topology such approach is called fibration.
To the other direction for a given morphism $(p: E \rightarrow B)$ which we understands as fibration with projection p,
we could get its dependent type as $(D: B \rightarrow Type)$ by calculation in every point $(b: B)$ its image of
projection p by using equality on elements of $B$. In type theory besides depndent pair $Sigma$ also
used the dependent product $Pi$. In encoding of dependent types with fibrations there is a correspondance
between elements of dependent and morphism-fibrations for projection $p$: such $(s: B \rightarrow E)$ that $s * p = I$.
The example of this implementaion could be seen in\\
\\
EXE\footnote{https://github.com/groupoid/exe/blob/master/prelude/macro.new/Mini.macro}
OM\footnote{https://github.com/groupoid/om/blob/master/priv/posets/sec2all}

\subsubsection{Induction}
The input for induction is a predicate — dependent type encoded with $(p: E \rightarrow B)$.
Induction needed additional information for predicate. The type of induction is defined
by set of inductive constructors. Induction is just a statement that on E we have the
structure of F-algebra of inductive type. Now we could apply recursor to E
getting the map $(I \rightarrow E)$ from initial object which in fact the section (fibre bundle) of fibration
and thus defines the dependent function which is a proved value of induction principle.
The example for $Bool$ could be found in\\
\\
EXE \footnote{https://github.com/groupoid/exe/blob/master/prelude/macro.new/Data.Bool.macro}
OM \footnote{https://github.com/groupoid/om/blob/master/priv/posets/Data/Bool/induc}\\

\subsection{Conclusion}
Summarizing we encode types of source lambda calculus with objects of selected category,
dependent types with fibrations, dependent function as fibrations, inductive types as
limits of identity functors on category of F-algebras.

\newpage
\begin{thebibliography}{9}


\subsection*{Category Theory}
\bibitem{maclane}    S.MacLane \textit{Categories for the Working Mathematician} 1972
\bibitem{lawvere}    W.Lawvere \textit{Conceptual Mathematics} 1997
\bibitem{curien}     P.Curien \textit{Category theory: a programming language-oriented introduction} 2008

\subsection*{Pure Type Systems}
\bibitem{martinlof}  P.Martin-Löf \textit{Intuitionistic Type Theory} 1984
\bibitem{coquand}    T.Coquand \textit{The Calculus of Constructions.} 1988
\bibitem{henk}       E.Meijer \textit{Henk: a typed intermediate language} 1997
\bibitem{barendregt} H.Barendregt \textit{Lambda Calculus With Types} 2010

\subsection*{Inductive Type Systems}
\bibitem{pfenning}   F.Pfenning \textit{Inductively defined types in the Calculus of Constructions} 1989
\bibitem{wadler}     P.Wadler \textit{Recursive types for free} 1990
\bibitem{gambino}    N.Gambino \textit{Wellfounded Trees and Dependent Polynomial Functors} 1995
\bibitem{dybjer}     P.Dybjer \textit{Inductive Famalies} 1997
\bibitem{jacobs}     B.Jacobs \textit{(Co)Algebras) and (Co)Induction} 1997
\bibitem{vene}       V.Vene \textit{Categorical programming with (co)inductive types} 2000
\bibitem{guevers}    H.Geuvers \textit{Dependent (Co)Inductive Types are Fibrational Dialgebras} 2015

% \subsection*{Pi Calculus}
% \bibitem{comm}       Robin Milner. \textit{A Calculus of Communicating Systems.} 1986.
% \bibitem{commpi}     Robin Milner. \textit{Communicating and Mobile Systems: The $\pi$-calculus.} 1999.
% \bibitem{polypi}     Robin Milner. \textit{The Polyadic $\pi$-Calculus: A Tutorial.} 1993.

\subsection*{Homotopy Type Systems}
\bibitem{streicher0} T.Streicher \textit{A groupoid model refutes uniqueness of identity proofs} 1994
\bibitem{streicher}  T.Streicher \textit{The Groupoid Interpretation of Type Theory} 1996
\bibitem{jacobs2}    B.Jacobs \textit{Categorical Logic and Type Theory} 1999
\bibitem{awodey}     S.Awodey \textit{Homotopy Type Theory and Univalent Foundations} 2013
\bibitem{huber}      S.Huber \textit{A Cubical Type Theory} 2015
\bibitem{joyal}      A.Joyal \textit{What is an elementary higher topos} 2014
\bibitem{mortberg}   A.Mortberg \textit{Cubical Type Theory: a constructive univalence axiom} 2017

\end{thebibliography}

\end{document}
