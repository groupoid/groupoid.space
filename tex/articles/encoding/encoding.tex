% copyright (c) 2015 Synrc Research Center

\documentclass{article}
\usepackage{amscd}
\usepackage{listings}
\usepackage[only,llbracket,rrbracket,llparenthesis,rrparenthesis]{stmaryrd}
\usepackage{amssymb,amsmath,amsthm,stmaryrd,mathrsfs,wasysym}
\usepackage{graphicx}
\usepackage{amsmath}
\usepackage{mathtools}
\usepackage{amssymb}
\usepackage{amsthm}
\usepackage{url}
\usepackage{tikz-cd}
\usepackage[utf8]{inputenc}
\newcommand*{\thead}[1]{\multicolumn{1}{c}{\bfseries #1}}
\lstset{basicstyle=\small,inputencoding=utf8}

\begin{document}

\title{Types II: Inductive Types and Encodings}
\author{Maksym Sokhatskyi $^1$}
\date{
    $^1$ National Technical University of Ukraine \\
    \small Igor Sikorsky Kyiv Polytechnical Institute\\
    \today
}

\maketitle

\begin{abstract}

Impredicative Encoding of Inductive Types in HoTT.
\\
\\
{\bf Keywords}: Formal Methods, Type Theory, Programming Languages,
          Theoretical Computer Science, Applied Mathematics,
          Cubical Type Theory, Martin-Löf Type Theory
\end{abstract}

\tableofcontents
\newpage

\section{Intro}

\section{Martin-Lof Type Theory}

\subsection{Universes}

\begin{equation}
\tag{sorts}
\dfrac
  {i : Nat}
  {Type_i}
\end{equation}

\begin{equation}
\tag{axioms}
\dfrac
  {i: Nat}
  {Type_i : Type_{i+1}}
\end{equation}

\begin{equation}
\tag{rules}
\dfrac
  {i : Nat,\ \ \ \ j : Nat}
  {Type_i \rightarrow Type_{j} : Type_{max(i,j)}}
\end{equation}

\subsection{Dependent Types}

\begin{equation}
\tag{subst}
\dfrac
  {\pi_1 : A\ \ \ \ u:A \vdash \pi_2 : B}
  {[\pi_1/u]\ \pi_2 : B}
\end{equation}

\begin{equation}
\tag{$\Pi$-formation}
\dfrac
  {x:A \vdash B : Type}
  {\Pi\ (x:A) \rightarrow B : Type}
\end{equation}

\begin{equation}
\tag{$\lambda$-intro}
\dfrac
  {x:A \vdash b : B}
  {\lambda\ (x:A) \rightarrow b : \Pi\ (x: A) \rightarrow B }
\end{equation}

\begin{equation}
\tag{$App$-elimination}
\dfrac
  {f: (\Pi\ (x:A) \rightarrow B)\ \ \ a: A}
  {f\ a : B\ [a/x]}
\end{equation}

\begin{equation}
\tag{$\beta$-computation}
\dfrac
  {x:A \vdash b: B\ \ \ a:A}
  {(\lambda\ (x:A) \rightarrow b)\ a = b\ [a/x] : B\ [a/x]}
\end{equation}

\subsection{Identity Types}

\begin{equation}
\tag{$Id$-formation}
\dfrac
  {a:A\ \ \ \ b:A\ \ \ \ A:Type}
  {Id(A,a,b) : Type}
\end{equation}

\begin{equation}
\tag{$Id$-intro}
\dfrac
  {a:A}
  {refl(A,a) : Id(A,a,a) }
\end{equation}

\begin{equation}
\tag{$J$-elimination}
\dfrac
  {p:Id(a,b)\ \ \ \ x,y:A\ \ \ \ u:Id(x,y) \vdash E:Type\ \ \ \ x:A \vdash d: E\ [x/y,\ refl(x)/u]}
  {J(a,b,p,(x,y,u)\ d) : E\ [a/x,\ b/y,\ p/u]}
\end{equation}

\begin{equation}
\tag{$Id$-computation}
\dfrac
  {a,x,y:A,\ \ \ \ u:Id(x,y) \vdash E:Type\ \ \ \ x:A \vdash d:E\ [x/y,\ refl(x)/u]}
  {J(a,a,refl(a),(x,y,u)\ d) = d\ [a/x] : E\ [a/y,\ refl(a)/u]}
\end{equation}

\subsection{n-Types}

A type $A$ is contractible, or a singleton, if there is $a : A$,
called the center of contraction, such that $a = x$ for all $x : A$:
A type $A$ is proposition if any x,y: A are equals.
A type is a Set if all equalities in A form a prop.
It is defined as recursive definition.

$$isContr = \sum_{a:A}\prod_{x:A} a =_A x,\ \ 
  isProp(A) = \prod_{x,y:A} x =_A y,\ \ 
  isSet = \prod_{x,y:A} isProp\ (x =_A y),\ \ $$
$$isGroupoid = \prod_{x,y:A} isSet\ (x =_A y),\ \ 
  PROP = \sum_{X:U}isProp(X),\ \ 
  SET = \sum_{X:U}isSet(X),...$$

The following types are inhabited: isSet PROP, isGroupoid SET.
All these functions are defined in {\bf path} module. As you can see
from definition there is a recurrent pattern which we encode in cubical syntax
as follows:

\begin{lstlisting}[mathescape=true]
${\bf data}$ N = Z | S (n: N)
n_grpd (A: U) (n: N): U = (a b: A) -> ((rec A a b) n) where
  rec (A: U) (a b: A): (k: N) -> U = ${\bf split}$
    Z -> Path A a b
    S n -> n_grpd (Path A a b) n

isContr (A: U): U = (x:A) * ((y: A) -> Equ A x y)
isProp      (A: U): U = n_grpd A Z
isSet       (A: U): U = n_grpd A (S Z)
isGroupoid  (A: U): U = n_grpd A (S (S Z))
PROP      : U = (X:U) * isProp X
SET       : U = (X:U) * isSet  X
GRPOUPOID : U = (X:U) * isGroupoid X
\end{lstlisting}

\subsection{Path Eliminators}

The very basic ground of type checker is heterogeneous equality $PathP$ and contructive
implementation of reflection rule as lambda over interval $[0,1]$ that
return constant value $a$ on all domain.

\begin{lstlisting}[mathescape=true]
Path (A:U)(a b:A):U = PathP (<i>A) a b
HeteroEqu (A B:U)(a:A)(b:B)(P:Equ U A B):U = Path P a b
refl (A:U)(a:A):Path A a a = <i> a
sym (A:U)(a b:A) (p: Path A a b): Path A b a = <i> p @ -i
transitivity (A: U)(a b c:A)(p: Path A a b) (q: Path A b c):
    Path A a c = comp (<i> Path A a (q @ i)) p []
\end{lstlisting}

$$trans : \prod_{p:A=_U B}^{A,B:U} \prod_{a:A} B,\ \ \ singleton : \prod_{x:A}^{A:U} \sum_{y:A} x =_A y $$

$$subst : \prod_{a,b:A}^{A:U,B:A\rightarrow U} \prod_{p: a =_A b} \prod_{e:B(a)} B(b), \ \ \ 
  congruence : \prod_{f:A\rightarrow B}^{A,B:U} \prod_{a,b:A} \prod_{p:a =_A b} f(a) =_B f(b) $$

Transport tranfers the element of type to another by given path equality of the types.
Substitution is like transport but for dependent functions values: by given dependent function
and path equality of points in the function domain we can replace the value of dependent function
in one point with value in the second point. Congruence states that for a given function
and for any two points in the function domain, that are connected, we can state that function
values in that points are equal.

\begin{lstlisting}[mathescape=true]
singl (A:U) (a:A): U = (x: A) * Path A a x
trans (A B:U) (p: Path U A B) (a: A): B = comp p a []
congruence (A B: U) (f:A->B) (a b: A)
           (p: Path A a b): Path B (f a) (f b)
           = <i> f (p @ i)

subst (A:U) (P:A->U) (a b: A)
      (p: Path A a b) (e: P a): P b
      = trans (P a) (P b) (congruence A U P a b p) e
\end{lstlisting}

These function are defined in {\bf proto\_path} module, and all of them
except singleton definition are underivable in MLTT.

\section{Encoding}

\subsection{Church Encoding}

You know Church encoding which also has its dependent alanolgue in CoC, however
in Coq it is imposible to detive Inductive Principle as type system lacks fixpoint
and functional extensionality. The example of working compiler of PTS languages are Om and Morte.
Assume we have Church encoded NAT:

\begin{lstlisting}[mathescape=true]
nat = (X:U) -> (X -> X) -> X -> X
\end{lstlisting}

where first parameter $(X -> X)$ is a $succ$, the second parameter $X$ is $zero$,
and the result of encoding is landed in X. Even if we encode the parameter

\begin{lstlisting}[mathescape=true]
list (A: U) = (X:U) -> X -> (A -> X) -> X
\end{lstlisting}

and paremeter A let's say live in 42 universe and X live in 2 universe, then by
the signature of encoding the term will be landed in X, thus 2 universe. In other words
such dependency is called impredicative displaying that landed term is not a predicate over parameters.
This means that Church encoding is incompatible with predicative type checkers with predicative
of predicative-cumulative hierarchies.

\subsection{Impredicative Encoding}

In HoTT n-types is encoded as n-groupoids, thus we need to add a predicate in which n-type
we would like to land the encoding:

\begin{lstlisting}[mathescape=true]
NAT (A: U) = (X:U) -> isSet X -> X -> (A -> X) -> X
\end{lstlisting}

Here we added isSet predicate. With this motto we can implement propositional
truncation by landing term in isProp or even HIT by langing in isGroupoid:

\begin{lstlisting}[mathescape=true]
TRUN (A:U) type = (X: U) -> isProp X -> (A -> X) -> X
S1 = (X:U) -> isGroupoid X -> ((x:X) -> Path X x x) -> X
MONOPLE (A:U) = (X:U) -> isSet X -> (A -> X) -> X
NAT = (X:U) -> isSet X -> X -> (A -> X) -> X
\end{lstlisting}

The main publication on this topic could be found at \cite{Awodey17} and \cite{Speight17}.

\section{The Unit Example}

Here we have the implementation of Unit impredicative encoding in HoTT.

\begin{lstlisting}[mathescape=true]
upPath     (X Y:U)(f:X->Y)(a:X->X): X -> Y = o X X Y f a
downPath   (X Y:U)(f:X->Y)(b:Y->Y): X -> Y = o X Y Y b f
naturality (X Y:U)(f:X->Y)(a:X->X)(b:Y->Y): U
  = Path (X->Y)(upPath X Y f a)(downPath X Y f b)

unitEnc': U = (X: U) -> isSet X -> X -> X
isUnitEnc (one: unitEnc'): U
  = (X Y:U)(x:isSet X)(y:isSet Y)(f:X->Y) ->
    naturality X Y f (one X x)(one Y y)

unitEnc: U = (x: unitEnc') * isUnitEnc x
unitEncStar: unitEnc = (\(X:U)(_:isSet X) ->
  idfun X,\(X Y: U)(_:isSet X)(_:isSet Y)->refl(X->Y))
unitEncRec  (C: U) (s: isSet C) (c: C): unitEnc -> C
  = \(z: unitEnc) -> z.1 C s c
unitEncBeta (C: U) (s: isSet C) (c: C)
  : Path C (unitEncRec C s c unitEncStar) c = refl C c
unitEncEta (z: unitEnc): Path unitEnc unitEncStar z = undefined
unitEncInd (P: unitEnc -> U) (a: unitEnc): P unitEncStar -> P a
  = subst unitEnc P unitEncStar a (unitEncEta a)
unitEncCondition (n: unitEnc'): isProp (isUnitEnc n)
  =  \ (f g: isUnitEnc n) ->
     <h> \ (x y: U) -> \ (X: isSet x) -> \ (Y: isSet y)
  -> \ (F: x -> y) -> <i> \ (R: x) -> Y (F (n x X R)) (n y Y (F R))
       (<j> f x y X Y F @ j R) (<j> g x y X Y F @ j R) @ h @ i
\end{lstlisting}

\bibliographystyle{plain}
\bibliography{encoding}

\end{document}
