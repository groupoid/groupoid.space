\documentclass{svproc}
\usepackage{url}
\def\UrlFont{\rmfamily}
\begin{document}
\mainmatter
\title{Applying Formal Methods to Smart Contract Languages and Blockchain}
\titlerunning{Hamiltonian Mechanics}
\author{Maxim Sokhatsky\inst{1} \and Pavlo Maslianko\inst{1}}
\authorrunning{Maxim Sokhatsky}
\tocauthor{Maxim Sokhatsky}
\institute{
National Technical University of Ukraine, \\
``Igor Sikorsky Kyiv Polytechnical Institute'', \\
Ukraine, \email{maxim@synrc.com}, WWW: \texttt{http://groupoid.space} \\
Address: 37, Prosp. Peremohy, Kyiv, Ukraine, 03056,
Tel.:  +380.44.2367989 }
\maketitle
\begin{abstract}
This paper shows our vision on the way to better contract languages and
trusted foundations of Blockchain protocols. The main problem in current
contract languages and its virtual machine implementations used in
Blockchain and Ethereum cryptocurrencies is that they are built upon stack-based
virtual machines while we propose pure lambda-calculus based computing
environments with proven theorems if its evaluations strategies and also a
corespondent high-level language with dependent types that
can be used to embed smart contract DSL into dependently typed language.
Also we will draw how formal verification could be applied to other
Blockchain subsystems including proof-of-work protocols and distributed ledger.
\keywords{formal methods, software verification, program languages}
\end{abstract}

\section{Formal Verification}
Formal verification is an ablity to prove properties in a form of theorems
of a given system which are formulated in a axiomatic manner.
While the origins of this approach led us to logic and automated
provers (AUTOMATH, ACL2 and later Coq, F*) now it seems applicable
to much wider spectrum of mathematical systems, including not only logic,
but also geometry and topology.

\section{Model Checkers}
There are many approaches to formal verification, depending of
origins and areas. E.g. one can prove properties (or we say, verify) for some model,
then prove the isomorphism between that model and the actual program. Such systems
are called model checkers and usually are just a tools and small steps in a process
of formal verification. Well known model checkers are TLA+ for formal
verification of distributed algorithms. Twelf for verification of computer languages, etc.
This apporach involves several (more that one) languages into verification process.

\section{Embeddable Languages}
The another approach is to use single unified language for proving properties of models
and language of actual computations. In that area dependently typed languages dominates
in all applications. MLTT with Sigma and Pi types was introduced in 1972 and from that
times developed enough to be used for extracting certified programs. Such process of
erasing type information (that are used by type-checkers) is called extraction.

\section{Comparing ULC and Stack-based machines}
The target language of extracted programs are exacly the same as the operating language
of smart contracts usually used in cryptocurrencies. So we can say that
it is untyped lambda interpreter with proven properies. However those


RIPEMD160
SHA1
SHA256
HASH160
HASH256


% https://en.bitcoin.it/w/images/en/7/70/Bitcoin_OpCheckSig_InDetail.png
% https://eprint.iacr.org/2016/1156.pdf
% http://antoine.delignat-lavaud.fr/doc/plas16.pdf
%
\begin{thebibliography}{6}
%

\bibitem {smit:wat}
Smith, T.F., Waterman, M.S.: Identification of common molecular subsequences.
J. Mol. Biol. 147, 195?197 (1981). \url{doi:10.1016/0022-2836(81)90087-5}

\bibitem {may:ehr:stein}
May, P., Ehrlich, H.-C., Steinke, T.: ZIB structure prediction pipeline:
composing a complex biological workflow through web services.
In: Nagel, W.E., Walter, W.V., Lehner, W. (eds.) Euro-Par 2006.
LNCS, vol. 4128, pp. 1148?1158. Springer, Heidelberg (2006).
\url{doi:10.1007/11823285_121}

\bibitem {fost:kes}
Foster, I., Kesselman, C.: The Grid: Blueprint for a New Computing Infrastructure.
Morgan Kaufmann, San Francisco (1999)

\bibitem {czaj:fitz}
Czajkowski, K., Fitzgerald, S., Foster, I., Kesselman, C.: Grid information services
for distributed resource sharing. In: 10th IEEE International Symposium
on High Performance Distributed Computing, pp. 181?184. IEEE Press, New York (2001).
\url{doi: 10.1109/HPDC.2001.945188}

\bibitem {fo:kes:nic:tue}
Foster, I., Kesselman, C., Nick, J., Tuecke, S.: The physiology of the grid: an open grid services architecture for distributed systems integration. Technical report, Global Grid
Forum (2002)

\bibitem {onlyurl}
National Center for Biotechnology Information. \url{http://www.ncbi.nlm.nih.gov}


\end{thebibliography}
\end{document}
