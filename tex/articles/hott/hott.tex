% copyright (c) 2018 Groupoid Infinity

\documentclass{article}
\usepackage{listings}
%\usepackage{geometry}
\usepackage{amsmath}
\usepackage{amssymb}
\usepackage{float}
\usepackage{amsthm}
\usepackage{url}
\usepackage{tikz-cd}
\usepackage[utf8]{inputenc}

\theoremstyle{definition}
\newtheorem{definition}{Definition}
\newtheorem{theorem}{Theorem}
\newcommand*{\incmap}{\hookrightarrow}
\newcommand*{\thead}[1]{\multicolumn{1}{c}{\bfseries #1}}
\lstset{basicstyle=\small,inputencoding=utf8}

\begin{document}

\title{Issue III: Homotopy Type Theory}
\author{Maxim Sokhatsky $^1$}
\date{ $^1$ National Technical University of Ukraine \\
       \small Igor Sikorsky Kyiv Polytechnical Institute \\
       \today }

\maketitle

\begin{abstract}
We present here main destinctive point of Homotopy Type Theory
as an extension of Martin-Löf Type Theory up to higher inductive types
which we will give in the next issue.
\\
\\
{\bf Keywords}: Homotopy Type Theory
\end{abstract}
\tableofcontents

\newpage
\section{Homotypy Type Theory}


Homotopy type theory to classical homotopy theory is like Euclidian
syntethic geometry (points, lines, axioms and deduction rules) to
analytical geometry with cartesian coordinates on $\mathbb{R}^n$ (geometric and algebraic)
\footnote{We will denote geometric, type theoretical and homotopy constants bold font ${\bf R}$ while
analitical will be denoted with double lined letters $\mathbb{R}$.}

\subsection{Homotopies}
The first higher equality we meet in homotopy theory is a notion of homotopy,
where we compare two functions or two path spaces (which is sort of dependent families).
The homotopy interval $\mathrm{I}=[0,1]$ is the perfect foundation for definition of homotopy.

\begin{definition} (Interval). Compact interval.
\begin{lstlisting}
data I = i0
       | i1
       | seg <i> [(i=0) -> i0,
                  (i=1) -> i1]
\end{lstlisting}
\end{definition}

You can think of ${\bf I}$ as isomorphism of equality type,
disregarding carriers on the edges. By mapping $i0,i1:{\bf I}$ to $x,y:A$ one can
obtain identity or equality type from classic type theory.

\begin{definition} (Interval Split).
The convertion function from $\mathrm{I}$ to a type of comparison
is a direct eliminator of interval. The interval is also known as one of
primitive higher inductive types which will be given in the next
{\bf Issue IV: Higher Inductive Types}.
\begin{lstlisting}
pathToHtpy (A: U) (x y: A) (p: Path A x y): I -> A
   = split { i0 -> x; i1 -> y; seg @ i -> p @ i }
\end{lstlisting}
\end{definition}

\begin{definition} (Homotopy). The homotopy between two function $f,g: X \rightarrow Y$
is a continuous map of cylinder $H : X \times {\bf I} \rightarrow Y$ such that
$$
\begin{cases}
H(x,0)=f(x), \\
H(x,1)=g(x).
\end{cases}
$$
\begin{lstlisting}
homotopy (X Y: U) (f g: X -> Y)
         (p: (x: X) -> Path Y (f x) (g x))
         (x: X): I -> Y = pathToHtpy Y (f x) (g x) (p x)
\end{lstlisting}
\end{definition}
\newpage
\subsection{Groupoid Interpretation}
The first text about groupoid interpretation of type theory can be found in Francois Lamarche:
A proposal about Foundations\footnote{http://www.cse.chalmers.se/~coquand/Proposal.pdf}. Then Martin Hofmann and Thomas Streicher wrote the initial
document on groupoid interpretation of type theory\footnote{Martin Hofmann and Thomas Streicher. The Groupoid Interpretation of Type Theory. 1996.}.


\begin{table}[H]
\begin{center}
\begin{tabular}{lccc}
\hline
{\bf Equality} & {\bf Homotopy} & {\bf $\infty$-Groupoid} \\
\hline
reflexivity  & constant path & identity morphism \\
symmetry     & inversion of path & inverse morphism \\
transitivity & concatenation of paths & composition of mopphisms \\
\hline
\end{tabular}
\end{center}
\end{table}

There is a deep connection between higher-dimentinal groupoids in category theory and
spaces in homotopy theory, equipped with some topology. The category or groupoid could
be built where the objects are particular spaces or types, and morphisms are path types
between these types, composition operation is a path concatenation. We can write this
groupoid here recalling that it should be category with inverted morphisms.

\begin{lstlisting}
cat: U = (A: U) * (A -> A -> U)
groupoid: U = (X: cat) * isCatGroupoid X
PathCat (X: U): cat = (X,\(x y:X)->Path X x y)
\end{lstlisting}

\begin{lstlisting}
isCatGroupoid (C: cat): U
  = (id: (x: C.1) -> C.2 x x)
  * (c: (x y z:C.1) -> C.2 x y -> C.2 y z -> C.2 x z)
  * (inv: (x y: C.1) -> C.2 x y -> C.2 y x)
  * (inv_left:  (x y: C.1) (p: C.2 x y) ->
    Path (C.2 x x) (c x y x p (inv x y p)) (id x))
  * (inv_right: (x y: C.1) (p: C.2 x y) ->
    Path (C.2 y y) (c y x y (inv x y p) p) (id y))
  * (left: (x y: C.1) (f: C.2 x y) ->
    Path (C.2 x y) (c x x y (id x) f) f)
  * (right: (x y: C.1) (f: C.2 x y) ->
    Path (C.2 x y) (c x y y f (id y)) f)
  * ((x y z w:C.1)(f:C.2 x y)(g:C.2 y z)(h:C.2 z w)->
    Path (C.2 x w) (c x z w (c x y z f g) h)
                   (c x y w f (c y z w g h)))
\end{lstlisting}

\newpage
\begin{lstlisting}
PathGrpd (X: U)
  : groupoid
  = ((Ob,Hom),id,c,sym X,compPathInv X,compInvPath X,L,R,Q) where
    Ob: U = X
    Hom (A B: Ob): U = Path X A B
    id (A: Ob): Path X A A = refl X A
    c (A B C: Ob) (f: Hom A B) (g: Hom B C): Hom A C
      = comp (<i> Path X A (g@i)) f []
\end{lstlisting}
From here should be clear what it meant to be groupoid interpretation
of path type in type theory. In the same way we can construct categories of $\prod$ and $\sum$ types.
In {\bf Issue VIII: Topos Theory} such categories will be given.

\subsection{Functional Extensionality}

\subsection{Fibration}

\subsection{Loop Spaces}

\subsection{Equivalence}

\subsection{Homotopy Type}

\subsection{Univalence}

\bibliographystyle{plain}
\bibliography{hott}

\end{document}

