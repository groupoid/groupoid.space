\chapter*{ВСТУП}

\epigraph{Присвячується Маші та Міші}{}

Тут дамо тему, предмет та мету роботи, а також дамо опис структури роботи.

\section*{Актуальність роботи}
Ціна помилок в індустрії надзвичайно велика. Наведемо відомі приклади:
1) Mars Climate Orbiter (1998),
   помилка\footnote{Mars Climate Orbiter Mishap Investigation Board Phase I Report November 10, 1999. \\
                    \url{https://llis.nasa.gov/llis_lib/pdf/1009464main1_0641-mr.pdf}}
   невідповідності типів британської метричної системи, коштувала 80 мільйонів фунтів стерлінгів.
   Невдача стала причиною переходу
   NASA\footnote{National Aeronautics and Space Administration, національна адміністрація аеронавтики та космосу США}
   повністю на метричну систему в 2007 році.
2) Ariane Rocket (1996),
   причина\footnote{ARIANE 5 Flight 501 Failure, \\
          \url{http://www-users.math.umn.edu/~arnold/disasters/ariane5rep.html}}
   катастрофи -- округлення 64-бітного дійсного числа до 16-бітного.
   Втрачені кошти на побудову ракети та запуск 500 мільйонів доларів.
3) Помилка в FPU в перших Pentium (1994), збитки на 300 мільйонів доларів.
4) Помилка\footnote{The Matter of Heartbleed. \\
                    \url{http://mdbailey.ece.illinois.edu/publications/imc14-heartbleed.pdf}}
   в SSL (heartbleed), оцінені збитки у розмірі 400 мільйонів доларів.
5) Помилка у логіці бізнес-контрактів EVM (неконтрольована рекурсія), збитки 50 мільйонів,
   що привело до появи верифікаторів та валідаторів
   контрактів\footnote{Vandal: A Scalable Security Analysis Framework for Smart Contracts, \\
                       \url{https://arxiv.org/pdf/1809.03981.pdf}},
             \footnote{Short Paper: Formal Verification of Smart Contracts, \\
                       \url{https://www.cs.umd.edu/~aseem/solidetherplas.pdf}}.
Більше того, і найголовніше, помилки у програмному забезпеченні можуть коштувати життя людей.

\section*{Формалізована постановка задачі}
Тут стисло розкривається об'єкт, предтем, мета та завдання дослідження,
та виділяються основні методи, які будуть використовуватися в роботі.

\subsection*{Об'єкт та предмет дослідження}
Об'єктом дослідження в широкому сенсі є множина
всіх формальних мов, та можливих зв'язків між ними.

Формально, об'єктом дослідження данної роботи є:
1) системи верифікації програмного забезпечення;
2) системи доведення теорем;
3) мови програмування;
4) операційні системи, які виконують обчислення в реальному часі;
3) їх поєднання, побудова формальної системи для
уніфікованого середовища, яке поєднує середовище
виконання та систему верифікації у єдину систему мов та засобів.

Предметом та методом дослідження такої системи мов є теорія типів,
як сучасний фундамент математики,
який стисло та компактно представляє не тільки теорію множин,
але і теорію категорій, алгебраїчну топологію та дифференціальну геометрію.
Теорія типів вивчає обчислювальні властивості мов та виділилася
в окрему науку Пером Мартіном-Льофом як запит на вакантне місце у
трикутнику теорій, які відповідають ізоморфізму
Каррі-Говарда-Ламбека (Логіки, Мови, Категорії).

\subsection*{Мета і завдання дослідження}
Одна з причина низького рівня впровадження у виробництво систем
верифікації -- це висока складність таких систем. Складні системи
верифікуються складно. Ми хочемо запропонувати спрощений
підхід до верифікації --- оснований на концепції компактних
та простих мовних ядер для створення специфікацій, моделей,
перевірки моделей, доведення теорем у теорії типів з кванторами.

Метою цього дослідження є побудова єдиної системи, яка поєднує середовище
викодання та систему верифікації програмного забезпечення. Це прикладне дослідження,
яке є сплавом фундаментальної математики та інженерних
систем з формальними методами верицікації.

Головним завданням цього дослідження є побудова мінімальної системи
мовних засобів для побудови ефективного циклу верифікації програмного
забезпечення та доведення теорем. Основні компоненти системи, як продукт дослідження:
1) інтерпретатор безтипового лямбда числення;
2) компактне ядро --- система з однією аксіомою;
3) мова з індуктивними типами;
4) мова з гомотопічним інтервалом $[0,1]$;
5) уніфікована базова бібліотека;
6) бібліотека математичних компонент.

\subsection*{Методи дослідження}
Існує багадо підходів для формальної специфікації,
верифікації та валідації, усі вони даються у розділі 1 та
бгрунтувується вибір методу моделювання з використанням
мови з залежними типами (теорії типів Мартіна-Льофа).
Для разкриття семантки цього методу використовується
категорний метод та категоріальна логіка -- теорія топосів.
Теорія категорій довела свою корисність не тільки для
математики\footnote{Близько 10 робіт медалістів премії Філдса грунтуються на категорних методах},
але і для програмного забезпечення\footnote{Категорні бібліотеки таких мов як Haskell, Scala тощо.}

\section*{Наукова новизна}
Автор спробував поглянути на проблему розширення, або мовного доповнення
топосів та категорій де відбуваються обчислення, з точки зору теорії типів,
та їх мовних синтаксисів та категорій. За допомогою спектрального розкладення на
елементарні мови, які репрезентують певні типи та описуються сигнатурами ізоморфізмів,
будується єдиний погляд на еволюцію мови та її покомпонентний аналіз.

В рамках розробленого фреймворку будується
архітектура системи доведення теорем та обчислювального
середовища (яке складається з верифікованого засобами Rust
інтерпретатора та операційної системи), яке побудовано за сучасними стандартами:
1) відсутність системи управління пам'яті у реальному часі (тільки на стадії компіляції);
2) автоматична векторизація за допомогою AVX інструкцій;
3) дані ніколи не копіюються;
4) в системі немає очікування, полінгу та даремного витрачання ресурсів.

Перший кубічний верифікатор, в якому аксіома унівалентності
Воєводського має конструктивну інтерпретацію, це cubicaltt авторства Андерса Мортберга (2017).
Гомотопічні системи типів та системи типів в зв'язаних топосах є сучасним поглядом на
конструктивну математику на шляху до формалізації таких моделей як інфінітезімальні околи,
нескінченно малі (які потрібні для моделювання послідовності Коші) тощо.

Також це перша глобальна робота по створенню базової кубічної бібліотеки
на базі розробленої концепції.

\section*{Практичні результати}
В результаті цього дослідження встановлено, що формалізація системи
програмування та доведення теорем можлива в рамках одніє уніфікованої
системи, та показано на прикладі спектр мовних засобі [ починаючи з інтерпретатора
та безтипового лямбда числення, операційної системи середовища виконання (розділ 2),
а також гомотопічної базової бібліотеки (розіл 3), здатної охопити майже всь
глибину математики (розділ 4) ] необхідних для побудови замкненої формальної системи
доведення теорем та виконання програм.

\section*{Особистий внесок здобувача}
Автор особисто створив інтерпретатор, декілька верифікаторів та базову бібліотеку
як приклади використання та моделювання системи доведення теорем. Автор також
розробив курс гомотопічної теорії типів, на якому здійнюється формалізація
певних розділів математики (теорія категорій, різні частини алгебраїчної
топології та диференціальної геометрії). Окрім того створений сайт, присвячений
документації по базовій бібліотеці для кубічного верифікатора гомотопічної мови
програмування. Також частина моделей розроблений автором попала в
апстрім кубічного верифікатора, як приклади використання.

\section*{Апробація результатів роботи}
Відбулися виступи на двох конференціях: MMCTSE, IAI.

\section*{Структура роботи та публікації}
Якшо коротко суть роботи зводиться до побудови системи, яка складається з:
i) середовища виконання; ii) формального інтерпретатора; iii) системи формальних мов
для доведення теорем математики, програмної інженерії та філософії.

\subsection*{Формальна верифікація}

У розділі 1 дається огляд існуючих рішень для доведення
властивостей систем та моделей, класифікуються мови програмування
та системи доведення теорем.

\subsection*{Концептуальна модель системи доведення теорем на основі гомотопічної теорії типів}
У розділі 2 розглядається повний стек формального програмного забезпечення
від віртуальної машини, байт-код інтерпретатора та середовища виконання
та планування процесів до формальної мови для доведення теорем (або сімейства мов).

\subsection*{Базова бібліотека}
У розділі 3 описується базова бібліотека, написана на самій потужній
формальній мові системи доведення теорем.

\subsection*{Математичні компоненти}
У розділі 4 пропонується ряд математичних моделей та теорій з використанням
базової бібліотеки розділу 3 та мови гомотопічної системи типів.

%\subsection*{Додатки}

%У додатках надаються приклади іншого використання фомальних мов та моделей,
%зокрема для мінімальної формальної мови, побудованої в рамках дисертації,
%та мови програмування Coq. А також дається приклад використання
%гомотопічної мови для формальної філософії.

