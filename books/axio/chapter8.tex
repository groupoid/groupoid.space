\chapter{Додаткові матеріали}
\epigraph{Присвячується майстрам тибетського буддизму}{Атіша, Нагарджуна, Лонгченпа}

У додатках ми використаємо три різних мови, та покаже два застосування
формальних мов: 1) формальна філософія (на мові $O_{HTS}$; 2) формальний ввід-вивід для
системної інженерії (на двох промислових мовах Erlang та OCaml).

\section{Формалізація мадх'яміки}

Зараз я дам вам відчути смак формальної філософії по-справжньому!
А то вам може здатися, що це канал з формальної математики, а не
формальної філософії. Я ж вважаю, що якщо формальна філософія не
спирається на формальну математику, то гріш ціна такій формальній філософії.

\begin{lstlisting}
module buddhism where
import path
\end{lstlisting}

Сьогодні ми будемо формалізувати поняття недвоїстості в буддизмі,
яке пов'язане одразу з багатьма концепціями на рівнях Сутри, Тантри
та Дзогчена: поняттям взаємозалежного виникнення та поняттям порожнечі
всіх феноменів (Сутра Праджняпараміти). Класичний приклад із
розчленовуванням тіла ставить питання, коли тіло перестає бути
людиною-істотою, якщо від нього почати відрубувати шматки м'яса
(ми буддисти любимо і лілеєм такі уявні образи-експерименти) або
іншими словами, щоб відрізнити тіло від не-тіла, нам потрібен
двомісний предикат (родина типів), функція, яка може ідентифікувати
конректні два еклемпляри тіла. Практично йдеться про ідентифікацію
двох об'єктів, тобто про звичайний тип-рівность Мартіна-Льофа.

За фреймворк візьмемо концепти Готтлоба Фреге, згідно з визначенням,
концепт - це предикат над об'єктом або, іншими словами, Пі-тип
Мартіна-Льофа, індексований тип, сім'я типів, тривіальне розшарування тощо.
Де об'єкт x з o належить концепту, якщо сам концепт, параметризований
цим об'єктом, населений p(o) : U (де p : concept o).

\begin{lstlisting}
concept (o: U): U
  = o -> U
\end{lstlisting}

Концепт p повинен надавати приклад чи контрприклад розрізнення,
тобто щоб визначити тіло це чи не тіло ще, поки ми його розчленовуємо,
нам потрібно як мінімум два шматки: тіло і не тіло як приклади ідентифікації.
Таким чином, недвоїстість може бути представлена як рівність між усіма
розшаруваннями (предекатами над об'єктами).

\begin{lstlisting}
nondual (o: U) (p: concept o): U
  = (x y: o) -> Path U (p x) (p y)
\end{lstlisting}

Отже, недвоїстість усуває різницю між прикладами і контрприкладами
на примордіальному рівні мандали MLTT, тобто ідентифікує всі концепти.
Сама ж ідентифікація класів об'єктів, які належать різним концептам ---
це умова, що стискає всі об'єкти в точку, або стягуваний простір, вершина
конуса мандали MLTT, або, іншими словами, порожнеча всіх феноменів виражена
як тип логічної одиниці, який містить лише один елемент.

\begin{lstlisting}
allpaths (o: U): U
  = (x y: o) -> Path o x y
\end{lstlisting}

Формулювання буддійської теореми недвоїстості, яка поширюється всі
типи учнів (тупих, середніх і тямущих), може звучати так: недвоїстість
концепту є спосіб ідентифікації його об'єктів. Сформулюємо цю саму теорему
в інший бік: спосіб ідентифікації об'єктів задає предикат неподвоїстості
концептів. Туди --- ((p: concept o) -> nondual o p) -> allpaths o,
Сюди --- allpaths o -> ((p: concept o) -> nondual o p). І доведемо її!
Як видно з сигнатур нам лише треба побудувати функцію транспорту між
двома просторами шляхів: (p x) $=_U$ (p y) і x $=_o$ y. Скористаємося
приведенням шляху до стрілки (coerce) та конгруентності (cong) з базової бібліотеки.

\begin{lstlisting}
encode (o:U): ((p: concept o) -> nondual o p) -> allpaths o
  = \(nd: (p: concept o) -> nondual o p) (a b: o)
  -> coerce(Path o a a)(Path o a b)(nd(\(z:o)->Path o a z)a b)(refl o a)
\end{lstlisting}

\begin{lstlisting}
decode (o:U): allpaths o -> ((p: concept o) -> nondual o p)
  = \(all: allpaths o)(p: concept o)(x y: o) -> cong o U p x y (all x y)
\end{lstlisting}

Як бачите, теоремка про порожнечу всіх феноменів вийшла на кілька рядків,
які демонструють: 1) основи формальної філософії та швидке занурення в область
математичної філософії; 2) гарний приклад до першого розділу HoTT на простір
шляхів та модуль path\footnote{\url{https://favonia.org/files/thesis.pdf}}.

\newpage
\section{Формалізація вводу-виводу для Coq/OCaml}

\begin{lstlisting}
CoInductive Co (E : Effect.t) : Type -> Type :=
    | Bind : forall (A B : Type), Co E A -> (A -> Co E B) -> Co E B
    | Split : forall (A : Type), Co E A -> Co E A -> Co E A
    | Join : forall (A B : Type), Co E A -> Co E B -> Co E (A * B).
    | Ret : forall (A : Type) (x : A), Co E A
    | Call : forall (command : Effect.command E),
                    Co E (Effect.answer E command)

Definition run (argv : list LString.t): Co effect unit :=
    ido! log (LString.s "What is your name?") in
    ilet! name := read_line in
    match name with
      | None => ret tt
      | Some name => log (LString.s "Hello " ++ name ++ LString.s "!")
    end.

Parameter infinity : nat.
Definition eval {A} (x : Co effect A) : Lwt.t A := eval_aux infinity x.
\end{lstlisting}

% \newpage
\begin{lstlisting}
Fixpoint eval_aux {A} (s: nat) (x: Co effect A) : Lwt.t A :=
  match s with
    | O => error tt
      | S s =>
    match x with
      | Bind _ _ x f => Lwt.bind   (eval_aux s x) (fun v_x => eval_aux s (f v_x))
      | Split _ x y  => Lwt.choose (eval_aux s x) (eval_aux s y)
      | Join _ _ x y => Lwt.join   (eval_aux s x) (eval_aux s y)
      | Ret _ v => Lwt.ret v
      | Call c => eval_command c
    end
  end.
\end{lstlisting}

\begin{lstlisting}
CoFixpoint handle_commands : Co effect unit :=
    ilet! name := read_line in
    match name with
      | None => ret tt
      | Some command =>
        ilet! result := log (LString.s "Input: "
                     ++ command ++ LString.s ".")
     in handle_commands
    end.

Definition launch (m: list LString.t -> Co effect unit): unit :=
    let argv := List.map String.to_lstring Sys.argv in
    Lwt.launch (eval (m argv)).

Definition corun (argv: list LString.t): Co effect unit :=
    handle_commands.

Definition main := launch corun.
\end{lstlisting}

\newpage
\section{Формалізація вводу-виводу для PTS/BEAM}

This work is expected to compile to a limited number of target platforms.
For now, Erlang, Haskell, and LLVM are awaiting.
Erlang version is expected to be used both on LING and BEAM Erlang virtual machines.
This language allows you to define trusted operations in System F and extract this routine to Erlang/OTP platform and plug as trusted resources.
As the example, we also provide infinite coinductive process creation and inductive shell that linked to Erlang/OTP IO functions directly.

\textbf{IO} protocol.
We can construct in pure type system the state machine based on (co)free monads driven by \textbf{IO/IOI} protocols.
Assume that \textbf{String} is a \textbf{List\ Nat} (as it is in Erlang natively), and three external constructors: getLine, putLine and pure.
We need to put correspondent implementations on host platform as parameters to perform the actual IO.

\begin{lstlisting}
String: Type = List Nat
data IO: Type =
     (getLine: (String -> IO) -> IO)
     (putLine: String -> IO)
     (pure: () -> IO)
\end{lstlisting}

%\newpage
\subsubsection{Infinity I/O Type}

Infinity I/O Type Spec.

\begin{lstlisting}
-- IOI/@: (r: U) [x: U] [[s: U] -> s -> [s -> #IOI/F r s] -> x] x
   \ (r : *)
-> \/ (x : *)
-> (\/ (s : *)
   -> s
   -> (s -> #IOI/F r s)
   -> x)
-> x
\end{lstlisting}

\begin{lstlisting}
-- IOI/F
   \ (a : *)
-> \ (State : *)
-> \/ (IOF : *)
-> \/ (PutLine_ : #IOI/data -> State -> IOF)
-> \/ (GetLine_ : (#IOI/data -> State) -> IOF)
-> \/ (Pure_ : a -> IOF)
-> IOF
\end{lstlisting}

\begin{lstlisting}
-- IOI/MkIO
   \ (r : *)
-> \ (s : *)
-> \ (seed : s)
-> \ (step : s -> #IOI/F r s)
-> \ (x : *)
-> \ (k : forall (s : *) -> s -> (s -> #IOI/F r s) -> x)
-> k s seed step
\end{lstlisting}

Infinite I/O Sample Program.

\begin{lstlisting}[mathescape=true]
-- Morte/corecursive
( \ (r: *1)
 -> ( (((#IOI/MkIO r) (#Maybe/@ #IOI/data)) (#Maybe/Nothing #IOI/data))
    ( \ (m: (#Maybe/@ #IOI/data))
     -> (((((#Maybe/maybe #IOI/data) m) ((#IOI/F r) (#Maybe/@ #IOI/data)))
           ( \ (str: #IOI/data)
            -> ((((#IOI/putLine r) (#Maybe/@ #IOI/data)) str)
                (#Maybe/Nothing #IOI/data))))
         (((#IOI/getLine r) (#Maybe/@ #IOI/data))
          (#Maybe/Just #IOI/data))))))
\end{lstlisting}

Erlang Coinductive Bindings.

\begin{lstlisting}[mathescape=true]
copure() ->
    fun (_) -> fun (IO) -> IO end end.

cogetLine() ->
    fun(IO) -> fun(_) ->
        L = ch:list(io:get_line("> ")),
        ch:ap(IO,[L]) end end.

coputLine() ->
    fun (S) -> fun(IO) ->
        X = ch:unlist(S),
        io:put_chars(": "++X),
        case X of "0\n" -> list([]);
                      _ -> corec() end end end.

corec() ->
    ap('Morte':corecursive(),
        [copure(),cogetLine(),coputLine(),copure(),list([])]).
\end{lstlisting}

\begin{lstlisting}[mathescape=true]

> om_extract:extract("priv/normal/IOI").
ok
> Active: module loaded: {reloaded,'IOI'}
\end{lstlisting}

\begin{lstlisting}[mathescape=true]
> om:corec().
> 1
: 1
> 0
: 0
#Fun<List.3.113171260>
\end{lstlisting}

\subsubsection{I/O Type}

I/O Type Spec.

\begin{lstlisting}[mathescape=true]
-- IO/@
   \ (a : *)
-> \/ (IO : *)
-> \/ (GetLine_ : (#IO/data -> IO) -> IO)
-> \/ (PutLine_ : #IO/data -> IO -> IO)
-> \/ (Pure_ : a -> IO)
-> IO
\end{lstlisting}

\begin{lstlisting}[mathescape=true]
-- IO/replicateM
   \ (n: #Nat/@)
-> \ (io: #IO/@ #Unit/@)
-> #Nat/fold n (#IO/@ #Unit/@)
               (#IO/[>>] io)
               (#IO/pure #Unit/@ #Unit/Make)
\end{lstlisting}

Guarded Recursion I/O Sample Program.

\begin{lstlisting}[mathescape=true]
-- Morte/recursive
((#IO/replicateM #Nat/Five)
 ((((#IO/[>>=] #IO/data) #Unit/@) #IO/getLine) #IO/putLine))
\end{lstlisting}

Erlang Inductive Bindings.

\begin{lstlisting}[mathescape=true]
pure() ->
    fun(IO) -> IO end.

getLine() ->
    fun(IO) -> fun(_) ->
        L = ch:list(io:get_line("> ")),
        ch:ap(IO,[L]) end end.

putLine() ->
    fun (S) -> fun(IO) ->
        io:put_chars(": "++ch:unlist(S)),
        ch:ap(IO,[S]) end end.

rec() ->
    ap('Morte':recursive(),
        [getLine(),putLine(),pure(),list([])]).
\end{lstlisting}


Here is example of Erlang/OTP shell running recursive example.

\begin{lstlisting}[mathescape=true]
> om:rec().
> 1
: 1
> 2
: 2
> 3
: 3
> 4
: 4
> 5
: 5
#Fun<List.28.113171260>
\end{lstlisting}

\newpage
\section{Словник термінів}

З області програмування:
\\
\\
Формальні методи --- \\
Система типізації --- \\
Типова сигнатура --- \\
Імплементація --- \\
Інтерпретатор --- \\
Мова програмування --- \\
Теорія типів --- \\
Компіляція --- \\
Базова бібліотека --- \\
Середовище виконання --- \\
Вищі мови програмування --- \\
BNF нотація --- \\
Синтаксичне дерево --- \\
Синтаксис --- \\
Семантика --- \\

\newpage
З обсласті математики:
\\
\\
STLC --- \\
Логіка першого порядку --- \\
Класичні логіки --- \\
Лямбда-числення --- \\
Лямбда-куб --- \\
Пі-числення --- \\
Модальні логіки --- \\
Основи математики --- \\
Математична (формальна) верифікація --- \\
Теорія категорій --- \\
Теорія топосів --- \\
Теорема Гьоделя про неповноту --- \\
Логіки вищий порядків --- \\
Ізоморфізм --- \\
Гомотопічна теорія типів --- \\
Диференціальна топологія --- \\
Диференціальна геометрія --- \\
Еквіваріантна модальна супер-гомотопічна система --- \\
Теорія залежних типів --- \\
Числення конструкцій --- \\
Декартово-замкнена категорія --- \\
Числення індуктивних конструкцій --- \\
Системи доведення теорем (прувери) --- \\
Фібраційні математичні прувери --- \\
PTS система --- \\
MLTT система --- \\
