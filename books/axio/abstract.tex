\frontmatter
\thispagestyle{empty}
\mbox{}\vspace{1in}
\noindent
\begin{flushright}
\vspace{0.5cm}
\Huge \textbf{AXIO/1} \\
\vspace{0.5cm}
{       \Large Формальне середовище виконання, \\
        \Large система вищих мов та базові бібліотеки \\
        \Large для програмування, доведення теорем \\
              і формальної філософії \\
}
\vspace{4em}
        \large Навчальний посібник курсу «Формальна математика» \\
\vspace{1em}
\vspace{4cm}
\hfill{\Large Максим Сохацький \\
              Кам'янець-Подільський---Київ \\
              26 травня 2021 \\
}
\vspace{0.3cm}
\hfill{}
\end{flushright}
\newpage
\noindent УДК 004.4, 004.6, 004.9\\
\noindent УДК 510.21, 510.24, 510.25\\
\noindent УДК 510.6\\
\noindent УДК 519.68\\
\epigraph{Присвячується Маші та Міші}

У роботі розказується про новий формальний підхід до
математичної верифікації та спробу автора у цій
парадигмі побудувати замкнену уніфіковану систему
формальних мов для програмування, математики та
філософії. В процесі розробки моделі такої системи
автору довелося апробувати частини її імплементації
для головних SML-подібних формальних академічних мов,
мови Erlang та інших (загалом 7 мов). За 10 років
автором було проаналізовані синтаксис та семантика
основних мов програмування (більше 50 мов) з різних
промислових та академічних доменів, 8 мов з яких
були особисто реалізовані автором. В роботі описані
8 мов уніфікованої мовної системи (концептуальна модель)
та представлені 2 їх імплементації.

Головним чином, натхнення було почерпнуте з
LISP-машин минулого, APL-систем, перших систем
доведення теорем таких як AUTOMATH, віртуальних
маших паралельної та узгодженої обробки нескінченних
процесів, таких як BEAM, та кубічних MLTT-пруверів.
Робота буде корисною всім аспірантам чистої та
прикладної математики, теоретичної інформатики,
а також інженерам цих спеціальностей для розуміння
природи обчислень.
\\
\\
\\
\\
\\
\\
\\
Постійне посилання твору: https://axiosis.top/axio/ \\
Видавець: Інститут формальної математики «Групоїд Інфініті» \\
\\
\\
{\textbf{ISBN: 978-617-8027-10-0} \hspace{2em}} \\ \\
Підготовлено до друку в м.Кам'янець-Подільський, 26 травня 2021 року.
\\
\\
\begin{tabular}{ll}
\textcopyright{} 2021 & Максим Сохацький
\end{tabular}


\newpage
\cleartorecto
\tableofcontents*
\mainmatter
