\documentclass{article}
\usepackage{indentfirst}
\usepackage{hyphenat}
\usepackage{listings}
\usepackage{amsmath}
\usepackage{amssymb}
\usepackage{amsthm}
\usepackage{url}
\usepackage{tikz-cd}
\usepackage[utf8]{inputenc}
\usepackage[english]{babel}

\ProvidesPackage{journal}
\usepackage{graphicx}
\usepackage{mathtools}
\usepackage{hyphenat}
\usepackage{hyperref}
\usepackage{adjustbox}
\usepackage{listings}
\usepackage{verbatim}
\usepackage{xcolor}
\usepackage{amsfonts}
\usepackage{amscd}
\usepackage{amsmath}
\usepackage{amssymb}
\usepackage{amsthm}
\usepackage{tikz}
\usepackage{tikz-cd}
\usepackage{url}
\usepackage[utf8]{inputenc}
\usepackage[english,ukrainian]{babel}
\usepackage{float}
\usepackage{url}
\usepackage{tikz}
\usepackage{tikz-cd}
\usepackage[utf8]{inputenc}
\usepackage{graphicx}
\usepackage[utf8]{inputenc}
\usepackage[T1]{fontenc}
\usepackage{lmodern}
\usepackage{tocloft}
\usepackage{hyperref}
\usepackage{xcolor}
\usepackage[only,llbracket,rrbracket,llparenthesis,rrparenthesis]{stmaryrd}

\usetikzlibrary{babel}

\newcommand*{\incmap}{\hookrightarrow}
\newcommand*{\thead}[1]{\multicolumn{1}{c}{\bfseries #1}}
\renewcommand{\Join}{\vee} % Join operation symbol
\newcommand{\tabstyle}[0]{\scriptsize\ttfamily\fontseries{l}\selectfont}

\lstset{
  basicstyle=\footnotesize,
  inputencoding=utf8,
  identifierstyle=,
  literate=
{𝟎}{{\ensuremath{\mathbf{0}}}}1
{𝟏}{{\ensuremath{\mathbf{1}}}}1
{≔}{{\ensuremath{\mathrm{:=}}}}1
{α}{{\ensuremath{\mathrm{\alpha}}}}1
{ᵂ}{{\ensuremath{^W}}}1
{β}{{\ensuremath{\mathrm{\beta}}}}1
{γ}{{\ensuremath{\mathrm{\gamma}}}}1
{δ}{{\ensuremath{\mathrm{\delta}}}}1
{ε}{{\ensuremath{\mathrm{\varepsilon}}}}1
{ζ}{{\ensuremath{\mathrm{\zeta}}}}1
{η}{{\ensuremath{\mathrm{\eta}}}}1
{θ}{{\ensuremath{\mathrm{\theta}}}}1
{ι}{{\ensuremath{\mathrm{\iota}}}}1
{κ}{{\ensuremath{\mathrm{\kappa}}}}1
{λ}{{\ensuremath{\mathrm{\lambda}}}}1
{μ}{{\ensuremath{\mathrm{\mu}}}}1
{ν}{{\ensuremath{\mathrm{\nu}}}}1
{ξ}{{\ensuremath{\mathrm{\xi}}}}1
{π}{{\ensuremath{\mathrm{\mathnormal{\pi}}}}}1
{ρ}{{\ensuremath{\mathrm{\rho}}}}1
{σ}{{\ensuremath{\mathrm{\sigma}}}}1
{τ}{{\ensuremath{\mathrm{\tau}}}}1
{φ}{{\ensuremath{\mathrm{\varphi}}}}1
{χ}{{\ensuremath{\mathrm{\chi}}}}1
{ψ}{{\ensuremath{\mathrm{\psi}}}}1
{ω}{{\ensuremath{\mathrm{\omega}}}}1
{Π}{{\ensuremath{\mathrm{\Pi}}}}1
{Γ}{{\ensuremath{\mathrm{\Gamma}}}}1
{Δ}{{\ensuremath{\mathrm{\Delta}}}}1
{Θ}{{\ensuremath{\mathrm{\Theta}}}}1
{Λ}{{\ensuremath{\mathrm{\Lambda}}}}1
{Σ}{{\ensuremath{\mathrm{\Sigma}}}}1
{Φ}{{\ensuremath{\mathrm{\Phi}}}}1
{Ξ}{{\ensuremath{\mathrm{\Xi}}}}1
{Ψ}{{\ensuremath{\mathrm{\Psi}}}}1
{Ω}{{\ensuremath{\mathrm{\Omega}}}}1
{ℵ}{{\ensuremath{\aleph}}}1
{≤}{{\ensuremath{\leq}}}1
{≥}{{\ensuremath{\geq}}}1
{≠}{{\ensuremath{\neq}}}1
{≈}{{\ensuremath{\approx}}}1
{≡}{{\ensuremath{\equiv}}}1
{≃}{{\ensuremath{\simeq}}}1
{≤}{{\ensuremath{\leq}}}1
{≥}{{\ensuremath{\geq}}}1
{∂}{{\ensuremath{\partial}}}1
{∆}{{\ensuremath{\triangle}}}1 % or \laplace?
{∫}{{\ensuremath{\int}}}1
{∑}{{\ensuremath{\mathrm{\Sigma}}}}1
{→}{{\ensuremath{\rightarrow}}}1
{⊥}{{\ensuremath{\perp}}}1
{∞}{{\ensuremath{\infty}}}1
{∂}{{\ensuremath{\partial}}}1
{∓}{{\ensuremath{\mp}}}1
{±}{{\ensuremath{\pm}}}1
{×}{{\ensuremath{\times}}}1
{⊕}{{\ensuremath{\oplus}}}1
{⊗}{{\ensuremath{\otimes}}}1
{⊞}{{\ensuremath{\boxplus}}}1
{∇}{{\ensuremath{\nabla}}}1
{√}{{\ensuremath{\sqrt}}}1
{⬝}{{\ensuremath{\cdot}}}1
{•}{{\ensuremath{\cdot}}}1
{∘}{{\ensuremath{\circ}}}1
{⁻}{{\ensuremath{^{-}}}}1
{▸}{{\ensuremath{\blacktriangleright}}}1
{★}{{\ensuremath{\star}}}1
{∧}{{\ensuremath{\wedge}}}1
{∨}{{\ensuremath{\vee}}}1
{¬}{{\ensuremath{\neg}}}1
{⊢}{{\ensuremath{\vdash}}}1
{⟨}{{\ensuremath{\langle}}}1
{⟩}{{\ensuremath{\rangle}}}1
{↦}{{\ensuremath{\mapsto}}}1
{→}{{\ensuremath{\rightarrow}}}1
{↔}{{\ensuremath{\leftrightarrow}}}1
{⇒}{{\ensuremath{\Rightarrow}}}1
{⟹}{{\ensuremath{\Longrightarrow}}}1
{⇐}{{\ensuremath{\Leftarrow}}}1
{⟸}{{\ensuremath{\Longleftarrow}}}1
{∩}{{\ensuremath{\cap}}}1
{∪}{{\ensuremath{\cup}}}1
{·}{{\ensuremath{\cdot}}}1
{ᵢ}{{\ensuremath{_i}}}1
{ⱼ}{{\ensuremath{_j}}}1
{₊}{{\ensuremath{_+}}}1
{ℑ}{{\ensuremath{\Im}}}1
{𝒢}{{\ensuremath{\mathcal{G}}}}1
{ℕ}{{\ensuremath{\mathbb{N}}}}1
{𝟘}{{\ensuremath{\mathbb{0}}}}1
{ℤ}{{\ensuremath{\mathbb{Z}}}}1
{ℝ}{{\ensuremath{\mathbb{R}}}}1
{⊂}{{\ensuremath{\subseteq}}}1
{⊆}{{\ensuremath{\subseteq}}}1
{⊄}{{\ensuremath{\nsubseteq}}}1
{⊈}{{\ensuremath{\nsubseteq}}}1
{⊃}{{\ensuremath{\supseteq}}}1
{⊇}{{\ensuremath{\supseteq}}}1
{⊅}{{\ensuremath{\nsupseteq}}}1
{⊉}{{\ensuremath{\nsupseteq}}}1
{∈}{{\ensuremath{\in}}}1
{∉}{{\ensuremath{\notin}}}1
{∋}{{\ensuremath{\ni}}}1
{∌}{{\ensuremath{\notni}}}1
{∅}{{\ensuremath{\emptyset}}}1
{∖}{{\ensuremath{\setminus}}}1
{†}{{\ensuremath{\dag}}}1
{ℕ}{{\ensuremath{\mathbb{N}}}}1
{ℤ}{{\ensuremath{\mathbb{Z}}}}1
{ℝ}{{\ensuremath{\mathbb{R}}}}1
{ℚ}{{\ensuremath{\mathbb{Q}}}}1
{ℂ}{{\ensuremath{\mathbb{C}}}}1
{⌞}{{\ensuremath{\llcorner}}}1
{⌟}{{\ensuremath{\lrcorner}}}1
{⦃}{{\ensuremath{ \{\!| }}}1
{⦄}{{\ensuremath{ |\!\} }}}1
{ᵁ}{{\ensuremath{^U}}}1
{₋}{{\ensuremath{_{-}}}}1
{₁}{{\ensuremath{_1}}}1
{₂}{{\ensuremath{_2}}}1
{₃}{{\ensuremath{_3}}}1
{₄}{{\ensuremath{_4}}}1
{₅}{{\ensuremath{_5}}}1
{₆}{{\ensuremath{_6}}}1
{₇}{{\ensuremath{_7}}}1
{₈}{{\ensuremath{_8}}}1
{₉}{{\ensuremath{_9}}}1
{₀}{{\ensuremath{_0}}}1
{¹}{{\ensuremath{^1}}}1
{ₙ}{{\ensuremath{_n}}}1
{ₘ}{{\ensuremath{_m}}}1
{↑}{{\ensuremath{\uparrow}}}1
{↓}{{\ensuremath{\downarrow}}}1
{▸}{{\ensuremath{\triangleright}}}1
{∀}{{\ensuremath{\forall}}}1
{∃}{{\ensuremath{\exists}}}1
{λ}{{\ensuremath{\mathrm{\lambda}}}}1
{=}{{\ensuremath{=}}}1
{<}{{\ensuremath{\textless}}}1
{>}{{\ensuremath{\textgreater}}}1
{_}{{$\_$}}1
{(}{(}1
{(}{(}1
{‖}{{\ensuremath{\Vert}}}1
{+}{{+}}1
{*}{{*}}1,
}

\theoremstyle{definition}
\newtheorem{definition}{Definition}
\newtheorem{theorem}{Theorem}
\newtheorem{lemma}{Lemma}
\newtheorem{example}{Example}



\begin{document}

\title{Issue IV: Higher Inductive Types}
\author{Maksym Sokhatskyi $^1$}
\date{ $^1$ National Technical University of Ukraine \\
       \small Igor Sikorsky Kyiv Polytechnic Institute \\
       May 4, 2019 }
\maketitle

\begin{abstract}
CW-complexes are central to both homotopy theory and homotopy type theory (HoTT) and are encoded
in cubical theorem-proving systems as higher inductive types (HIT), similar to recursive
trees for (co)inductive types. We explore the basic primitives of homotopy theory,
which are considered as a foundational basis in theorem-proving systems. \\
\indent {\bf Keywords}: Homotopy Theory, Type Theory
\end{abstract}


\ifincludeTOC
  \tableofcontents
\else
  \addtocontents{toc}{\protect\newpage}
\fi

\section{CW-Complexes}
CW-complexes are spaces constructed by attaching cells of various
dimensions. In HoTT, they are encoded as higher inductive types (HIT),
where cells are constructors for points and paths.

\begin{definition} (Cell Attachment).
The attachment of an $n$-cell to a space \( X \)
along \( f : S^{n-1} \to X \) is a pushout:
\[
\begin{tikzcd}
S^{n-1} \arrow[r, "f"] \arrow[d, "\iota"] & X \arrow[d, "j"] \\
D^n \arrow[r, "g"]                        & X \cup_f D^n \\
\end{tikzcd}
\]
Here, \( \iota : S^{n-1} \incmap D^n \) is the boundary inclusion,
and \( X \cup_f D^n \) is the pushout that attaches an $n$-cell to \( X \) via \( f \). The result depends on
the homotopy class of \( f \).
\end{definition}

\begin{definition} (CW-Complex).
A CW-complex is a space \( X \), constructed inductively by attaching cells, with a skeletal filtration:
\begin{itemize}
    \item \((-1)\)-skeleton: \( X_{-1} = \varnothing \).
    \item For \( n \geq 0 \), the $n$-skeleton \( X_n \) is obtained by attaching $n$-cells to \( X_{n-1} \). For
      indices \( J_n \) and maps \( \{ f_j : S^{n-1} \to X_{n-1}
      \}_{j \in J_n} \), \( X_n \) is the pushout:
    \[
    \begin{tikzcd}
    \coprod_{j \in J_n} S^{n-1} \arrow[r, "\coprod f_j"] \arrow[d, "\coprod \iota_j"] & X_{n-1} \arrow[d, "i_n"] \\
    \coprod_{j \in J_n} D^n \arrow[r, "\coprod g_j"] & X_n
    \end{tikzcd}
    \]
    where \( \coprod_{j \in J_n} S^{n-1} \), \( \coprod_{j \in
    J_n} D^n \) are disjoint unions, and \( i_n : X_{n-1}
    \incmap X_n \) is the inclusion.
    \item \( X \) is the colimit:
    \[
    \varnothing = X_{-1} \incmap X_0 \incmap X_1 \incmap \dots
    \incmap X,
    \]
    where \( X_n \) is the $n$-skeleton, and \( X = \text{colim}_{n
    \to \infty} X_n \). The sequence is the skeletal filtration.
\end{itemize}
In HoTT, CW-complexes are higher inductive types (HIT)
with constructors for cells and paths for attachment.
\end{definition}

\newpage
\subsection{Introduction: Countable Constructors}
Some HITs require an infinite number of constructors for spaces,
such as Eilenberg-MacLane spaces or the infinite sphere \( S^\infty \).
\begin{lstlisting}[mathescape=true]
def S$^\infty$ : U
 := inductive { base
              | loop (n: $\mathbb{N}$) : base $\equiv$ base
              }
\end{lstlisting}
Challenges include type checking, computation, and expressiveness.

Agda Cubical uses cubical primitives to handle HITs, supporting
infinite constructors via HITs indexed by natural numbers, as colimits.

\subsection{Motivation: Higher Inductive Types}
HITs in HoTT enable direct encoding of topological spaces, such as
CW-complexes. In homotopy theory, spaces are constructed by attaching
cells via attaching maps. HoTT views types as spaces, elements
as points, and equalities as paths, making HITs a natural choice.
Standard inductive types cannot capture higher homotopies, but
HITs allow constructors for points and paths. For example, the circle \( S^1 \) (Definition 2) has a base point and a loop, encoding its
fundamental group \( \mathbb{Z} \). HITs avoid the use of multiple
quotient spaces, preserving the synthetic nature of HoTT. In cubical type theory, paths are
intervals (e.g., \( <i> \)) with computational content, unlike
propositional equalities, enabling efficient type checking in tools
such as Agda Cubical.

\subsection{Metatheory: Cohesive Topoi}

\subsubsection{Geometric Proofs}
$$
\Re \dashv \Im \dashv \&
$$
For differential geometry, type theory incorporates primitive axioms of
categorical meta-theoretical models of three Schreiber-Shulman functors:
infinitesimal neighborhood (\(\Im\)), reduced modality (\(\Re\)), and
infinitesimal discrete neighborhood (\(\&\)).

\subsubsection{Flat Proofs}
\subsubsection{Sharp Proofs}
\subsubsection{Bose Proofs}
\subsubsection{Fermi Proofs}
\subsubsection{Linear Proofs}

$$
\otimes \dashv x \dashv\ \multimap
$$

For engineering applications (e.g., Milner's $\pi$-calculus, quantum computing)
and linear type theory, type theory embeds linear proofs based on the adjunction
of the tensor and linear function spaces: $(A \otimes B) \multimap A \simeq A \multimap (B \multimap C)$,
represented in a symmetric monoidal category $\mathbf{D}$ for a functor $[A,B]$ as:
$\mathbf{D}(A \otimes B, C) \simeq \mathbf{D}(A, [B,C])$.

\section{Higher Inductive Types}
CW-complexes are central to HoTT and appear in cubical type checkers as HITs.
Unlike inductive types (recursive trees), HITs encode CW-complexes, capturing
points (0-cells) and higher paths (n-cells). The definition of an HIT specifies a CW-complex
through cubical composition, an initial algebra in the cubical model.

\newpage
\subsection{Suspension}
The suspension \(\Sigma A\) of a type \( A \) is a higher inductive type that
constructs a new type by adding two points, called poles, and paths connecting
each point of \( A \) to these poles. It is a fundamental construction in homotopy theory,
often used to shift homotopy groups, e.g., obtaining \( S^{n+1} \) from \( S^n \).

\begin{definition} (Formation).
For any type \( A : \mathcal{U} \), there exists a suspension type \( \Sigma A : \mathcal{U} \).
\end{definition}

\begin{definition} (Constructors).
For a type \( A : \mathcal{U} \), the suspension \( \Sigma A : \mathcal{U} \)
is generated by the following higher inductive compositional structure:
\[
\Sigma :=
\begin{cases}
\text{north} \\
\text{south} \\
\text{merid} : (a : A) \rightarrow \text{north} \equiv \text{south} \\
\end{cases}
\]
\begin{lstlisting}[mathescape=true]
def $\Sigma$ (A: U) : U
 := inductive { north
              | south
              | merid (a: A) : north $\equiv$ south
              }
\end{lstlisting}
\end{definition}

\begin{theorem} (Elimination).
For a family of types \( B : \Sigma A \to \mathcal{U} \), points \( n : B(\text{north}) \), \( s : B(\text{south}) \), and a family
of dependent paths
\[
  m : \Pi (a : A), \text{PathOver}(B,\text{merid}(a),n,s),
\]
there exists a dependent map \( \text{Ind}_{\Sigma A} : (x : \Sigma A) \to B(x) \), such that:
\[
\begin{cases}
\text{Ind}_{\Sigma A}(\text{north}) = n \\
\text{Ind}_{\Sigma A}(\text{south}) = s \\
\text{Ind}_{\Sigma A}(\text{merid}(a,i)) = m(a,i)
\end{cases}
\]
\begin{lstlisting}[mathescape=true]
def PathOver (B: $\Sigma$ A -> U) (a: A) (n: B north) (s: B south) : U
  := PathP ($\lambda$ i , B (merid a @ i)) n s

def Ind$_\Sigma A$ (A: U) (B: $\Sigma$ A -> U) (n: B north) (s: B south)
    (m: (a: A) -> PathOver B (merid a) n s) : (x: $\Sigma$ A) -> B x
 := split { north -> n | south -> s | merid a @ i -> m a @ i }
\end{lstlisting}
\end{theorem}

\begin{theorem} (Computation).
\[
    \text{Ind}_\Sigma A(\text{north}) = n \\
    \text{Ind}_\Sigma A(\text{south}) = s \\
    \text{Ind}_\Sigma A(\text{merid}(a,i)) = m(a,i)
\]
\begin{lstlisting}[mathescape=true]
def $\Sigma$-$\beta$ (A: U) (B: $\Sigma A$ -> U) (n: B north) (s: B south)
    (m: (a: A) -> PathOver B (merid a) n s) (x: $\Sigma$ A)
  : Path (B x) ($\Sigma$-I A B n s m x)
    split { north -> n | south -> s | merid a @ i -> m a @ i }
\end{lstlisting}
\end{theorem}

\begin{theorem} (Uniqueness).
Any two maps \( h_1, h_2 : (x : \Sigma A) \to B(x) \) are homotopic if they agree
on \( \text{north} \), \( \text{south} \), and \( \text{merid} \), i.e.,
if \( h_1(\text{north}) = h_2(\text{north}) \), \( h_1(\text{south}) = h_2(\text{south}) \),
and \( h_1(\text{merid } a) = h_2(\text{merid } a) \) for all \( a : A \).
\end{theorem}

\subsection{Pushout}
The pushout (amalgamation) is a higher inductive type that constructs a type by
gluing two types \( A \) and \( B \) along a common type \( C \) via maps \( f : C \to A \)
and \( g : C \to B \). It is a fundamental construction in homotopy theory,
used to model cell attachment and cofibrant objects, generalizing
the topological notion of a pushout.

\begin{definition} (Formation).
For types \( A, B, C : \mathcal{U} \) and maps \( f : C \to A \), \( g : C \to B \),
there exists a pushout \( \sqcup(A,B,C,f,g) : \mathcal{U} \).
\end{definition}

\begin{definition} (Constructors).
The pushout is generated by the following higher inductive compositional structure:
\[
\sqcup :=
\begin{cases}
\text{po$_1$} : A \to \sqcup(A,B,C,f,g) \\
\text{po$_2$} : B \to \sqcup(A,B,C,f,g) \\
\text{po$_3$} : (c : C) \to \text{po$_1$}(f(c)) \equiv \text{po$_2$}(g(c))
\end{cases}
\]
\begin{lstlisting}[mathescape=true]
def $\sqcup$ (A B C: U) (f: C -> A) (g: C -> B) : U
 := inductive { po$_1$ (a: A)
              | po$_2$ (b: B)
              | po$_3$ (c: C) : po$_1$(f(c)) $\equiv$ po$_2$(g(c))
              }
\end{lstlisting}
\end{definition}

\begin{theorem} (Elimination).
For a type \( D : \mathcal{U} \), maps \( u : A \to D \), \( v : B \to D \),
and a family of paths \( p : (c : C) \to u(f(c)) \equiv v(g(c)) \),
there exists a map \( \text{Ind}_\sqcup : \sqcup(A,B,C,f,g) \to D \), such that:
\[
\begin{cases}
\text{Ind}_\sqcup(\text{po$_1$}(a)) = u(a) \\
\text{Ind}_\sqcup(\text{po$_2$}(b)) = v(b) \\
\text{Ind}_\sqcup(\text{po$_3$}(c,i)) = p(c,i)
\end{cases}
\]
\begin{lstlisting}[mathescape=true]
def PathOver (A B C : U) (f : C $\rightarrow$ A) (g : C $\rightarrow$ B)
    (D : $\sqcup$ A B C f g $\rightarrow$ U)
    (c : C) (u : D (po$_1$ (f c))) (v : D (po$_2$ (g c))) : U
 := PathP ($\lambda$ i, D (po$_3$ c i)) u v

def Ind$_\sqcup$ : (A B C : U) (f : C $\rightarrow$ A) (g : C $\rightarrow$ B)
    (D : $\sqcup$ A B C f g $\rightarrow$ U)
    (u : (a : A) $\rightarrow$ D (po$_1$ a))
    (v : (b : B) $\rightarrow$ D (po$_2$ b))
    (p : (c : C) $\rightarrow$ PathOver D c (u (f c)) (v (g c)))
  : (x : $\sqcup$ A B C f g) $\rightarrow$ D x
 := split { po$_1$ a $\rightarrow$ u a | po$_2$ b $\rightarrow$ v b | po$_3$ c @ i $\rightarrow$ p c @ i }
\end{lstlisting}
\end{theorem}

\begin{theorem} (Computation).
For \( x : \sqcup(A,B,C,f,g) \),
\[
\begin{cases}
\text{Ind$_\sqcup$}(\text{po$_1$}(a)) \equiv u(a) \\
\text{Ind$_\sqcup$}(\text{po$_2$}(b)) \equiv v(b) \\
\text{Ind$_\sqcup$}(\text{po$_3$}(c,i)) \equiv p(c,i)
\end{cases}
\]
\end{theorem}

\begin{theorem} (Uniqueness).
Any two maps \( u, v : \sqcup(A,B,C,f,g) \to D \) are homotopic
if they agree on \( \text{po}_1 \), \( \text{po}_2 \), and \( \text{po}_3 \), i.e.,
if \( u(\text{po}_1(a)) = v(\text{po}_1(a)) \) for all \( a : A \), \( u(\text{po}_2(b)) = v(\text{po}_2(b)) \)
for all \( b : B \), and \( u(\text{po}_3(c)) = v(\text{po}_3(c)) \) for all \( c : C \).
\end{theorem}

\begin{example} (Cell Attachment)
The pushout models the attachment of an \( n \)-cell to a space \( X \).
Given \( f : S^{n-1} \to X \) and inclusion \( g : S^{n-1} \to D^n \),
the pushout \( \sqcup(X,D^n,S^{n-1},f,g) \) is
the space \( X \cup_f D^n \), attaching an \( n \)-disk to \( X \) along \( f \).
\[
\begin{tikzcd}
S^{n-1} \arrow[r, "f"] \arrow[d, "g"] & X \arrow[d] \\
D^n \arrow[r] & X \cup_f D^n
\end{tikzcd}
\]
\end{example}

\subsection{Spheres}
Spheres are higher inductive types with higher-dimensional paths,
representing fundamental topological spaces.

\begin{definition} (Pointed n-Spheres)
The \( n \)-sphere \( S^n \) is defined recursively as a type in
the universe \( \mathcal{U} \) using general recursion over dimensions:
\[
\mathbb{S}^n :=
\begin{cases}
\text{point} : \mathbb{S}^n, \\
\text{surface} :\ <i_1,...i_n> [\ (i_1=0) \rightarrow \text{point}, (i_1=1) \rightarrow \text{point},\ ... \\
\hspace{3.47cm} (i_n=0) \rightarrow \text{point}, (i_n=1) \rightarrow \text{point}\ ]
\end{cases}
\]
\end{definition}

\begin{definition} (n-Spheres via Suspension)
The \( n \)-sphere \( S^n \) is defined recursively as a type in
the universe \( \mathcal{U} \) using general recursion over natural
numbers \( \mathbb{N} \). For each \( n \in \mathbb{N} \),
the type \( S^n : \mathcal{U} \) is defined as:
\[
\mathbb{S}^n :=
\begin{cases}
S^0 = \mathbf{2}, \\
S^{n+1} = \Sigma(S^n).
\end{cases}
\]
\begin{lstlisting}[mathescape=true]
def sphere : $\mathbb{N}$ $\to$ U := $\mathbb{N}\text{-iter}$ U $\mathbf{2}$ $\Sigma$
\end{lstlisting}
This iterative definition applies the suspension functor \( \Sigma \)
to the base type \( \mathbf{2} \) (0-sphere) \( n \) times to obtain \( S^n \).
\end{definition}

\begin{example} (Sphere as CW-Complex)
The \( n \)-sphere \( S^n \) can be constructed as a CW-complex with
one 0-cell and one \( n \)-cell:
\[
\begin{cases}
X_0 = \{ \text{base} \}, \text{ one point} \\
X_k = X_0 \text{ for } 0 < k < n, \text{ no additional cells} \\
X_n: \text{Attachment of an } n\text{-cell to } X_{n-1} = \{ \text{base} \} \text{ along } f : S^{n-1} \to \{ \text{base} \}
\end{cases}
\]
The constructor \( \text{cell} \) attaches the boundary of the \( n \)-cell
to the base point, yielding the type \( S^n \).
\end{example}

\newpage
\subsection{Hub and Spokes}
The hub and spokes construction \( \odot \) defines an \( n \)-truncation,
ensuring that the type has no non-trivial homotopy groups above
dimension \( n \). It models the type as a CW-complex with
a hub (central point) and spokes (paths to points).

\begin{definition} (Formation).
For types \( S, A : \mathcal{U} \), there exists a hub and spokes type \( \odot\ (S,A) : \mathcal{U} \).
\end{definition}

\begin{definition} (Constructors).
The hub and spokes type is freely generated by the following higher inductive compositional structure:
\[
\odot :=
\begin{cases}
\text{base} : A \to \odot\ (S,A) \\
\text{hub} : (S \to \odot\ (S,A)) \to \odot\ (S,A) \\
\text{spoke} : (f : S \to \odot\ (S,A)) \to (s : S) \to \text{hub}(f) \equiv f(s)
\end{cases}
\]
\begin{lstlisting}[mathescape=true]
def $\odot$ (S A: U) : U
 := inductive { base (x: A)
              | hub (f: S -> $\odot$ S A)
              | spoke (f: S -> $\odot$ S A) (s:S) : hub f $\equiv$ f s
              }
\end{lstlisting}
\end{definition}

\begin{theorem} (Elimination).
For a family of types \( P : \text{HubSpokes} \, S \, A \to \mathcal{U} \),
maps \( \text{pbase} : (x : A) \to P(\text{base} \, x) \),
\( \text{phub} : (f : S \to \text{HubSpokes} \, S \, A) \to P(\text{hub} \, f) \),
and a family of paths
\( \text{pspoke} : (f : S \to \text{HubSpokes} \, S \, A) \to (s : S) \to \text{PathP} \, (<i> P(\text{spoke} \, f \, s \, @ \, i)) \, (\text{phub} \, f) \, (P(f \, s)) \),
there exists a map \( \text{hubSpokesInd} : (z : \text{HubSpokes} \, S \, A) \to P(z) \), such that:
\[
\begin{cases}
\text{Ind}_\odot \, (\text{base} \, x) = \text{pbase} \, x \\
\text{Ind}_\odot \, (\text{hub} \, f) = \text{phub} \, f \\
\text{Ind}_\odot \, (\text{spoke} \, f \, s \, @ \, i) = \text{pspoke} \, f \, s \, @ \, i \\
\end{cases}
\]
\end{theorem}

\newpage
\subsection{Truncation}
\subsubsection*{Set Truncation}
\begin{definition} (Formation).
Set truncation (0-truncation), denoted \( \| A \|_0 \),
ensures that the type is a set, with homotopy groups vanishing above dimension 0.
\end{definition}

\begin{definition} (Constructors).
For \( A : \mathcal{U} \), \( \| A \|_0 : \mathcal{U} \) is defined by the following higher
inductive compositional structure:
\[
\| \_ \|_0 :=
\begin{cases}
\text{inc} : A \to \| A \|_0 \\
\text{squash} : (a, b : \| A \|_0) \to (p, q : a \equiv b) \to p \equiv q
\end{cases}
\]
\begin{lstlisting}[mathescape=true]
def $\|$_$\|_0$ (A: U) : U
 := inductive { inc (a: A)
              | squash (a b: $\|$A$\|_0$) (p q: Path ($\|$A$\|_0$) a b)
                <i j> [ (i = 0) -> p @ j, (i = 1) -> q @ j,
                        (j = 0) -> a,     (j = 1) -> b ]
              }
\end{lstlisting}
\end{definition}

\begin{theorem} (Elimination \( \| A \|_0 \))
For a set \( B : \mathcal{U} \) (i.e., \( \text{isSet}(B) \)), and a map \( f : A \to B \),
there exists \( \text{setTruncRec} : \| A \|_0 \to B \), such that \( \text{Ind}_{\|A\|_0}(\text{inc}(a)) = f(a) \).
\end{theorem}

\subsubsection*{Groupoid Truncation}
\begin{definition} (Formation).
Groupoid truncation (1-truncation), denoted \( \| A \|_1 \), ensures that
the type is a 1-groupoid, with homotopy groups vanishing above dimension 1.
\end{definition}

\begin{definition} (Constructors).
For \( A : \mathcal{U} \), \( \| A \|_1 : \mathcal{U} \) is defined by the following higher
inductive compositional structure:
\[
\| \_ \|_1 :=
\begin{cases}
\text{inc} : A \to \| A \|_1 \\
\text{squash} : (a, b : \| A \|_1) \to (p, q : a \equiv b) \to (r, s : p \equiv q) \to r \equiv s
\end{cases}
\]
\begin{lstlisting}[mathescape=true]
def $\|$_$\|_1$ (A: U) : U
 := inductive { inc (a: A)
              | squash (a b: $\|$A$\|_1$) (p q: Path ($\|$A$\|_1$) a b)
                (r s: Path (Path ($\|$A$\|_1$) a b) p q) <i j k>
                [ (i = 0) -> r @ j @ k, (i = 1) -> s @ j @ k,
                  (j = 0) -> p @ k,     (j = 1) -> q @ k,
                  (k = 0) -> a,         (k = 1) -> b ]
              }
\end{lstlisting}
\end{definition}

\begin{theorem} (Elimination \( \| A \|_1 \))
For a 1-groupoid \( B : \mathcal{U} \) (i.e., \( \text{isGroupoid}(B) \)),
and a map \( f : A \to B \), there exists \( \text{Ind}_{\|A\|_1} : \|A\|_1 \to B \),
such that \( \text{Ind}_{\|A\|_1}(\text{inc}(a)) = f(a) \).
\end{theorem}

\newpage
\subsection{Quotients}
\subsubsection*{Set Quotient Spaces}
Quotient spaces are a powerful computational tool in type theory, embedded
in the core of Lean.
\begin{definition} (Formation).
Set quotient spaces construct a type \( A \), quotiented by
a relation \( R : A \to A \to \mathcal{U} \), ensuring that the result is a set.
\end{definition}

\begin{definition} (Constructors).
For a type \( A : \mathcal{U} \) and a relation \( R : A \to A \to \mathcal{U} \),
the set quotient space \( A / R : \mathcal{U} \) is freely generated by the following
higher inductive compositional structure:
\[
A / R :=
\begin{cases}
\text{quot} : A \to A / R \\
\text{ident} : (a, b : A) \to R(a,b) \to \text{quot}(a) \equiv \text{quot}(b) \\
\text{trunc} : (a, b : A / R) \to (p, q : a \equiv b) \to p \equiv q
\end{cases}
\]
\begin{lstlisting}[mathescape=true]
def / (A: U) (R: A -> A -> U) : U
  := inductive { quot (a: A)
               | ident (a b: A) (r: R a b) : quot(a) $\equiv$ quot(b)
               | trunc (a b : / A R) (p q : Path (/ A R) a b)
                 <i j> [ (i = 0) -> p @ j , (i = 1) -> q @ j ,
                         (j = 0) -> a ,     (j = 1) -> b ]
               }
\end{lstlisting}
\end{definition}

\begin{theorem} (Elimination).
For a family of types \( B : A/R \to \mathcal{U} \) with \( \text{isSet}(B x) \),
and maps \( f : (x : A) \to B(\text{quot}(x)) \),
\( g : (a, b : A) \to (r : R(a,b)) \to \text{PathP} (<i> B(\text{ident}(a,b,r)\ @\ i)) (f(a)) (f(b)) \),
there exists \( \text{Ind}_{A/R} : \Pi (x: A/R), B(x) \), such that \( \text{Ind}_{A/R}(\text{quot}(a)) = f(a) \).
\end{theorem}

\subsubsection*{Groupoid Quotient Spaces}
\begin{definition} (Formation).
Groupoid quotient spaces extend set quotient spaces to produce a 1-groupoid,
including constructors for higher paths. Groupoid quotient spaces
construct a type \( A \), quotiented by a relation \( R : A \to A \to \mathcal{U} \),
ensuring that the result is a groupoid.
\end{definition}

\begin{definition} (Constructors).
For a type \( A : \mathcal{U} \) and a relation \( R : A \to A \to \mathcal{U} \),
the groupoid quotient space \( A // R : \mathcal{U} \) includes
constructors for points, paths, and higher paths, ensuring a 1-groupoid structure.
\end{definition}

\newpage
\subsection{Wedge}
The wedge of two pointed types \( A \) and \( B \), denoted \( A \vee B \),
is a higher inductive type representing the union of \( A \)
and \( B \) with identified base points. Topologically,
it corresponds to \( A \times \{ y_0 \} \cup \{ x_0 \} \times B \),
where \( x_0 \) and \( y_0 \) are the base points of \( A \) and \( B \), respectively.

\begin{definition} (Formation).
For pointed types \( A, B : \text{pointed} \), the wedge \( A \vee B : \mathcal{U} \).
\end{definition}

\begin{definition} (Constructors).
The wedge is generated by the following higher inductive compositional structure:
\[
\vee :=
\begin{cases}
\text{winl} : A.1 \to A \vee B \\
\text{winr} : B.1 \to A \vee B \\
\text{wglue} : \text{winl} (A.2) \equiv \text{winr}(B.2)
\end{cases}
\]
\begin{lstlisting}[mathescape=true]
def $\vee$ (A : pointed) (B : pointed) : U
 := inductive { winl (a : A.1)
              | winr (b : B.1)
              | wglue : winl(A.2) $\equiv$ winr(B.2)
              }
\end{lstlisting}
\end{definition}

\begin{theorem} (Elimination).
For a type \( P : A \vee B \mathcal{U} \), maps \( f : A.1 \to C \),
\( g : B.1 \to C \), and a path \( p : \text{PathOverlue}(P,f(A.2),g (B.2)) \),
there exists a map \( \text{Ind}_\vee : A \vee B \to C \), such that:
\[
\begin{cases}
\text{Ind}(\text{winl}(a)) = f(a) \\
\text{Ind}(\text{winr}(b)) = g(b) \\
\text{Ind}(\text{wglue}(x)) = p(x)
\end{cases}
\]
\begin{lstlisting}[mathescape=true]
def PathOverGlue : (P : A $\vee$ B → U)
    (p : P (inl (A.2))) (q : P (inr (B.2))) : U
 := PathP (λ i → P (wglue i)) p q

def Ind$_\vee$ (A B : pointed) (C : U) (f : A.1 -> C) (g : B.1 -> C)
    (p : Path C (f A.2) (g B.2))
  : A $\vee$ B -> C
 := split { winl a -> f a | winr b -> g b | wglue @ x -> p @ x }
\end{lstlisting}
\end{theorem}

\begin{theorem} (Computation).
For \( z : \text{Wedge } A B \),
\[
\begin{cases}
\text{Ind}_\vee(\text{winl } a) \equiv f(a) \\
\text{Ind}_\vee(\text{winr } b) \equiv g(b) \\
\text{Ind}_\vee(\text{wglue } @ \, x) \equiv p \, @ \, x
\end{cases}
\]
\end{theorem}

\begin{theorem} (Uniqueness).
Any two maps \( h_1, h_2 : \text{Wedge } A B \to C \) are homotopic if
they agree on \( \text{winl} \), \( \text{winr} \), and \( \text{wglue} \),
i.e., if \( h_1(\text{winl } a) = h_2(\text{winl } a) \) for
all \( a : A.1 \), \( h_1(\text{winr } b) = h_2(\text{winr } b) \) for
all \( b : B.1 \), and \( h_1(\text{wglue}) = h_2(\text{wglue}) \).
\end{theorem}

\newpage
\subsection{Smash Product}
The smash product of two pointed types \( A \) and \( B \), denoted \( A \wedge B \),
is a higher inductive type that quotients the product \( A \times B \) by
the pushout \( A \sqcup B \). It represents the space \( A \times B / (A \times \{ y_0 \} \cup \{ x_0 \} \times B) \),
collapsing the wedge to a single point.

\begin{definition} (Formation).
For pointed types \( A, B : \text{pointed} \), the smash product \( A \wedge B : \mathcal{U} \).
\end{definition}

\begin{definition} (Constructors).
The smash product is generated by the following higher inductive compositional structure:
\[
A \wedge B :=
\begin{cases}
\text{basel} : A \wedge B \\
\text{baser} : A \wedge B \\
\text{proj} (x: A.1) (y: B.1) : A \wedge B \\
\text{gluel} (a: A.2) : \text{proj}(a,B.2) \equiv \text{basel} \\
\text{gluer} (b: B.2) : \text{proj}(A.2,b) \equiv \text{baser} \\
\end{cases}
\]
\begin{lstlisting}[mathescape=true]
def $\wedge$ (A : pointed) (B : pointed) : U
 := inductive { basel
              | baser
              | proj (a : A.1) (b : B.1)
              | gluel (a : A.2) : proj(a,B.2) $\equiv$ basel
              | gluer (a : B.2) : proj(A.2,b) $\equiv$ baser
              }
\end{lstlisting}
\end{definition}

\begin{theorem} (Elimination).
For a family of types \( P : \text{Smash } A \, B \to \mathcal{U} \),
points \( \text{pbasel} : P(\text{basel}) \),
         \( \text{pbaser} : P(\text{baser}) \), maps
\( \text{pproj} : (x : A.1) \to (y : B.1) \to P \, (\text{proj} \, x \, y) \),
and a family of paths
\( \text{pgluel} : (a : A.1) \to \text{pproj}(a,B.2) \equiv \text{pbasel} \),
\( \text{pgluer} : (b : B.1) \to \text{pproj}(A.2,b) \equiv \text{pbaser} \),
there exists a map \( \text{Ind}_\wedge : (z : A \wedge B) \to P(z) \), such that:
\[
\begin{cases}
\text{Ind}_\wedge \, (\text{basel}) = \text{pbasel} \\
\text{Ind}_\wedge \, (\text{baser}) = \text{pbaser} \\
\text{Ind}_\wedge \, (\text{proj} \, x \, y) = \text{pproj} \, x \, y \\
\text{Ind}_\wedge \, (\text{gluel} \, a \, @ \, i) = \text{pgluel} \, a \, @ \, i \\
\text{Ind}_\wedge \, (\text{gluer} \, b \, @ \, i) = \text{pgluer} \, b \, @ \, i \\
\end{cases}
\]
\begin{lstlisting}[mathescape=true]
def Ind$_\wedge$ (A B : pointed) (P : A $\wedge$ B -> U)
    (pbasel : P basel) (pbaser : P baser)
    (pproj : (x : A.1) -> (y : B.1) -> P (proj x y))
    (pgluel : (a : A.1) -> PathP (<i> P (gluel a @ i)) (pproj a B.2) pbasel)
    (pgluer : (b : B.1) -> PathP (<i> P (gluer b @ i)) (pproj A.2 b) pbaser)
  : (z : A $\wedge$ B) -> P z
 := split { basel -> pbasel | baser -> pbaser | proj x y -> pproj x y
          | gluel a @ i -> pgluel a @ i | gluer b @ i -> pgluer b @ i }
\end{lstlisting}
\end{theorem}

\begin{theorem} (Computation).
For a family of types \( P : A \wedge B \to \mathcal{U} \),
points \( \text{pbasel} : P(\text{basel}) \),
\( \text{pbaser} : P(\text{baser}) \),
map \( \text{pproj} : (x : A.1) \to (y : B.1) \to P(\text{proj} \, x \, y) \),
and families of paths
\( \text{pgluel} : (a : A.1) \to \text{PathP} \, (<i> P(\text{gluel} \, a \, @ \, i)) \, (\text{pproj} \, a \, B.2) \, \text{pbasel} \),
\( \text{pgluer} : (b : B.1) \to \text{PathP} \, (<i> P(\text{gluer} \, b \, @ \, i)) \, (\text{pproj} \, A.2 \, b) \, \text{pbaser} \),
the map \( \text{Ind}_\wedge : (z : A \wedge B) \to P(z) \) satisfies all equations for all variants
of the predicate \( P \):
\[
\begin{cases}
\text{Ind}_\wedge \, (\text{basel}) \equiv \text{pbasel} \\
\text{Ind}_\wedge \, (\text{baser}) \equiv \text{pbaser} \\
\text{Ind}_\wedge \, (\text{proj} \, x \, y) \equiv \text{pproj} \, x \, y \\
\text{Ind}_\wedge \, (\text{gluel} \, a \, @ \, i) \equiv \text{pgluel} \, a \, @ \, i \\
\text{Ind}_\wedge \, (\text{gluer} \, b \, @ \, i) \equiv \text{pgluer} \, b \, @ \, i
\end{cases}
\]
\end{theorem}

\begin{theorem} (Uniqueness).
For a family of types \( P : A \wedge B \to \mathcal{U} \),
and maps \( h_1, h_2 : (z : A \wedge B) \to P(z) \),
if there exist paths
\( e_{\text{basel}} : h_1(\text{basel}) \equiv h_2(\text{basel}) \),
\( e_{\text{baser}} : h_1(\text{baser}) \equiv h_2(\text{baser}) \),
\( e_{\text{proj}} : (x : A.1) \to (y : B.1) \to h_1(\text{proj} \, x \, y) \equiv h_2(\text{proj} \, x \, y) \),
\( e_{\text{gluel}} : (a : A.1) \to \text{PathP} \, (<i> h_1(\text{gluel} \, a \, @ \, i) \equiv h_2(\text{gluel} \, a \, @ \, i)) \, (e_{\text{proj}} \, a \, B.2) \, e_{\text{basel}} \),
\( e_{\text{gluer}} : (b : B.1) \to \text{PathP} \, (<i> h_1(\text{gluer} \, b \, @ \, i) \equiv h_2(\text{gluer} \, b \, @ \, i)) \, (e_{\text{proj}} \, A.2 \, b) \, e_{\text{baser}} \),
then \( h_1 \equiv h_2 \), i.e., there exists a path \( (z : A \wedge B) \to h_1(z) \equiv h_2(z) \).
\end{theorem}

\newpage
\subsection{Join}
The join of two types \( A \) and \( B \), denoted \( A \Join B \),
is a higher inductive type that constructs a type by
joining each point of \( A \) to each point of \( B \) via a path.
Topologically, it corresponds to the join of spaces, forming
a space that interpolates between \( A \) and \( B \).

\begin{definition} (Formation).
For types \( A, B : \mathcal{U} \), the join \( A * B : \mathcal{U} \).
\end{definition}

\begin{definition} (Constructors).
The join is generated by the following higher inductive compositional structure:
\[
A \Join B :=
\begin{cases}
\text{joinl} : A \to A \Join B \\
\text{joinr} : B \to A \Join B \\
\text{join} (a : A) (b : B) : \text{joinl}(a) \equiv \text{joinr}(b)
\end{cases}
\]
\begin{lstlisting}[mathescape=true]
def $\Join$ (A : U) (B : U) : U
 := inductive { joinl (a: A)
              | joinr (b: B)
              | join (a: A) (b: B) : joinl(a) $\equiv$ joinr(b)
              }
\end{lstlisting}
\end{definition}

\begin{theorem} (Elimination).
For a type \( C : \mathcal{U} \), maps \( f : A \to C \), \( g : B \to C \),
and a family of paths \( h : (a : A) \to (b : B) \to f(a) \equiv g(b) \),
there exists a map \( \text{Ind}_\Join : A \Join B \to C \), such that:
\[
\begin{cases}
\text{Ind}_\Join(\text{joinl}(a)) = f(a) \\
\text{Ind}_\Join(\text{joinr}(b)) = g(b) \\
\text{Ind}_\Join(\text{join}(a,b,i)) = h(a,b,i) \\
\end{cases}
\]
\begin{lstlisting}[mathescape=true]
def Ind$_\Join$ (A B C : U) (f : A -> C) (g : B -> C)
    (h : (a : A) -> (b : B) -> Path C (f a) (g b))
  : A $\Join$ B -> C
 := split { joinl a -> f a
          | joinr b -> g b
          | join a b @ i -> h a b @ i
          }
\end{lstlisting}
\end{theorem}

\begin{theorem} (Computation).
For all \( z : A \Join B \), and predicate \( P \), the rules of \( \text{Ind}_\Join \) hold for all parameters of the predicate \( P \).
\end{theorem}

\begin{theorem} (Uniqueness).
Any two maps \( h_1, h_2 : A \Join B \to C \) are homotopic
if they agree on \( \text{joinl} \), \( \text{joinr} \), and \( \text{join} \).
\end{theorem}

\newpage
\subsection{Colimit}
Colimits construct the limit of a sequence of types, connected by maps,
e.g., propositional truncations.

\begin{definition} (Colimit)
For a sequence of types \( A : \text{nat} \to \mathcal{U} \) and
maps \( f : (n : \mathbb{N}) \to A n \to A(\text{succ}(n)) \),
the colimit type \( \text{colimit}(A,f) : \mathcal{U} \).
\[
\text{colim} :=
\begin{cases}
\text{ix} : (n : \text{nat}) \to A n \to \text{colimit}(A,f) \\
\text{gx} : (n : \text{nat}) \to (a : A(n)) \to \text{ix} (\text{succ}(n),f(n,a)) \equiv \text{ix}(n,a)
\end{cases}
\]
\begin{lstlisting}
def colimit (A : nat -> U) (f : (n : nat) -> A n -> A (succ n)) : U
 := inductive { ix (n : nat) (x: A n)
              | gx (n : nat) (a: A n)
                <i> [ (i=0) -> ix (succ n) (f n a),
                      (i=1) -> ix n a ]
              }
\end{lstlisting}
\end{definition}

\begin{theorem} (Elimination \( \text{colimit} \))
For a type \( P : \text{colimit } A f \to \mathcal{U} \),
with \( p : (n : \text{nat}) \to (x : A n) \to P(\text{ix}(n, x)) \)
and \( q : (n : \text{nat}) \to (a : A n) \to \text{PathP} (\langle i \rangle P(\text{gx}(n, a) @ i)) (p (\text{succ } n) (f n a)) (p n a) \),
there exists \( i : \Pi_{x:\text{colimit } A f} P(x) \), such that \( i(\text{ix}(n, x)) = p n x \).
\end{theorem}

\newpage
\subsection{Coequalizers}
\subsubsection*{Coequalizer}
The coequalizer of two maps \( f, g : A \to B \) is a higher inductive
type (HIT) that constructs a type consisting of elements in \( B \),
where \( f \) and \( g \) agree, along with paths ensuring this equality.
It is a fundamental construction in homotopy theory,
capturing the subspace of \( B \) where \( f(a) = g(a) \) for \( a : A \).

\begin{definition} (Formation).
For types \( A, B : \mathcal{U} \) and maps \( f, g : A \to B \),
the coequalizer \( \text{coeq } A B f g : \mathcal{U} \).
\end{definition}

\begin{definition} (Constructors).
The coequalizer is generated by the following higher inductive compositional structure:
\[
\text{Coeq} :=
\begin{cases}
\text{inC} : B \to \text{Coeq}(A,B,f,g) \\
\text{glueC} : (a : A) \to \text{inC}(f(a)) \equiv \text{inC}(g(a))
\end{cases}
\]
\begin{lstlisting}[mathescape=true]
def Coeq (A B: U) (f g: A -> B) : U
 := inductive { inC (b: B)
              | glueC (a: A) : inC (f a) $\equiv$ inC (g a)
              }
\end{lstlisting}
\end{definition}

\begin{theorem} (Elimination).
For a type \( C : \mathcal{U} \), map \( h : B \to C \), and a family of paths \( y : (x : A) \to \text{Path}_C (h (f x), h (g x)) \), there exists a map \( \text{coequRec} : \text{coeq } A B f g \to C \), such that:
\[
\begin{cases}
\text{coequRec}(\text{inC}(x)) = h(x) \\
\text{coequRec}(\text{glueC}(x,i)) = y(x,i)
\end{cases}
\]
\begin{lstlisting}
def coequRec (A B C : U) (f g : A -> B) (h: B -> C)
    (y: (x : A) -> Path C (h (f x)) (h (g x)))
  : (z : coeq A B f g) -> C
 := split { inC x -> h x | glueC x @ i -> y x @ i }
\end{lstlisting}
\end{theorem}

\begin{theorem} (Computation).
For \( z : \text{coeq } A B f g \),
\[
\begin{cases}
\text{coequRec}(\text{inC } x) \equiv h(x) \\
\text{coequRec}(\text{glueC } x \, @ \, i) \equiv y(x) \, @ \, i
\end{cases}
\]
\end{theorem}

\begin{theorem} (Uniqueness).
Any two maps \( h_1, h_2 : \text{coeq } A B f g \to C \) are homotopic
if they agree on \( \text{inC} \) and \( \text{glueC} \), i.e.,
if \( h_1(\text{inC } x) = h_2(\text{inC } x) \) for
all \( x : B \) and \( h_1(\text{glueC } a) = h_2(\text{glueC } a) \) for all \( a : A \).
\end{theorem}

\begin{example} (Coequalizer as Subspace)
The coequalizer \( \text{coeq } A B f g \) represents the subspace
of \( B \), where \( f(a) = g(a) \). For example, if \( A = B = \mathbb{R} \)
and \( f(x) = x^2 \), \( g(x) = x \), the coequalizer captures the points
where \( x^2 = x \), i.e., \( \{0, 1\} \).
\end{example}

\newpage
\subsubsection*{Path Coequalizer}
The path coequalizer is a higher inductive type that generalizes
the coequalizer to handle pairs of paths in \( B \). Given
a map \( p : A \to (b_1, b_2 : B) \times (\text{Path}_B (b_1, b_2)) \times (\text{Path}_B (b_1, b_2)) \),
it constructs a type where elements of \( A \) generate pairs
of paths between points in \( B \), with paths connecting
the endpoints of these paths.

\begin{definition} (Formation).
For types \( A, B : \mathcal{U} \) and a map \( p : A \to (b_1, b_2 : B) \times (b_1 \equiv b_2) \times (b_1 \equiv b_2) \),
there exists a path coequalizer \( \text{Coeq}_\equiv(A,B,p) : \mathcal{U} \).
\end{definition}

\begin{definition} (Constructors).
The path coequalizer is generated by the following higher inductive compositional structure:
\[
\text{Coequ}_\equiv :=
\begin{cases}
\text{inP} : B \to \text{Coeq}_\equiv(A,B,p) \\
\text{glueP} : (a : A) \to \text{inP}(p(a).2.2.1 @ 0) \equiv \text{inP}(p(a).2.2.2 @ 1)
\end{cases}
\]
\begin{lstlisting}[mathescape=true]
data Coeq-$\equiv$ (A B: U) (p : A -> $\Sigma$ (b1 b2: B), b1 $\equiv$ b2 $\times$ b1 $\equiv$ b2)
   = inP (b: B)
   | glueP (a:A) <i> [(i=0) -> inP ((p a).2.2.1 @ 0),
                      (i=1) -> inP ((p a).2.2.2 @ 1) ]
\end{lstlisting}
\end{definition}

\begin{theorem} (Elimination).
For a type \( C : \mathcal{U} \), map \( h : B \to C \),
and a family of paths \( y : (a : A) \to h(p(a).2.2.1  @ 0) \equiv h(p(a).2.2.2 @ 1) \),
there exists a map \( \text{Ind-Coequ}_\equiv : \text{Coeq}_\equiv(A,B,p) \to C \), such that:
\[
\begin{cases}
\text{coequPRec}(\text{inP}(b)) = h(b) \\
\text{coequPRec}(\text{glueP}(a,i)) = y(a,i)
\end{cases}
\]
\begin{lstlisting}[mathescape=true]
def Ind-Coequ$_\equiv$ (A B C : U)
    (p : A -> $\Sigma$ (b1 b2: B) (x: Path B b1 b2), Path B b1 b2)
    (h: B -> C) (y: (a : A) -> Path C (h (((p a).2.2.1) @ 0)) (h (((p a).2.2.2) @ 1)))
  : (z : coeqP A B p) -> C
 := split { inP b -> h b | glueP a @ i -> y a @ i }
\end{lstlisting}
\end{theorem}

\begin{theorem} (Computation).
For \( z : \text{coeqP } A B p \),
\[
\begin{cases}
\text{coequPRec}(\text{inP } b) \equiv h(b) \\
\text{coequPRec}(\text{glueP } a \, @ \, i) \equiv y(a) \, @ \, i
\end{cases}
\]
\end{theorem}

\begin{theorem} (Uniqueness).
Any two maps \( h_1, h_2 : \text{coeqP } A B p \to C \) are homotopic
if they agree on \( \text{inP} \) and \( \text{glueP} \), i.e.,
if \( h_1(\text{inP } b) = h_2(\text{inP } b) \) for all \( b : B \)
and \( h_1(\text{glueP } a) = h_2(\text{glueP } a) \) for all \( a : A \).
\end{theorem}

\newpage
\subsection{K(G,n)}
Eilenberg-MacLane spaces \( K(G,n) \) have a single non-trivial
homotopy group \( \pi_n(K(G,n)) = G \). They are defined using truncations and suspensions.

\begin{definition} (K(G,n))
For an abelian group \( G : \text{abgroup} \), the type \( KGn(G) : \text{nat} \to \mathcal{U} \).
\[
K(G,n) :=
\begin{cases}
n = 0 \leadsto \text{discreteTopology}(G) \\
n \geq 1 \leadsto \|\Sigma^{n-1}(K1'(G.1,G.2.1))\|_{n}
\end{cases}
\]
\begin{lstlisting}[mathescape=true]
def KGn (G: abgroup) : $\mathbf{N}$ -> U
 := split { zero -> discreteTopology G
          | succ n -> nTrunc ($\Sigma$ (K1' (G.1,G.2.1)) n) (succ n)
          }
\end{lstlisting}
\end{definition}

\begin{theorem} (Elimination \( KGn \))
For \( n \geq 1 \), a type \( B : \mathcal{U} \) with \( \text{isNGroupoid}(B, \text{succ } n) \),
and a map \( f : \text{suspension}(K1' G) \to B \), there exists
\( \text{rec}_{KGn} : KGn G (\text{succ } n) \to B \),
defined via \( \text{nTruncRec} \).
\end{theorem}

\newpage
\subsection{Localization}
Localization constructs an \( F \)-local type from a type \( X \),
with respect to a family of maps \( F_A : S(a) \to T(a) \).

\begin{definition} (Localization Modality)
For a family of maps \( F_A : S(a) \to T(a) \),
the \( F \)-localization \( L^{AST}_F(X) : \mathcal{U} \).
\[
L^A_F(X) :=
\begin{cases}
\text{center} : X \to L_{F_A}(X) \\
\text{ext} (a : A) \to (S(a) \to L_{F_A}(X)) : T(a) \to L_{F_A}(X) \\
\text{isExt} (a : A) (f : S(a) \to L_{F_A}(X)) \to (s : S(a)) : \text{ext}(a,f,F(a,s)) \equiv f(s) \\
\text{extEq} (a : A) (g, h : T(a) \to L_{F_A}(X)) \\
\indent (p : (s : S(a)) \to g(F(a,s)) \equiv h(F(a,s))) \\
\indent (t : T(a)) : g(t) \equiv h(t) \\
\text{isExtEq} : (a : A) (g, h : T(a) \to L_{F_A}(X)) \\
\indent  (p : (s : S(a)) \to g(F(a,s)) \equiv h(F(a,s))) \\
\indent (s : S(a)) : \text{extEq}(a,g,h,p,F(a,s) \equiv p(s)
\end{cases}
\]
\begin{lstlisting}
data Localize (A X: U) (S T: A -> U) (F : (x:A) -> S x -> T x)
   = center (x: X)
   | ext (a: A) (f: S a -> Localize A X S T F) (t: T a)
   | isExt (a: A) (f: S a -> Localize A X S T F) (s: S a) <i>
     [ (i=0) -> ext a f (F a s) , (i=1) -> f s ]
   | extEq (a: A) (g h: T a -> Localize A X S T F)
     (p: (s : S a) -> Path (Localize A X S T F) (g (F a s)) (h (F a s)))
     (t : T a) <i> [ (i=0) -> g t , (i=1) -> h t ]
   | isExtEq (a: A) (g h : T a -> Localize A X S T F)
     (p: (s : S a) -> Path (T a -> Localize A X S T F) (g (F a s)) (h (F a s)))
     (s : S a) <i> [ (i=0) -> extEq a g h p (F a s) , (i=1) -> p s ]
\end{lstlisting}
\end{definition}

\begin{theorem} (Localization Induction)
For any \( P : \Pi_{X:U} L_{F_A}(X) \to U \) with \( \{n, r, s\} \), satisfying coherence conditions,
there exists \( i : \Pi_{x:L_{F_A}(X)} P(x) \), such that \( i \cdot \text{center}_X = n \).
\end{theorem}

\subsection*{Conclusion}
HITs directly encode CW-complexes in HoTT, bridging topology and type theory.
They enable the analysis and manipulation of homotopical types.

\newpage
\bibliographystyle{plain}
\begin{thebibliography}{9}

\bibitem{univalent} The Univalent Foundations Program, \emph{Homotopy Type Theory: Univalent Foundations of Mathematics}, IAS, 2013.
\bibitem{cubicaltt} C. Cohen, T. Coquand, S. Huber, A. Mörtberg, \emph{Cubical Type Theory}, Journal of Automated Reasoning, 2018.
\bibitem{agda-cubical} A. Mörtberg et al., \emph{Agda Cubical Library}, \url{https://github.com/agda/cubical}, 2023.
\bibitem{hott-book} M. Shulman, \emph{Higher Inductive Types in HoTT}, \url{https://arxiv.org/abs/1705.07088}, 2017.
\bibitem{localization} J. D. Christensen, M. Opie, E. Rijke, L. Scoccola, \emph{Localization in Homotopy Type Theory}, \url{https://arxiv.org/pdf/1807.04155.pdf}, 2018.
\bibitem{modalities} E. Rijke, M. Shulman, B. Spitters, \emph{Modalities in Homotopy Type Theory}, \url{https://arxiv.org/pdf/1706.07526v6.pdf}, 2017.
\bibitem{synthetic-spectra} M. Riley, E. Finster, D. R. Licata, \emph{Synthetic Spectra via a Monadic and Comonadic Modality}, \url{https://arxiv.org/pdf/2102.04099.pdf}, 2021.

\end{thebibliography}

\end{document}