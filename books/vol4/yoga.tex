\documentclass{article}
\usepackage[utf8]{inputenc}
\usepackage[ukrainian]{babel}
\usepackage{amsmath, amssymb, amsthm}

\theoremstyle{plain}
\newtheorem{theorem}{Теорема}[section]
\newtheorem{lemma}[theorem]{Лема}
\newtheorem{example}{Example}
\newtheorem{corollary}[theorem]{Наслідок}
\theoremstyle{definition}
\newtheorem{definition}{Визначення}[section]
\newtheorem{remark}{Зауваження}[section]

\begin{document}

\title{Issue XXXV: Grothendieck Yogas}
\author{Namdak Tonpa}
\date{\today}

\maketitle

\begin{abstract}
Ця стаття присвячена огляду функторіальних йог Гротендіка,
зокрема шести функторів, когезивних топосів та їхньої ролі
в теорії похідних категорій. Ми розглядаємо основні концепції,
такі як когомології та їх узагальнення, а також зв’язок із сучасною
алгебраїчною геометрією та мотивною гомотопічною теорією.
Стаття базується на сучасних джерелах, зокрема на лекціях
Мартіна Галлауера про шестифункторний формалізм.
\end{abstract}

\ifincludeTOC
  \tableofcontents
\fi

\section{Grothendieck Yogas}

Шестифункторний формалізм Гротендіка є одним із ключових інструментів сучасної алгебраїчної геометрії, що дозволяє узагальнити класичні когомологічні теорії та застосовувати їх у різних контекстах, від топології до мотивної гомотопічної теорії. Цей формалізм, розроблений Александром Гротендіком, включає шість основних операцій (функторів), які діють на категорії пучків або їх узагальнень, забезпечуючи багатий набір інструментів для вивчення геометричних об’єктів.

У цій статті ми зосередимося на трьох основних аспектах:
\begin{enumerate}
    \item \textbf{Чому шестифункторний формалізм важливий?} Він узагальнює когомології, дозволяючи працювати з відносною точкою зору та застосовувати їх у складних геометричних контекстах.
    \item \textbf{Що таке шестифункторний формалізм?} Ми розглянемо основні функтори та їх властивості, такі як локалізація, дуальність та відносна чистота.
    \item \textbf{Як його конструюють?} Ми обговоримо методи побудови формалізму, зокрема через системи коефіцієнтів та когезивні топоси.
\end{enumerate}

\subsection{Узагальнення когомологій}

Когомології є фундаментальним інструментом у топології та алгебраїчній геометрії. Наприклад, для топологічного простору \( X \) числа Бетті \( b_n(X) \) вимірюють кількість \( n \)-вимірних дірок, але гомології \( \mathrm{H}_n(X) \) є багатшим інваріантом, оскільки містять інформацію про цикли та границі.

\begin{example}
Для різноманіття \( X \) над скінченним полем \( k = \mathbb{F}_q \), \(\zeta\)-функція \( \zeta_X(T) \) кодує кількість раціональних точок. Гротендік показав, що властивості цієї функції випливають із \(\ell\)-адичних когомологій \( \mathrm{H}^*(X_{\bar{k}}; \mathbb{Q}_\ell) \), які, у свою чергу, походять із похідної категорії \( \mathrm{D}_c^b(X_{\bar{k}}; \mathbb{Q}_\ell) \).
\end{example}

Шестифункторний формалізм узагальнює ці ідеї, дозволяючи працювати з категоріями рівня, які керують поведінкою когомологій.

\subsection{Відносна точка зору}

Гротендік наголошував на важливості відносної точки зору, де замість окремих об’єктів (наприклад, схем) розглядаються морфізми між ними. Це дозволяє вивчати когомології не ізольовано, а разом із дією морфізмів:

\[
f^*: \mathrm{H}^*(Y) \to \mathrm{H}^*(X).
\]

\begin{remark}
Навіть для однієї схеми \( X \) часто необхідно розглядати когомології пов’язаних об’єктів, наприклад, при індукції за розмірністю або розбитті на простіші частини.
\end{remark}

\subsection{Основні функтори}

Шестифункторний формалізм складається з шести основних функторів, які діють на категорії пучків (або їх узагальнень, таких як похідні категорії):

\begin{itemize}
    \item \( f^* \): обернений образ (pull-back),
    \item \( f_* \): прямий образ (push-forward),
    \item \( f_! \): прямий образ із компактною підтримкою,
    \item \( f^! \): винятковий обернений образ,
    \item \( \otimes \): тензорний добуток,
    \item \( \underline{\mathrm{Hom}} \): внутрішній гом.
\end{itemize}

Ці функтори пов’язані між собою ад’юнкціями:

\[
f^* \dashv f_*, \quad f_! \dashv f^!.
\]

\begin{definition}
Для простору \( X \) (наприклад, топологічного простору або схеми) категорія \( C(X) \) є замкненою тензорною триангулятивною категорією, оснащеною операціями \( \otimes \) та \( \underline{\mathrm{Hom}} \). Для морфізму \( f: X \to Y \) визначено ад’юнкції \( f^* \dashv f_* \), \( f_! \dashv f^! \), а також природну трансформацію \( f_! \to f_* \).
\end{definition}

\subsection{Когезивні топоси}

Когезивні топоси є природним контекстом для шестифункторного формалізму, оскільки вони забезпечують категоріальну структуру, яка підтримує геометричні та когомологічні операції. Топос \( \mathcal{E} \) називається когезивним, якщо він має набір ад’юнктних функторів, що моделюють геометричні трансформації.

\begin{example}
Категорія пучків \( \mathrm{Sh}(X) \) на топологічному просторі \( X \) є когезивним топосом, де \( f^* \) та \( f_* \) відповідають оберненим і прямим образам.
\end{example}

У контексті алгебраїчної геометрії когезивні топоси часто виникають як категорії пучків на схемах або стеках, оснащені додатковими структурами, такими як стабільні \(\infty\)-категорії.

\subsection{Роль абелевих категорій}

Абелеві категорії відіграють фундаментальну роль у шестифункторному формалізмі,
оскільки вони є основою для побудови похідних категорій, які використовуються
для опису пучків та їх когомологій. Абелева категорія — це категорія,
в якій морфізми мають ядра та кокернали, а кожна монада та епіморфізм є нормальними.
Типовим прикладом є категорія абелевих пучків \( \mathrm{Ab}(X) \) на
топологічному просторі \( X \) або категорія когерентних пучків на схемі.

У шестифункторному формалізмі абелеві категорії, такі як \( \mathrm{Sh}(X) \),
слугують вихідним пунктом для визначення функторів \( f^* \) та \( f_* \).
Наприклад, для неперервного відображення \( f: X \to Y \), функтор прямого
образу \( f_* \mathscr{F} \) визначається через секції \( \Gamma(f^{-1}(U), \mathscr{F}) \),
де \( \mathscr{F} \in \mathrm{Sh}(X) \), а \( f^* \) є його лівою ад’юнктою. Однак, щоб врахувати гомотопічні властивості та виняткову функторіальність (\( f_! \), \( f^! \)), необхідно перейти до похідних категорій \( \mathrm{D}(\mathrm{Sh}(X)) \), які будуються з абелевих категорій шляхом локалізації за квазіізоморфізмами.

\begin{remark}
Абелеві категорії забезпечують строгу алгебраїчну структуру, але їх обмеження (наприклад, відсутність природної триангулятивної структури) роблять похідні категорії більш придатними для шестифункторного формалізму, особливо в контексті \(\ell\)-адичних або мотивних пучків.
\end{remark}

\subsection{Похідні категорії}

Похідна категорія \( \mathrm{D}(\mathrm{Sh}(X)) \) пучків на просторі \( X \) є природним узагальненням категорії пучків, що враховує гомотопічні властивості. Вона дозволяє працювати з похідними функторами, такими як:

\[
\mathrm{R}^n f_*(\mathscr{F}) \simeq \mathrm{H}^n(X; \mathscr{F}), \quad \mathrm{R}^n f_!(\mathscr{F}) \simeq \mathrm{H}_c^n(X; \mathscr{F}).
\]

\begin{example}
Для \(\ell\)-адичних пучків на схемі \( X \) похідна категорія \( \mathrm{D}_c^b(X; \mathbb{Q}_\ell) \) є основою для \(\ell\)-адичних когомологій, які використовувалися для доведення гіпотез Вейля.
\end{example}

\subsection{Конструкція шестифункторного формалізму}

Конструкція шестифункторного формалізму є складним завданням, яке часто потребує значних зусиль. Одним із ключових викликів є побудова виняткової функторіальності (\( f_! \), \( f^! \)).

\begin{remark}
За Делінем, для морфізму \( f: X \to Y \) можна використати компактифікацію Нагати, щоб розкласти \( f \) на відкрите вкладення \( j \) та власний морфізм \( p \):

\[
f = p \circ j, \quad f_! := p_* j_\sharp.
\]

Ця конструкція вимагає доведення незалежності від вибору факторизації та існування правої ад’юнкти \( f^! \).
\end{remark}

\subsection{Застосування в мотивній гомотопічній теорії}

Мотивна гомотопічна теорія, розроблена Морелем і Воєводським, використовує шестифункторний формалізм для узагальнення класичних гомотопічних теорій на алгебраїчні схеми. Категорія \( \mathrm{SH}(X) \) стабільних мотивних гомотопічних пучків є прикладом системи коефіцієнтів, яка підтримує всі шість функторів.

\begin{example}
Для поля \( k \) категорія \( \mathrm{DM}(k; \mathbb{Q}) \) геометричних мотивів Воєводського еквівалентна компактній частині \( \mathrm{DM}_\mathrm{B}(k) \), що є основою для раціональних мотивних когомологій.
\end{example}

\subsection{Висновки}

Шестифункторний формалізм Гротендіка є потужним інструментом, який узагальнює когомології та дозволяє працювати з відносними інваріантами в алгебраїчній геометрії. Його зв’язок із когезивними топосами та похідними категоріями відкриває нові можливості для дослідження складних геометричних структур. У майбутньому цей формалізм, ймовірно, залишатиметься ключовим у розвитку мотивної гомотопічної теорії та інших областей математики.

\begin{thebibliography}{9}
\bibitem{Gallauer2021}
Gallauer, M. (2021). An introduction to six-functor formalisms. \textit{arXiv:2112.10456v1}.

\bibitem{Ayoub2007}
Ayoub, J. (2007). Les six opérations de Grothendieck et le formalisme des cycles évanescents dans le monde motivique. \textit{Astérisque}, 314-315.

\bibitem{CD2019}
Cisinski, D.-C., \& Déglise, F. (2019). Triangulated categories of mixed motives. \textit{Springer Monographs in Mathematics}.
\end{thebibliography}

\end{document}