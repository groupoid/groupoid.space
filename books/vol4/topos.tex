% copyright (c) 2018 Groupoid Infinity

\documentclass{article}
\usepackage{amsmath}
\usepackage{amssymb}
\usepackage{amsthm}
\usepackage{url}
\usepackage{tikz}
\usepackage{tikz-cd}
\usetikzlibrary{matrix}
\usetikzlibrary{babel}
\usepackage{listings}
\usepackage[utf8]{inputenc}

\theoremstyle{definition}
\newtheorem{definition}{Definition}
\newtheorem{theorem}{Theorem}

\lstset{basicstyle=\tiny,inputencoding=utf8}
\lstMakeShortInline[mathescape=true,columns=fixed]|

\def\mapright#1{\xrightarrow{{#1}}}
\def\mapleft#1{\xleftarrow{{#1}}}
\def\mapup#1{\Big\uparrow\rlap{\raise2pt{\scriptstyle{#1}}}}
\def\mapupl#1{\Big\uparrow\llap{\raise2pt{\scriptstyle{#1}}}}
\def\mapdown#1{\Big\downarrow\rlap{\raise2pt{\scriptstyle{#1}}}}
\def\mapdownl#1{\Big\downarrow\llap{\raise2pt{\scriptstyle{#1}}}}
\def\mapdiagl#1{\vcenter{\searrow}\rlap{\raise2pt{\scriptstyle{#1}}}}
\def\mapdiagr#1{\vcenter{\swarrow}\rlap{\raise2pt{\scriptstyle{#1}}}}
\def\under{\underline{\hspace{0.25cm}}}

\lstset{basicstyle=\small,inputencoding=utf8}

\begin{document}

\title{Issue XXXVI: Topos on Category of Sets}
\author{Maxim Sokhatsky}
\date{ $^1$ National Technical University of Ukraine \\
       \small Igor Sikorsky Kyiv Polytechnical Institute\\
       \today }

\maketitle

\begin{abstract}

The purpose of this work is to clarify all topos definitions using type theory.
Not much efforts was done to give all the examples, but one example, a topos
on category of sets, is constructively presented at the finale.

As this cricial example definition is used in presheaf definition,
the construction of category of sets is a mandatory excercise for any topos library.
We propose here cubicaltt\footnote{Cubical Type Theory, http://github.com/mortberg/cubicaltt}
version of elementary topos on category of sets for demonstration of categorical
semantics (from logic perspective) of the fundamental notion of set theory in mathematics.

Other disputed foundations for set theory could be taken as:
ZFC, NBG, ETCS. We will disctinct syntetically: i) category theory;
ii) set theory in univalent foundations; iii) topos theory, grothendieck topos,
elementary topos. For formulation of definitions and theorems only Martin-Löf
Type Theory is requested. The proofs involve cubical type checker primitives.

{\bf Keywords}: Homotopy Type Theory, Topos Theory
\end{abstract}

\ifincludeTOC
  \tableofcontents
\fi

\newpage

\section{Topos Theory}

One can admit two topos theory lineages. One lineage takes its roots from published
by Jean Leray in 1945 initial work on sheaves and spectral sequences. Later this lineage
was developed by Henri Paul Cartan, André Weil. The peak of lineage was settled
with works by Jean-Pierre Serre, Alexander Grothendieck, and Roger Godement.

Second remarkable lineage take its root from William Lawvere and
Myles Tierney. The main contribution is the reformulation of Grothendieck topology
by using subobject classifier.

\subsection{Set Theory}

Here is given the $\infty$-groupoid model of sets.

\begin{definition} (Mere proposition, $\mathrm{PROP}$).
A type P is a mere proposition if for all $x,y: P$ we have $x=y$:
$$
    \mathrm{isProp}(P) = \prod_{x,y:P}(x=y).
$$
\end{definition}

\begin{definition} (0-type).
A type A is a 0-type is for all $x,y: A$ and $p,q: x =_A y$ we have $p = q$.
\end{definition}

\begin{definition} (1-type).
A type A is a 1-type if for all $x,y: A$ and $p,q: x =_A y$ and $r,s:p =_{=_A} q$, we have $r = s$.
\end{definition}

\begin{definition} (A set of elements, $\mathrm{SET}$).
A type A is a $\mathrm{SET}$ if for all $x,y: A$ and $p,q: x = y$, we have $p = q$:
$$
    \mathrm{isSet}(A) = \prod_{x,y:A}\prod_{p,q:x=y}(p=q).
$$
\end{definition}

\begin{definition}
\begin{lstlisting}
data N = Z  | S (n: N)

n_grpd (A: U) (n: N): U = (a b: A) -> rec A a b n where
  rec (A: U) (a b: A) : (k: N) -> U
    = split { Z -> Path A a b ; S n -> n_grpd (Path A a b) n }

isContr (A: U): U = (x: A) * ((y: A) -> Path A x y)
isProp  (A: U): U = n_grpd A Z
isSet   (A: U): U = n_grpd A (S Z)
PROP  : U = (X:U) * isProp X
SET   : U = (X:U) * isSet X
\end{lstlisting}
\end{definition}

\begin{definition} ($\Pi$-Contractability).
If fiber is set thene path space between any sections is contractible.
\begin{lstlisting}
setPi (A: U) (B: A -> U) (h: (x: A) -> isSet (B x)) (f g: Pi A B)
      (p q: Path (Pi A B) f g)
    : Path (Path (Pi A B) f g) p q
\end{lstlisting}
\end{definition}

\begin{definition} ($\Sigma$-Contractability).
If fiber is set then $\Sigma$ is set.
\begin{lstlisting}
setSig (A:U) (B: A -> U) (base: isSet A)
       (fiber: (x:A) -> isSet (B x)) : isSet (Sigma A B)
\end{lstlisting}
\end{definition}

\begin{definition} (Unit type, $1$).
The unit $1$ is a type with one element.
\begin{lstlisting}
data unit = tt
unitRec (C: U) (x: C): unit -> C = split tt -> x
unitInd (C: unit -> U) (x: C tt): (z:unit) -> C z
  = split tt -> x
\end{lstlisting}
\end{definition}

\begin{theorem} (Category of Sets, $\mathbf{Set}$).
Sets forms a Category.
All compositional theorems proved by using reflection rule of internal language.
The proof that $\mathrm{Hom}$ forms a set is taken through $\Pi$-contractability.
\begin{lstlisting}
Set: precategory = ((Ob,Hom),id,c,HomSet,L,R,Q) where
  Ob: U = SET
  Hom (A B: Ob): U = A.1 -> B.1
  id (A: Ob): Hom A A = idfun A.1
  c (A B C: Ob) (f: Hom A B) (g: Hom B C): Hom A C
    = o A.1 B.1 C.1 g f
  HomSet (A B: Ob): isSet (Hom A B) = setFun A.1 B.1 B.2
  L (A B:Ob) (f:Hom A B): Path (Hom A B)(c A A B (id A)f)f
    = refl (Hom A B) f
  R (A B:Ob) (f:Hom A B): Path (Hom A B)(c A B B f(id B))f
    = refl (Hom A B) f
  Q (A B C D: Ob) (f:Hom A B) (g:Hom B C) (h:Hom C D)
    : Path (Hom A D) (c A C D (c A B C f g) h)
                     (c A B D f (c B C D g h))
    = refl (Hom A D) (c A B D f (c B C D g h))
\end{lstlisting}
\end{theorem}

\subsection{Topological Structure}

Topos theory extends category theory with notion of topological
structure but reformulated in a categorical way
as a category of sheaves on a site or as one that has cartesian closure
and subobject classifier. We give here two definitions.

\begin{definition} (Topology). The topological structure on $\mathrm{A}$
(or topology) is a subset $S \in \mathrm{A}$ with following properties:
i) any finite union of subsets of $\mathrm{S}$ is belong to $\mathrm{S}$;
ii) any finite intersection of subsets of $\mathrm{S}$ is belong to $\mathrm{S}$.
Subets of $\mathrm{S}$ are called open sets of family $\mathrm{S}$.
\begin{lstlisting}
Structure topology (A : Type) := {
          open :> (A -> Prop) -> Prop;
          empty_open: open (empty _);
          full_open:  open (full _);
          inter_open: forall u,
                      open u -> forall v, open v
                             -> open (inter A u v) ;
          union_open: forall s, (subset _ s open)
                             -> open (union A s) }.
\end{lstlisting}
For fully functional general topology theorems and Zorn lemma you can refer to
the Coq library \footnote{https://github.com/verimath/topology}{topology} by Daniel Schepler.
\end{definition}

\subsection{Grothendieck Topos}

Grothendieck Topology is a calculus of coverings which generalizes the algebra
of open covers of a topological space, and can exist on much more general categories.
There are three variants of Grothendieck topology definition:
i) sieves; ii) coverage; iii) covering families.
A category have one of these three is called a Grothendieck site.

{\bf Examples}: Zariski, flat, étale, Nisnevich topologies.

A sheaf is a presheaf (functor from opposite category to category of sets) which
satisties patching conditions arising from Grothendieck topology, and applying
the associated sheaf functor to preashef forces compliance with these conditions.

The notion of Grothendieck topos is a geometric flavour of topos theory,
where topos is defined as category of sheaves on a Grothendieck site with
geometric moriphisms as adjoint pairs of functors between topoi, that
satisfy exactness properties. \cite{Jardine15}

As this flavour of topos theory uses category of sets as a prerequisite,
the formal construction of set topos is cricual in doing sheaf topos theory.

\begin{definition} (Sieves).
Sieves are a family of subfunctors
$$\def\mapright#1{\xrightarrow{{#1}}}
  \def\mapdown#1{\Big\downarrow\rlap{\raise2pt{\scriptstyle{#1}}}}
  \def\mapdiagl#1{\vcenter{\searrow}\rlap{\raise2pt{\scriptstyle{#1}}}}
  \def\mapdiagr#1{\vcenter{\swarrow}\rlap{\raise2pt{\scriptstyle{#1}}}}
  R \subset Hom_C(\under,U), U \in \mathrm{C},
$$
such that following axioms hold: i) (base change) If $R \subset Hom_C(\under,U)$ is covering
and $\phi : V \rightarrow U$ is a morphism of $\mathrm{C}$, then the subfuntor
$$
   \phi^{-1}(R) = \{ \gamma : W \rightarrow V \| \phi \cdot \gamma \in R \}
$$
is covering for $V$; ii) (local character) Suppose that $R,R' \subset Hom_C(\under,U)$ are
subfunctors and R is covering. If $\phi^{-1}(R')$ is covering for
all $\phi : V \rightarrow U$ in $R$, then $R'$ is covering; iii)
$Hom_C(\under,U)$ is covering for all $U \in \mathrm{C}$.
\end{definition}

\newpage
\begin{definition} (Coverage).
A coverage is a function assigning
to each $\mathrm{Ob}_C$ the family of morphisms $\{f_i : U_i \rightarrow U \}_{i\in I}$ called
covering families, such that for any $g: V \rightarrow U$ exist
a covering family $\{h:V_j \rightarrow V\}_{j \in J}$ such that
each composite $h_j \circ g$ factors some $f_i$:
\begin{tikzcd}
 & V_j \arrow[r, "k"] \arrow[d, "h"]
 & U_i \arrow[d, "f_i"] \\
 & V \arrow[r, "g"] & U
\end{tikzcd}
\begin{lstlisting}
Co (C: precategory) (cod: carrier C) : U
  = (dom: carrier C)
  * (hom C dom cod)

Delta (C: precategory) (d: carrier C) : U
  = (index: U)
  * (index -> Co C d)

Coverage (C: precategory): U
  = (cod: carrier C)
  * (fam: Delta C cod)
  * (coverings: carrier C -> Delta C cod -> U)
  * (coverings cod fam)
\end{lstlisting}
\end{definition}

\begin{definition} (Grothendieck Topology).
Suppose category $\mathrm{C}$ has all pullbacks.
Since $\mathrm{C}$ is small, a pretopology on $\mathrm{C}$ consists of families of sets of
morphisms
$$
    \{ \phi_\alpha : U_\alpha \rightarrow U \}, U \in \mathrm{C},
$$
called covering families, such that following axioms hold:
i) suppose that $\phi_\alpha : U_\alpha \rightarrow U$ is a covering family
and that $\psi : V \rightarrow U$ is a morphism of $\mathrm{C}$.
Then the collection $V \times_U U_\alpha \rightarrow V$ is a cvering family for $V$.
ii) If $\{\phi_\alpha : U_\alpha \rightarrow U \}$ is covering,
and $\{\gamma_{\alpha,\beta} : W_{\alpha,\beta} \rightarrow U_\alpha \}$
is covering for all $\alpha$, then the family of composites
$$
    W_{\alpha,\beta} \mapright{\gamma_{\alpha,\beta}} U_\alpha \mapright{\phi_\alpha} U
$$
is covering; iii) The family $\{1: U \rightarrow U\}$ is covering for all $U \in \mathrm{C}$.
\end{definition}

\begin{definition} (Site).
Site is a category having either a coverage, grothendieck topology, or sieves.
\begin{lstlisting}
site (C: precategory): U
  = (C: precategory) * Coverage C
\end{lstlisting}
\end{definition}

\begin{definition} (Presheaf).
Presheaf of a category $\mathrm{C}$ is a functor from opposite category to category of sets:
$\mathrm{C}^{op} \rightarrow \mathrm{Set}$.
\begin{lstlisting}
presheaf (C: precategory): U
  = catfunctor (opCat C) Set
\end{lstlisting}
\end{definition}

\begin{definition} (Presheaf Category, {\bf PSh}).
Presheaf category {\bf PSh} for a site $\mathrm{C}$ is category
were objects are presheaves and morphisms are natural transformations
of presheaf functors.
\end{definition}

\begin{definition} (Sheaf). Sheaf is a presheaf on a site. In other words
a presheaf $F : \mathrm{C}^{op} \rightarrow \mathbf{Set}$ such that the
cannonical map of inverse limit
$$
    F(U) \rightarrow \lim^{\leftarrow}_{V \to U \in R}{F(V)}
$$
is an isomorphism for each covering sieve $R \subset Hom_C(\_,U)$.
Equivalently, all induced functions
$$
    Hom_C(Hom_C(\_,U),F) \rightarrow Hom_C(R,F)
$$
should be bejections.
\begin{lstlisting}
sheaf (C: precategory): U
  = (S: site C)
  * presheaf S.1
\end{lstlisting}
\end{definition}

\begin{definition} (Sheaf Category, {\bf Sh}).
Sheaf category {\bf Sh}
is a category where objects are sheaves and morphisms are
natural transformation of sheves. Sheaf category is a full subcategory
of category of presheaves {\bf PSh}.
\end{definition}

\begin{definition} (Grothendieck Topos).
Topos is the category of sheaves {\bf Sh}$(C,J)$ on a site $\mathrm{C}$ with topology $J$.
\end{definition}

\begin{theorem} (Giraud).
A category $\mathrm{C}$ is a Grothiendieck topos iff it has following properties:
i) has all finite limits;
ii) has small disjoint coproducts stable under pullbacks;
iii) any epimorphism is coequalizer;
iv) any equivalence relation $R \rightarrow E$ is a kernel pair and has a quotient;
v) any coequalizer $R \rightarrow E \rightarrow Q$ is stably exact;
vi) there is a set of objects that generates $\mathrm{C}$.
\end{theorem}

\begin{definition} (Geometric Morphism). Suppose that $\mathrm{C}$ and $\mathrm{D}$
are Grothendieck sites. A geometric morphism
$$
    f : {\bf Sh}(C) \rightarrow {\bf Sh}(D)
$$
consist of functors $f_* : {\bf Sh}(C) \rightarrow {\bf Sh}(D)$ and
$f^* : {\bf Sh}(D) \rightarrow {\bf Sh}(C)$ such that $f^*$ is
left adjoint to $f_*$ and $f^*$ preserves finite limits. The left adjoint $f^*$ is called
the inverse image functor, while $f_*$ is called the direct image. The inverse image functor
$f^*$ is left and right exact in the sense that it preserves all finite
colimits and limits, respectively.
\end{definition}

\begin{definition} (Cohesive Topos). A topos $E$ is a cohesive topos over a base topos $S$,
if there is a geometric morphism $(p^*,p_*): E \rightarrow S$, such that:
i) exists adjunction $p^! \vdash p_*$ and $ p^! \dashv p_*$;
ii) $p^*$ and $p^!$ are full faithful; iii) $p_!$ preserves finite products.
\end{definition}
This quadruple defines adjoint triple:
$$
\int \dashv \flat \dashv \sharp
$$


\newpage
\subsection{Elementary Topos}

Giraud theorem was a synonymical topos definition involved only topos
properties but not a site properties. That was step forward on
predicative definition. The other step was made by Lawvere and Tierney,
by removing explicit dependance on categorical model of set
theory (as category of set is used in definition of presheaf). This information
was hidden into subobject classifier which was well defined through
categorical pullback and property of being cartesian
closed (having lambda calculus as internal language).

Elementary topos doesn't involve 2-categorical modeling, so we can construct
set topos without using functors and natural transformations
(what we need in geometrical topos theory flavour). This flavour of topos
theory more suited for logic needs rather that geometry, as its set properties
are hidden under the predicative predicative pullback definition of subobject classifier
rather that functorial notation of presheaf functor. So we can simplify proofs
at the homotopy levels, not to lift everything to 2-categorical model.

\begin{definition} (Monomorphism).
An morphism $f : Y \rightarrow Z $ is a monic or mono
if for any object $X$ and every pair of parralel morphisms $g_1,g_2: X \rightarrow Y$ the
$$
    f \circ g_1 = f \circ g_2 \rightarrow g_1 = g_2.
$$
More abstractly, f is mono if for any $X$ the $\mathrm{Hom}(X,\_)$ takes it to an injective
function between hom sets $\mathrm{Hom}(X,Y) \rightarrow \mathrm{Hom}(X,Z)$.
\begin{lstlisting}
mono (P: precategory) (Y Z: carrier P) (f: hom P Y Z): U
   = (X: carrier P) (g1 g2: hom P X Y)
  -> Path (hom P X Z) (compose P X Y Z g1 f)
                      (compose P X Y Z g2 f)
  -> Path (hom P X Y) g1 g2
\end{lstlisting}
\end{definition}

\begin{definition} (Subobject Classifier\cite{Johnstone14}).
In category $\mathrm{C}$ with finite limits,
a subobject classifier is a monomorphism $true: 1 \rightarrow \Omega$ out of terminal
object $\mathrm{1}$, such that for any mono $U \rightarrow X$ there is a unique
morphism $\chi_U : X \rightarrow \Omega$ and pullback diagram:
\begin{tikzcd}
 & U \arrow[r, "k"] \arrow[d, ""]
 & 1 \arrow[d, "true"] \\
 & X \Omega \arrow[r, "\chi_U"] & \Omega
\end{tikzcd}
\begin{lstlisting}
subobjectClassifier (C: precategory): U
  = (omega: carrier C)
  * (end: terminal C)
  * (trueHom: hom C end.1 omega)
  * (chi: (V X: carrier C) (j: hom C V X) -> hom C X omega)
  * (square: (V X: carrier C) (j: hom C V X) -> mono C V X j
    -> hasPullback C (omega,(end.1,trueHom),(X,chi V X j)))
  * ((V X: carrier C) (j: hom C V X) (k: hom C X omega)
    -> mono C V X j
    -> hasPullback C (omega,(end.1,trueHom),(X,k))
    -> Path (hom C X omega) (chi V X j) k)
\end{lstlisting}
\end{definition}

\begin{theorem} (Category of Sets has Subobject Classifier).
\end{theorem}

\begin{definition} (Cartesian Closed Categories).
The category $\mathrm{C}$ is called cartesian closed if exists all:
i) terminals; ii) products; iii) exponentials. Note that this definition
lacks beta and eta rules which could be found in embedding $\mathbf{MLTT}$.
\begin{lstlisting}
isCCC (C: precategory): U
  = (Exp:   (A B: carrier C) -> carrier C)
  * (Prod:  (A B: carrier C) -> carrier C)
  * (Apply: (A B: carrier C) -> hom C (Prod (Exp A B) A) B)
  * (P1:    (A B: carrier C) -> hom C (Prod A B) A)
  * (P2:    (A B: carrier C) -> hom C (Prod A B) B)
  * (Term:  terminal C)
  * unit
\end{lstlisting}
\end{definition}

\begin{theorem} (Category of Sets is cartesian closed).
As you can see from exp and pro we internalize $\Pi$ and $\Sigma$ types as $\mathrm{SET}$ instances,
the $\mathrm{isSet}$ predicates are provided with contractability.
Exitense of terminals is proved by $\mathrm{propPi}$. The same technique you
can find in $\mathbf{MLTT}$ embedding.
\begin{lstlisting}
cartesianClosure : isCCC Set
  = (expo,prod,appli,proj1,proj2,term,tt) where
    exp (A B: SET): SET = (A.1   -> B.1, setFun A.1 B.1 B.2)
    pro (A B: SET): SET = (prod A.1 B.1, setSig A.1 (\(_ : A.1)
                          -> B.1) A.2 (\(_ : A.1) -> B.2))
    expo:  (A B: SET) -> SET = \(A B: SET) -> exp A B
    prod:  (A B: SET) -> SET = \(A B: SET) -> pro A B
    appli: (A B: SET) -> hom Set (pro (exp A B) A) B
        = \(A B: SET) -> \(x:(pro(exp A B)A).1)-> x.1 x.2
    proj1: (A B: SET) -> hom Set (pro A B) A
        = \(A B: SET) (x: (pro A B).1) -> x.1
    proj2: (A B: SET) -> hom Set (pro A B) B
        = \(A B: SET) (x: (pro A B).1) -> x.2
    unitContr (x: SET) (f: x.1 -> unit) : isContr (x.1 -> unit)
      = (f, \(z: x.1 -> unit) -> propPi x.1 (\(_:x.1)->unit)
           (\(x:x.1) -> propUnit) f z)
    term: terminal Set = ((unit,setUnit),
           \(x: SET) -> unitContr x (\(z: x.1) -> tt))
\end{lstlisting}
Note that rules of cartesian closure forms a type theoretical langage
called lambda calculus.
\end{theorem}

\begin{definition} (Elementary Topos).
Topos is a precategory which is cartesian closed and has subobject classifier.
\begin{lstlisting}
Topos (cat: precategory) : U
  = (cartesianClosure: isCCC cat)
  * subobjectClassifier cat
\end{lstlisting}
\end{definition}

\begin{theorem} (Topos Definitions).
Any Grothendieck topos is an elementary topos too.
The proof is sligthly based on results of Giraud theorem.
\end{theorem}

\begin{theorem} (Category of Sets forms a Topos).
There is a cartesian closure and subobject classifier for a categoty of sets.
\begin{lstlisting}
internal : Topos Set
         = (cartesianClosure,hasSubobject)
\end{lstlisting}
\end{theorem}

\begin{theorem} (Freyd). Main theorem of topos theory\cite{Goldblatt14}.
For any topos $C$ and any $b : \mathrm{Ob}_C$ relative category $C\downarrow b$ is also a topos.
And for any arrow $f: a \rightarrow b$ inverse image functor $f^*: C\downarrow b \rightarrow c\downarrow a$
has left adjoint $\sum_f$ and right adjoin $\prod_f$.
\end{theorem}

\section*{Conclusion}

We gave here constructive definition of topology as finite unions and
intersections of open subsets. Then make this definition categorically
compatible by introducing Grothendieck topology in three different forms: sieves, coverage,
and covering families. Then we defined an elementary topos and introduce category of sets,
and proved that {\bf Set} is cartesian closed, has object classifier and thus a topos.

This intro could be considered as a formal introduction to topos theory (at least of the level of first chapter)
and you may evolve this library to your needs or ask to help porting or developing your application
of topos theory to a particular formal construction.

\bibliographystyle{plain}
\bibliography{topos}

\end{document}

