\documentclass{article}
\usepackage[utf8]{inputenc}
\usepackage{amsmath,amssymb,amsthm}

\ProvidesPackage{journal}
\usepackage{graphicx}
\usepackage{mathtools}
\usepackage{hyphenat}
\usepackage{hyperref}
\usepackage{adjustbox}
\usepackage{listings}
\usepackage{verbatim}
\usepackage{xcolor}
\usepackage{amsfonts}
\usepackage{amscd}
\usepackage{amsmath}
\usepackage{amssymb}
\usepackage{amsthm}
\usepackage{tikz}
\usepackage{tikz-cd}
\usepackage{url}
\usepackage[utf8]{inputenc}
\usepackage[english,ukrainian]{babel}
\usepackage{float}
\usepackage{url}
\usepackage{tikz}
\usepackage{tikz-cd}
\usepackage[utf8]{inputenc}
\usepackage{graphicx}
\usepackage[utf8]{inputenc}
\usepackage[T1]{fontenc}
\usepackage{lmodern}
\usepackage{tocloft}
\usepackage{hyperref}
\usepackage{xcolor}
\usepackage[only,llbracket,rrbracket,llparenthesis,rrparenthesis]{stmaryrd}

\usetikzlibrary{babel}

\newcommand*{\incmap}{\hookrightarrow}
\newcommand*{\thead}[1]{\multicolumn{1}{c}{\bfseries #1}}
\renewcommand{\Join}{\vee} % Join operation symbol
\newcommand{\tabstyle}[0]{\scriptsize\ttfamily\fontseries{l}\selectfont}

\lstset{
  basicstyle=\footnotesize,
  inputencoding=utf8,
  identifierstyle=,
  literate=
{𝟎}{{\ensuremath{\mathbf{0}}}}1
{𝟏}{{\ensuremath{\mathbf{1}}}}1
{≔}{{\ensuremath{\mathrm{:=}}}}1
{α}{{\ensuremath{\mathrm{\alpha}}}}1
{ᵂ}{{\ensuremath{^W}}}1
{β}{{\ensuremath{\mathrm{\beta}}}}1
{γ}{{\ensuremath{\mathrm{\gamma}}}}1
{δ}{{\ensuremath{\mathrm{\delta}}}}1
{ε}{{\ensuremath{\mathrm{\varepsilon}}}}1
{ζ}{{\ensuremath{\mathrm{\zeta}}}}1
{η}{{\ensuremath{\mathrm{\eta}}}}1
{θ}{{\ensuremath{\mathrm{\theta}}}}1
{ι}{{\ensuremath{\mathrm{\iota}}}}1
{κ}{{\ensuremath{\mathrm{\kappa}}}}1
{λ}{{\ensuremath{\mathrm{\lambda}}}}1
{μ}{{\ensuremath{\mathrm{\mu}}}}1
{ν}{{\ensuremath{\mathrm{\nu}}}}1
{ξ}{{\ensuremath{\mathrm{\xi}}}}1
{π}{{\ensuremath{\mathrm{\mathnormal{\pi}}}}}1
{ρ}{{\ensuremath{\mathrm{\rho}}}}1
{σ}{{\ensuremath{\mathrm{\sigma}}}}1
{τ}{{\ensuremath{\mathrm{\tau}}}}1
{φ}{{\ensuremath{\mathrm{\varphi}}}}1
{χ}{{\ensuremath{\mathrm{\chi}}}}1
{ψ}{{\ensuremath{\mathrm{\psi}}}}1
{ω}{{\ensuremath{\mathrm{\omega}}}}1
{Π}{{\ensuremath{\mathrm{\Pi}}}}1
{Γ}{{\ensuremath{\mathrm{\Gamma}}}}1
{Δ}{{\ensuremath{\mathrm{\Delta}}}}1
{Θ}{{\ensuremath{\mathrm{\Theta}}}}1
{Λ}{{\ensuremath{\mathrm{\Lambda}}}}1
{Σ}{{\ensuremath{\mathrm{\Sigma}}}}1
{Φ}{{\ensuremath{\mathrm{\Phi}}}}1
{Ξ}{{\ensuremath{\mathrm{\Xi}}}}1
{Ψ}{{\ensuremath{\mathrm{\Psi}}}}1
{Ω}{{\ensuremath{\mathrm{\Omega}}}}1
{ℵ}{{\ensuremath{\aleph}}}1
{≤}{{\ensuremath{\leq}}}1
{≥}{{\ensuremath{\geq}}}1
{≠}{{\ensuremath{\neq}}}1
{≈}{{\ensuremath{\approx}}}1
{≡}{{\ensuremath{\equiv}}}1
{≃}{{\ensuremath{\simeq}}}1
{≤}{{\ensuremath{\leq}}}1
{≥}{{\ensuremath{\geq}}}1
{∂}{{\ensuremath{\partial}}}1
{∆}{{\ensuremath{\triangle}}}1 % or \laplace?
{∫}{{\ensuremath{\int}}}1
{∑}{{\ensuremath{\mathrm{\Sigma}}}}1
{→}{{\ensuremath{\rightarrow}}}1
{⊥}{{\ensuremath{\perp}}}1
{∞}{{\ensuremath{\infty}}}1
{∂}{{\ensuremath{\partial}}}1
{∓}{{\ensuremath{\mp}}}1
{±}{{\ensuremath{\pm}}}1
{×}{{\ensuremath{\times}}}1
{⊕}{{\ensuremath{\oplus}}}1
{⊗}{{\ensuremath{\otimes}}}1
{⊞}{{\ensuremath{\boxplus}}}1
{∇}{{\ensuremath{\nabla}}}1
{√}{{\ensuremath{\sqrt}}}1
{⬝}{{\ensuremath{\cdot}}}1
{•}{{\ensuremath{\cdot}}}1
{∘}{{\ensuremath{\circ}}}1
{⁻}{{\ensuremath{^{-}}}}1
{▸}{{\ensuremath{\blacktriangleright}}}1
{★}{{\ensuremath{\star}}}1
{∧}{{\ensuremath{\wedge}}}1
{∨}{{\ensuremath{\vee}}}1
{¬}{{\ensuremath{\neg}}}1
{⊢}{{\ensuremath{\vdash}}}1
{⟨}{{\ensuremath{\langle}}}1
{⟩}{{\ensuremath{\rangle}}}1
{↦}{{\ensuremath{\mapsto}}}1
{→}{{\ensuremath{\rightarrow}}}1
{↔}{{\ensuremath{\leftrightarrow}}}1
{⇒}{{\ensuremath{\Rightarrow}}}1
{⟹}{{\ensuremath{\Longrightarrow}}}1
{⇐}{{\ensuremath{\Leftarrow}}}1
{⟸}{{\ensuremath{\Longleftarrow}}}1
{∩}{{\ensuremath{\cap}}}1
{∪}{{\ensuremath{\cup}}}1
{·}{{\ensuremath{\cdot}}}1
{ᵢ}{{\ensuremath{_i}}}1
{ⱼ}{{\ensuremath{_j}}}1
{₊}{{\ensuremath{_+}}}1
{ℑ}{{\ensuremath{\Im}}}1
{𝒢}{{\ensuremath{\mathcal{G}}}}1
{ℕ}{{\ensuremath{\mathbb{N}}}}1
{𝟘}{{\ensuremath{\mathbb{0}}}}1
{ℤ}{{\ensuremath{\mathbb{Z}}}}1
{ℝ}{{\ensuremath{\mathbb{R}}}}1
{⊂}{{\ensuremath{\subseteq}}}1
{⊆}{{\ensuremath{\subseteq}}}1
{⊄}{{\ensuremath{\nsubseteq}}}1
{⊈}{{\ensuremath{\nsubseteq}}}1
{⊃}{{\ensuremath{\supseteq}}}1
{⊇}{{\ensuremath{\supseteq}}}1
{⊅}{{\ensuremath{\nsupseteq}}}1
{⊉}{{\ensuremath{\nsupseteq}}}1
{∈}{{\ensuremath{\in}}}1
{∉}{{\ensuremath{\notin}}}1
{∋}{{\ensuremath{\ni}}}1
{∌}{{\ensuremath{\notni}}}1
{∅}{{\ensuremath{\emptyset}}}1
{∖}{{\ensuremath{\setminus}}}1
{†}{{\ensuremath{\dag}}}1
{ℕ}{{\ensuremath{\mathbb{N}}}}1
{ℤ}{{\ensuremath{\mathbb{Z}}}}1
{ℝ}{{\ensuremath{\mathbb{R}}}}1
{ℚ}{{\ensuremath{\mathbb{Q}}}}1
{ℂ}{{\ensuremath{\mathbb{C}}}}1
{⌞}{{\ensuremath{\llcorner}}}1
{⌟}{{\ensuremath{\lrcorner}}}1
{⦃}{{\ensuremath{ \{\!| }}}1
{⦄}{{\ensuremath{ |\!\} }}}1
{ᵁ}{{\ensuremath{^U}}}1
{₋}{{\ensuremath{_{-}}}}1
{₁}{{\ensuremath{_1}}}1
{₂}{{\ensuremath{_2}}}1
{₃}{{\ensuremath{_3}}}1
{₄}{{\ensuremath{_4}}}1
{₅}{{\ensuremath{_5}}}1
{₆}{{\ensuremath{_6}}}1
{₇}{{\ensuremath{_7}}}1
{₈}{{\ensuremath{_8}}}1
{₉}{{\ensuremath{_9}}}1
{₀}{{\ensuremath{_0}}}1
{¹}{{\ensuremath{^1}}}1
{ₙ}{{\ensuremath{_n}}}1
{ₘ}{{\ensuremath{_m}}}1
{↑}{{\ensuremath{\uparrow}}}1
{↓}{{\ensuremath{\downarrow}}}1
{▸}{{\ensuremath{\triangleright}}}1
{∀}{{\ensuremath{\forall}}}1
{∃}{{\ensuremath{\exists}}}1
{λ}{{\ensuremath{\mathrm{\lambda}}}}1
{=}{{\ensuremath{=}}}1
{<}{{\ensuremath{\textless}}}1
{>}{{\ensuremath{\textgreater}}}1
{_}{{$\_$}}1
{(}{(}1
{(}{(}1
{‖}{{\ensuremath{\Vert}}}1
{+}{{+}}1
{*}{{*}}1,
}

\theoremstyle{definition}
\newtheorem{definition}{Definition}
\newtheorem{theorem}{Theorem}
\newtheorem{lemma}{Lemma}
\newtheorem{example}{Example}



\begin{document}

\title{Issue XLII: The Cosmic Cube}
\author{Namdak Tonpa}
\date{\today}

\maketitle

\begin{abstract}
The Cosmic Cube is a conceptual framework that organizes various forms of higher category theory, homotopy theory, and type theory along three independent structural axes: strictness, groupoidality, and stability. In this article, we articulate the homotopy-theoretic and computational significance of the cube, map its vertices to familiar categorical and type-theoretic structures, and propose a unifying perspective relevant to both category theorists and type theorists.
\end{abstract}

\ifincludeTOC
  \tableofcontents
\fi

\section{The Cosmic Cube}
The development of higher category theory, homotopy type theory (HoTT), and related computational systems reveals a landscape structured by three key dimensions:

\begin{itemize}
\item \textbf{Strictness}: distinguishing between strict and weak composition laws.
\item \textbf{Groupoidality}: determining whether morphisms are invertible.
\item \textbf{Stability}: whether the theory admits additive or stable (symmetric monoidal) structure.
\end{itemize}

The Cosmic Cube organizes the eight possible combinations of these properties, resulting in a conceptual taxonomy of type theories, logical systems, and homotopy-theoretic models.

\subsection{Axes}

Each axis of the cube represents a binary structural distinction:

\begin{enumerate}
\item \textbf{Groupoidality}: Passing from general $n\$-categories to $n\$-groupoids, reflecting the invertibility of morphisms.
\item \textbf{Strictness}: Moving from weak higher categories to strictly associative and unital structures.
\item \textbf{Stability}: Enhancing categories with stable or symmetric monoidal structure, reflecting additivity or loop space objects.
\end{enumerate}

\subsection{Vertices}

Each vertex of the cube corresponds to a combination of the above properties and can be interpreted both categorically and computationally. We describe these as follows:

\begin{center}
\begin{tabular}{|l|l|l|}
\hline
\textbf{Configuration} & \textbf{Model} \\
\hline
($\Delta$, $\mathbf{2}$, $\mathbf{1}$) & Simply typed $\lambda$-calculus (STLC) \\
($\Delta$, $\mathbf{2}$, $\mathbb{H}$) & $\lambda$-calculus, resource-sensitive computation \\
($\Delta$, $\mathbb{N}$, $\mathbf{1}$) & Homotopy Type Theory (HoTT) \\
($\Delta$, $\mathbb{N}$, $\mathbb{H}$) & Linear HoTT \\
($\nabla$, $\mathbf{2}$, $\mathbf{1}$) & Modal STLC \\
($\nabla$, $\mathbf{2}$, $\mathbb{H}$) & $\infty$-toposes, QFT \\
($\nabla$, $\mathbb{N}$, $\mathbf{1}$) & Synthetic Differential Geometry (Modal HoTT) \\
($\nabla$, $\mathbb{N}$, $\mathbb{H}$) & Modal Linear HoTT \\
\hline
\end{tabular}
\end{center}

\subsection{Homotopy-Theoretic Realization}

The cube also arises naturally from the classification of higher-categorical structures:

\begin{itemize}
\item \textbf{Strict $\infty$-categories}: basic directed homotopy theory.
\item \textbf{Strict $\infty$-groupoids}: modeled by crossed complexes.
\item \textbf{Stable $\infty$-groupoids}: spectra (e.g., infinite loop spaces).
\item \textbf{Strictly stable strict $\infty$-groupoids}: chain complexes (via Dold-Kan correspondence).
\end{itemize}

The inclusions among these structures (e.g., from chain
complexes to spectra, or from strict to weak groupoids)
correspond to forgetful functors or structure-preserving
embeddings (e.g., via the nerve, stabilization, or $\Omega^\infty$).

\subsection{Computational Interpretation}

From the viewpoint of type theory and programming languages:

\begin{itemize}
\item \textbf{Strictness} governs syntactic vs coherent compositions.
\item \textbf{Groupoidality} relates to equality vs higher identity types.
\item \textbf{Stability} corresponds to additivity or quantum effects.
\end{itemize}

Thus, the Cosmic Cube serves not only as a classification of categorical models, but also as a blueprint for designing new type theories with specific logical and computational properties.

\section{The Colored Spectra}

The Colored Spectra consists of five flavours (layers):
1) Fibrational (Pi, Sigma),
2) Identificational (Strip Equality, Path Spaces),
3) Polynomial (W, Higher Inductive Types),
4) Modal (Bose, Fermi, Flat, Sharp),
5) Localizational (Local, Graded, Stable, Simplicial).
\\
\\
\noindent \textbf{Fibrational}:
\begin{itemize}
\item \text{Functional Spaces} ($\mathbf{\Pi}$)
\item \text{Contextual Spaces} ($\mathbf{\Sigma}$)
\end{itemize}

\noindent \textbf{Identificational}:
\begin{itemize}
\item \text{Strict Equality} ($\mathbf{=}$)
\item \text{Path Spaces} ($\mathbf{\equiv}$),
      \text{Homotopy} ($\mathbf{\sim}$),
      \text{Equivalence} ($\mathbf{\simeq}$),
      \text{Isomorphism} ($\mathbf{\cong}$)
\item \text{Bisimulation} ($\mathbf{\approx}$)
\item \text{Gluing} ($\mathbf{Glue}$)
\item \text{Quotient} ($\mathbf{Q}$)
\end{itemize}

\noindent \textbf{Polynomial}:
\begin{itemize}
\item \text{$\Pi$W-Pretopos}: \text{Induction} ($\mathbf{W}$), \text{Coinduction} ($\mathbf{M}$)
\item \text{CIC}: \text{Induction} ($\mathbf{CIC}$), \text{Induction} ($\mathbf{CCoIC}$)
\item \text{CW-Complexes}: \text{Induction} ($\mathbf{CW}$), \text{Coinduction} ($\mathbf{CM}$)
\end{itemize}

\noindent \textbf{Modal}:
\begin{itemize}
\item \text{Fermi} ($\Im$)
\item \text{Bose} ($\bigcirc$)
\item \text{Flat} ($\flat$)
\item \text{Sharp} ($\sharp$)
\end{itemize}

\noindent \textbf{Localizational}:
\begin{itemize}
\item \text{Stable Homotopy Theory} ($\mathbf{Sp}$)
\item \text{Simplicial Homotopy Theory} ($\mathbb{\Delta}^n$)
\item \text{Local Homotopy Theory} ($\mathbb{A}^1$)
\item \text{Graded Homotopy Theory} ($\mathbb{Z}_2$)
\end{itemize}

\subsection{Conclusion}
The Cosmic Cube provides a unifying language for relating different regions
of categorical and homotopical logic. It highlights deep dualities (such as
LCCC vs SMC), computational distinctions (classical vs quantum),
and modalities (discrete, cohesive, stable) that structure modern
type theories and their semantics.

\begin{thebibliography}{9}
\bibitem{baes} John C. Baez, \emph{What n-Categories Should Be Like}, , 2002.
\end{thebibliography}


\end{document}

