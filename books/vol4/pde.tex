\documentclass{article}
\usepackage[utf8]{inputenc}
\usepackage{amsmath, amssymb, amsthm}
\usepackage{mathrsfs}
\usepackage{mathtools}
\usepackage{url}

\theoremstyle{plain}
\newtheorem{theorem}{Theorem}[section]
\newtheorem{definition}{Definition}[section]
\newtheorem{remark}{Remark}[section]

\newcommand{\cat}[1]{\mathbf{#1}}
\newcommand{\Hom}{\mathrm{Hom}}
\newcommand{\id}{\mathrm{id}}
\newcommand{\op}{\mathrm{op}}
\newcommand{\dR}{\mathrm{dR}}
\newcommand{\CE}{\mathrm{CE}}

\begin{document}

% Title and introduction
\title{Solving Partial }
\author{}
\date{}
\maketitle

\begin{abstract}
This lecture presents a categorical framework for Green's functions, Stokes--Ostrogradsky theorems, and Fredholm and Volterra integral equations in the context of partial differential equations (PDEs). We incorporate the de Rham theorem, simplicial de Rham complexes, and synthetic differential geometry, drawing on insights from nLab to unify local differential and global integral structures within the category of smooth toposes and differential graded algebras.
\end{abstract}

\section{Introduction: Mathematics as a Categorical Architecture}

% Framing the categorical perspective
In the spirit of structural mathematics, partial differential equations (PDEs) are not merely computational tools but objects within a categorical framework, where morphisms reveal universal properties of local-to-global transitions. Green's functions, Stokes--Ostrogradsky theorems, and Fredholm and Volterra integral equations, together with the de Rham theorem and synthetic differential geometry, form a cohesive structure in the category of functional spaces and smooth toposes. This lecture elucidates these connections, emphasizing homological and operator-theoretic perspectives, and integrates insights from synthetic differential geometry to provide an axiomatic foundation for differential and integral structures \cite{nLab_synthetic_diff_geom}.

\section{Green's Functions: Kernels of Inverse Morphisms}

% Defining Green's function categorically
\begin{definition}
Let \(\cat{E}\) and \(\cat{F}\) be Hilbert spaces (e.g., \(L_2(\Omega)\) or Sobolev spaces \(H^k(\Omega)\)), and let \(L: \cat{E} \to \cat{F}\) be a linear differential operator on a smooth manifold \(\Omega\). A \emph{Green's function} \(G: \Omega \times \Omega \to \mathbb{R}\) (or \(\mathbb{C}\)) for \(L\) is a morphism satisfying:
\[
L_x G(x, \xi) = \delta_\xi,
\]
where \(\delta_\xi \in \cat{F}^*\) is the Dirac delta functional, defined by \(\langle \delta_\xi, f \rangle = f(\xi)\) for all \(f \in \cat{F}\).
\end{definition}

% Categorical role of Green's functions
The Green's function induces an integral operator \(K: \cat{F} \to \cat{E}\), given by:
\[
(K f)(x) = \int_\Omega G(x, \xi) f(\xi) \, d\mu(\xi),
\]
which acts as a right inverse to \(L\), i.e., \(L K = \id_{\cat{F}}\) on a suitable subspace of \(\cat{F}\). In the category of Hilbert spaces, \(G\) is a kernel in \(\Hom(\cat{F}^*, \cat{E})\), embodying the duality between \(\cat{F}\) and its dual \(\cat{F}^*\).

\section{Stokes--Ostrogradsky Theorems: Homological Isomorphisms}

% Defining the generalized Stokes theorem
\begin{theorem}[Generalized Stokes Theorem]
Let \(M\) be an oriented smooth \(n\)-manifold with boundary \(\partial M\), and let \(\omega \in \Omega^{n-1}(M)\) be a differential \((n-1)\)-form. Then:
\[
\int_M d\omega = \int_{\partial M} \omega,
\]
where \(d: \Omega^{n-1}(M) \to \Omega^n(M)\) is the exterior derivative \cite{nLab_stokes_theorem}.
\end{theorem}

% Special cases
\begin{itemize}
    \item \emph{Ostrogradsky--Gauss Theorem}: For a vector field \(\mathbf{F}\) on \(\Omega \subset \mathbb{R}^n\),
    \[
    \int_\Omega \nabla \cdot \mathbf{F} \, dV = \oint_{\partial \Omega} \mathbf{F} \cdot \mathbf{n} \, dS.
    \]
    \item \emph{Stokes Theorem}: For a vector field \(\mathbf{F}\) on a surface \(S \subset \mathbb{R}^3\),
    \[
    \int_S (\nabla \times \mathbf{F}) \cdot \mathbf{n} \, dS = \oint_{\partial S} \mathbf{F} \cdot d\mathbf{l}.
    \]
\end{itemize}

% Categorical interpretation
In the category of differential forms \(\cat{\Omega}(M)\), the Stokes theorem is an isomorphism in the de Rham complex, reflecting the exactness of the sequence:
\[
\cdots \to \Omega^{k-1}(M) \xrightarrow{d} \Omega^k(M) \xrightarrow{d} \Omega^{k+1}(M) \to \cdots.
\]
This isomorphism connects local differential structures to global integral structures, enabling the derivation of integral representations for PDE solutions.

\section{de Rham Theorem and Complex}

% Defining the de Rham complex
\begin{definition}[de Rham Complex]
Let \(M\) be a smooth manifold. The \emph{de Rham complex} is the cochain complex of differential forms:
\[
\Omega^\bullet(M): \Omega^0(M) \xrightarrow{d} \Omega^1(M) \xrightarrow{d} \Omega^2(M) \xrightarrow{d} \cdots,
\]
where \(\Omega^k(M)\) is the space of smooth \(k\)-forms, and \(d\) is the exterior derivative, satisfying \(d \circ d = 0\). The \emph{de Rham cohomology} is the cohomology of this complex:
\[
H^k_{\dR}(M) = \frac{\ker(d: \Omega^k(M) \to \Omega^{k+1}(M))}{\mathrm{im}(d: \Omega^{k-1}(M) \to \Omega^k(M))}.
\]
\end{definition}

% de Rham theorem
\begin{theorem}[de Rham Theorem]
For a smooth manifold \(M\), the de Rham cohomology \(H^k_{\dR}(M)\) is isomorphic to the singular cohomology \(H^k(M; \mathbb{R})\) with real coefficients:
\[
H^k_{\dR}(M) \cong H^k(M; \mathbb{R}).
\]
The isomorphism is induced by the integration map:
\[
\omega \mapsto \left( \sigma \mapsto \int_\sigma \omega \right),
\]
where \(\sigma: \Delta^k \to M\) is a singular \(k\)-simplex \cite{nLab_de_Rham_theorem}.
\end{theorem}

% Categorical perspective
In the category of sheaves on \(M\), the de Rham complex is a complex of abelian sheaves:
\[
\bar{\mathbf{B}}^k \mathbb{R} = (C^\infty(-, \mathbb{R}) \xrightarrow{d_{\dR}} \Omega^1(-) \xrightarrow{d_{\dR}} \cdots \xrightarrow{d_{\dR}} \Omega^k_{\mathrm{closed}}(-)).
\]
The de Rham theorem establishes an equivalence in the derived category of sheaves, linking differential forms to topological invariants. In synthetic differential geometry, this complex is internalized in a smooth topos, where the Kock-Lawvere axiom ensures the existence of infinitesimals, facilitating intuitive reasoning about differentials \cite{nLab_synthetic_diff_geom}.

\section{Simplicial de Rham Complex}

% Defining the simplicial de Rham complex
\begin{definition}[Simplicial de Rham Complex]
Let \(X_\bullet: \Delta^{\op} \to \cat{Diff}\) be a simplicial manifold in the category of smooth manifolds \(\cat{Diff}\). The \emph{simplicial de Rham complex} is the total complex of the double complex:
\[
\Omega^p(X_q) \xrightarrow{\sum_i (-1)^i \delta_i^*} \Omega^p(X_{q+1}),
\]
\[
\Omega^p(X_q) \xrightarrow{d_{\dR}} \Omega^{p+1}(X_q),
\]
where \(\delta_i: X_{q+1} \to X_q\) are face maps, and \(d_{\dR}\) is the de Rham differential \cite{nLab_simplicial_de_Rham}.
\end{definition}

% Connection to ∞-Lie theory
The simplicial de Rham complex is quasi-isomorphic to the Chevalley-Eilenberg algebra of the infinitesimal path \(\infty\)-groupoid \(\mathbf{\Pi}_{\inf}(X_\bullet)\), reflecting the structure of a geometric \(\infty\)-Lie groupoid. This complex is crucial for equivariant de Rham cohomology and rational homotopy theory, where:
\[
C^\infty(\mathbf{\Pi}_{\inf}(X_\bullet)) \simeq \mathrm{Tot} \, \Omega^\bullet(X_\bullet).
\]

\section{Synthetic Differential Geometry: Axiomatic Framework}

% Defining synthetic differential geometry
\begin{definition}[Smooth Topos]
A \emph{smooth topos} \(\cat{T}\) is a topos equipped with a line object \(R\) satisfying the Kock-Lawvere axiom: for any \(f: D \to R\), where \(D = \{ x \in R \mid x^2 = 0 \}\) is the infinitesimal object, there exist unique \(a, b \in R\) such that:
\[
f(x) = a + b x \quad \text{for all } x \in D.
\]
A smooth topos models synthetic differential geometry, allowing intuitive reasoning with infinitesimals \cite{nLab_synthetic_diff_geom}.
\end{definition}

% Integration in synthetic differential geometry
In a smooth topos, the integration axiom posits that the shape modality \(\int\) and flat modality \(\flat\) define a cohesive structure:
\[
\int \dashv \flat \dashv \sharp.
\]
The de Rham complex in synthetic differential geometry is internalized as a complex of objects in \(\cat{T}\), where differential forms are morphisms from infinitesimal thickenings. The integration map in this context is a morphism:
\[
\int: \Omega^k(M) \to \Omega^k(\int M),
\]
inducing a synthetic analogue of the de Rham theorem \cite{nLab_synthetic_diff_geom}.

\section{Integral Equations: Operator Algebras}

% Defining Fredholm and Volterra equations
\begin{definition}
Let \(\cat{E}\) be a Hilbert space, and let \(K: \cat{E} \to \cat{E}\) be a compact linear operator.
\begin{itemize}
    \item A \emph{Fredholm integral equation of the second kind} is:
    \[
    u = f + \lambda K u,
    \]
    where \(f \in \cat{E}\), \(\lambda \in \mathbb{C}\), and \(u \in \cat{E}\).
    \item A \emph{Volterra integral equation of the second kind} is:
    \[
    u(x) = f(x) + \lambda \int_a^x K(x, \xi) u(\xi) \, d\xi,
    \]
    where the integral has a variable upper bound.
\end{itemize}
\end{definition}

% Connection to Green's functions
The kernel \(K(x, \xi)\) is often the Green's function \(G(x, \xi)\) or its derivatives. For a PDE \(L u = f\) with boundary conditions, the solution may satisfy:
\[
u(x) = \int_\Omega G(x, \xi) f(\xi) \, d\xi + \int_{\partial \Omega} K(x, \xi) u(\xi) \, dS.
\]
In synthetic differential geometry, such equations are internalized as morphisms in \(\cat{T}\), with \(K\) defined via the infinitesimal structure of \(R\).

\section{Structural Unity: Local-Global Duality}

The categorical framework unifies these concepts:
\begin{itemize}
    \item Green's functions are kernels in \(\Hom(\cat{F}^*, \cat{E})\).
    \item Stokes--Ostrogradsky theorems are isomorphisms in the de Rham complex.
    \item The de Rham theorem links differential forms to singular cohomology via integration.
    \item Simplicial de Rham complexes extend this to \(\infty\)-groupoids.
    \item Synthetic differential geometry provides an axiomatic framework for infinitesimals and integration.
    \item Fredholm and Volterra equations are algebraic representations of these morphisms.
\end{itemize}

This reflects a Grothendieckian local-to-global duality, where PDE solutions are cohomology classes in the derived category of sheaves or modules over the ring of differential operators \cite{nLab_de_Rham_theorem, nLab_synthetic_diff_geom}.

\section{Conclusion}

This lecture presents a categorical synthesis of PDE theory, integrating Green's functions, Stokes--Ostrogradsky theorems, de Rham complexes, synthetic differential geometry, and integral equations. Future lectures will:
\begin{enumerate}
    \item Construct Green's functions and integral operators in smooth toposes.
    \item Explore homological properties of the de Rham and simplicial de Rham complexes.
    \item Analyze synthetic differential geometry and its integration axioms.
    \item Synthesize these into a unified framework for PDEs.
\end{enumerate}

% Bibliography
\begin{thebibliography}{9}
\bibitem{nLab_de_Rham_theorem}
nLab, \emph{de Rham theorem}, \url{https://ncatlab.org/nlab/show/de+Rham+theorem}, accessed 2024-11-21.
\bibitem{nLab_stokes_theorem}
nLab, \emph{Stokes theorem}, \url{https://ncatlab.org/nlab/show/Stokes+theorem}, accessed 2025-05-11.
\bibitem{nLab_synthetic_diff_geom}
nLab, \emph{synthetic differential geometry}, \url{https://ncatlab.org/nlab/show/synthetic+differential+geometry}, accessed 2025-05-09.
\bibitem{nLab_simplicial_de_Rham}
nLab, \emph{simplicial de Rham complex}, \url{https://ncatlab.org/nlab/show/simplicial+de+Rham+complex}, accessed 2025-02-13.
\end{thebibliography}

\end{document}
