\documentclass{article}
\usepackage{amsmath, amssymb, amsthm}
\usepackage{enumitem}
\usepackage{graphicx}
\usepackage{tikz}

\ProvidesPackage{journal}
\usepackage{caption}
\usepackage{hyperref}
\usepackage{epigraph}
\usepackage{listings}
\usepackage{amsmath}
\usepackage{amssymb}
\usepackage{amsthm}
\usepackage{tikz}
\usepackage{url}
\usepackage[english,ukrainian]{babel}
\usepackage[utf8]{inputenc}
\usepackage[T1]{fontenc}
\usepackage[only,llbracket,rrbracket,llparenthesis,rrparenthesis]{stmaryrd}

\newcommand*{\incmap}{\hookrightarrow}
\newcommand*{\thead}[1]{\multicolumn{1}{c}{\bfseries #1}}
\renewcommand{\Join}{\vee} % Join operation symbol
\newcommand{\tabstyle}[0]{\scriptsize\ttfamily\fontseries{l}\selectfont}

\lstset{
  basicstyle=\footnotesize,
  inputencoding=utf8,
  identifierstyle=,
  literate=
{𝟎}{{\ensuremath{\mathbf{0}}}}1
{𝟏}{{\ensuremath{\mathbf{1}}}}1
{≔}{{\ensuremath{\mathrm{:=}}}}1
{α}{{\ensuremath{\mathrm{\alpha}}}}1
{ᵂ}{{\ensuremath{^W}}}1
{β}{{\ensuremath{\mathrm{\beta}}}}1
{γ}{{\ensuremath{\mathrm{\gamma}}}}1
{δ}{{\ensuremath{\mathrm{\delta}}}}1
{ε}{{\ensuremath{\mathrm{\varepsilon}}}}1
{ζ}{{\ensuremath{\mathrm{\zeta}}}}1
{η}{{\ensuremath{\mathrm{\eta}}}}1
{θ}{{\ensuremath{\mathrm{\theta}}}}1
{ι}{{\ensuremath{\mathrm{\iota}}}}1
{κ}{{\ensuremath{\mathrm{\kappa}}}}1
{λ}{{\ensuremath{\mathrm{\lambda}}}}1
{μ}{{\ensuremath{\mathrm{\mu}}}}1
{ν}{{\ensuremath{\mathrm{\nu}}}}1
{ξ}{{\ensuremath{\mathrm{\xi}}}}1
{π}{{\ensuremath{\mathrm{\mathnormal{\pi}}}}}1
{ρ}{{\ensuremath{\mathrm{\rho}}}}1
{σ}{{\ensuremath{\mathrm{\sigma}}}}1
{τ}{{\ensuremath{\mathrm{\tau}}}}1
{φ}{{\ensuremath{\mathrm{\varphi}}}}1
{χ}{{\ensuremath{\mathrm{\chi}}}}1
{ψ}{{\ensuremath{\mathrm{\psi}}}}1
{ω}{{\ensuremath{\mathrm{\omega}}}}1
{Π}{{\ensuremath{\mathrm{\Pi}}}}1
{Γ}{{\ensuremath{\mathrm{\Gamma}}}}1
{Δ}{{\ensuremath{\mathrm{\Delta}}}}1
{Θ}{{\ensuremath{\mathrm{\Theta}}}}1
{Λ}{{\ensuremath{\mathrm{\Lambda}}}}1
{Σ}{{\ensuremath{\mathrm{\Sigma}}}}1
{Φ}{{\ensuremath{\mathrm{\Phi}}}}1
{Ξ}{{\ensuremath{\mathrm{\Xi}}}}1
{Ψ}{{\ensuremath{\mathrm{\Psi}}}}1
{Ω}{{\ensuremath{\mathrm{\Omega}}}}1
{ℵ}{{\ensuremath{\aleph}}}1
{≤}{{\ensuremath{\leq}}}1
{≥}{{\ensuremath{\geq}}}1
{≠}{{\ensuremath{\neq}}}1
{≈}{{\ensuremath{\approx}}}1
{≡}{{\ensuremath{\equiv}}}1
{≃}{{\ensuremath{\simeq}}}1
{≤}{{\ensuremath{\leq}}}1
{≥}{{\ensuremath{\geq}}}1
{∂}{{\ensuremath{\partial}}}1
{∆}{{\ensuremath{\triangle}}}1 % or \laplace?
{∫}{{\ensuremath{\int}}}1
{∑}{{\ensuremath{\mathrm{\Sigma}}}}1
{→}{{\ensuremath{\rightarrow}}}1
{⊥}{{\ensuremath{\perp}}}1
{∞}{{\ensuremath{\infty}}}1
{∂}{{\ensuremath{\partial}}}1
{∓}{{\ensuremath{\mp}}}1
{±}{{\ensuremath{\pm}}}1
{×}{{\ensuremath{\times}}}1
{⊕}{{\ensuremath{\oplus}}}1
{⊗}{{\ensuremath{\otimes}}}1
{⊞}{{\ensuremath{\boxplus}}}1
{∇}{{\ensuremath{\nabla}}}1
{√}{{\ensuremath{\sqrt}}}1
{⬝}{{\ensuremath{\cdot}}}1
{•}{{\ensuremath{\cdot}}}1
{∘}{{\ensuremath{\circ}}}1
{⁻}{{\ensuremath{^{-}}}}1
{▸}{{\ensuremath{\blacktriangleright}}}1
{★}{{\ensuremath{\star}}}1
{∧}{{\ensuremath{\wedge}}}1
{∨}{{\ensuremath{\vee}}}1
{¬}{{\ensuremath{\neg}}}1
{⊢}{{\ensuremath{\vdash}}}1
{⟨}{{\ensuremath{\langle}}}1
{⟩}{{\ensuremath{\rangle}}}1
{↦}{{\ensuremath{\mapsto}}}1
{→}{{\ensuremath{\rightarrow}}}1
{↔}{{\ensuremath{\leftrightarrow}}}1
{⇒}{{\ensuremath{\Rightarrow}}}1
{⟹}{{\ensuremath{\Longrightarrow}}}1
{⇐}{{\ensuremath{\Leftarrow}}}1
{⟸}{{\ensuremath{\Longleftarrow}}}1
{∩}{{\ensuremath{\cap}}}1
{∪}{{\ensuremath{\cup}}}1
{·}{{\ensuremath{\cdot}}}1
{ᵢ}{{\ensuremath{_i}}}1
{ⱼ}{{\ensuremath{_j}}}1
{₊}{{\ensuremath{_+}}}1
{ℑ}{{\ensuremath{\Im}}}1
{𝒢}{{\ensuremath{\mathcal{G}}}}1
{ℕ}{{\ensuremath{\mathbb{N}}}}1
{𝟘}{{\ensuremath{\mathbb{0}}}}1
{ℤ}{{\ensuremath{\mathbb{Z}}}}1
{ℝ}{{\ensuremath{\mathbb{R}}}}1
{⊂}{{\ensuremath{\subseteq}}}1
{⊆}{{\ensuremath{\subseteq}}}1
{⊄}{{\ensuremath{\nsubseteq}}}1
{⊈}{{\ensuremath{\nsubseteq}}}1
{⊃}{{\ensuremath{\supseteq}}}1
{⊇}{{\ensuremath{\supseteq}}}1
{⊅}{{\ensuremath{\nsupseteq}}}1
{⊉}{{\ensuremath{\nsupseteq}}}1
{∈}{{\ensuremath{\in}}}1
{∉}{{\ensuremath{\notin}}}1
{∋}{{\ensuremath{\ni}}}1
{∌}{{\ensuremath{\notni}}}1
{∅}{{\ensuremath{\emptyset}}}1
{∖}{{\ensuremath{\setminus}}}1
{†}{{\ensuremath{\dag}}}1
{ℕ}{{\ensuremath{\mathbb{N}}}}1
{ℤ}{{\ensuremath{\mathbb{Z}}}}1
{ℝ}{{\ensuremath{\mathbb{R}}}}1
{ℚ}{{\ensuremath{\mathbb{Q}}}}1
{ℂ}{{\ensuremath{\mathbb{C}}}}1
{⌞}{{\ensuremath{\llcorner}}}1
{⌟}{{\ensuremath{\lrcorner}}}1
{⦃}{{\ensuremath{ \{\!| }}}1
{⦄}{{\ensuremath{ |\!\} }}}1
{ᵁ}{{\ensuremath{^U}}}1
{₋}{{\ensuremath{_{-}}}}1
{₁}{{\ensuremath{_1}}}1
{₂}{{\ensuremath{_2}}}1
{₃}{{\ensuremath{_3}}}1
{₄}{{\ensuremath{_4}}}1
{₅}{{\ensuremath{_5}}}1
{₆}{{\ensuremath{_6}}}1
{₇}{{\ensuremath{_7}}}1
{₈}{{\ensuremath{_8}}}1
{₉}{{\ensuremath{_9}}}1
{₀}{{\ensuremath{_0}}}1
{¹}{{\ensuremath{^1}}}1
{ₙ}{{\ensuremath{_n}}}1
{ₘ}{{\ensuremath{_m}}}1
{↑}{{\ensuremath{\uparrow}}}1
{↓}{{\ensuremath{\downarrow}}}1
{▸}{{\ensuremath{\triangleright}}}1
{∀}{{\ensuremath{\forall}}}1
{∃}{{\ensuremath{\exists}}}1
{λ}{{\ensuremath{\mathrm{\lambda}}}}1
{=}{{\ensuremath{=}}}1
{<}{{\ensuremath{\textless}}}1
{>}{{\ensuremath{\textgreater}}}1
{_}{{$\_$}}1
{(}{(}1
{(}{(}1
{‖}{{\ensuremath{\Vert}}}1
{+}{{+}}1
{*}{{*}}1,
}

\setlength{\parindent}{15pt}

\theoremstyle{definition}
\newtheorem{definition}{Definition}
\newtheorem{theorem}{Theorem}
\newtheorem{lemma}{Lemma}
\newtheorem{example}{Example}

\definecolor{LightGray}{rgb}{0.8,0.8,0.8}
\definecolor{LightGray25}{rgb}{0.9,0.9,0.9}
\definecolor{ZimaBlue}{HTML}{A1D8FC}

\newif\ifincludeTOC
\includeTOCtrue

\lefthyphenmin=1
\hyphenpenalty=100
\tolerance=11000
\emergencystretch=1em
\hfuzz=2pt
\vfuzz=2pt



\begin{document}

\title{Monads and Descent}
\author{Jean B\'enabou and Jacques Roubaud \\ \small Communicated by Henri Cartan}
\date{Received 5 January 1970}

\maketitle

\begin{abstract}
Using category theory, we interpret descent data to determine, in very general settings, whether a morphism is a descent morphism or an effective descent morphism.
\end{abstract}

\section{Chevalley Bifibrations and Descent}

Let $\mathrm{P} : \mathbf{M} \to \mathbf{A}$ denote a bifibrant functor \cite{Grothendieck1959}. For an object $A \in \mathbf{A}$, let $\mathbf{M}(A)$ denote the fibre over $A$. We assume that $\mathbf{A}$ has fibred products.

\subsection{Monad Associated with an Arrow}

Let $a : A_1 \to A_0$ be an arrow in $\mathbf{A}$. Denote by
\[
a^* : \mathbf{M}(A_0) \to \mathbf{M}(A_1) \quad \text{[resp. } a_* : \mathbf{M}(A_1) \to \mathbf{M}(A_0)\text{]}
\]
the inverse image functor (resp. direct image functor), and
\[
\eta^a : \mathrm{Id}_{\mathbf{M}(A_1)} \to a^* a_*; \quad \varepsilon^a : a_* a^* \to \mathrm{Id}_{\mathbf{M}(A_0)}
\]
the canonical natural transformations making $a_*$ a left adjoint to $a^*$. This adjunction defines \cite{Linton1969} on $\mathbf{M}(A_1)$ the monad $\mathbf{T}^a = (T^a, \mu^a, \eta^a)$, where
\[
T^a = a^* a_* : \mathbf{M}(A_1) \to \mathbf{M}(A_1), \quad \mu^a = a^* \varepsilon^a a_* : T^a \circ T^a \to T^a.
\]
Let $\mathbf{M}^a$ denote the category $\mathbf{M}(A_1)^{(\mathbf{T}^a)}$ of algebras over the monad $\mathbf{T}^a$, and let
\[
\mathrm{U}^{\mathbf{T}^a} : \mathbf{M}^a \to \mathbf{M}(A_1), \quad \Phi^a : \mathbf{M}(A_0) \to \mathbf{M}^a
\]
be the canonical functors.

\subsection{Chevalley Property}

\begin{definition}
\label{def:chevalley}
The functor $\mathrm{P}$ is a \emph{Chevalley functor} if it satisfies the following property (C):
\begin{itemize}
    \item[(C)] For every commutative diagram in $\mathbf{M}$
    \[
    \begin{tikzpicture}[baseline=(current bounding box.center)]
        \node (M1) at (0,2) {$M_1$};
        \node (M2) at (2,2) {$M_2$};
        \node (M3) at (0,0) {$M_3$};
        \node (M4) at (2,0) {$M_4$};
        \draw[->] (M1) to node[above] {$k_1$} (M2);
        \draw[->] (M1) to node[left] {$\gamma$} (M3);
        \draw[->] (M2) to node[right] {$\gamma'$} (M4);
        \draw[->] (M3) to node[below] {$k_0$} (M4);
    \end{tikzpicture}
    \]
    whose image under $\mathrm{P}$ is a cartesian square in $\mathbf{A}$, if $\gamma$ and $\gamma'$ are cartesian and $k_0$ is cocartesian, then $k_1$ is cocartesian.
\end{itemize}
\end{definition}

\subsection{Characterization of Descent Data}

Assume henceforth that $\mathrm{P} : \mathbf{M} \to \mathbf{A}$ is a Chevalley functor. Let $a : A_1 \to A_0$ be an arrow in $\mathbf{A}$. Let $A_2$ be the fibred product $A_1 \times_{A_0} A_1$, with canonical projections $a_1, a_2 : A_2 \to A_1$. The property (C) defines, for every object $M_1 \in \mathbf{M}(A_1)$, a canonical bijection, natural in $M_1$,
\[
\operatorname{Hom}_{\mathbf{M}(A_2)}(a_1^*(M_1), a_2^*(M_1)) \to \operatorname{Hom}_{\mathbf{M}(A_1)}(T^a(M_1), M_1),
\]
denoted $\varphi \mapsto \mathrm{K}^a(\varphi)$.

\begin{lemma}
\label{lem:descent}
An arrow $\varphi : a_1^*(M_1) \to a_2^*(M_1)$ such that $\mathrm{P}(\varphi) = \mathrm{id}_{A_2}$ is a descent datum if and only if $\mathrm{K}^a(\varphi)$ is an algebra over the monad $\mathbf{T}^a$.
\end{lemma}

Let $\mathrm{D}(a)$ denote the category of descent data relative to $a$, and let
\[
\Psi^a : \mathbf{M}(A_0) \to \mathrm{D}(a), \quad \mathrm{U}^a : \mathrm{D}(a) \to \mathbf{M}(A_1)
\]
be the canonical functors.

\begin{theorem}
\label{thm:equivalence}
The correspondence $\varphi \mapsto \mathrm{K}^a(\varphi)$ induces an equivalence of categories $\mathrm{K}^a : \mathrm{D}(a) \to \mathbf{M}^a$, making the following diagram commute:
\[
\begin{tikzpicture}
    \node (M0) at (0,2) {$\mathbf{M}(A_0)$};
    \node (Da) at (2,2) {$\mathrm{D}(a)$};
    \node (Ma) at (4,2) {$\mathbf{M}^a$};
    \node (M1) at (2,0) {$\mathbf{M}(A_1)$};
    \draw[->] (M0) to node[above] {$\Psi^a$} (Da);
    \draw[->] (Da) to node[above] {$\mathrm{K}^a$} (Ma);
    \draw[->] (M0) to node[left] {$\Phi^a$} (Ma);
    \draw[->] (Da) to node[left] {$\mathrm{U}^a$} (M1);
    \draw[->] (Ma) to node[right] {$\mathrm{U}^{\mathbf{T}^a}$} (M1);
\end{tikzpicture}
\]
\end{theorem}

\begin{proposition}
\label{prop:universal}
The correspondence $\varphi \mapsto \mathrm{K}^a(\varphi)$ is universal. Precisely, for an arrow $b_0 : A_0' \to A_0$ in $\mathbf{A}$, consider the change-of-base diagram in $\mathbf{A}$:
\[
\begin{tikzpicture}
    \node (A2') at (0,2) {$A_2'$};
    \node (A1') at (2,2) {$A_1'$};
    \node (A0') at (0,0) {$A_0'$};
    \node (A2) at (4,2) {$A_2$};
    \node (A1) at (6,2) {$A_1$};
    \node (A0) at (4,0) {$A_0$};
    \draw[->] (A2') to node[above] {$b_2$} (A2);
    \draw[->] (A1') to node[above] {$b_1$} (A1);
    \draw[->] (A0') to node[below] {$b_0$} (A0);
    \draw[->] (A2') to node[left] {$a_1'$} (A1');
    \draw[->] (A2) to node[left] {$a_1$} (A1);
    \draw[->] (A2') to node[right] {$a_2'$} (A1');
    \draw[->] (A2) to node[right] {$a_2$} (A1);
    \draw[->] (A1') to node[below] {$a'$} (A0');
    \draw[->] (A1) to node[below] {$a$} (A0);
\end{tikzpicture}
\]
For $M_1 \in \mathbf{M}(A_1)$ and $\varphi : a_1^*(M_1) \to a_2^*(M_1)$ in $\mathbf{M}(A_2)$,
\[
\mathrm{K}^{a'}(b_2^*(\varphi)) = b_1^*(\mathrm{K}^a(\varphi)).
\]
\end{proposition}

In particular, taking $A_0' = A_1$ and $b_0 = a$, if $\varphi$ is a descent datum, then $b_2^*(\varphi)$ is an effective descent datum. The converse holds, yielding:

\begin{corollary}
\label{cor:effective-descent}
An arrow $\varphi : a_1^*(M_1) \to a_2^*(M_1) \in \mathbf{M}(A_2)$ is a descent datum if and only if its inverse image $b_2^*(\varphi)$ under the canonical change of base $b_0 = a : A_0' = A_1 \to A_0$ is an effective descent datum.
\end{corollary}

This eliminates the need for the ``cocycle condition'' in subsequent arguments.

\section{First Applications}

Using Theorem \ref{thm:equivalence}, Beck's criterion \cite{Linton1969} provides necessary and sufficient conditions for $\Psi^a$ to be faithful, fully faithful, or an equivalence of categories, in terms of commutation and reflection of certain cokernels by $a^*$.

\begin{proposition}
\label{prop:adj-left}
If cokernels of pairs of arrows exist in $\mathbf{M}(A_0)$, then $\Psi^a$ has a left adjoint.
\end{proposition}

\begin{proposition}
\label{prop:faithful}
The functor $\Psi^a$ is faithful if and only if $a^*$ is faithful.
\end{proposition}

\begin{proposition}
\label{prop:fully-faithful}
If $a^*$ reflects cokernels, then $\Psi^a$ is fully faithful. In particular, if all fibres of $\mathbf{M}$ are abelian, then
\[
\Psi^a \text{ faithful} \iff \Psi^a \text{ fully faithful} \iff a^* \text{ faithful}.
\]
\end{proposition}

\begin{definition}
\label{def:faithfully-flat}
An arrow $a : A_1 \to A_0$ is \emph{faithfully flat} if $a^*$ commutes with cokernels and reflects isomorphisms.
\end{definition}

\begin{proposition}
\label{prop:equivalence}
If $a : A_1 \to A_0$ is faithfully flat and cokernels exist in $\mathbf{M}(A_0)$, then $\Psi^a$ is an equivalence of categories.
\end{proposition}

\section{First Examples of Chevalley Functors}

\begin{enumerate}
    \item If $\mathbf{A}$ is the dual of the category of commutative rings and $\mathbf{M}$ is the dual of the category of modules over varying commutative rings, the obvious functor $\mathrm{P} : \mathbf{M} \to \mathbf{A}$ is Chevalley.
    \item If $\mathbf{A}$ is a category with fibred products and $\mathbf{M} = \mathbf{Fl}(\mathbf{A})$ is the category of arrows in $\mathbf{A}$, the ``target'' functor $\mathrm{P} : \mathbf{M} \to \mathbf{A}$ is Chevalley.
    \item If $\mathrm{P} : \mathbf{M} \to \mathbf{A}$ and $\mathrm{Q} : \mathbf{N} \to \mathbf{M}$ are Chevalley, their composite $\mathrm{P} \circ \mathrm{Q}$ is Chevalley.
    \item If $\mathrm{P} : \mathbf{M} \to \mathbf{A}$ is Chevalley and $\mathbf{I}$ is any category, the functor $\mathrm{P}^{\mathbf{I}} : \mathbf{M}^{\mathbf{I}} \to \mathbf{A}^{\mathbf{I}}$ is Chevalley.
    \item In a cartesian diagram of categories
    \[
    \begin{tikzpicture}
        \node (X') at (0,2) {$\mathbf{X}'$};
        \node (M) at (2,2) {$\mathbf{M}$};
        \node (X) at (0,0) {$\mathbf{X}$};
        \node (A) at (2,0) {$\mathbf{A}$};
        \draw[->] (X') to node[above] {$f^*$} (M);
        \draw[->] (X') to node[left] {$g$} (X);
        \draw[->] (M) to node[right] {$\mathrm{P}$} (A);
        \draw[->] (X) to node[below] {$f$} (A);
    \end{tikzpicture}
    \]
    if $\mathbf{X}$ has fibred products, $f$ preserves fibred products, and $\mathrm{P}$ is Chevalley, then $f^*(\mathrm{P})$ is Chevalley.
\end{enumerate}

In a future publication, we will provide further examples of Chevalley categories and more precise criteria for determining whether $\Psi^a$ is faithful, fully faithful, or an equivalence when the fibres of $\mathbf{M}$ are algebraic categories (e.g., categories of modules).

\begin{thebibliography}{9}
\bibitem{Grothendieck1959} A. Grothendieck, Cat\'egories fibr\'ees et descente, \emph{S\'eminaire Bourbaki}, 1959.
\bibitem{Linton1969} F. E. J. Linton, Applied functorial semantics II, \emph{Springer Lecture Notes in Mathematics}, No. 80, 1969.
\bibitem{Chevalley1964} C. Chevalley, S\'eminaire sur la descente, 1964--1965 (unpublished).
\end{thebibliography}

\end{document}
