\documentclass{article}
\usepackage[utf8]{inputenc}
\usepackage{amsmath,amssymb,amsthm,mathtools}
\usepackage{tikz-cd}

\ProvidesPackage{journal}
\usepackage{graphicx}
\usepackage{mathtools}
\usepackage{hyphenat}
\usepackage{hyperref}
\usepackage{adjustbox}
\usepackage{listings}
\usepackage{verbatim}
\usepackage{xcolor}
\usepackage{amsfonts}
\usepackage{amscd}
\usepackage{amsmath}
\usepackage{amssymb}
\usepackage{amsthm}
\usepackage{tikz}
\usepackage{tikz-cd}
\usepackage{url}
\usepackage[utf8]{inputenc}
\usepackage[english,ukrainian]{babel}
\usepackage{float}
\usepackage{url}
\usepackage{tikz}
\usepackage{tikz-cd}
\usepackage[utf8]{inputenc}
\usepackage{graphicx}
\usepackage[utf8]{inputenc}
\usepackage[T1]{fontenc}
\usepackage{lmodern}
\usepackage{tocloft}
\usepackage{hyperref}
\usepackage{xcolor}
\usepackage[only,llbracket,rrbracket,llparenthesis,rrparenthesis]{stmaryrd}

\usetikzlibrary{babel}

\newcommand*{\incmap}{\hookrightarrow}
\newcommand*{\thead}[1]{\multicolumn{1}{c}{\bfseries #1}}
\renewcommand{\Join}{\vee} % Join operation symbol
\newcommand{\tabstyle}[0]{\scriptsize\ttfamily\fontseries{l}\selectfont}

\lstset{
  basicstyle=\footnotesize,
  inputencoding=utf8,
  identifierstyle=,
  literate=
{𝟎}{{\ensuremath{\mathbf{0}}}}1
{𝟏}{{\ensuremath{\mathbf{1}}}}1
{≔}{{\ensuremath{\mathrm{:=}}}}1
{α}{{\ensuremath{\mathrm{\alpha}}}}1
{ᵂ}{{\ensuremath{^W}}}1
{β}{{\ensuremath{\mathrm{\beta}}}}1
{γ}{{\ensuremath{\mathrm{\gamma}}}}1
{δ}{{\ensuremath{\mathrm{\delta}}}}1
{ε}{{\ensuremath{\mathrm{\varepsilon}}}}1
{ζ}{{\ensuremath{\mathrm{\zeta}}}}1
{η}{{\ensuremath{\mathrm{\eta}}}}1
{θ}{{\ensuremath{\mathrm{\theta}}}}1
{ι}{{\ensuremath{\mathrm{\iota}}}}1
{κ}{{\ensuremath{\mathrm{\kappa}}}}1
{λ}{{\ensuremath{\mathrm{\lambda}}}}1
{μ}{{\ensuremath{\mathrm{\mu}}}}1
{ν}{{\ensuremath{\mathrm{\nu}}}}1
{ξ}{{\ensuremath{\mathrm{\xi}}}}1
{π}{{\ensuremath{\mathrm{\mathnormal{\pi}}}}}1
{ρ}{{\ensuremath{\mathrm{\rho}}}}1
{σ}{{\ensuremath{\mathrm{\sigma}}}}1
{τ}{{\ensuremath{\mathrm{\tau}}}}1
{φ}{{\ensuremath{\mathrm{\varphi}}}}1
{χ}{{\ensuremath{\mathrm{\chi}}}}1
{ψ}{{\ensuremath{\mathrm{\psi}}}}1
{ω}{{\ensuremath{\mathrm{\omega}}}}1
{Π}{{\ensuremath{\mathrm{\Pi}}}}1
{Γ}{{\ensuremath{\mathrm{\Gamma}}}}1
{Δ}{{\ensuremath{\mathrm{\Delta}}}}1
{Θ}{{\ensuremath{\mathrm{\Theta}}}}1
{Λ}{{\ensuremath{\mathrm{\Lambda}}}}1
{Σ}{{\ensuremath{\mathrm{\Sigma}}}}1
{Φ}{{\ensuremath{\mathrm{\Phi}}}}1
{Ξ}{{\ensuremath{\mathrm{\Xi}}}}1
{Ψ}{{\ensuremath{\mathrm{\Psi}}}}1
{Ω}{{\ensuremath{\mathrm{\Omega}}}}1
{ℵ}{{\ensuremath{\aleph}}}1
{≤}{{\ensuremath{\leq}}}1
{≥}{{\ensuremath{\geq}}}1
{≠}{{\ensuremath{\neq}}}1
{≈}{{\ensuremath{\approx}}}1
{≡}{{\ensuremath{\equiv}}}1
{≃}{{\ensuremath{\simeq}}}1
{≤}{{\ensuremath{\leq}}}1
{≥}{{\ensuremath{\geq}}}1
{∂}{{\ensuremath{\partial}}}1
{∆}{{\ensuremath{\triangle}}}1 % or \laplace?
{∫}{{\ensuremath{\int}}}1
{∑}{{\ensuremath{\mathrm{\Sigma}}}}1
{→}{{\ensuremath{\rightarrow}}}1
{⊥}{{\ensuremath{\perp}}}1
{∞}{{\ensuremath{\infty}}}1
{∂}{{\ensuremath{\partial}}}1
{∓}{{\ensuremath{\mp}}}1
{±}{{\ensuremath{\pm}}}1
{×}{{\ensuremath{\times}}}1
{⊕}{{\ensuremath{\oplus}}}1
{⊗}{{\ensuremath{\otimes}}}1
{⊞}{{\ensuremath{\boxplus}}}1
{∇}{{\ensuremath{\nabla}}}1
{√}{{\ensuremath{\sqrt}}}1
{⬝}{{\ensuremath{\cdot}}}1
{•}{{\ensuremath{\cdot}}}1
{∘}{{\ensuremath{\circ}}}1
{⁻}{{\ensuremath{^{-}}}}1
{▸}{{\ensuremath{\blacktriangleright}}}1
{★}{{\ensuremath{\star}}}1
{∧}{{\ensuremath{\wedge}}}1
{∨}{{\ensuremath{\vee}}}1
{¬}{{\ensuremath{\neg}}}1
{⊢}{{\ensuremath{\vdash}}}1
{⟨}{{\ensuremath{\langle}}}1
{⟩}{{\ensuremath{\rangle}}}1
{↦}{{\ensuremath{\mapsto}}}1
{→}{{\ensuremath{\rightarrow}}}1
{↔}{{\ensuremath{\leftrightarrow}}}1
{⇒}{{\ensuremath{\Rightarrow}}}1
{⟹}{{\ensuremath{\Longrightarrow}}}1
{⇐}{{\ensuremath{\Leftarrow}}}1
{⟸}{{\ensuremath{\Longleftarrow}}}1
{∩}{{\ensuremath{\cap}}}1
{∪}{{\ensuremath{\cup}}}1
{·}{{\ensuremath{\cdot}}}1
{ᵢ}{{\ensuremath{_i}}}1
{ⱼ}{{\ensuremath{_j}}}1
{₊}{{\ensuremath{_+}}}1
{ℑ}{{\ensuremath{\Im}}}1
{𝒢}{{\ensuremath{\mathcal{G}}}}1
{ℕ}{{\ensuremath{\mathbb{N}}}}1
{𝟘}{{\ensuremath{\mathbb{0}}}}1
{ℤ}{{\ensuremath{\mathbb{Z}}}}1
{ℝ}{{\ensuremath{\mathbb{R}}}}1
{⊂}{{\ensuremath{\subseteq}}}1
{⊆}{{\ensuremath{\subseteq}}}1
{⊄}{{\ensuremath{\nsubseteq}}}1
{⊈}{{\ensuremath{\nsubseteq}}}1
{⊃}{{\ensuremath{\supseteq}}}1
{⊇}{{\ensuremath{\supseteq}}}1
{⊅}{{\ensuremath{\nsupseteq}}}1
{⊉}{{\ensuremath{\nsupseteq}}}1
{∈}{{\ensuremath{\in}}}1
{∉}{{\ensuremath{\notin}}}1
{∋}{{\ensuremath{\ni}}}1
{∌}{{\ensuremath{\notni}}}1
{∅}{{\ensuremath{\emptyset}}}1
{∖}{{\ensuremath{\setminus}}}1
{†}{{\ensuremath{\dag}}}1
{ℕ}{{\ensuremath{\mathbb{N}}}}1
{ℤ}{{\ensuremath{\mathbb{Z}}}}1
{ℝ}{{\ensuremath{\mathbb{R}}}}1
{ℚ}{{\ensuremath{\mathbb{Q}}}}1
{ℂ}{{\ensuremath{\mathbb{C}}}}1
{⌞}{{\ensuremath{\llcorner}}}1
{⌟}{{\ensuremath{\lrcorner}}}1
{⦃}{{\ensuremath{ \{\!| }}}1
{⦄}{{\ensuremath{ |\!\} }}}1
{ᵁ}{{\ensuremath{^U}}}1
{₋}{{\ensuremath{_{-}}}}1
{₁}{{\ensuremath{_1}}}1
{₂}{{\ensuremath{_2}}}1
{₃}{{\ensuremath{_3}}}1
{₄}{{\ensuremath{_4}}}1
{₅}{{\ensuremath{_5}}}1
{₆}{{\ensuremath{_6}}}1
{₇}{{\ensuremath{_7}}}1
{₈}{{\ensuremath{_8}}}1
{₉}{{\ensuremath{_9}}}1
{₀}{{\ensuremath{_0}}}1
{¹}{{\ensuremath{^1}}}1
{ₙ}{{\ensuremath{_n}}}1
{ₘ}{{\ensuremath{_m}}}1
{↑}{{\ensuremath{\uparrow}}}1
{↓}{{\ensuremath{\downarrow}}}1
{▸}{{\ensuremath{\triangleright}}}1
{∀}{{\ensuremath{\forall}}}1
{∃}{{\ensuremath{\exists}}}1
{λ}{{\ensuremath{\mathrm{\lambda}}}}1
{=}{{\ensuremath{=}}}1
{<}{{\ensuremath{\textless}}}1
{>}{{\ensuremath{\textgreater}}}1
{_}{{$\_$}}1
{(}{(}1
{(}{(}1
{‖}{{\ensuremath{\Vert}}}1
{+}{{+}}1
{*}{{*}}1,
}

\theoremstyle{definition}
\newtheorem{definition}{Definition}
\newtheorem{theorem}{Theorem}
\newtheorem{lemma}{Lemma}
\newtheorem{example}{Example}



\begin{document}

\title{Issue XLI: Local Homotopy Type Theory}
\author{Maksym Sokhatskyi $^1$}
\date{ $^1$ National Technical University of Ukraine \\
       \small Igor Sikorsky Kyiv Polytechnical Institute \\
       \today }

\maketitle

\begin{abstract}
\end{abstract}

{\bf Keywords}: Motivic Stable Homotopy Theory

\ifincludeTOC
  \tableofcontents
\fi

\section{Local Homotopy Type Theory}

Motivic homotopy theory, introduced by Morel and Voevodsky, extends classical homotopy
theory to the setting of algebraic geometry, treating schemes as analogous to topological spaces.
A central object in this framework is the category of \emph{T-spectra}, which generalizes
the notion of spectra in stable homotopy theory to the motivic context, where
the circle \( S^1 \) is replaced by the Tate object \( \T = \Aone / (\Aone \setminus \{0\}) \).
John F. Jardine’s work on motivic symmetric spectra provides a categorical model for the
motivic stable category, equipped with a symmetric monoidal smash product, enabling rich
interactions between algebraic and topological structures \cite{Jardine2000MotivicSS}.

This article formalizes the category of T-spectra, emphasizing Jardine’s contributions.
We define T-spectra and symmetric T-spectra, describe their model category structure,
and present key theorems on stable equivalences and monoidal properties. Applications
to algebraic geometry, such as the study of motivic cohomology and algebraic K-theory, are discussed.

We assume familiarity with basic category theory and algebraic geometry. Below, we outline essential concepts.

\begin{definition}
Let \( S \) be a Noetherian scheme of finite Krull dimension. The category \( \Sm_S \) consists of smooth schemes of finite type over \( S \), with morphisms being scheme morphisms over \( S \).
\end{definition}

\begin{definition}
A \emph{simplicial presheaf} on \( \Sm_S \) is a contravariant functor from \( \Sm_S \) to the category of simplicial sets. The category of simplicial presheaves, denoted \( \Spt(\Sm_S) \), is equipped with a proper closed simplicial model structure, as constructed by Morel and Voevodsky \cite{MorelVoevodsky1999}.
\end{definition}

\begin{remark}
The \emph{Nisnevich topology} on \( \Sm_S \), denoted \( \Nis \), is a Grothendieck topology coarser than the Zariski topology but finer than the étale topology. It is crucial for defining the motivic model category.
\end{remark}

\subsection{Definition of T-Spectra}

In motivic homotopy theory, the Tate object \( \T \) plays the role of the suspension functor. We define T-spectra as follows.

\begin{definition}
A \emph{T-spectrum} over a scheme \( S \) is a sequence of pointed simplicial presheaves \( E = \{ E_n \}_{n \geq 0} \) on \( \Sm_S \), equipped with structure maps \( \sigma_n: \T \wedge E_n \to E_{n+1} \), where \( \wedge \) denotes the smash product of pointed presheaves. The category of T-spectra, denoted \( \Spt^\T_S \), has morphisms given by sequences of maps \( f_n: E_n \to F_n \) compatible with the structure maps.
\end{definition}

\begin{example}
The \emph{motivic sphere spectrum} \( \Ssphere \) is a T-spectrum with \( \Ssphere_n = \T^{\wedge n} \), where \( \T^{\wedge n} \) is the \( n \)-fold smash product of \( \T \), and structure maps given by the identity.
\end{example}

\subsection{Symmetric T-Spectra}

Jardine’s work focuses on symmetric T-spectra, which incorporate symmetric group actions to define a robust smash product.

\begin{definition}
A \emph{symmetric T-spectrum} over \( S \) is a T-spectrum \( E = \{ E_n \}_{n \geq 0} \) where each \( E_n \) is equipped with an action of the symmetric group \( \Sigma_n \), and the structure maps \( \sigma_n: \T \wedge E_n \to E_{n+1} \) are \( \Sigma_n \)-equivariant with respect to the trivial action on \( \T \). The category of symmetric T-spectra is denoted \( \Spt^{\T, \Sigma}_S \).
\end{definition}

\begin{remark}
The symmetric structure allows for a well-defined internal smash product, making \( \Spt^{\T, \Sigma}_S \) a symmetric monoidal category \cite{Jardine2000MotivicSS}.
\end{remark}

\subsection{Model Category Structure}

Jardine establishes a model category structure on \( \Spt^\T_S \) and \( \Spt^{\T, \Sigma}_S \).

\begin{theorem}
The category \( \Spt^\T_S \) admits a proper closed simplicial model structure where:
\begin{itemize}
    \item \emph{Weak equivalences} are maps \( f: E \to F \) inducing isomorphisms on stable homotopy groups in the Nisnevich topology.
    \item \emph{Cofibrations} are monomorphisms.
    \item \emph{Fibrations} are defined via the right lifting property with respect to trivial cofibrations.
\end{itemize}
A Bousfield localization of this model structure with respect to stable weak equivalences yields the \emph{motivic stable category} \cite{Jardine2000MotivicSS}.
\end{theorem}

\begin{theorem}
The category \( \Spt^{\T, \Sigma}_S \) is a cofibrantly generated, symmetric monoidal model category satisfying the monoid axiom. The smash product \( \wedge \) is an internal symmetric monoidal structure, with unit the sphere spectrum \( \Ssphere \).
\end{theorem}

\subsection{Key Theorems}

Jardine’s results provide a categorical foundation for motivic stable homotopy theory.

\begin{theorem}
The motivic stable category, obtained as the homotopy category of \( \Spt^{\T, \Sigma}_S \), is equivalent to the localization of \( \Spt^\T_S \) at stable weak equivalences. Stable equivalences in this category are stable homotopy isomorphisms in the Nisnevich topology \cite{Jardine2000MotivicSS}.
\end{theorem}

\begin{theorem}
The symmetric smash product \( \wedge \) on \( \Spt^{\T, \Sigma}_S \) is associative, commutative, and unital up to homotopy, making the motivic stable category a symmetric monoidal category with unit \( \Ssphere \).
\end{theorem}

\begin{example}
The Eilenberg-MacLane spectrum \( \HA \) for an abelian group \( A \) is a symmetric T-spectrum, with \( \HA_n = K(A, n) \), the simplicial presheaf representing motivic cohomology. Its homotopy groups recover motivic cohomology groups.
\end{example}

\subsection{Conclusion}

T-spectra provide a framework for studying phenomena in algebraic geometry and stable homotopy theory:
- \emph{Motivic Cohomology}: The spectrum \( \HA \) represents motivic cohomology, connecting algebraic cycles to homotopy theory.
- \emph{Algebraic K-Theory}: Voevodsky’s motivic spectrum for K-theory, refined by Jardine’s symmetric structures, links K-theory to stable homotopy \cite{Jardine2000MotivicSS}.
- \emph{Algebraic Geometry}: The motivic stable category facilitates the study of Gysin triangles and oriented spectra, generalizing classical results in algebraic topology \cite{Jardine2000MotivicSS}.

The category of T-spectra, as developed by Jardine, provides a powerful framework for motivic homotopy theory. By equipping T-spectra with a symmetric monoidal smash product and a robust model category structure, Jardine’s work bridges algebraic geometry and stable homotopy theory. Future directions include exploring representability theorems for presheaves of spectra and applications to topological modular forms \cite{Jardine2011RepresentabilityTF}.

\bibliographystyle{plain}
\begin{thebibliography}{9}
\bibitem{Jardine2000MotivicSS} Jardine, J.F., \emph{Motivic symmetric spectra}, Documenta Mathematica, 2000.[](https://www.semanticscholar.org/paper/Motivic-symmetric-spectra-Jardine/6971f0bfc4423bc6adbf3c85cec59e8761d27757)
\bibitem{Jardine2011RepresentabilityTF} Jardine, J.F., \emph{Representability theorems for presheaves of spectra}, Journal of Pure and Applied Algebra, 2011, vol. 215, pp. 77--88.[](https://www.semanticscholar.org/paper/Representability-theorems-for-presheaves-of-spectra-Jardine/bafdec48a84fe467a4becaae929540c34af2da76)
\bibitem{MorelVoevodsky1999} Morel, F., Voevodsky, V., \emph{A¹-homotopy theory of schemes}, Publications Mathématiques de l’I.H.É.S., 1999.
\end{thebibliography}

\end{document}
