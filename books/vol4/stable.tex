\documentclass{article}
\usepackage[utf8]{inputenc}
\usepackage{amsmath, amssymb, amsthm}
\usepackage{enumitem}
\usepackage{booktabs}

\ProvidesPackage{journal}
\usepackage{caption}
\usepackage{hyperref}
\usepackage{epigraph}
\usepackage{listings}
\usepackage{amsmath}
\usepackage{amssymb}
\usepackage{amsthm}
\usepackage{tikz}
\usepackage{url}
\usepackage[english,ukrainian]{babel}
\usepackage[utf8]{inputenc}
\usepackage[T1]{fontenc}
\usepackage[only,llbracket,rrbracket,llparenthesis,rrparenthesis]{stmaryrd}

\newcommand*{\incmap}{\hookrightarrow}
\newcommand*{\thead}[1]{\multicolumn{1}{c}{\bfseries #1}}
\renewcommand{\Join}{\vee} % Join operation symbol
\newcommand{\tabstyle}[0]{\scriptsize\ttfamily\fontseries{l}\selectfont}

\lstset{
  basicstyle=\footnotesize,
  inputencoding=utf8,
  identifierstyle=,
  literate=
{𝟎}{{\ensuremath{\mathbf{0}}}}1
{𝟏}{{\ensuremath{\mathbf{1}}}}1
{≔}{{\ensuremath{\mathrm{:=}}}}1
{α}{{\ensuremath{\mathrm{\alpha}}}}1
{ᵂ}{{\ensuremath{^W}}}1
{β}{{\ensuremath{\mathrm{\beta}}}}1
{γ}{{\ensuremath{\mathrm{\gamma}}}}1
{δ}{{\ensuremath{\mathrm{\delta}}}}1
{ε}{{\ensuremath{\mathrm{\varepsilon}}}}1
{ζ}{{\ensuremath{\mathrm{\zeta}}}}1
{η}{{\ensuremath{\mathrm{\eta}}}}1
{θ}{{\ensuremath{\mathrm{\theta}}}}1
{ι}{{\ensuremath{\mathrm{\iota}}}}1
{κ}{{\ensuremath{\mathrm{\kappa}}}}1
{λ}{{\ensuremath{\mathrm{\lambda}}}}1
{μ}{{\ensuremath{\mathrm{\mu}}}}1
{ν}{{\ensuremath{\mathrm{\nu}}}}1
{ξ}{{\ensuremath{\mathrm{\xi}}}}1
{π}{{\ensuremath{\mathrm{\mathnormal{\pi}}}}}1
{ρ}{{\ensuremath{\mathrm{\rho}}}}1
{σ}{{\ensuremath{\mathrm{\sigma}}}}1
{τ}{{\ensuremath{\mathrm{\tau}}}}1
{φ}{{\ensuremath{\mathrm{\varphi}}}}1
{χ}{{\ensuremath{\mathrm{\chi}}}}1
{ψ}{{\ensuremath{\mathrm{\psi}}}}1
{ω}{{\ensuremath{\mathrm{\omega}}}}1
{Π}{{\ensuremath{\mathrm{\Pi}}}}1
{Γ}{{\ensuremath{\mathrm{\Gamma}}}}1
{Δ}{{\ensuremath{\mathrm{\Delta}}}}1
{Θ}{{\ensuremath{\mathrm{\Theta}}}}1
{Λ}{{\ensuremath{\mathrm{\Lambda}}}}1
{Σ}{{\ensuremath{\mathrm{\Sigma}}}}1
{Φ}{{\ensuremath{\mathrm{\Phi}}}}1
{Ξ}{{\ensuremath{\mathrm{\Xi}}}}1
{Ψ}{{\ensuremath{\mathrm{\Psi}}}}1
{Ω}{{\ensuremath{\mathrm{\Omega}}}}1
{ℵ}{{\ensuremath{\aleph}}}1
{≤}{{\ensuremath{\leq}}}1
{≥}{{\ensuremath{\geq}}}1
{≠}{{\ensuremath{\neq}}}1
{≈}{{\ensuremath{\approx}}}1
{≡}{{\ensuremath{\equiv}}}1
{≃}{{\ensuremath{\simeq}}}1
{≤}{{\ensuremath{\leq}}}1
{≥}{{\ensuremath{\geq}}}1
{∂}{{\ensuremath{\partial}}}1
{∆}{{\ensuremath{\triangle}}}1 % or \laplace?
{∫}{{\ensuremath{\int}}}1
{∑}{{\ensuremath{\mathrm{\Sigma}}}}1
{→}{{\ensuremath{\rightarrow}}}1
{⊥}{{\ensuremath{\perp}}}1
{∞}{{\ensuremath{\infty}}}1
{∂}{{\ensuremath{\partial}}}1
{∓}{{\ensuremath{\mp}}}1
{±}{{\ensuremath{\pm}}}1
{×}{{\ensuremath{\times}}}1
{⊕}{{\ensuremath{\oplus}}}1
{⊗}{{\ensuremath{\otimes}}}1
{⊞}{{\ensuremath{\boxplus}}}1
{∇}{{\ensuremath{\nabla}}}1
{√}{{\ensuremath{\sqrt}}}1
{⬝}{{\ensuremath{\cdot}}}1
{•}{{\ensuremath{\cdot}}}1
{∘}{{\ensuremath{\circ}}}1
{⁻}{{\ensuremath{^{-}}}}1
{▸}{{\ensuremath{\blacktriangleright}}}1
{★}{{\ensuremath{\star}}}1
{∧}{{\ensuremath{\wedge}}}1
{∨}{{\ensuremath{\vee}}}1
{¬}{{\ensuremath{\neg}}}1
{⊢}{{\ensuremath{\vdash}}}1
{⟨}{{\ensuremath{\langle}}}1
{⟩}{{\ensuremath{\rangle}}}1
{↦}{{\ensuremath{\mapsto}}}1
{→}{{\ensuremath{\rightarrow}}}1
{↔}{{\ensuremath{\leftrightarrow}}}1
{⇒}{{\ensuremath{\Rightarrow}}}1
{⟹}{{\ensuremath{\Longrightarrow}}}1
{⇐}{{\ensuremath{\Leftarrow}}}1
{⟸}{{\ensuremath{\Longleftarrow}}}1
{∩}{{\ensuremath{\cap}}}1
{∪}{{\ensuremath{\cup}}}1
{·}{{\ensuremath{\cdot}}}1
{ᵢ}{{\ensuremath{_i}}}1
{ⱼ}{{\ensuremath{_j}}}1
{₊}{{\ensuremath{_+}}}1
{ℑ}{{\ensuremath{\Im}}}1
{𝒢}{{\ensuremath{\mathcal{G}}}}1
{ℕ}{{\ensuremath{\mathbb{N}}}}1
{𝟘}{{\ensuremath{\mathbb{0}}}}1
{ℤ}{{\ensuremath{\mathbb{Z}}}}1
{ℝ}{{\ensuremath{\mathbb{R}}}}1
{⊂}{{\ensuremath{\subseteq}}}1
{⊆}{{\ensuremath{\subseteq}}}1
{⊄}{{\ensuremath{\nsubseteq}}}1
{⊈}{{\ensuremath{\nsubseteq}}}1
{⊃}{{\ensuremath{\supseteq}}}1
{⊇}{{\ensuremath{\supseteq}}}1
{⊅}{{\ensuremath{\nsupseteq}}}1
{⊉}{{\ensuremath{\nsupseteq}}}1
{∈}{{\ensuremath{\in}}}1
{∉}{{\ensuremath{\notin}}}1
{∋}{{\ensuremath{\ni}}}1
{∌}{{\ensuremath{\notni}}}1
{∅}{{\ensuremath{\emptyset}}}1
{∖}{{\ensuremath{\setminus}}}1
{†}{{\ensuremath{\dag}}}1
{ℕ}{{\ensuremath{\mathbb{N}}}}1
{ℤ}{{\ensuremath{\mathbb{Z}}}}1
{ℝ}{{\ensuremath{\mathbb{R}}}}1
{ℚ}{{\ensuremath{\mathbb{Q}}}}1
{ℂ}{{\ensuremath{\mathbb{C}}}}1
{⌞}{{\ensuremath{\llcorner}}}1
{⌟}{{\ensuremath{\lrcorner}}}1
{⦃}{{\ensuremath{ \{\!| }}}1
{⦄}{{\ensuremath{ |\!\} }}}1
{ᵁ}{{\ensuremath{^U}}}1
{₋}{{\ensuremath{_{-}}}}1
{₁}{{\ensuremath{_1}}}1
{₂}{{\ensuremath{_2}}}1
{₃}{{\ensuremath{_3}}}1
{₄}{{\ensuremath{_4}}}1
{₅}{{\ensuremath{_5}}}1
{₆}{{\ensuremath{_6}}}1
{₇}{{\ensuremath{_7}}}1
{₈}{{\ensuremath{_8}}}1
{₉}{{\ensuremath{_9}}}1
{₀}{{\ensuremath{_0}}}1
{¹}{{\ensuremath{^1}}}1
{ₙ}{{\ensuremath{_n}}}1
{ₘ}{{\ensuremath{_m}}}1
{↑}{{\ensuremath{\uparrow}}}1
{↓}{{\ensuremath{\downarrow}}}1
{▸}{{\ensuremath{\triangleright}}}1
{∀}{{\ensuremath{\forall}}}1
{∃}{{\ensuremath{\exists}}}1
{λ}{{\ensuremath{\mathrm{\lambda}}}}1
{=}{{\ensuremath{=}}}1
{<}{{\ensuremath{\textless}}}1
{>}{{\ensuremath{\textgreater}}}1
{_}{{$\_$}}1
{(}{(}1
{(}{(}1
{‖}{{\ensuremath{\Vert}}}1
{+}{{+}}1
{*}{{*}}1,
}

\setlength{\parindent}{15pt}

\theoremstyle{definition}
\newtheorem{definition}{Definition}
\newtheorem{theorem}{Theorem}
\newtheorem{lemma}{Lemma}
\newtheorem{example}{Example}

\definecolor{LightGray}{rgb}{0.8,0.8,0.8}
\definecolor{LightGray25}{rgb}{0.9,0.9,0.9}
\definecolor{ZimaBlue}{HTML}{A1D8FC}

\newif\ifincludeTOC
\includeTOCtrue

\lefthyphenmin=1
\hyphenpenalty=100
\tolerance=11000
\emergencystretch=1em
\hfuzz=2pt
\vfuzz=2pt



\begin{document}

\title{Issue XXXV: Cohomology and Spectra}
\author{Maksym Sokhatskyi $^1$}
\date{ $^1$ National Technical University of Ukraine \\
       \small Igor Sikorsky Kyiv Polytechnical Institute \\
       \today }

\maketitle

\begin{abstract}
This article presents formal definitions and theorems for ordinary and
generalized cohomology theories, unstable and stable spectra,
and spectral sequences in Abelian categories, including the Serre,
Atiyah-Hirzebruch, Leray, Eilenberg-Moore, Hochschild-Serre,
Filtered Complex, Chromatic, Adams, and Bockstein spectral sequences.
We define slopes, sheets, coordinates, quadrants, complex filtrations,
and double complexes. Additionally, we explore the categorical foundations
of cohomology theories and spectra, including their relationships to algebra,
homological algebra, and stable homotopy theory, through isomorphisms,
 analogies, and instances.
\end{abstract}

\ifincludeTOC
\tableofcontents
\fi

\newpage
\section{Stable Homotopy Type Theory}

\subsection{Ordinary Cohomology Theories}

\begin{definition}
An \emph{ordinary cohomology theory} on the category of topological spaces and pairs is a contravariant functor \( H^*(-; G): \text{Top}^{\text{op}} \to \text{GrAb} \), assigning to each pair \( (X, A) \) a sequence of abelian groups \( \{ H^n(X, A; G) \}_{n \in \Z} \), with coefficient group \( G \), satisfying:
\begin{enumerate}
    \item \emph{Homotopy}: If \( f \simeq g: (X, A) \to (Y, B) \), then \( f^* = g^*: H^n(Y, B; G) \to H^n(X, A; G) \).
    \item \emph{Exactness}: For \( (X, A) \), there is a long exact sequence: \( \cdots \to H^n(X, A; G) \to H^n(X; G) \to H^n(A; G) \xrightarrow{\delta} H^{n+1}(X, A; G) \to \cdots \)
    \item \emph{Excision}: For \( U \subset A \) with \( \overline{U} \subset \text{int}(A) \), the inclusion \( (X \setminus U, A \setminus U) \hookrightarrow (X, A) \) induces isomorphisms \( H^n(X, A; G) \cong H^n(X \setminus U, A \setminus U; G) \).
    \item \emph{Additivity}: For \( X = \bigsqcup X_i \), \( H^n(X; G) \cong \bigoplus H^n(X_i; G) \).
    \item \emph{Dimension}: For a point \( \pt \), \( H^n(\pt; G) = \begin{cases} G & n = 0 \\ 0 & n \neq 0 \end{cases} \).
\end{enumerate}
\end{definition}

\subsection{Generalized Cohomology Theories}

\begin{definition}
A \emph{generalized cohomology theory} is a contravariant
functor \( h^*: \text{Top}^{\text{op}} \to \text{GrAb} \),
assigning to each pair \( (X, A) \) a
sequence \( \{ h^n(X, A) \}_{n \in \Z} \), satisfying:
\begin{enumerate}
    \item \emph{Homotopy}, \emph{Exactness}, \emph{Excision}, and \emph{Additivity} as in \textbf{Definition 1}.
    \item \emph{Suspension}: There is a natural isomorphism \( h^n(X, A) \cong h^{n+1}(\Sigma X, \Sigma A) \), where \( \Sigma \) is the reduced suspension.
\end{enumerate}
The groups \( h^n(\pt) \) form a graded ring, the coefficients of \( h^* \).
\end{definition}

\begin{theorem}
Every generalized cohomology theory \( h^* \) is representable
by a spectrum \( E = \{ E_n, \sigma_n: \Sigma E_n \to E_{n+1} \} \),
with \( h^n(X) \cong [X, E_n]_* \), where \( [-, -]_* \) denotes
pointed homotopy classes.
\end{theorem}

\subsection{Unstable and Stable Spectra}

\begin{definition}
A \emph{spectrum} is a sequence of pointed spaces \( \{ E_n \}_{n \in I} \),
where \( I \subseteq \Z \), with structure maps \( \sigma_n: \Sigma E_n \to E_{n+1} \). It is:
\begin{itemize}
    \item \emph{Unstable} if \( I \subseteq \Z_{\geq 0} \).
    \item \emph{Stable} if \( I = \Z \) and each \( \sigma_n \) is a homotopy equivalence.
\end{itemize}
\end{definition}

\begin{theorem}
For an unstable spectrum \( E \), the functor \( X \mapsto [X, E_n]_* \) defines a cohomology theory on spaces of dimension \( \leq n \). For a stable spectrum \( E \), the functor \( h^n(X) = [X, E_n]_* \) defines a generalized cohomology theory.
\end{theorem}

\subsection{Categorical Interpretation}

This section explores the categorical foundations of ordinary and generalized cohomology theories, their associated spectra, and the relationships between algebraic and topological categories through isomorphisms, analogies, and instances. We formalize these structures and highlight their categorical nuances isomorphisms and non-isomorphic relationships, drawing on frameworks like algebra, homological algebra, and stable homotopy theory.

\begin{definition}
The \emph{category of spectra}, denoted \( \Spectra \), is the category whose objects are stable spectra \( E = \{ E_n, \sigma_n: \Sigma E_n \to E_{n+1} \} \), where \( E_n \) are pointed spaces and \( \sigma_n \) are homotopy equivalences. Morphisms are collections of maps \( f_n: E_n \to F_n \) compatible with structure maps. The \emph{stable homotopy category} is the localization of \( \Spectra \) at weak equivalences (maps inducing isomorphisms on homotopy groups).
\end{definition}

\begin{definition}
An \emph{ordinary cohomology theory} is a functor \( H^*(-; G): \text{Top}^{\text{op}} \to \text{GrAb} \) satisfying the Eilenberg-Steenrod axioms (Definition 1). Categorically, it is represented by the Eilenberg-MacLane spectrum \( \HA \), where \( A \in \Ab \), with \( H^n(X; A) \cong [X, \HA_n]_* \).
\end{definition}

\begin{definition}
A \emph{generalized cohomology theory} is a functor \( h^*: \text{Top}^{\text{op}} \to \text{GrAb} \) satisfying the axioms of Definition 2. It is representable in \( \Spectra \), with \( h^n(X) \cong [X, E_n]_* \) for a spectrum \( E \).
\end{definition}

\begin{theorem}[Brown Representability]
Every generalized cohomology theory \( h^* \) on \( \text{Top} \) is representable by a spectrum \( E \in \Spectra \), i.e., there exists \( E \) such that \( h^n(X) \cong [X, E_n]_* \) for all \( X \in \text{Top} \).
\end{theorem}

\begin{theorem}
The stable homotopy category \( \Spectra \) is a triangulated category, with distinguished triangles corresponding to cofiber sequences. It is equivalent to the category of spectra localized at weak equivalences.
\end{theorem}

\begin{theorem}
The functor \( A \mapsto \HA \) from \( \Ab \) to \( \Spectra \), mapping an abelian group to its Eilenberg-MacLane spectrum, is faithful but not full. The induced functor on ordinary cohomology theories to generalized cohomology theories is an embedding of categories.
\end{theorem}

\subsubsection{Algebraic and Spectral Correspondences}

Mathematics is unified through \emph{isomorphisms} (categorical equivalences),
\emph{analogies} (functorial similarities), and \emph{instances} (specific subcategories or objects).
We present a correspondence table linking Algebra (\( \Ab \)), Homological Algebra (\( \Ch(\Z) \)),
Ordinary Cohomology, K-Theory, Superalgebra, and Stable Spectra (\( \Spectra \)).

\begin{definition}[Isomorphism]
An \textbf{isomorphism} in a category \(\mathcal{C}\) is
a morphism \(f: A \to B\) with an inverse \(g: B \to A\)
such that \(g \circ f = \text{id}_A\) and \(f \circ g = \text{id}_B\).
For categories, an isomorphism is an equivalence, i.e.,
a functor \(F: \mathcal{C} \to \mathcal{D}\) with a quasi-inverse \(G: \mathcal{D} \to \mathcal{C}\).
\end{definition}

\begin{definition}[Analogy]
A non-isomorphic \textbf{analogy} is a structural similarity between objects or
categories, captured by functors that preserve some properties but not all,
ensuring no categorical equivalence.
\end{definition}

\begin{definition}[Instance]
An \textbf{instance} is a specific object or subcategory within a broader category,
embedded via a faithful functor. A column in the table is an instance of another
if its structures are special cases of the latter’s, maintaining non-isomorphic
distinctions from other categories.
\end{definition}

\begin{table}[h]
\centering
\caption{Algebraic and Spectral Correspondences}
\begin{tabular}{l|c|c|c|c}
\toprule
Category & Object & Ring & Initial Unit & Operations \\
\midrule
Algebra & Abelian group & Ring & \( \Z \) & \( \oplus, \otimes \) \\
Homological Algebra & Chain complex & dg-ring & \( \Z[0] \) & \( \oplus, \otimes \) \\
Superalgebra & \( \Z/2\Z \)-graded Ab & \( \Z/2\Z \)-graded Ring & \( \Z \) & \( \oplus, \otimes \) \\
Ordinary Cohomology & Cohomology \( H^*(-; A) \) & Graded ring & \( H^*(-; \Z) \) & \( \oplus, \otimes \) \\
Complex K-Theory & Graded abelian group & Graded ring & \( \KU \) & \( \vee, \wedge \) \\
Real K-Theory & Graded abelian group & Graded ring & \( \KO \) & \( \vee, \wedge \) \\
Stable Spectra & Stable spectrum & Ring spectrum & \( \Ssphere \) & \( \vee, \wedge \) \\
\bottomrule
\end{tabular}
\end{table}

\begin{itemize}
    \item \emph{Isomorphisms}: Rare, e.g., \( \Ab \cong \Mod_\Z \). Most relationships are non-isomorphic.
    \item \emph{Analogies}: The tensor product \( \otimes \) in \( \Ab \) and smash product \( \wedge \) in \( \Spectra \) are analogous, but \( \Ab \not\cong \Spectra \) due to \( \Spectra \)’s triangulated structure.
    \item \emph{Instances}: \( \KU \), \( \KO \), and \( \HA \) are instances of \( \Spectra \). Superalgebra is an instance of \( \Ab \) via the forgetful functor.
\end{itemize}

\begin{example}
The functor \( A \mapsto A[0] \) embeds \( \Ab \) into \( \Ch(\Z) \), but \( \Ch(\Z) \not\cong \Ab \) due to differentials. Similarly, \( \HA: \Ab \to \Spectra \) embeds abelian groups as Eilenberg-MacLane spectra, but \( \Spectra \)’s stable phenomena (e.g., suspension equivalences) distinguish it.
\end{example}

\begin{remark}
Non-isomorphic analogies require careful handling. Conflating \( \wedge \) in \( \Spectra \) with \( \otimes \) in \( \Ab \) can lead to errors in spectral sequence computations, as \( \wedge \) introduces higher Tor terms.
\end{remark}

\newpage
\subsection{Spectral Sequences}

\begin{definition}
A \emph{spectral sequence} in an Abelian category \( \mathcal{A} \) is a collection of objects \( \{ E_r^{p,q} \}_{r \geq 1, p, q \in \Z} \), \( E_r^{p,q} \in \mathcal{A} \), with differentials:
\[
d_r^{p,q}: E_r^{p,q} \to E_r^{p + a_r, q + b_r},
\]
such that:
\begin{enumerate}
    \item \( d_r \circ d_r = 0 \).
    \item \( E_{r+1}^{p,q} = H^{p,q}(E_r, d_r) = \ker(d_r^{p,q}) / \im(d_r^{p - a_r, q - b_r}) \).
    \item There exists a graded object \( H^n \in \mathcal{A} \) with filtration \( F_p H^{p+q} \subseteq H^{p+q} \), such that:
    \[
    E_\infty^{p,q} \cong F_p H^{p+q} / F_{p-1} H^{p+q}.
    \]
\end{enumerate}
The sequence is \emph{first-quadrant} if \( E_r^{p,q} = 0 \) for \( p < 0 \) or \( q < 0 \).
\end{definition}

\begin{definition}
The \emph{\( r \)-th sheet} of a spectral sequence is the collection \( \{ E_r^{p,q} \}_{p,q} \). The indices \( (p, q) \) are \emph{coordinates}, with \( p \) the filtration degree and \( q \) the complementary degree, satisfying total degree \( n = p + q \). The \emph{slope} of \( d_r: E_r^{p,q} \to E_r^{p+r, q-r+1} \) is \( \frac{-r+1}{r} \).
\end{definition}

\begin{definition}
A \emph{filtered complex} in \( \mathcal{A} = \text{Ab} \) is a chain complex \( (C_*, \partial) \) with a filtration \( \cdots \subseteq F_{p-1} C_n \subseteq F_p C_n \subseteq F_{p+1} C_n \subseteq \cdots \), compatible with \( \partial \). A \emph{double complex} is a bigraded object \( C_{p,q} \) with differentials \( d^h: C_{p,q} \to C_{p-1,q} \), \( d^v: C_{p,q} \to C_{p,q-1} \), satisfying \( d^h d^h = d^v d^v = d^h d^v + d^v d^h = 0 \). The total complex is \( \text{Tot}(C)_n = \bigoplus_{p+q=n} C_{p,q} \).
\end{definition}

\begin{theorem}
A filtered complex \( (C_*, F_p) \) induces a spectral sequence with:
\[
E_0^{p,q} = F_p C_{p+q} / F_{p-1} C_{p+q}, \quad E_1^{p,q} = H_{p+q}(F_p C / F_{p-1} C) \implies H_{p+q}(C).
\]
A double complex \( C_{p,q} \) with filtration by \( p \)-index induces:
\[
E_1^{p,q} = H_q^v(C_{p,*}), \quad d_1 = H(d^h) \implies H_{p+q}(\text{Tot}(C)).
\]
\end{theorem}

\newpage
\subsubsection{Serre Spectral Sequence}

\begin{theorem}
For a fibration \( F \to E \to B \) with \( B \) path-connected, there exists a first-quadrant spectral sequence:
\[
E_2^{p,q} = H^p(B; H^q(F; \Z)) \implies H^{p+q}(E; \Z),
\]
with \( d_r: E_r^{p,q} \to E_r^{p+r, q-r+1} \).
\end{theorem}

\subsubsection{Atiyah-Hirzebruch Spectral Sequence}

\begin{theorem}
For a generalized cohomology theory \( h^* \) and a CW-complex \( X \), there exists a spectral sequence:
\[
E_2^{p,q} = H^p(X; h^q(\pt)) \implies h^{p+q}(X),
\]
with \( d_r: E_r^{p,q} \to E_r^{p+r, q-r+1} \).
\end{theorem}

\subsubsection{Leray Spectral Sequence}

\begin{theorem}
For a continuous map \( f: X \to Y \) and a sheaf \( \mathcal{F} \) on \( X \), there exists a spectral sequence:
\[
E_2^{p,q} = H^p(Y; R^q f_* \mathcal{F}) \implies H^{p+q}(X; \mathcal{F}),
\]
with \( d_r: E_r^{p,q} \to E_r^{p+r, q-r+1} \).
\end{theorem}

\subsubsection{Eilenberg-Moore Spectral Sequence}

\begin{theorem}
For a pullback diagram with fibration \( F \to E \to B \), there exists a spectral sequence:
\[
E_2^{p,q} = \Tor_{H_*(B)}^{p,q}(H_*(F), R) \implies H_{p+q}(F; R),
\]
with \( d_r: E_r^{p,q} \to E_r^{p-r, q+r-1} \).
\end{theorem}

\subsubsection{Hochschild-Serre Spectral Sequence}

\begin{theorem}
For a group extension \( 1 \to N \to G \to Q \to 1 \), there exists a spectral sequence:
\[
E_2^{p,q} = H^p(Q; H^q(N; R)) \implies H^{p+q}(G; R),
\]
with \( d_r: E_r^{p,q} \to E_r^{p+r, q-r+1} \).
\end{theorem}

\subsubsection{Spectral Sequence of a Filtered Complex}

\begin{theorem}
For a filtered complex \( (C_*, F_p) \), there exists a spectral sequence:
\[
E_1^{p,q} = H_{p+q}(F_p C / F_{p-1} C) \implies H_{p+q}(C),
\]
with \( d_r: E_r^{p,q} \to E_r^{p-r, q+r-1} \).
\end{theorem}

\subsubsection{Chromatic Spectral Sequence}

\begin{theorem}
For a spectrum \( X \), there exists a spectral sequence:
\[
E_1^{n,k} = \pi_{n-k}(L_{K(k)} X) \implies \pi_{n-k}(X),
\]
where \( L_{K(k)} X \) is the localization at the \( k \)-th Morava K-theory, with \( d_r: E_r^{n,k} \to E_r^{n+1,k-r} \).
\end{theorem}

\subsubsection{Adams Spectral Sequence}

\begin{theorem}
For a spectrum \( X \) and prime \( p \), there exists a spectral sequence:
\[
E_2^{s,t} = \Ext_A^{s,t}(\Hom_*(X, \Z/p), \Z/p) \implies \pi_{t-s}(X_{(p)}),
\]
where \( A \) is the Steenrod algebra, with \( d_r: E_r^{s,t} \to E_r^{s+r, t+r-1} \).
\end{theorem}

\subsubsection{Bockstein Spectral Sequence}

\begin{theorem}
For a short exact sequence \( 0 \to R \to R' \to R'' \to 0 \) of coefficient rings, there exists a spectral sequence:
\[
E_1^{p,q} = H^{p+q}(X; R'') \implies H^{p+q}(X; R),
\]
with \( d_r: E_r^{p,q} \to E_r^{p+1,q-r} \).
\end{theorem}

\bibliographystyle{plain}
\begin{thebibliography}{9}
\bibitem{hatcher} Hatcher, A., \emph{Algebraic Topology}, Cambridge University Press, 2002.
\bibitem{switzer} Switzer, R., \emph{Algebraic Topology - Homology and Homotopy}, Springer, 1975.
\bibitem{kato} Kato, G., \emph{The Heart of Cohomology}, Springer, 2006.
\bibitem{adams} Adams, J.F., \emph{Stable Homotopy and Generalized Homology}, University of Chicago Press, 1974.
\end{thebibliography}

\end{document}
