\documentclass{article}
\usepackage[utf8]{inputenc}
\usepackage{amsmath, amssymb, amsthm}
\usepackage{enumitem}
\usepackage{booktabs}

\theoremstyle{plain}
\newtheorem{theorem}{Theorem}
\newtheorem{lemma}{Lemma}
\theoremstyle{definition}
\newtheorem{definition}{Definition}
\newtheorem{example}{Example}
\newtheorem{remark}{Remark}

\ProvidesPackage{journal}
\usepackage{graphicx}
\usepackage{mathtools}
\usepackage{hyphenat}
\usepackage{hyperref}
\usepackage{adjustbox}
\usepackage{listings}
\usepackage{verbatim}
\usepackage{xcolor}
\usepackage{amsfonts}
\usepackage{amscd}
\usepackage{amsmath}
\usepackage{amssymb}
\usepackage{amsthm}
\usepackage{tikz}
\usepackage{tikz-cd}
\usepackage{url}
\usepackage[utf8]{inputenc}
\usepackage[english,ukrainian]{babel}
\usepackage{float}
\usepackage{url}
\usepackage{tikz}
\usepackage{tikz-cd}
\usepackage[utf8]{inputenc}
\usepackage{graphicx}
\usepackage[utf8]{inputenc}
\usepackage[T1]{fontenc}
\usepackage{lmodern}
\usepackage{tocloft}
\usepackage{hyperref}
\usepackage{xcolor}
\usepackage[only,llbracket,rrbracket,llparenthesis,rrparenthesis]{stmaryrd}

\usetikzlibrary{babel}

\newcommand*{\incmap}{\hookrightarrow}
\newcommand*{\thead}[1]{\multicolumn{1}{c}{\bfseries #1}}
\renewcommand{\Join}{\vee} % Join operation symbol
\newcommand{\tabstyle}[0]{\scriptsize\ttfamily\fontseries{l}\selectfont}

\lstset{
  basicstyle=\footnotesize,
  inputencoding=utf8,
  identifierstyle=,
  literate=
{𝟎}{{\ensuremath{\mathbf{0}}}}1
{𝟏}{{\ensuremath{\mathbf{1}}}}1
{≔}{{\ensuremath{\mathrm{:=}}}}1
{α}{{\ensuremath{\mathrm{\alpha}}}}1
{ᵂ}{{\ensuremath{^W}}}1
{β}{{\ensuremath{\mathrm{\beta}}}}1
{γ}{{\ensuremath{\mathrm{\gamma}}}}1
{δ}{{\ensuremath{\mathrm{\delta}}}}1
{ε}{{\ensuremath{\mathrm{\varepsilon}}}}1
{ζ}{{\ensuremath{\mathrm{\zeta}}}}1
{η}{{\ensuremath{\mathrm{\eta}}}}1
{θ}{{\ensuremath{\mathrm{\theta}}}}1
{ι}{{\ensuremath{\mathrm{\iota}}}}1
{κ}{{\ensuremath{\mathrm{\kappa}}}}1
{λ}{{\ensuremath{\mathrm{\lambda}}}}1
{μ}{{\ensuremath{\mathrm{\mu}}}}1
{ν}{{\ensuremath{\mathrm{\nu}}}}1
{ξ}{{\ensuremath{\mathrm{\xi}}}}1
{π}{{\ensuremath{\mathrm{\mathnormal{\pi}}}}}1
{ρ}{{\ensuremath{\mathrm{\rho}}}}1
{σ}{{\ensuremath{\mathrm{\sigma}}}}1
{τ}{{\ensuremath{\mathrm{\tau}}}}1
{φ}{{\ensuremath{\mathrm{\varphi}}}}1
{χ}{{\ensuremath{\mathrm{\chi}}}}1
{ψ}{{\ensuremath{\mathrm{\psi}}}}1
{ω}{{\ensuremath{\mathrm{\omega}}}}1
{Π}{{\ensuremath{\mathrm{\Pi}}}}1
{Γ}{{\ensuremath{\mathrm{\Gamma}}}}1
{Δ}{{\ensuremath{\mathrm{\Delta}}}}1
{Θ}{{\ensuremath{\mathrm{\Theta}}}}1
{Λ}{{\ensuremath{\mathrm{\Lambda}}}}1
{Σ}{{\ensuremath{\mathrm{\Sigma}}}}1
{Φ}{{\ensuremath{\mathrm{\Phi}}}}1
{Ξ}{{\ensuremath{\mathrm{\Xi}}}}1
{Ψ}{{\ensuremath{\mathrm{\Psi}}}}1
{Ω}{{\ensuremath{\mathrm{\Omega}}}}1
{ℵ}{{\ensuremath{\aleph}}}1
{≤}{{\ensuremath{\leq}}}1
{≥}{{\ensuremath{\geq}}}1
{≠}{{\ensuremath{\neq}}}1
{≈}{{\ensuremath{\approx}}}1
{≡}{{\ensuremath{\equiv}}}1
{≃}{{\ensuremath{\simeq}}}1
{≤}{{\ensuremath{\leq}}}1
{≥}{{\ensuremath{\geq}}}1
{∂}{{\ensuremath{\partial}}}1
{∆}{{\ensuremath{\triangle}}}1 % or \laplace?
{∫}{{\ensuremath{\int}}}1
{∑}{{\ensuremath{\mathrm{\Sigma}}}}1
{→}{{\ensuremath{\rightarrow}}}1
{⊥}{{\ensuremath{\perp}}}1
{∞}{{\ensuremath{\infty}}}1
{∂}{{\ensuremath{\partial}}}1
{∓}{{\ensuremath{\mp}}}1
{±}{{\ensuremath{\pm}}}1
{×}{{\ensuremath{\times}}}1
{⊕}{{\ensuremath{\oplus}}}1
{⊗}{{\ensuremath{\otimes}}}1
{⊞}{{\ensuremath{\boxplus}}}1
{∇}{{\ensuremath{\nabla}}}1
{√}{{\ensuremath{\sqrt}}}1
{⬝}{{\ensuremath{\cdot}}}1
{•}{{\ensuremath{\cdot}}}1
{∘}{{\ensuremath{\circ}}}1
{⁻}{{\ensuremath{^{-}}}}1
{▸}{{\ensuremath{\blacktriangleright}}}1
{★}{{\ensuremath{\star}}}1
{∧}{{\ensuremath{\wedge}}}1
{∨}{{\ensuremath{\vee}}}1
{¬}{{\ensuremath{\neg}}}1
{⊢}{{\ensuremath{\vdash}}}1
{⟨}{{\ensuremath{\langle}}}1
{⟩}{{\ensuremath{\rangle}}}1
{↦}{{\ensuremath{\mapsto}}}1
{→}{{\ensuremath{\rightarrow}}}1
{↔}{{\ensuremath{\leftrightarrow}}}1
{⇒}{{\ensuremath{\Rightarrow}}}1
{⟹}{{\ensuremath{\Longrightarrow}}}1
{⇐}{{\ensuremath{\Leftarrow}}}1
{⟸}{{\ensuremath{\Longleftarrow}}}1
{∩}{{\ensuremath{\cap}}}1
{∪}{{\ensuremath{\cup}}}1
{·}{{\ensuremath{\cdot}}}1
{ᵢ}{{\ensuremath{_i}}}1
{ⱼ}{{\ensuremath{_j}}}1
{₊}{{\ensuremath{_+}}}1
{ℑ}{{\ensuremath{\Im}}}1
{𝒢}{{\ensuremath{\mathcal{G}}}}1
{ℕ}{{\ensuremath{\mathbb{N}}}}1
{𝟘}{{\ensuremath{\mathbb{0}}}}1
{ℤ}{{\ensuremath{\mathbb{Z}}}}1
{ℝ}{{\ensuremath{\mathbb{R}}}}1
{⊂}{{\ensuremath{\subseteq}}}1
{⊆}{{\ensuremath{\subseteq}}}1
{⊄}{{\ensuremath{\nsubseteq}}}1
{⊈}{{\ensuremath{\nsubseteq}}}1
{⊃}{{\ensuremath{\supseteq}}}1
{⊇}{{\ensuremath{\supseteq}}}1
{⊅}{{\ensuremath{\nsupseteq}}}1
{⊉}{{\ensuremath{\nsupseteq}}}1
{∈}{{\ensuremath{\in}}}1
{∉}{{\ensuremath{\notin}}}1
{∋}{{\ensuremath{\ni}}}1
{∌}{{\ensuremath{\notni}}}1
{∅}{{\ensuremath{\emptyset}}}1
{∖}{{\ensuremath{\setminus}}}1
{†}{{\ensuremath{\dag}}}1
{ℕ}{{\ensuremath{\mathbb{N}}}}1
{ℤ}{{\ensuremath{\mathbb{Z}}}}1
{ℝ}{{\ensuremath{\mathbb{R}}}}1
{ℚ}{{\ensuremath{\mathbb{Q}}}}1
{ℂ}{{\ensuremath{\mathbb{C}}}}1
{⌞}{{\ensuremath{\llcorner}}}1
{⌟}{{\ensuremath{\lrcorner}}}1
{⦃}{{\ensuremath{ \{\!| }}}1
{⦄}{{\ensuremath{ |\!\} }}}1
{ᵁ}{{\ensuremath{^U}}}1
{₋}{{\ensuremath{_{-}}}}1
{₁}{{\ensuremath{_1}}}1
{₂}{{\ensuremath{_2}}}1
{₃}{{\ensuremath{_3}}}1
{₄}{{\ensuremath{_4}}}1
{₅}{{\ensuremath{_5}}}1
{₆}{{\ensuremath{_6}}}1
{₇}{{\ensuremath{_7}}}1
{₈}{{\ensuremath{_8}}}1
{₉}{{\ensuremath{_9}}}1
{₀}{{\ensuremath{_0}}}1
{¹}{{\ensuremath{^1}}}1
{ₙ}{{\ensuremath{_n}}}1
{ₘ}{{\ensuremath{_m}}}1
{↑}{{\ensuremath{\uparrow}}}1
{↓}{{\ensuremath{\downarrow}}}1
{▸}{{\ensuremath{\triangleright}}}1
{∀}{{\ensuremath{\forall}}}1
{∃}{{\ensuremath{\exists}}}1
{λ}{{\ensuremath{\mathrm{\lambda}}}}1
{=}{{\ensuremath{=}}}1
{<}{{\ensuremath{\textless}}}1
{>}{{\ensuremath{\textgreater}}}1
{_}{{$\_$}}1
{(}{(}1
{(}{(}1
{‖}{{\ensuremath{\Vert}}}1
{+}{{+}}1
{*}{{*}}1,
}

\theoremstyle{definition}
\newtheorem{definition}{Definition}
\newtheorem{theorem}{Theorem}
\newtheorem{lemma}{Lemma}
\newtheorem{example}{Example}



\begin{document}

\title{Issue XXXV: Categories of Spectra}
\author{Maksym Sokhatskyi $^1$}
\date{ $^1$ National Technical University of Ukraine \\
       \small Igor Sikorsky Kyiv Polytechnical Institute \\
       \today }

\maketitle

\begin{abstract}
The structural unity of mathematics is revealed through isomorphisms, analogies, and instances (type element),
connecting frameworks like algebra, homological algebra, and stable homotopy theory.
This article examines these relationships via a correspondence table spanning Algebra,
Homological Algebra, Eilenberg-MacLane Spectrum, Ordinary Cohomology, Superalgebra, K-Theory, and Stable Spectra.
We focus on analogies, elucidating their categorical nuances, formal properties,
and pitfalls in interpretation, using rigorous examples to underscore their significance in modern mathematics.
\end{abstract}

\ifincludeTOC
\tableofcontents
\fi

\section{Stable Spectra}

Mathematics thrives on connections between distinct categories,
where \textbf{isomorphisms} provide exact equivalences, \textbf{analogies}
capture structural similarities, and \textbf{instances} represent
specific realizations. These concepts are central to a correspondence
table linking Algebra (abelian groups), Homological Algebra (chain complexes),
Eilenberg-MacLane Spectrum, Ordinary Cohomology, Superalgebra, K-Theory, and
Stable Spectra (stable homotopy category).

This article explores the categorical distinctions and subtleties
of isomorphisms, analogies, and instances. We provide formal
examples, discuss implications for fields like derived categories and
stable homotopy theory, and highlight pitfalls specific to non-isomorphic
relationships.

\subsection{Definitions and Categorical Context}

We define our key terms within a category-theoretic framework, emphasizing non-isomorphic structures.

\begin{definition}[Isomorphism]
An \textbf{isomorphism} in a category \(\mathcal{C}\) is a morphism \(f: A \to B\) with an inverse \(g: B \to A\) such that \(g \circ f = \text{id}_A\) and \(f \circ g = \text{id}_B\). For categories, an isomorphism is an equivalence, i.e., a functor \(F: \mathcal{C} \to \mathcal{D}\) with a quasi-inverse \(G: \mathcal{D} \to \mathcal{C}\).
\end{definition}

\begin{example}
In \(\Ab\) (abelian groups), the map \(f: \Z \to 2\Z\), \(f(n) = 2n\), is an isomorphism. Categorically, \(\Ab \cong \Mod_\Z\), but \(\Ab \not\cong \Spectra\), as the latter is triangulated.
\end{example}

\begin{definition}[Non-Isomorphic Analogy]
A non-isomorphic \textbf{analogy} is a structural similarity between objects or categories, captured by functors that preserve some properties but not all, ensuring no categorical equivalence.
\end{definition}

\begin{example}
The tensor product \(\otimes\) in \(\Ab\) and the smash product \(\wedge\) in \(\Spectra\) are analogous monoidal operations, but \(\Ab \not\cong \Spectra\) due to the topological structure of \(\Spectra\).
\end{example}

\begin{definition}[Instance]
An \textbf{instance} is a specific object or subcategory within a broader category, embedded via a faithful functor. A column in the table is an instance of another if its structures are special cases of the latter’s, maintaining non-isomorphic distinctions from other categories.
\end{definition}

\begin{example}
The category of \(\Z/2\Z\)-graded abelian groups is an instance of \(\Ab\), embedded via the forgetful functor. The K-theory spectra \(KU\) and \(KO\) are instances of \(\Spectra\), distinct from general spectra.
\end{example}

\begin{remark}
One must distinguish isomorphisms (equivalences), analogies (functorial similarities),
and instances (specific embeddings of elements into types). Focusing on non-isomorphic
structures requires attention to categorical distinctions, as assuming equivalence
can derail research in stable homotopy or supergeometry.
\end{remark}

\subsection{Multidimentional Framework Signatues}

The table organizes structures across seven categories, with Stable Spectra as the primary topological framework, replacing the isomorphic Higher Algebra. We focus on non-isomorphic relationships:

\begin{table}[h]
\centering
\caption{Algebras and Stable Spectra}
\begin{tabular}{l|c|c|c|c}
\toprule
Category & Object & Ring & Initial Unit & Operations \\
\midrule
Algebra & Abelian group & Ring & \(\Z\) & \(\oplus, \otimes\) \\
Homological Algebra & Chain complex & dg-ring & \(\Z[0]\) & \(\oplus, \otimes\) \\
Superalgebra & \(\Z/2\Z\)-graded Ab & \(\Z/2\Z\)-graded Ring & \(\Z\) & \(\oplus, \otimes\) \\
Ordinary Cohomology & Cohomology \(H^*(-; A)\) & Graded ring & \(H^*(-; \Z)\) & \(\oplus, \otimes\) \\
\hline
Eilenberg-MacLane & Abelian group \(A\) & Ring \(A\) & \(\HA\) & \(\oplus/\vee, \otimes/\wedge\) \\
Complex K-Theory & Graded abelian group & Graded ring & \(KU\) & \(\vee, \wedge\) \\
Real K-Theory & Graded abelian group & Graded ring & \(KO\) & \(\vee, \wedge\) \\
Stable Spectra & Stable spectrum & Ring spectrum & \(\Ssphere\) & \(\vee, \wedge\) \\
\bottomrule
\end{tabular}
\end{table}

Relationships include:
- Isomorphisms: Rare, as we focus on non-isomorphic structures.
- Non-Isomorphic Analogies: Structural similarities without equivalence.
- Isomorphic Instances: Columns as specific subcategories (e.g., K-Theory as an instance of Stable Spectra).

\subsection{Categories}

Non-isomorphic analogies highlight similarities between categorically
distinct structures, requiring careful distinction.

\subsection{Objects}
The progression from abelian groups to chain complexes to stable spectra illustrates non-isomorphic analogies:
- Algebra to Homological Algebra: The functor \(A \mapsto A[0]\) embeds \(\Ab\) into \(\Ch(\Z)\), mapping an abelian group \(A\) to a chain complex concentrated in degree 0. However, \(\Ch(\Z) \not\cong \Ab\), as chain complexes have differentials yielding homology \(H_n(C)\). The operations \(\oplus\) and \(\otimes\) are analogous, but homological structures (e.g., long exact sequences) distinguish \(\Ch(\Z)\).
- Homological Algebra to Stable Spectra: The Dold-Kan correspondence maps non-negatively graded chain complexes to simplicial abelian groups, extendable to spectra via the Eilenberg-MacLane functor. Stable spectra in \(\Spectra\) have homotopy groups \(\pi_n(E)\) analogous to \(H_n(C)\), but \(\Spectra\)’s triangulated structure and stable phenomena (e.g., suspension equivalences) ensure \(\Spectra \not\cong \Ch(\Z)\). The wedge sum \(\vee\) and smash product \(\wedge\) mirror \(\oplus\) and \(\otimes\), but topological complexities (e.g., Künneth spectral sequence) confirm non-isomorphism.

\begin{example}
Consider a chain complex \(C = (\dots \to 0 \to \Z \xrightarrow{2} \Z \to 0)\) with \(H_0(C) \cong \Z/2\Z\). The spectrum \(H(\Z/2\Z)\) has \(\pi_0 \cong \Z/2\Z\), but higher \(\pi_n\) involve stable homotopy groups, unlike \(C\)’s algebraic homology.
\end{example}

Subtlety: The topological nature of \(\Spectra\) introduces phenomena (e.g., higher Tor terms in \(\wedge\)) absent in \(\Ch(\Z)\). Applying algebraic exactness to spectral cofiber sequences can lead to errors in derived category or stable homotopy computations.

\subsection{Cohomology Theories}
The cohomology theory \(H^*(-; A)\) in Ordinary Cohomology,
defined by \(HA\), is analogous to \(K^*(-)\) and \(KO^*(-)\) in K-Theory,
defined by \(KU\) and \(KO\). Both are generalized cohomology theories,
but \(H^*(-; A)\) is degree-specific, while \(K^*(-)\) is 2-periodic
(\(K^n(X) \cong K^{n+2}(X)\)) and \(KO^*(-)\) is 8-periodic, ensuring non-isomorphism.

\begin{example}
For \(X = S^1\), \(H^0(S^1; \Z) \cong \Z\), \(H^1(S^1; \Z) \cong \Z\), with higher groups zero. In contrast, \(K^0(S^1) \cong \Z \oplus \Z\) (vector bundle classes), with \(K^1(S^1) \cong \Z\), reflecting periodicity. The cup product in \(H^*(-; \Z)\) parallels \(K^*(-)\)’s ring structure, but periodicity distinguishes them.
\end{example}

Subtlety: Periodicity in \(KU\) and \(KO\) requires tools like the Atiyah-Hirzebruch spectral sequence, unlike \(H^*(-; A)\). Assuming similar behavior can lead to errors in bundle classification or index theorems.

\subsection{Instances of Stable Spectra}
Stable Spectra (\(\Spectra\)) encompasses all stable spectra. Several columns are instances:
- Eilenberg-MacLane Spectrum: The spectrum \(HA\) for \(A \in \Ab\) has \(\pi_0(HA) = A\), \(\pi_n(HA) = 0\) for \(n \neq 0\). It is an instance of \(\Spectra\), embedded via \(A \mapsto HA\).
- Ordinary Cohomology: The cohomology theory \(H^*(-; A)\), defined by \(HA\), is an instance of \(\Spectra\), arising from a specific spectrum.
- K-Theory: The spectra \(KU\) and \(KO\) are specific objects in \(\Spectra\), with \(\pi_n(KU) \cong \Z\) (n even), 0 (n odd), and \(\pi_n(KO)\) 8-periodic. This column is an instance of \(\Spectra\).

\begin{example}
The spectrum \(KU\) defines \(K^0(X)\), the group of virtual vector bundles on \(X\). Its 2-periodicity distinguishes it from \(HA\), but both are spectra in \(\Spectra\).
\end{example}

Subtlety: Instances like \(KU\) have unique properties (e.g., Bott periodicity) not shared by general spectra. Accounting for these is critical in applications, such as K-theory’s role in operator algebras versus \(HA\)’s role in singular cohomology.

\subsection{Instances of Algebra}
- Superalgebra: A \(\Z/2\Z\)-graded abelian group \(A = A_0 \oplus A_1\) is an instance of \(\Ab\), embedded via the forgetful functor. The grading introduces sign rules in the tensor product.
- Eilenberg-MacLane Spectrum: The abelian group \(A\) defining \(HA\) is an instance of \(\Ab\), though \(HA\) is a spectrum.

\begin{example}
The graded group \(\Z \oplus \Z\) (even/odd parts) is an abelian group with \(\Z/2\Z\)-grading. Its tensor product uses signs (\(a \otimes b \mapsto (-1)^{\deg a \cdot \deg b} b \otimes a\)), unlike \(\Z \oplus \Z\) in \(\Ab\).
\end{example}

Subtlety: The grading in Superalgebra affects operations (e.g., supercommutativity), distinguishing it from \(\Ab\). Ignoring signs can lead to errors in supergeometry or supersymmetry.

\subsection{Research Implications}

Researchers must navigate non-isomorphic structures carefully:
1. Conflating Analogies with Isomorphisms: Assuming \(\wedge\) in \(\Spectra\) mirrors \(\otimes\) in \(\Ab\) can lead to incorrect spectral sequence computations, as \(\wedge\) introduces higher Tor terms.
2. Overgeneralizing Instances: Treating \(KU\) as a generic spectrum ignores its 2-periodicity, critical in K-theory applications.
3. Misapplying Algebraic Tools: Using exact sequences in \(\Spectra\) without considering cofiber sequences can yield invalid results in stable homotopy theory.
4. Neglecting Derived Structures: In Homological Algebra, derived functors (\(\text{Tor}\), \(\text{Ext}\)) distinguish \(\Ch(\Z)\) from \(\Ab\). In Superalgebra, graded structures require sign conventions.
5. Overlooking Topological Phenomena: The periodicity of \(KU\) and \(KO\) or the triangulated structure of \(\Spectra\) are absent in \(\Ab\) or \(\Ch(\Z)\). Ignoring these can lead to errors in cohomology computations.

These nuances inform research:
- Derived Categories: Non-isomorphic analogies between \(\Ch(\Z)\) and \(\Spectra\) guide derived category constructions, where instances like Ordinary Cohomology inform derived functors.
- Stable Homotopy Theory: The instance of K-Theory in Stable Spectra drives chromatic homotopy theory, leveraging periodicity.
- Supergeometry: The instance of Superalgebra in \(\Ab\) underpins super manifolds, requiring careful graded structure handling.
- Noncommutative Geometry: Analogies between Algebra and K-Theory inspire noncommutative K-theory, connecting to operator algebras.

\subsection{Conclusion}

The correspondence table reveals the unity of mathematics through isomorphisms, analogies,
and instances (type theory), with Stable Spectra as the primary topological framework.
Non-isomorphic analogies, like abelian groups to spectra, highlight categorical evolution,
while instances, like K-Theory in Stable Spectra, ground general frameworks in specific cases.
Researchers must navigate these distinctions—avoiding conflations and respecting categorical
contexts to advance fields from derived geometry to stable homotopy theory.

\end{document}
