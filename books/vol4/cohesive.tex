\documentclass{article}
\usepackage{amsmath,amssymb,amsthm,mathtools}
\usepackage{tikz-cd}

\begin{document}

\title{Issue XL: Cohesive Topos}
\author{Максим Сохацький $^1$}
\date{ $^1$ Національний технічний університет України \\
       \small Київський політехнічний інститут імені Ігоря Сікорського \\
       \today }

\maketitle

\begin{abstract}

Formal definition of Cohesive Topos.

{\bf Keywords}: Topos Theory \\
\end{abstract}

\ifincludeTOC
  \tableofcontents
\fi

\section{Cohesive Topos Theory}

\subsection{Preliminaries}
A \textbf{category} $\mathcal{C}$ consists of:
\begin{itemize}
  \item A class of \textbf{objects}, $\mathrm{Ob}(\mathcal{C})$,
  \item A class of \textbf{morphisms}, $\mathrm{Hom}_{\mathcal{C}}(X,Y)$, for each pair $X,Y \in \mathrm{Ob}(\mathcal{C})$,
  \item Composition maps $\circ: \mathrm{Hom}(Y,Z) \times \mathrm{Hom}(X,Y) \to \mathrm{Hom}(X,Z)$,
  \item Identity morphisms $\mathrm{id}_X \in \mathrm{Hom}(X,X)$ for each $X$,
\end{itemize}
satisfying associativity and identity laws.


A \textbf{functor} $F: \mathcal{C} \to \mathcal{D}$ assigns to each:
\begin{itemize}
  \item Object $X \in \mathcal{C}$ an object $F(X) \in \mathcal{D}$,
  \item Morphism $f: X \to Y$ a morphism $F(f): F(X) \to F(Y)$,
\end{itemize}
such that $F(\mathrm{id}_X) = \mathrm{id}_{F(X)}$ and $F(g \circ f) = F(g) \circ F(f)$.


A \textbf{natural transformation} $\eta: F \Rightarrow G$ between functors $F, G: \mathcal{C} \to \mathcal{D}$ consists of morphisms $\eta_X: F(X) \to G(X)$ such that for every $f: X \to Y$ in $\mathcal{C}$,
\[
\begin{tikzcd}
F(X) \arrow[r, "\eta_X"] \arrow[d, "F(f)"'] & G(X) \arrow[d, "G(f)"] \\
F(Y) \arrow[r, "\eta_Y"'] & G(Y)
\end{tikzcd}
\]
commutes.

An \textbf{adjunction} between categories $\mathcal{C}$ and $\mathcal{D}$ consists of functors
\[
F: \mathcal{C} \leftrightarrows \mathcal{D} : G
\]
and natural transformations (unit $\eta$ and counit $\varepsilon$)
\[
\eta: \mathrm{Id}_{\mathcal{C}} \Rightarrow G \circ F, \quad \varepsilon: F \circ G \Rightarrow \mathrm{Id}_{\mathcal{D}}
\]
satisfying the triangle identities.

\subsection{Topos}
A \textbf{topos} $\mathcal{E}$ is a category that:
\begin{itemize}
  \item Has all finite limits and colimits,
  \item Is Cartesian closed: has exponential objects $[X,Y]$,
  \item Has a subobject classifier $\Omega$.
\end{itemize}

\subsection{Geometric Morphism}
A \textbf{geometric morphism} $f: \mathcal{E} \to \mathcal{F}$ between topoi consists of an adjoint pair
\[
f^*: \mathcal{F} \leftrightarrows \mathcal{E} : f_*
\]
with $f^* \dashv f_*$, where $f^*$ preserves finite limits (i.e., is left exact).

\newpage
\subsection{Cohesive Topos}
A \textbf{cohesive topos} is a topos $\mathcal{E}$ equipped with a quadruple of adjoint functors:
\[
\Pi \dashv \Delta \dashv \Gamma \dashv \nabla : \mathcal{E} \leftrightarrows \mathbf{Set}
\]
such that:
\begin{itemize}
  \item $\Gamma$ is the global sections functor,
  \item $\Delta$ is the constant sheaf functor,
  \item $\nabla$ sends a set to a codiscrete object,
  \item $\Pi$ is the shape or fundamental groupoid functor,
  \item $\Delta$ and $\nabla$ are fully faithful,
  \item $\Delta$ preserves finite limits,
  \item $\Pi$ preserves finite products (in some variants).
\end{itemize}

\subsection{Cohesive Adjunction Diagram and Modalities}

\[
\begin{tikzcd}
\mathcal{E}
  \arrow[r, shift left=3, "\Pi"]
  \arrow[r, shift right=3, "\nabla"']
  \arrow[r, phantom, "\dashv" description]
  & \mathbf{Set}
  \arrow[l, shift left=1.5, "\Delta"]
  \arrow[l, shift right=1.5, "\Gamma"']
  \arrow[l, phantom, "\dashv" description]
\end{tikzcd}
\]

\[
\begin{tikzcd}[column sep=large]
\mathcal{E}
  \arrow[r, bend left=40, "\int", ""{name=s, below}]
  \arrow[r, bend right=40, "\sharp"'{name=t}]
  & \mathcal{E}
  \arrow[Rightarrow, from=s, to=t, "\flat" description]
\end{tikzcd}
\]


\newpage
\subsection{Cohesive Modalities}

The above adjoint quadruple canonically induces a triple of endofunctors on $\mathcal{E}$:
\[
(\int \dashv \flat \dashv \sharp) : \mathcal{E} \to \mathcal{E}
\]
defined as follows:
\begin{align*}
\int &\coloneqq \Delta \circ \Pi \\
\flat &\coloneqq \Delta \circ \Gamma \\
\sharp &\coloneqq \nabla \circ \Gamma
\end{align*}
This yields an \textbf{adjoint triple} of endofunctors on $\mathcal{E}$:
\[
\int \dashv \flat \dashv \sharp
\]

These are:
\begin{itemize}
  \item $\int$ — the \textbf{shape modality}: captures the fundamental shape or homotopy type,
  \item $\flat$ — the \textbf{flat modality}: forgets cohesive structure while remembering discrete shape,
  \item $\sharp$ — the \textbf{sharp modality}: codiscretizes the structure, reflecting the full cohesion.
\end{itemize}

Each of these is an \textbf{idempotent} (co)monad, hence a \emph{modality} in the internal language (type theory) of $\mathcal{E}$.


\newpage
\subsection{Differential Cohesion}

A \textbf{differential cohesive topos} is a cohesive topos $\mathcal{E}$ equipped with an additional adjoint triple of endofunctors:
\[
(\Re \dashv \Im \dashv \&) : \mathcal{E} \to \mathcal{E}
\]
These are:
\begin{itemize}
  \item $\Re$: the \textbf{reduction modality} — forgets nilpotents,
  \item $\Im$: the \textbf{infinitesimal shape modality} — retains infinitesimal data,
  \item $\&$: the \textbf{infinitesimal flat modality} — reflects formally smooth structure.
\end{itemize}

Important object classes:
\begin{itemize}
  \item An object $X$ is \textbf{reduced} if $\Re(X) \cong X$.
  \item It is \textbf{coreduced} if $\&(X) \cong X$.
  \item It is \textbf{formally smooth} if the unit map $X \to \& X$ is an effective epimorphism.
\end{itemize}

\textbf{Formally étale maps} are those morphisms \( f: X \to Y \) such that the square
\[
\begin{tikzcd}
X \arrow[r] \arrow[d, "f"'] & \Im X \arrow[d, "\Im(f)"] \\
Y \arrow[r] & \Im Y
\end{tikzcd}
\]
is a pullback.

\newpage
\subsection{Graded Differential Cohesion}

In \textbf{graded differential cohesion}, such as used in synthetic supergeometry, one introduces an adjoint triple:

% \def\fermionic\rightrightarrows
% \def\bosonic\rightsquigarrow

10) $\rightrightarrows\ \dashv\ \rightsquigarrow\ \dashv\ Rh$

\[
(\rightrightarrows \dashv \rightsquigarrow \dashv Rh) : \mathcal{E} \to \mathcal{E}
\]

These are:
\begin{itemize}
  \item $\rightrightarrows$: the \textbf{fermionic modality} — captures anti-commuting directions,
  \item $\rightsquigarrow$: the \textbf{bosonic modality} — filters out fermionic directions,
  \item $Rh$: the \textbf{rheonomic modality} — encodes constraint structures.
\end{itemize}

These modal operators form part of the internal logic of supergeometric or supersymmetric type theories.

\newpage
\subsection{Adjoint String of Identity Modalities}
In Homotopy Type Theory (HoTT), identity systems (Contractible, Strict Id, Quotient, Isomorphism, Path = Equivalence) are modeled as modalities in the $\infty$-topos $\mathcal{E} = \infty\text{Grp}$. We construct an adjoint quadruple extending the Jacobs-Lawvere triple $C \dashv \text{Id}_A \dashv Q(-/\sim)$, incorporating Isomorphism and Path = Equivalence. The modalities are ordered by adjointness: Contractible $\leq$ Strict Id $\leq$ Quotient $\leq$ Isomorphism $\leq$ Path = Equivalence, reflecting their structure in HoTT, where Strict Id, Quotient, and Path = Equivalence are mere propositions for h-sets, while Isomorphism is not.

Homotopy Type Theory (HoTT) provides a framework for reasoning about equality via the identity type $\text{Id}_A(x,y)$. In the $\infty$-topos $\mathcal{E} = \infty\text{Grp}$, identity systems are modalities (monads), ordered by adjointness. The classical Jacobs-Lawvere adjunction triple $C \dashv \text{Id}_A \dashv Q(-/\sim)$ captures Contractible, Strict Id, and Quotient. We extend this to a quadruple, including Isomorphism and Path = Equivalence, respecting the HoTT equivalence of Path and Equivalence and the propositional nature of Strict Id, Quotient, and Path = Equivalence for h-sets.

\begin{definition}
In HoTT, the identity systems are:
\begin{itemize}
    \item \textbf{Contractible}: (-1)-truncated types, mere propositions.
    \item \textbf{Strict Id}: $\text{Id}_A(x,y)$ for h-sets (0-truncated), a mere proposition.
    \item \textbf{Quotient}: Set-quotients $A / \sim$, 0-truncated, equivalent to Strict Id.
    \item \textbf{Isomorphism}: $\text{iso}_A(x,y)$, a triple $(f, g, p)$, not a mere proposition.
    \item \textbf{Path = Equiv}: $\text{Id}_A(x,y) \simeq (x \simeq y)$, equivalent in HoTT.
\end{itemize}
\end{definition}

In $\mathcal{E} = \infty\text{Grp}$, we define categories:
\begin{itemize}
    \item $\mathcal{E}_{\text{contr}} = \mathcal{E}_{\leq -1}$: Mere propositions.
    \item $\mathcal{E}_{\text{strict}} = \mathcal{E}_{\leq 0} \cong \text{Set}$: h-sets (Strict Id).
    \item $\mathcal{E}_{\text{quot}} = \mathcal{E}_{\leq 0} \cong \text{Set}$: h-sets (Quotient).
    \item $\mathcal{E}_{\text{iso}} \cong \mathcal{E}$: $\infty$-groupoids with isomorphisms.
    \item $\mathcal{E}_{\text{path/equiv}} \cong \mathcal{E}$: $\infty$-groupoids with paths/equivalences.
\end{itemize}

\newpage
The Jacobs-Lawvere triple $C \dashv \text{Id}_A \dashv Q(-/\sim)$ is extended to an adjoint quadruple:

\[
\mathcal{E}_{\text{contr}} \xrightarrow{F_4} \mathcal{E}_{\text{strict}} \xrightarrow{F_3} \mathcal{E}_{\text{quot}} \xrightarrow{F_2} \mathcal{E}_{\text{iso}} \xrightarrow{F_1} \mathcal{E}_{\text{path/equiv}}
\]

\begin{theorem}
The functors form an adjoint quadruple with adjunctions:
\[
F_4 \dashv U_4, \quad F_3 \dashv U_3, \quad F_2 \dashv U_2, \quad F_1 \dashv U_1
\]
\end{theorem}

\begin{itemize}
    \item \textbf{$F_4: \mathcal{E}_{\text{contr}} \to \mathcal{E}_{\text{strict}}$}: Inclusion of (-1)-truncated objects into 0-truncated objects. Right adjoint $U_4$: (-1)-truncation, $U_4(X) = \|X\|_{-1}$.
    \item \textbf{$F_3: \mathcal{E}_{\text{strict}} \to \mathcal{E}_{\text{quot}}$}: Canonical map to quotient structure, viewing h-sets as quotiented by trivial relations. Right adjoint $U_3$: Inverse map preserving h-set structure.
    \item \textbf{$F_2: \mathcal{E}_{\text{quot}} \to \mathcal{E}_{\text{iso}}$}: Inclusion of h-sets into $\mathcal{E}$, $\text{core}(X) \cong X$. Right adjoint $U_2$: 0-truncation, $U_2(X) = \|X\|_0$.
    \item \textbf{$F_1: \mathcal{E}_{\text{iso}} \to \mathcal{E}_{\text{path/equiv}}$}: Canonical inclusion of $\infty$-groupoids with isomorphisms into full $\infty$-groupoids with paths/equivalences. Right adjoint $U_1$: Core map, preserving isomorphism structure.
\end{itemize}

The adjunctions induce the ordering:
\[
\text{Contractible} \leq \text{Strict Id} \leq \text{Quotient} \leq \text{Isomorphism} \leq \text{Path = Equivalence}
\]
\begin{itemize}
    \item \textbf{Contractible}: Coarsest, mere propositions ((-1)-truncated).
    \item \textbf{Strict Id}: h-sets, $\text{Id}_A(x,y)$ is a mere proposition.
    \item \textbf{Quotient}: Equivalent to Strict Id, 0-truncated set-quotients.
    \item \textbf{Isomorphism}: $\text{iso}_A(x,y)$ is not a mere proposition for general types.
    \item \textbf{Path = Equivalence}: Finest, full $\infty$-groupoid structure, equivalent via univalence.
\end{itemize}

The adjoint quadruple extends the Jacobs-Lawvere triple, capturing the structure of identity systems in HoTT. The ordering reflects their increasing complexity, with Strict Id, Quotient, and Path = Equivalence collapsing to mere propositions for h-sets, while Isomorphism retains higher structure. Future work could explore these adjunctions in other $\infty$-topoi or specific CTT models.

\end{document}
