\documentclass{article}
\usepackage{amsmath,amssymb,amsthm}

\ProvidesPackage{journal}
\usepackage{graphicx}
\usepackage{mathtools}
\usepackage{hyphenat}
\usepackage{hyperref}
\usepackage{adjustbox}
\usepackage{listings}
\usepackage{verbatim}
\usepackage{xcolor}
\usepackage{amsfonts}
\usepackage{amscd}
\usepackage{amsmath}
\usepackage{amssymb}
\usepackage{amsthm}
\usepackage{tikz}
\usepackage{tikz-cd}
\usepackage{url}
\usepackage[utf8]{inputenc}
\usepackage[english,ukrainian]{babel}
\usepackage{float}
\usepackage{url}
\usepackage{tikz}
\usepackage{tikz-cd}
\usepackage[utf8]{inputenc}
\usepackage{graphicx}
\usepackage[utf8]{inputenc}
\usepackage[T1]{fontenc}
\usepackage{lmodern}
\usepackage{tocloft}
\usepackage{hyperref}
\usepackage{xcolor}
\usepackage[only,llbracket,rrbracket,llparenthesis,rrparenthesis]{stmaryrd}

\usetikzlibrary{babel}

\newcommand*{\incmap}{\hookrightarrow}
\newcommand*{\thead}[1]{\multicolumn{1}{c}{\bfseries #1}}
\renewcommand{\Join}{\vee} % Join operation symbol
\newcommand{\tabstyle}[0]{\scriptsize\ttfamily\fontseries{l}\selectfont}

\lstset{
  basicstyle=\footnotesize,
  inputencoding=utf8,
  identifierstyle=,
  literate=
{𝟎}{{\ensuremath{\mathbf{0}}}}1
{𝟏}{{\ensuremath{\mathbf{1}}}}1
{≔}{{\ensuremath{\mathrm{:=}}}}1
{α}{{\ensuremath{\mathrm{\alpha}}}}1
{ᵂ}{{\ensuremath{^W}}}1
{β}{{\ensuremath{\mathrm{\beta}}}}1
{γ}{{\ensuremath{\mathrm{\gamma}}}}1
{δ}{{\ensuremath{\mathrm{\delta}}}}1
{ε}{{\ensuremath{\mathrm{\varepsilon}}}}1
{ζ}{{\ensuremath{\mathrm{\zeta}}}}1
{η}{{\ensuremath{\mathrm{\eta}}}}1
{θ}{{\ensuremath{\mathrm{\theta}}}}1
{ι}{{\ensuremath{\mathrm{\iota}}}}1
{κ}{{\ensuremath{\mathrm{\kappa}}}}1
{λ}{{\ensuremath{\mathrm{\lambda}}}}1
{μ}{{\ensuremath{\mathrm{\mu}}}}1
{ν}{{\ensuremath{\mathrm{\nu}}}}1
{ξ}{{\ensuremath{\mathrm{\xi}}}}1
{π}{{\ensuremath{\mathrm{\mathnormal{\pi}}}}}1
{ρ}{{\ensuremath{\mathrm{\rho}}}}1
{σ}{{\ensuremath{\mathrm{\sigma}}}}1
{τ}{{\ensuremath{\mathrm{\tau}}}}1
{φ}{{\ensuremath{\mathrm{\varphi}}}}1
{χ}{{\ensuremath{\mathrm{\chi}}}}1
{ψ}{{\ensuremath{\mathrm{\psi}}}}1
{ω}{{\ensuremath{\mathrm{\omega}}}}1
{Π}{{\ensuremath{\mathrm{\Pi}}}}1
{Γ}{{\ensuremath{\mathrm{\Gamma}}}}1
{Δ}{{\ensuremath{\mathrm{\Delta}}}}1
{Θ}{{\ensuremath{\mathrm{\Theta}}}}1
{Λ}{{\ensuremath{\mathrm{\Lambda}}}}1
{Σ}{{\ensuremath{\mathrm{\Sigma}}}}1
{Φ}{{\ensuremath{\mathrm{\Phi}}}}1
{Ξ}{{\ensuremath{\mathrm{\Xi}}}}1
{Ψ}{{\ensuremath{\mathrm{\Psi}}}}1
{Ω}{{\ensuremath{\mathrm{\Omega}}}}1
{ℵ}{{\ensuremath{\aleph}}}1
{≤}{{\ensuremath{\leq}}}1
{≥}{{\ensuremath{\geq}}}1
{≠}{{\ensuremath{\neq}}}1
{≈}{{\ensuremath{\approx}}}1
{≡}{{\ensuremath{\equiv}}}1
{≃}{{\ensuremath{\simeq}}}1
{≤}{{\ensuremath{\leq}}}1
{≥}{{\ensuremath{\geq}}}1
{∂}{{\ensuremath{\partial}}}1
{∆}{{\ensuremath{\triangle}}}1 % or \laplace?
{∫}{{\ensuremath{\int}}}1
{∑}{{\ensuremath{\mathrm{\Sigma}}}}1
{→}{{\ensuremath{\rightarrow}}}1
{⊥}{{\ensuremath{\perp}}}1
{∞}{{\ensuremath{\infty}}}1
{∂}{{\ensuremath{\partial}}}1
{∓}{{\ensuremath{\mp}}}1
{±}{{\ensuremath{\pm}}}1
{×}{{\ensuremath{\times}}}1
{⊕}{{\ensuremath{\oplus}}}1
{⊗}{{\ensuremath{\otimes}}}1
{⊞}{{\ensuremath{\boxplus}}}1
{∇}{{\ensuremath{\nabla}}}1
{√}{{\ensuremath{\sqrt}}}1
{⬝}{{\ensuremath{\cdot}}}1
{•}{{\ensuremath{\cdot}}}1
{∘}{{\ensuremath{\circ}}}1
{⁻}{{\ensuremath{^{-}}}}1
{▸}{{\ensuremath{\blacktriangleright}}}1
{★}{{\ensuremath{\star}}}1
{∧}{{\ensuremath{\wedge}}}1
{∨}{{\ensuremath{\vee}}}1
{¬}{{\ensuremath{\neg}}}1
{⊢}{{\ensuremath{\vdash}}}1
{⟨}{{\ensuremath{\langle}}}1
{⟩}{{\ensuremath{\rangle}}}1
{↦}{{\ensuremath{\mapsto}}}1
{→}{{\ensuremath{\rightarrow}}}1
{↔}{{\ensuremath{\leftrightarrow}}}1
{⇒}{{\ensuremath{\Rightarrow}}}1
{⟹}{{\ensuremath{\Longrightarrow}}}1
{⇐}{{\ensuremath{\Leftarrow}}}1
{⟸}{{\ensuremath{\Longleftarrow}}}1
{∩}{{\ensuremath{\cap}}}1
{∪}{{\ensuremath{\cup}}}1
{·}{{\ensuremath{\cdot}}}1
{ᵢ}{{\ensuremath{_i}}}1
{ⱼ}{{\ensuremath{_j}}}1
{₊}{{\ensuremath{_+}}}1
{ℑ}{{\ensuremath{\Im}}}1
{𝒢}{{\ensuremath{\mathcal{G}}}}1
{ℕ}{{\ensuremath{\mathbb{N}}}}1
{𝟘}{{\ensuremath{\mathbb{0}}}}1
{ℤ}{{\ensuremath{\mathbb{Z}}}}1
{ℝ}{{\ensuremath{\mathbb{R}}}}1
{⊂}{{\ensuremath{\subseteq}}}1
{⊆}{{\ensuremath{\subseteq}}}1
{⊄}{{\ensuremath{\nsubseteq}}}1
{⊈}{{\ensuremath{\nsubseteq}}}1
{⊃}{{\ensuremath{\supseteq}}}1
{⊇}{{\ensuremath{\supseteq}}}1
{⊅}{{\ensuremath{\nsupseteq}}}1
{⊉}{{\ensuremath{\nsupseteq}}}1
{∈}{{\ensuremath{\in}}}1
{∉}{{\ensuremath{\notin}}}1
{∋}{{\ensuremath{\ni}}}1
{∌}{{\ensuremath{\notni}}}1
{∅}{{\ensuremath{\emptyset}}}1
{∖}{{\ensuremath{\setminus}}}1
{†}{{\ensuremath{\dag}}}1
{ℕ}{{\ensuremath{\mathbb{N}}}}1
{ℤ}{{\ensuremath{\mathbb{Z}}}}1
{ℝ}{{\ensuremath{\mathbb{R}}}}1
{ℚ}{{\ensuremath{\mathbb{Q}}}}1
{ℂ}{{\ensuremath{\mathbb{C}}}}1
{⌞}{{\ensuremath{\llcorner}}}1
{⌟}{{\ensuremath{\lrcorner}}}1
{⦃}{{\ensuremath{ \{\!| }}}1
{⦄}{{\ensuremath{ |\!\} }}}1
{ᵁ}{{\ensuremath{^U}}}1
{₋}{{\ensuremath{_{-}}}}1
{₁}{{\ensuremath{_1}}}1
{₂}{{\ensuremath{_2}}}1
{₃}{{\ensuremath{_3}}}1
{₄}{{\ensuremath{_4}}}1
{₅}{{\ensuremath{_5}}}1
{₆}{{\ensuremath{_6}}}1
{₇}{{\ensuremath{_7}}}1
{₈}{{\ensuremath{_8}}}1
{₉}{{\ensuremath{_9}}}1
{₀}{{\ensuremath{_0}}}1
{¹}{{\ensuremath{^1}}}1
{ₙ}{{\ensuremath{_n}}}1
{ₘ}{{\ensuremath{_m}}}1
{↑}{{\ensuremath{\uparrow}}}1
{↓}{{\ensuremath{\downarrow}}}1
{▸}{{\ensuremath{\triangleright}}}1
{∀}{{\ensuremath{\forall}}}1
{∃}{{\ensuremath{\exists}}}1
{λ}{{\ensuremath{\mathrm{\lambda}}}}1
{=}{{\ensuremath{=}}}1
{<}{{\ensuremath{\textless}}}1
{>}{{\ensuremath{\textgreater}}}1
{_}{{$\_$}}1
{(}{(}1
{(}{(}1
{‖}{{\ensuremath{\Vert}}}1
{+}{{+}}1
{*}{{*}}1,
}

\theoremstyle{definition}
\newtheorem{definition}{Definition}
\newtheorem{theorem}{Theorem}
\newtheorem{lemma}{Lemma}
\newtheorem{example}{Example}



\begin{document}

\title{Issue XXXIII: Structure Preserving Theorems}
\author{Namdak Tonpa}
\date{\today}

\maketitle

\begin{abstract}
This article unifies algebra and geometry by characterizing algebra
as the domain of homomorphisms preserving structure and geometry as
the domain of inverse images of homomorphisms preserving structure.
We introduce two new theorems: the Homomorphism Preservation Theorem
(HPT) for Algebraic Categories and the Inverse Image Preservation
Theorem (IIPT) for Geometric Categories. These build on foundational
results like the First Isomorphism Theorem, Continuity Theorem,
Pullback Theorem, Stone Duality, Gelfand Duality, and Adjoint Functor
Theorem. Aimed at advanced graduate students, this exposition uses
category theory to illuminate the algebraic-geometric duality.
\end{abstract}

\ifincludeTOC
  \tableofcontents
\fi


\section{Algebra and Geometry}
Algebra and geometry, foundational to pure mathematics,
differ in focus: algebra on abstract structures and their
transformations, geometry on spatial properties and invariants.
We propose a unifying perspective: algebra is defined by homomorphisms
preserving structure, and geometry by the inverse images of
homomorphisms preserving structure. This article formalizes
this view through two explicit theorems—the Homomorphism
Preservation Theorem (HPT) for Algebraic Categories and
the Inverse Image Preservation Theorem (IIPT) for Geometric
Categories—building on established results. Assuming familiarity
with category theory, algebraic topology, and commutative algebra,
we provide a framework for graduate students to explore these fields’ interplay.

\subsection{Homomorphisms in Algebra}
\begin{definition}
Let \(\mathcal{C}\) be a category, and let \(A, B\) be objects in \(\mathcal{C}\). A \emph{homomorphism} \(\phi: A \to B\) is a morphism in \(\mathcal{C}\) that preserves the structure defined by the category’s operations and relations.
\end{definition}

In algebraic categories (e.g., \(\mathbf{Grp}\), \(\mathbf{Ring}\), \(\mathbf{Mod}_R\)), homomorphisms preserve operations like group multiplication or module scalar multiplication.

\begin{example}
In \(\mathbf{Grp}\), a group homomorphism \(\phi: G \to H\) satisfies \(\phi(g_1 g_2) = \phi(g_1) \phi(g_2)\) for all \(g_1, g_2 \in G\), preserving the group operation.
\end{example}

\begin{theorem}[First Isomorphism Theorem]
Let \(\phi: G \to H\) be a group homomorphism with kernel \(K = \ker(\phi)\). Then \(G/K \cong \text{im}(\phi)\).
\end{theorem}

\begin{theorem}[Universal Property of Free Objects]
In an algebraic category (e.g., \(\mathbf{Grp}\), \(\mathbf{Ring}\)), for a free object \(F(X)\) on a set \(X\), any map \(f: X \to A\) (where \(A\) is an object) extends uniquely to a homomorphism \(\phi: F(X) \to A\).
\end{theorem}

We now introduce a theorem encapsulating the algebraic perspective.

\begin{theorem}[Homomorphism Preservation Theorem for Algebraic Categories]
Let \(\mathcal{C}\) be an algebraic category (e.g., \(\mathbf{Grp}\), \(\mathbf{Ring}\), \(\mathbf{Mod}_R\)) with a forgetful functor \(U: \mathcal{C} \to \mathbf{Set}\). For any surjective homomorphism \(\phi: A \to B\) in \(\mathcal{C}\) with kernel \(K\) (a normal subobject), there exists an isomorphism \(\psi: A/K \to B\) such that \(\psi \circ \pi = \phi\), where \(\pi: A \to A/K\) is the canonical projection. Moreover, any object \(A\) can be generated by a free object \(F(X)\) via a surjective homomorphism whose structure is preserved by \(\phi\).
\end{theorem}

\begin{proof}
The first part follows from the First Isomorphism Theorem \cite{lang}: for a surjective homomorphism \(\phi: A \to B\) with kernel \(K\), the quotient \(A/K \cong B\) via the isomorphism \(\psi: aK \mapsto \phi(a)\). The second part follows from the Universal Property of Free Objects \cite{mac}: for any object \(A\), there exists a set \(X\) and a free object \(F(X)\) with a surjective homomorphism \(\eta: F(X) \to A\), and any homomorphism \(\phi: A \to B\) extends the structure-preserving maps from \(F(X)\).
\end{proof}

\begin{remark}
The HPT formalizes that homomorphisms in algebraic categories preserve structure forward, inducing isomorphisms on quotients and respecting generators, unifying the First Isomorphism Theorem and Universal Property. The name avoids confusion with the Structure-Identity Principle in category theory \cite{mac}.
\end{remark}

\subsection{Homomorphisms in Geometry}
Geometry emphasizes spaces where structure is preserved under inverse images of homomorphisms, as in \(\mathbf{Top}\) or \(\mathbf{Sch}\).

\begin{definition}
Let \(\phi: X \to Y\) be a morphism in a category \(\mathcal{C}\). The \emph{inverse image} of a subobject \(S \subseteq Y\) (if it exists) is the subobject \(\phi^{-1}(S) \subseteq X\) defined via the pullback of \(S \hookrightarrow Y\) along \(\phi\).
\end{definition}

\begin{example}
In \(\mathbf{Top}\), a continuous map \(\phi: X \to Y\) ensures that \(\phi^{-1}(V) \subseteq X\) is open for every open set \(V \subseteq Y\).
\end{example}

\begin{theorem}[Continuity in Topology]
A function \(\phi: X \to Y\) between topological spaces is continuous if and only if for every open set \(V \subseteq Y\), the inverse image \(\phi^{-1}(V)\) is open in \(X\).
\end{theorem}

\begin{theorem}[Pullback Theorem in Sheaf Theory]
For a morphism \(\phi: X \to Y\) in a category with sheaves (e.g., \(\mathbf{Top}\), \(\mathbf{Sch}\)), the inverse image functor \(\phi^{-1}: \text{Sh}(Y) \to \text{Sh}(X)\) is exact, preserving the structure of sheaves.
\end{theorem}

We now define a theorem for geometric categories.

\begin{theorem}[Inverse Image Preservation Theorem for Geometric Categories]
Let \(\mathcal{C}\) be a geometric category (e.g., \(\mathbf{Top}\), \(\mathbf{Sch}\)) with pullbacks. For any morphism \(\phi: X \to Y\) in \(\mathcal{C}\), the inverse image functor \(\phi^{-1}: \text{Sub}(Y) \to \text{Sub}(X)\) preserves the lattice structure of subobjects. If \(\mathcal{C}\) admits sheaves, \(\phi^{-1}: \text{Sh}(Y) \to \text{Sh}(X)\) is exact and preserves sheaf isomorphisms, ensuring that the geometric structure of \(Y\) is reflected in \(X\).
\end{theorem}

\begin{proof}
In \(\mathbf{Top}\), the Continuity Theorem \cite{munkres} ensures that \(\phi: X \to Y\) is continuous if and only if \(\phi^{-1}(V)\) is open for every open set \(V \subseteq Y\), so \(\phi^{-1}\) preserves the lattice of open sets. In categories with sheaves (e.g., \(\mathbf{Top}\), \(\mathbf{Sch}\)), the Pullback Theorem \cite{kashiwara} guarantees that \(\phi^{-1}: \text{Sh}(Y) \to \text{Sh}(X)\) is exact, preserving sheaf structures. For schemes, \(\phi^{-1}\) maps prime ideals to prime ideals \cite{hart}, preserving geometric properties. Since \(\phi^{-1}\) is functorial and preserves monomorphisms, it maintains isomorphisms of subobjects or sheaves.
\end{proof}

\begin{remark}
The IIPT captures the geometric essence of inverse images preserving structure, unifying the Continuity Theorem and Pullback Theorem. The name distinguishes it from the Structure-Identity Principle \cite{mac}.
\end{remark}

\begin{example}
For a morphism of schemes \(\phi: X \to Y\), the inverse image of a prime ideal under the induced map on stalks is prime, preserving geometric structure \cite{hart}.
\end{example}

\subsection{Categorical Unification}
Category theory bridges algebra and geometry through dualities, where the HPT and IIPT interplay.

\begin{theorem}[Stone Duality]
The category of Boolean algebras, \(\mathbf{BoolAlg}\), is dually equivalent to the category of Stone spaces, \(\mathbf{Stone}\), via the spectrum functor.
\end{theorem}

\begin{theorem}[Gelfand Duality]
The category of commutative \(C^*\)-algebras is dually equivalent to the category of compact Hausdorff spaces via the spectrum functor.
\end{theorem}

\begin{theorem}[Adjoint Functor Theorem]
In a complete category, a functor has a left adjoint if it preserves limits, and a right adjoint if it preserves colimits.
\end{theorem}

\begin{remark}
Stone and Gelfand Dualities \cite{johnstone, takesaki} connect algebraic homomorphisms (HPT) to geometric inverse images (IIPT). The Adjoint Functor Theorem \cite{mac} underpins dualities like Spec, where algebraic and geometric structures are preserved \cite{hart}.
\end{remark}

\begin{example}
The Spec functor maps a ring homomorphism \(\phi: R \to S\) to a morphism \(\Spec S \to \Spec R\), with inverse images of prime ideals preserving geometric structure.
\end{example}

\subsection{Applications and Implications}
The HPT and IIPT, supported by prior results, impact advanced research:

\begin{itemize}
    \item \textbf{Algebraic Topology}: The HPT governs homology maps, while the IIPT defines covering spaces.
    \item \textbf{Algebraic Geometry}: The IIPT underpins étale cohomology via inverse images, while the HPT applies to ring homomorphisms.
    \item \textbf{Category Theory}: Stone, Gelfand, and Adjoint Functor Theorems reveal algebra-geometry correspondences.
\end{itemize}

\begin{corollary}
In any category with pullbacks, \(\phi^{-1}: \text{Sub}(Y) \to \text{Sub}(X)\) preserves subobject lattices, as per the IIPT.
\end{corollary}

\subsection{Conclusion}
The Homomorphism Preservation Theorem and Inverse Image Preservation Theorem formalize that algebra preserves structure via homomorphisms and geometry via inverse images. Building on the First Isomorphism Theorem, Continuity Theorem, Pullback Theorem, and dualities, these theorems unify pure mathematics. Graduate students are encouraged to apply this framework to algebraic topology, algebraic geometry, and category theory, deepening their research.

\bibliographystyle{amsplain}
\begin{thebibliography}{9}
\bibitem{lang} Lang, S., \emph{Algebra}, Graduate Texts in Mathematics, Springer, 2002.
\bibitem{mac} Mac Lane, S., \emph{Categories for the Working Mathematician}, Graduate Texts in Mathematics, Springer, 1998.
\bibitem{hart} Hartshorne, R., \emph{Algebraic Geometry}, Graduate Texts in Mathematics, Springer, 1977.
\bibitem{munkres} Munkres, J. R., \emph{Topology}, 2nd ed., Prentice Hall, 2000.
\bibitem{kashiwara} Kashiwara, M., Schapira, P., \emph{Sheaves on Manifolds}, Springer, 1990.
\bibitem{johnstone} Johnstone, P. T., \emph{Stone Spaces}, Cambridge University Press, 1982.
\bibitem{takesaki} Takesaki, M., \emph{Theory of Operator Algebras I}, Springer, 2002.
\end{thebibliography}

\end{document}
