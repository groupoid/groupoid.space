\section{Свідомість}

Формальна філософія якщо й має чимось займатися, то лише кодуванням різних
моделей свідомості (різного ступеня фрічєства, чому фрічєство взагалі
допускається скажу пізніше). Не здаватимуся політкоректним, більшість
філософів, які вивчаються в контексті свого предмета, я вважаю душевно
хворими людьми (і не бачу нічого в цьому поганого). Якщо ви можете уявити
будь-яку модель на MLTT і більш сучасних типових системах, і можете розповісти,
як ця модель кодує якийсь феномен, ви вже стаєте личинкою формального
міні-філософа. Зазвичай що складніша модель, то більше вписувалося її
як твір розглядатимуть інші формальні художники.

Багато хто критично ставиться до сучасних моделей АI, тому що вони
надто прості, щоб повірити, що там може зародитись якась автономна
свідомість. Щоб дати можливість системі зародитися і здобути якусь свободу,
ми повинні надати цій системі якийсь простір, і глибина цього простору
повинна перебувати в мові цієї системи. У MLTT таку глибину, яка закриває
навіть гьоделівські питання, надає ієрархія всесвітів. Причому вона виникає
незалежно від наших примх, типи зобов'язані десь перебувати, тому ми в мові
виділяємо контейнер для типів $U_n$, і кажемо, що цей тип містить усі типи
(наочно це демонструє індукція-рекурсія, в інших випадках потрібно вірити
тайпчекеру, що всі формейшин рули ядра живуть в $U_n$. Природно постає питання
межі послідовності $U_i$ прагне до $U_\omega$. А далі послідовність недоступних
кардиналів $U_\omega: U_{\omega+1}$. Всесвіт Махло є щось на зразок такої згортки цієї
послідовності. Така глибина дає певний спокій, що глибшого простору для
мови ми не запропонуємо для нашої моделі свідомості, тому що ми просто не
знаємо про це нічого. Інше відображення цієї глибини можна знайти в теорії
інфініті категорій, теорії інфініті топосів та їх фізичним моделям ізоморфізмів,
різних версіям теорії струн. Подих такого простору типів з відкритим дном
і контрактибл типом у вершині конуса --- це та мандала де знаходяться всі
малюнки всіх формальних філософів.

\newpage
Тепер про інформаційний тракт свідомості на нижніх рівнях, які вже можна
помацати у вигляді AI. Якщо припустити, що ландшафт моделей всіх можливих
мереж описується різними видами комплексів (симпліціальними, клітинними),
а їх інваріанти задаються гомологіями і гомотопічними типами, то така
глибина теж цілком сумісна з поточними методами, а гомологічна алгебра
вже застосовується в мережевій інженерії. Такий простір вимагає застосування
методів топології алгебри і створює нову глибину де може зародитися мислення.
Якщо стисло, то тут ідея така, що є якийсь генератор свідомості, який постійно
будує сам різні топології мереж, сам їх навчає, і сам веде реєстр цього поля
мереж, яке вбудовується в сам простір, як типи вбудовуються у всесвіт.

Третя більша частина, яка зараз відсутня в моделях свідомості, це фізична
комутативна математика, виняткові і класичні групи Лі, просто тому, що ми
себе виявили в цьому просторі, і очевидно, це якось пов'язано з мисленням.
Як і що тут вбудовувати мені незрозуміло, але здається, що тут спливає щось
на кшталт чакр або рівнів буття якоїсь мета-істоти, яка задає уречевлення
всієї мандали у видимому нам світі, який інкрустований іншими різноманіттями
як прикрасами основного простору.

\normalsize
