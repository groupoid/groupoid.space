\section{Модельні категорії Квілена}

PhD Деніела Квілена була присвячена диференціальним рівнянням,
але відразу після цього він перевівся в МІТ і почав працювати
в алгебраїній топології, під впливом Дена Кана. Через три роки
він видає Шпрінгеровські лекції з математики "Гомотопічна алгебра",
яка назавжди трансформувала алегбраїчну топологію від вивчення
топологічних просторів з точністю до гомотопій до загального
інструменту, що застосовується в інших галузях математики.

Модельні категорії вперше були успішно застосовані Воєводським
на підтвердження кон'юнктури Мілнора (для 2) і потім мотивної
кон'юнктури Блоха-Като (для n). Для доказу для 2 була побудована
зручна гомотопічна стабільна категорія узагальнених схем. Інфініті
категорії Джояля, досить добре досліджені Лур'є, є прямим
узагальненням модельних категорій.

До часу, коли Квіллен написав "Гомотопічну алгебру", вже було
деяке уявлення про те, як має виглядати теорія гомотопій.
Починаємо ми з категорії С та колекції морфізмів W --- слабкими
еквівалентностями. Завдання вправи інвертувати W морфізму щоб
отримати гомотопічну категорію. Хотілося б мати спосіб, щоб
можна було конструтувати похідні функтори. Для топологічного
простору X, його апроксимації LX і слабкої еквівалентності LX -> X це
означає, що ми повинні замінити X на LX. Це аналогічно до заміни
модуля або ланцюгового комплексу на проективну резольвенту.
Подвійним чином, для симпліційної множини K, Кан комплексу
RK, і слабкої еквілентності K -> RK ми повинні замінити K
на RK. У цьому випадку це аналогічно до заміни ланцюгового
комплексу ін'єктивною резольвентою.

\begin{lstlisting}
 modelStructure (C: category): U
   = (fibrations: fib C)
   * (cofibrations: cofib C)
   * (weakEqivalences: weak C)
   * unit
\end{lstlisting}

Таким чином Квілену потрібно було окрім поняття слабкої еквівалентності
ще й поняття розшарованого (RK) та корозшарованого (LX) об'єктів.
Ключовий інстайт з топології тут наступний, в неабелевих ситуаціях
об'єкти не надають достатньої структури поняття точної послідовності.
Тому стало зрозуміло, що для відновлення структури необхідно ще два
класи морфізмів: розшарування та корозшарування на додаток до слабких
еквівалентностей, яким ми повинні інвертувати для розбудови гомотопічної
категорії. Природно ці три колекції морфізом повинні задовольняти
набору умов, званих аксіомами модельних категорій: 1) наявність
малих лімітів і колимітів, 2) правило 3-для-2, 3) правило ректрактів,
4) правило підйому, 5) правило факторизації.

Цікавою властивістю модельних категорій є те, що дуальні до них
категорії перевертають розшарування та корозшарування, таким чином
реалізуючи дуальність Екманна-Хілтона. Розшарування та корозшарування
пов'язані, тому взаємовизначені. Корозшарування є морфізми, що мають
властивість лівого гомотопічного підйому по відношенню до ациклічних
розшарування і розшарування є морфизми, що мають властивість правого
гомотопічного підйому по відношенню до ациклічних кофібрацій.

Основним застосуванням модельних категорій у роботі Квілена було
присвячено категоріям топологічних просторів. Для топологічних
просторів існує дві модельні категорії: Квілена (1967) та Строма (1972).
Перша як розшарований використовує розшарування Серра, а як корозшаровування
морфізму які мають лівий гомотопічний підйом по відношенню до ациклічних
розшарування Серра, еквівалентно це ретракти відповідних CW-комплексів,
а як слабка еквівалентність виступає слабка гомотопічна. Друга модель
Строма як розшарування використовуються розшарування Гуревича, як
корозшарування стандартні корозшаровування, і як слабка еквівалентність ---
сильна гомотопічна еквівалентність.

\begin{lstlisting}
  quillen67
    : modelStructure Top
    = ( serreFibrations ,
        retractsCW ,
        weakHomotopyEquivalence )
\end{lstlisting}

\newpage
\begin{lstlisting}
  strom1972
    : modelStructure Top
    = ( hurewiczFibrations ,
        cofibrations ,
        strongHomotopyEquivalence )
\end{lstlisting}
    
Найпростіші модельні категорії можна побудувати для категорії множин,
де кількість ізоморфних моделей зростає до дев'яти. Наведемо деякі
конфігурації модельних категорій для категорії множин:

\begin{lstlisting}
  set0: modelStructure Set = (all,all,bijections)
  set1: modelStructure Set = (bijections,all,all)
  set2: modelStructure Set = (all,bijections,all)
  set3: modelStructure Set = (surjections,injections,all)
  set4: modelStructure Set = (injections,surjections,all)
\end{lstlisting}
    
У контексті модельних категорій визначаються сполучення Квілена,
лівий і правий функтори Квілена, Квілен еквівалентності, лівий і
правий похідні функтори, розширення Ріді (оскільки в загальному
випадку ліміти та коліміти не існують у гомотопічних категоріях
визначених на модельних категоріях, то модельні категорії Наприклад
є категорії С, і Ріді категорії J то J -> C має всю необхідну структуру
для існування гомотопічних (ко-)лімітів.

Для переходу від модельних категорій до інфініті категорій [або ($\infty$,1)-категорій]
необхідно перейти до категорій де морфізми утворюють не множини,
а симпліційні множини. Потім можна переходити до локалізації.

\begin{lstlisting}
  simplicial
    : modelStructure sSet
    = ( kanComplexes ,
        monos ,
        simplicialBijections )
\end{lstlisting}
    
Але для нас, для програмістів найцікавішими є модельні категорії
симпліціальних множин та модельні категорії кубічних множин, саме
в цьому сеттингу написано CCHM пейпер 2016 року, де показано модельну
структуру категорії кубічних множин.

\newpage
\begin{lstlisting}
  cubical
    : modelStructure cSet
    = ( kanComplexes ,
        monos ,
        geometricRealisation )
\end{lstlisting}

де cSet = [$\Box^{op}$,Set], а $\Box$ --- категорія
збагачена структурою алгебри де Моргана.

