\section{Теорії Янга-Міллса}

З точки зору алгебраїчної топології.

\subsection*{Електромагнетизм (Фотон)}

Теорія квантової електродинаміки розвинулася в 1930—1940-х рр.,
де унітарна група перетворень відіграє головну роль $U(1)$.
Шредінгер показав, що група $U(1)$ викликає фазовий
зсув $e^{i\theta}$ в електромагнітному полі,
що відповідає збереженню електричного заряду, зокрема при
розповсюдженні світла.

Електромагнітне поле може бути описано як вектор
потенціал $A\mu$ і тензор $F\mu\nu$. Зв'язок між групою $U(1)$
і перетвореннями Лоренца полягає в тому, що калібрувальне
перетворення електромагнітного потенціалу $A\mu$ при $U(1)$
аналогічне перетворенню просторово-часових координат при
перетвореннях Лоренца. В обох випадках ці перетворення
гарантують, що основна фізика залишається незмінною.

\subsection*{Теорії Янг-Міллса}

Калібровочна теорія поля --- це тип теорії поля де
Лангранжін, а значить і динамічна система загалом,
є інваріантним відносно локальних трансформацій
згідно певного гладкої сім'ї операторів (Груп Лі).

Теорія Янга-Міллса -- це калібровочна квантова теорія поля,
де головну роль відіграє спеціальна унітарна група $SU(n)$,
або більш загально, довільна компактна група Лі.

\subsection*{Слабка взаємодія (W і Z бозони)}

Група $SU(2)$ формалізує інваріант ізоспіна при колізіях,
спричиненими сильними взаємодіями.
Многовид $S^3$ є дифеоморфізмом до групи $SU(2)$,
який показує, що $SU(2)$ (многовид) є однозв'язним і що $S^3$
може бути наділений структурою компактної зв'язної групи Лі.

\subsection*{Сильна взаємодія}

Квантова хромодинаміка є неабелевою калібровочною теорією
поля на локальній (калібрувальній) групі симетрії під назвою $SU(3)$.
Її топологічну структуру можна зрозуміти
зауваживши, що $SU(3)$ діє транзитивно на одиничній сфері
$S^{5}$ у $\mathbb{C}^3 ≅ \mathbb{R}^6$. Стабілізатор
довільної точки сфери ізоморфний до $SU(2)$, яка топологічно є 3-сферою. Це
показує, що SU(3) є розшаруванням над базою $S^5$ з розшаруванням $S^3$.
Так як розшарування і бази просто-з'єднані,
тоді просто-зв'язність $SU(3)$ випливає з
стандартного топологічного результату
(довга точна послідовність гомотопічних
груп для пучків розшарувань).
