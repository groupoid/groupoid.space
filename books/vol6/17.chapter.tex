\section{Мадг'яміка в MLTT баченні}

У тибетській традиції є таке поняття --- Дхармадхату --- простор всіх Дхарм. Це простір всіх думок, які існують поза контекстами. Уявити складно -- ясна річ. Але у нас є MLTT теорія, яка допоможе це уявити.

Уявіть собі простір всіх мов програмування --- ясна річ всі знають про лямбда числення. Але не багато ж рубістів або ПХП програмісти замислюються, що вони пишуть мовою простору --- який з точністю до бітів моделюється теорією типів Мартіна Льофа (треба тільки всесвіт правильно відконфігурувати, це навчилися робити в 2001 році, коли Coq все дружно писали). Звичайно я сам у це спочатку не вірив, і думав, що є все-таки обмеження і що Данило Майстер, який вручну все сам робить замість того, щоб екстрактити це з Інфініті Топоса робить недаремну працю. Мені довелося написати прувер щоб зрозуміти --- таки так, кожен Данило Майстер, який вважає себе інженером, робить тимчасову і марну працю, даючи ярлики феноменам не бачачи їхньої суті.

Простір типів безмежний і всі типи одночасно живуть у цьому нескінченномірному просторі --- де кожен його вимір нашаровується один на одного, а самі типи утворюють патерни, схожі на гомотопічні групи. Оскільки ізоморфних типів у просторі нескінченного топосу вистачає, то MLTT теж є щось схоже на матрицю гомотопічних груп. Вони мають різні імена і можуть у принципі не проходити карбування на рівність тощо, але при компіляції в нетимізовану лямбду ізоморфні типи генеруватимуть сумісний код.

Всі типи мають чітку логіку заселення простору нескінченного топосу, подібно до того, як жителі самсари заселяють шість лок. Починається все з нижнього дна пекла --- типу Bottom. І потім по цеглині, починаючи з Unit(), потім A -> A, потім Nat, потім Stream, потім List, потім. і так далі аж до інфініті-групоїда, потім все починає повторюватися і візерунок починає змінюватися. Де у цього візерунка дірки я поки що не бачу, і яка у нього структура, але ніби відчуваю трохи подих цієї мандали. Ця мандала доступна в принципі всім програмістам, які можуть писати цикли, складати числа, більше рівня не потрібно. Всю цю математику можна переформулювати так, щоб MLTT викладати 11-річним дітям --- вважайте, що експеримент розпочато.

KLONG тибетською - це простір, простір всіх феноменів, KLONG CHEN NYING THIG широке простір серцевої сутності. Так само як Інфініті Топос - це простір всіх MLTT типів, який поєднує континуум і дискретний, а також ковтає всю математику з усіма її логіками і теоріями, тому що всі математики розмовляють вже давно цією мовою кванторів і нескінченних всесвітів.

Так ось, так само, як усі мови народжені з цього простору, так само і думки всі народжуються з одного простору, і всі вони пройменовані, так само, як пройменовані всі типи у всіх всесвітах. Народжуються тут умовно, тому що List не народжується, його можна виявити зрозуміти, але він завжди присутній у Топосі у всіх рівнях всесвіту. Усі типи є нествореними, доки їх не оголосить програміст як аксіом MLTT. Але якщо їх розглядати через призму нечистого бачення --- нетипізованого лямбду обчислення, то розглянути їх не вийде або дуже складно (Алонсо Черч та Боєм із Берардуччі не змогли цього зробити). Ну а аналогія просвітління - це вихід у цю мандалу, де видно всі типи відразу, або куди не кинеш погляд, бачиш скрізь схожі патерни. Послухай функціональних програмістів --- натуральні ж шизофреніки, бачать якісь катаморфізми там, де звичайна людина бачить while чи for; бачать групи, де звичайна людина бачить record; чому звичайні інженери дали вже тисячу ярликів. Знаєте, є таке в духовних практиках бути остороненим і не вішати ярлики на феномени, бо це безглуздо. Це теж саме що List/Cons і Stream/Mk --- мільйони програмістів дали різні назви цим штукам, але ці штуки існують самі по собі і побачити їх можна при індуктивному розгляді заповнення простору типами починаючи з Unit, і зробити це можна вже з третього кроку.

Взагалі аналогії настільки міцні, що я можу в священних текстах Тибету замінювати Дхармадхату на "Інфініті Топос", SEMS або Розум на "правила висновку" (до речі один з перекладів SEMS - це обчислювач) або на "компютешинал аксіоми", CHOS або DHARMA (ФО). Процес мислення --- це те, як ви конструюєте теореми, як ви вибудовуєте рекурсивні ланцюжки, тобто. як ви композитуєте думки (рекурсія та індукція). Якщо у типу немає рекурсора, ви не можете пустити думку з цього феномену. А пустити думку щодо феномену означає його усвідомлення тобто. звільнення в дхармакаї, що впринципі відповідає етимології слова елімінатор.

Пора покласти край суперечкам різних шкіл Мадхьямікі (моделі сватантрик) та описати всі ці логіки у вигляді MLTT програм.

