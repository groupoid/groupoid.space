\documentclass{article}
\usepackage{indentfirst}
\usepackage{listings}
\usepackage{amsmath}
\usepackage{amssymb}
\usepackage{url}
\usepackage{tikz-cd}
\usepackage[utf8]{inputenc}

\setlength{\parindent}{15pt}

\lstset{basicstyle=\footnotesize,inputencoding=utf8}

\begin{document}

\title{Issue X: The Robin Language}
\author{Maksym Sokhatskyi $^1$}
\date{ $^1$ National Technical University of Ukraine \\
       \small Igor Sikorsky Kyiv Polytechnical Institute \\
       \today }

\maketitle

\begin{abstract}

У цій статті Формальна Тензорна Мова'' або Лінійні Типи для Лінійної Алгебри''
розглядаються лінійні системи типів,
які є природним розширенням STLC для роботи з тензорами (структурами з лінійними алгебраїчними операціями),
та розподіленим у просторі та часі програмуванням.

Основні роботи для ознайомлення з темою: Ling, Guarded Cubical,
A Fibrational Framework for Substructural and Modal Logics,
APL-like interpreter in Rust, Futhark, NumLin.

{\bf Keywords}: Interaction Networks, Symmetric Monoidal Categories
\end{abstract}

\ifincludeTOC
  \tableofcontents
\fi

\newpage

\epigraph{Присвячується автору ML}{Робіну Мілнеру}

\section{The Robin Language}

\subsection{Пі-числення і лінійні типи}
Вперше семантика Пі-числення була представлена Мілнером разом з ML мовою,
за шо він дістав премію Тюрінга (один з небагатьох хто заслужено).
Якшо коротко то Пі-числення отримується з Лямбда-числення шляхом перетворення
кожної змінної в нескінченний стрім.

\paragraph{Приклад 1: факторіал}
Наприклад у нас є факторіал записаний таким чином:
\begin{lstlisting}
fac(0: int) -> 1
fac(x: int) -> x*fac(x -1)
\end{lstlisting}

Переписуємо його так шоб замість скалярного аргументу він споживав стрім аргументів, і результатом:
Альтернативна версія на стрімах:
\begin{lstlisting}
factorial(x: stream int): stream int -> result.set(x*fac(x.get()-1))
\end{lstlisting}

На відміну від попереднього факторіала, цей факторіал споживає
довільну кількість аргументів і для кожного з них виштовхує в
результуючий стрім результат обчислення факторіалу (використовуючи
попередню функцію). Цей новий факторіал на стрімах представляє собою
формалізацію нескінченного процесу який можна запустити,
це процес підключиться до черги аргументів, яку буде споживати
і до черги результату, куди буде виплювовути обчислення.

\paragraph{Приклад 2: скалярний добуток}
Функції можуть мати довільну кількість параметрів, всі ці параметри
-- це черги з яких нескінченний процес споживає повідомлення-аргументи
і випльовує їх в результуючу чергу-стрім. Наприклад лінійна
функція яка обчислює скалярний добуток трьохвимірних векторів виглядатиме так:

\begin{lstlisting}
dot3D(x: stream int, y: stream int): stream int ->
[x1, x2, x3] = x.get(3)
[y1, y2, y3] = y.get(3)
result.set(x1y1+x2y2+x2*y3)
\end{lstlisting}

Слід розрізняти лінійність як алгебраїне поняття і лінійнісь в Пі-численні.
В Пі-численні, а також в лінійній логіці Жана-Іва Жирара лінійність означає
шо змінна може бути використана тільки один раз, після чого курсор черги
зсувається і його неможливо буде вже вернути в попередню позицію після того
як якийсь процес прочитає це значення геттером. Саме така семантика присутня
в цих прикладах, зокрема в аксесорах get і set.

При реальних обочисленнях може статися так, що значення прочитане з черги
потрібно одразу двом функціям, тому природньо надати можливість закешувати
це значення, або іншими словами створити його копію x.duplicate для передачі
по мережі далі іншим функціям-процесам. Так само варто приділити увагу
деструктору пам'яті коли це значення вже використане усіма учасниками
і більше не потрібно нікому x.free.

\subsection{BLAS примітиви в ядрі}

Для реальних промислових обчислень скалярні добутки не рахують руками,
а є примітивами високооптимізованих бібліотек за допомогою SPIRAL чи вручну
закодовні. В статті NumLin автори зосереджуються на 1-му та 3-му рівню
BLAS, а це включає наступні примітиви для BLAS рівня 1:
1) Sum of vector magnitudes (Asum);
2) Scalar-vector product (Axpy);
3) Dot product (Dotp);
4) Modified Givens plane rotation of points (Rotm);
5) Vector-scalar product (Scal);
6) Index of the maximum absolute value element of a vector (Amax).
Так наступні примітиви для BLAS рівня 3:
1) Computes a matrix-matrix product with general matrices (Gemm);
2) Computes a matrix-matrix product where one input matrix is symmetric (Symm);
3) Performs a symmetric rank-k update (Syrk);
4) Декомпозиція Холецького (Posv)
Огляд примітивів рівня 1:

Також зауважимо шо єдиними типами даних які є в BLAS це Int і Float,
а також нам знадобляться хелпери типу Transpose і Size.
Тому синтаксичне дерево вбудованих примітивів BLAS буде виглядати так:</p>

\begin{lstlisting}
data Arith = Add | Sub | Mul | Div | Eq | Lt | Gt
data Builtin
   = Intop (a: Arith) | Floatop (a: Arith)    -- SIMD types
   | Get | Set | Duplicate | Free             -- linearity
   | Transpose | Size                         -- matrices
   | Asum | Axpy | Dotp | Rotm | Scal | Amax  -- BLAS Level 1
   | Symm | Gemm | Syrk | Posv                -- BLAS Level 3
\end{lstlisting}

\subsection{Лінійне лямбда числення}
Лінійне лямбда числення має всього три контексти: 1) Часткових дозволів,
2) Контекст лінійних змінних, 3) контекст звичайного лямбда числення.

Як мною було показано в QPL ми можемо одночасно мати два лямбда числення:
стандартне і лінійне на стрімах, однак тут ми просто будуємо звичайне лінійне лямбда
числення виділяючи його з основного дерева. Тензори в пам'яті містять додаткову
інформацію про часткові дозволи Fraction, якшо Fraction = 1
то мається на увазі повний ownership, якшо Fraction = 1/2 то частковий, шо означає
шо дві частин програми мають доступ до ньго, часткові дозволеності можна об'єднувати
в процесі нормалізації аж до повного ownership (Fraction = 1). Тут Pair представляє собою
лінійну пару, Fun --- лінійну функцію, Consume --- споживання змінної
перенос її з лінійного контексту в звичайний.

\begin{lstlisting}
data Fraction = Z | S (_: Fraction)
data Dimension = Vector | Matrix | Stream | Table
data Linear
   = Empty | Unit | Bool
   | Int | Float
   | Tensor (a: Fraction) (x: Dimension)
   | Pair (a b: Linear) | Fun (a b: Linear)
   | Consume (a: Linear) | All (a: Var) (b: Linear)
\end{lstlisting}

\paragraph{Приклад 3: лінійна регресія}
$$
  Posv : matrix \multimap matrix \multimap matrix \otimes matrix \\
  \beta = (X^TX)^{-1}X^Ty
$$

Програма, яка обчислює лінійну регресію спочатку визначає розмір матриці $x$,
потім створює в пам'яті нову матриці $X^T y$ і $x^Tx$,
після чого обчислює за допомогою $Posv$ і цих двох матриць безпосередньо результат.

\begin{lstlisting}
Linear_Regression(x y: matrix float) ->
(n, m) = Size x
xy = Tensor (m, 1) { Transpose(x) * y }
xTx = Tensor (m, m) { Transpose(x) * x }
(w, cholesky) = Posv xTx xy
Free w
result.emit(cholesky)
\end{lstlisting}

\subsection{AST результуючої мови}
Повне дерево виразів:
\begin{lstlisting}
data Exp
= Variable (: Var)
| Prim (: Builtin)
| Star | True | False
| Int (: nat) | Float (: float)
| Lambda (a: Var) (b: Linear) (c: Exp)
| App (a b: Exp) | Pair (a b: Var) (c d: Exp)
| Consume (a: Var) (b c: Exp)
| Gen (a: Var) (b: Exp) | Spec (a: Exp) (b: Fraction)
| Fix (a b: Var) (c d: Linear) (e: Exp)
| If (a b c: Exp) | Let (a: Var) (b c: Exp)
\end{lstlisting}

\paragraph{Приклад 4: Одновимірна конволюція для часових рядів}
\begin{lstlisting}
SimpleConvolution1D (i: int) (n : int) (x0: float)
(write: vector float) (weights: vector float): vector float ->
if n = i then result.emit(write)
a = [w0,w1,w2] = weights.get(0,3)
b = [x0,x1,x2] = [ x0 | write.get(i,2) ]
write.set(i, Dotp a b)
SimpleConvolution1D (i + 1) n x1 write weights

test ->
write   = [10, 50, 60, 10, 20, 30, 40]
weights = [1/3, 1/3, 1/3, 1/3, 1/3, 1/3, 1/3]
cnn = SimpleConvolution1D 0 6 10 write weights
[10.0, 40.0, 40.0, 30.0, 19.999999999999996, 30.0, 40.0] = cnn
\end{lstlisting}

\paragraph{Самоперевірка на Elixir}
\begin{lstlisting}
write   = [10.0, 50.0, 60.0, 10.0, 20.0, 30.0, 40.0]
weights = [1/3, 1/3, 1/3, 1/3, 1/3, 1/3, 1/3]
result  = CNN.conv1D(1,6,10.0,write,weights)

initial: [10.0,50.0,60.0,10.0,20.0,30.0,40.0]
result: [10.0,40.0,40.0,30.0,19.999999999999996,30.0,40.0]
\end{lstlisting}

\end{document}

