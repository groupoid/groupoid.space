\documentclass{article}
\usepackage{indentfirst}
\usepackage{listings}
\usepackage{amsmath}
\usepackage{amssymb}
\usepackage{amsthm}
\usepackage{url}
\usepackage{tikz-cd}
\usepackage[utf8]{inputenc}

\ProvidesPackage{journal}
\usepackage{graphicx}
\usepackage{mathtools}
\usepackage{hyphenat}
\usepackage{hyperref}
\usepackage{adjustbox}
\usepackage{listings}
\usepackage{verbatim}
\usepackage{xcolor}
\usepackage{amsfonts}
\usepackage{amscd}
\usepackage{amsmath}
\usepackage{amssymb}
\usepackage{amsthm}
\usepackage{tikz}
\usepackage{tikz-cd}
\usepackage{url}
\usepackage[utf8]{inputenc}
\usepackage[english,ukrainian]{babel}
\usepackage{float}
\usepackage{url}
\usepackage{tikz}
\usepackage{tikz-cd}
\usepackage[utf8]{inputenc}
\usepackage{graphicx}
\usepackage[utf8]{inputenc}
\usepackage[T1]{fontenc}
\usepackage{lmodern}
\usepackage{tocloft}
\usepackage{hyperref}
\usepackage{xcolor}
\usepackage[only,llbracket,rrbracket,llparenthesis,rrparenthesis]{stmaryrd}

\usetikzlibrary{babel}

\newcommand*{\incmap}{\hookrightarrow}
\newcommand*{\thead}[1]{\multicolumn{1}{c}{\bfseries #1}}
\renewcommand{\Join}{\vee} % Join operation symbol
\newcommand{\tabstyle}[0]{\scriptsize\ttfamily\fontseries{l}\selectfont}

\lstset{
  basicstyle=\footnotesize,
  inputencoding=utf8,
  identifierstyle=,
  literate=
{𝟎}{{\ensuremath{\mathbf{0}}}}1
{𝟏}{{\ensuremath{\mathbf{1}}}}1
{≔}{{\ensuremath{\mathrm{:=}}}}1
{α}{{\ensuremath{\mathrm{\alpha}}}}1
{ᵂ}{{\ensuremath{^W}}}1
{β}{{\ensuremath{\mathrm{\beta}}}}1
{γ}{{\ensuremath{\mathrm{\gamma}}}}1
{δ}{{\ensuremath{\mathrm{\delta}}}}1
{ε}{{\ensuremath{\mathrm{\varepsilon}}}}1
{ζ}{{\ensuremath{\mathrm{\zeta}}}}1
{η}{{\ensuremath{\mathrm{\eta}}}}1
{θ}{{\ensuremath{\mathrm{\theta}}}}1
{ι}{{\ensuremath{\mathrm{\iota}}}}1
{κ}{{\ensuremath{\mathrm{\kappa}}}}1
{λ}{{\ensuremath{\mathrm{\lambda}}}}1
{μ}{{\ensuremath{\mathrm{\mu}}}}1
{ν}{{\ensuremath{\mathrm{\nu}}}}1
{ξ}{{\ensuremath{\mathrm{\xi}}}}1
{π}{{\ensuremath{\mathrm{\mathnormal{\pi}}}}}1
{ρ}{{\ensuremath{\mathrm{\rho}}}}1
{σ}{{\ensuremath{\mathrm{\sigma}}}}1
{τ}{{\ensuremath{\mathrm{\tau}}}}1
{φ}{{\ensuremath{\mathrm{\varphi}}}}1
{χ}{{\ensuremath{\mathrm{\chi}}}}1
{ψ}{{\ensuremath{\mathrm{\psi}}}}1
{ω}{{\ensuremath{\mathrm{\omega}}}}1
{Π}{{\ensuremath{\mathrm{\Pi}}}}1
{Γ}{{\ensuremath{\mathrm{\Gamma}}}}1
{Δ}{{\ensuremath{\mathrm{\Delta}}}}1
{Θ}{{\ensuremath{\mathrm{\Theta}}}}1
{Λ}{{\ensuremath{\mathrm{\Lambda}}}}1
{Σ}{{\ensuremath{\mathrm{\Sigma}}}}1
{Φ}{{\ensuremath{\mathrm{\Phi}}}}1
{Ξ}{{\ensuremath{\mathrm{\Xi}}}}1
{Ψ}{{\ensuremath{\mathrm{\Psi}}}}1
{Ω}{{\ensuremath{\mathrm{\Omega}}}}1
{ℵ}{{\ensuremath{\aleph}}}1
{≤}{{\ensuremath{\leq}}}1
{≥}{{\ensuremath{\geq}}}1
{≠}{{\ensuremath{\neq}}}1
{≈}{{\ensuremath{\approx}}}1
{≡}{{\ensuremath{\equiv}}}1
{≃}{{\ensuremath{\simeq}}}1
{≤}{{\ensuremath{\leq}}}1
{≥}{{\ensuremath{\geq}}}1
{∂}{{\ensuremath{\partial}}}1
{∆}{{\ensuremath{\triangle}}}1 % or \laplace?
{∫}{{\ensuremath{\int}}}1
{∑}{{\ensuremath{\mathrm{\Sigma}}}}1
{→}{{\ensuremath{\rightarrow}}}1
{⊥}{{\ensuremath{\perp}}}1
{∞}{{\ensuremath{\infty}}}1
{∂}{{\ensuremath{\partial}}}1
{∓}{{\ensuremath{\mp}}}1
{±}{{\ensuremath{\pm}}}1
{×}{{\ensuremath{\times}}}1
{⊕}{{\ensuremath{\oplus}}}1
{⊗}{{\ensuremath{\otimes}}}1
{⊞}{{\ensuremath{\boxplus}}}1
{∇}{{\ensuremath{\nabla}}}1
{√}{{\ensuremath{\sqrt}}}1
{⬝}{{\ensuremath{\cdot}}}1
{•}{{\ensuremath{\cdot}}}1
{∘}{{\ensuremath{\circ}}}1
{⁻}{{\ensuremath{^{-}}}}1
{▸}{{\ensuremath{\blacktriangleright}}}1
{★}{{\ensuremath{\star}}}1
{∧}{{\ensuremath{\wedge}}}1
{∨}{{\ensuremath{\vee}}}1
{¬}{{\ensuremath{\neg}}}1
{⊢}{{\ensuremath{\vdash}}}1
{⟨}{{\ensuremath{\langle}}}1
{⟩}{{\ensuremath{\rangle}}}1
{↦}{{\ensuremath{\mapsto}}}1
{→}{{\ensuremath{\rightarrow}}}1
{↔}{{\ensuremath{\leftrightarrow}}}1
{⇒}{{\ensuremath{\Rightarrow}}}1
{⟹}{{\ensuremath{\Longrightarrow}}}1
{⇐}{{\ensuremath{\Leftarrow}}}1
{⟸}{{\ensuremath{\Longleftarrow}}}1
{∩}{{\ensuremath{\cap}}}1
{∪}{{\ensuremath{\cup}}}1
{·}{{\ensuremath{\cdot}}}1
{ᵢ}{{\ensuremath{_i}}}1
{ⱼ}{{\ensuremath{_j}}}1
{₊}{{\ensuremath{_+}}}1
{ℑ}{{\ensuremath{\Im}}}1
{𝒢}{{\ensuremath{\mathcal{G}}}}1
{ℕ}{{\ensuremath{\mathbb{N}}}}1
{𝟘}{{\ensuremath{\mathbb{0}}}}1
{ℤ}{{\ensuremath{\mathbb{Z}}}}1
{ℝ}{{\ensuremath{\mathbb{R}}}}1
{⊂}{{\ensuremath{\subseteq}}}1
{⊆}{{\ensuremath{\subseteq}}}1
{⊄}{{\ensuremath{\nsubseteq}}}1
{⊈}{{\ensuremath{\nsubseteq}}}1
{⊃}{{\ensuremath{\supseteq}}}1
{⊇}{{\ensuremath{\supseteq}}}1
{⊅}{{\ensuremath{\nsupseteq}}}1
{⊉}{{\ensuremath{\nsupseteq}}}1
{∈}{{\ensuremath{\in}}}1
{∉}{{\ensuremath{\notin}}}1
{∋}{{\ensuremath{\ni}}}1
{∌}{{\ensuremath{\notni}}}1
{∅}{{\ensuremath{\emptyset}}}1
{∖}{{\ensuremath{\setminus}}}1
{†}{{\ensuremath{\dag}}}1
{ℕ}{{\ensuremath{\mathbb{N}}}}1
{ℤ}{{\ensuremath{\mathbb{Z}}}}1
{ℝ}{{\ensuremath{\mathbb{R}}}}1
{ℚ}{{\ensuremath{\mathbb{Q}}}}1
{ℂ}{{\ensuremath{\mathbb{C}}}}1
{⌞}{{\ensuremath{\llcorner}}}1
{⌟}{{\ensuremath{\lrcorner}}}1
{⦃}{{\ensuremath{ \{\!| }}}1
{⦄}{{\ensuremath{ |\!\} }}}1
{ᵁ}{{\ensuremath{^U}}}1
{₋}{{\ensuremath{_{-}}}}1
{₁}{{\ensuremath{_1}}}1
{₂}{{\ensuremath{_2}}}1
{₃}{{\ensuremath{_3}}}1
{₄}{{\ensuremath{_4}}}1
{₅}{{\ensuremath{_5}}}1
{₆}{{\ensuremath{_6}}}1
{₇}{{\ensuremath{_7}}}1
{₈}{{\ensuremath{_8}}}1
{₉}{{\ensuremath{_9}}}1
{₀}{{\ensuremath{_0}}}1
{¹}{{\ensuremath{^1}}}1
{ₙ}{{\ensuremath{_n}}}1
{ₘ}{{\ensuremath{_m}}}1
{↑}{{\ensuremath{\uparrow}}}1
{↓}{{\ensuremath{\downarrow}}}1
{▸}{{\ensuremath{\triangleright}}}1
{∀}{{\ensuremath{\forall}}}1
{∃}{{\ensuremath{\exists}}}1
{λ}{{\ensuremath{\mathrm{\lambda}}}}1
{=}{{\ensuremath{=}}}1
{<}{{\ensuremath{\textless}}}1
{>}{{\ensuremath{\textgreater}}}1
{_}{{$\_$}}1
{(}{(}1
{(}{(}1
{‖}{{\ensuremath{\Vert}}}1
{+}{{+}}1
{*}{{*}}1,
}

\theoremstyle{definition}
\newtheorem{definition}{Definition}
\newtheorem{theorem}{Theorem}
\newtheorem{lemma}{Lemma}
\newtheorem{example}{Example}



\begin{document}

\title{Issue VI: Cartesian Interpreter}
\author{Maksym Sokhatskyi $^1$}
\date{ $^1$ National Technical University of Ukraine \\
       \small Igor Sikorsky Kyiv Polytechnical Institute \\
       \today }

\maketitle

\begin{abstract}

Minimal language for sequential computations in cartesian closed categories.
\\
\\
{\bf Keywords}: Lambda Calculus, Cartesian Closed Categories
\end{abstract}

\ifincludeTOC
  \tableofcontents
\fi

\newpage

\epigraph{Присвячується автору типізованого лямбда числення}{Алонзо Чорчу}

\section{The Alonzo Language}

Мова програмування $\mathbf{Alonzo}$ — це чиста нетипізована мова, що є внутрішньою мовою декартово-замкнених категорій. Вона базується на лямбда-численні, розширеному парами, проєкціями та термінальним об’єктом, забезпечуючи мінімальну модель для обчислень у категорійному контексті.

\subsection{Синтаксис}

Терми $\mathbf{Alonzo}$ складаються зі змінних, лямбда-абстракцій, застосувань, пар, проєкцій (першої та другої) та термінального об’єкта. Це мінімальна мова, що підтримує обчислення через бета-редукцію та проєкції.

\begin{lstlisting}[mathescape=true]
I = #identifier
O = I | ( O ) | O O | λ I -> O | O , O | O.1 | O.2 | 1
\end{lstlisting}

\begin{lstlisting}[mathescape=true]
type term =
  | Var of string
  | Lam of string * term
  | App of term * term
  | Pair of term * term
  | Fst of term
  | Snd of term
  | Unit
\end{lstlisting}

\subsection{Правила обчислень}

Основними правилами обчислень у $\mathbf{Alonzo}$ є бета-редукція для лямбда-абстракцій та правила проєкцій для пар. Термінальний об’єкт є незвідним.

\begin{lstlisting}[mathescape=true]
App (Lam (x, b), a) → subst x a b
Fst (Pair (t1, t2)) → t1
Snd (Pair (t1, t2)) → t2
\end{lstlisting}

$$
\frac{(\lambda x . b)\ a}{b [a/x]} \\[10pt]
\frac{\text{fst}\ \langle t_1, t_2 \rangle}{t_1} \\[10pt]
\frac{\text{snd}\ \langle t_1, t_2 \rangle}{t_2}
$$

\newpage
\subsection{Підстановка}

\begin{lstlisting}[mathescape=true]
let rec subst x s = function
  | Var y -> if x = y then s else Var y
  | Lam (y, t) when x <> y -> Lam (y, subst x s t)
  | App (f, a) -> App (subst x s f, subst x s a)
  | Pair (t1, t2) -> Pair (subst x s t1, subst x s t2)
  | Fst t -> Fst (subst x s t)
  | Snd t -> Snd (subst x s t)
  | Unit -> Unit
  | t -> t
\end{lstlisting}

\subsection{Рівність}

\begin{lstlisting}[mathescape=true]
let rec equal t1 t2 =
  match t1, t2 with
  | Var x, Var y -> x = y
  | Lam (x, b), Lam (y, b') -> equal b (subst y (Var x) b')
  | Lam (x, b), t -> equal b (App (t, Var x))
  | t, Lam (x, b) -> equal (App (t, Var x)) b
  | App (f1, a1), App (f2, a2) -> equal f1 f2 && equal a1 a2
  | Pair (t1, t2), Pair (t1', t2') -> equal t1 t1' && equal t2 t2'
  | Fst t, Fst t' -> equal t t'
  | Snd t, Snd t' -> equal t t'
  | Unit, Unit -> true
  | _ -> false
\end{lstlisting}

\subsection{Редукція}

\begin{lstlisting}[mathescape=true]
let rec reduce = function
  | App (Lam (x, b), a) -> subst x a b
  | App (f, a) -> App (reduce f, reduce a)
  | Pair (t1, t2) -> Pair (reduce t1, reduce t2)
  | Fst (Pair (t1, t2)) -> t1
  | Fst t -> Fst (reduce t)
  | Snd (Pair (t1, t2)) -> t2
  | Snd t -> Snd (reduce t)
  | Unit -> Unit
  | t -> t
\end{lstlisting}

\subsection{Нормалізація}

\begin{lstlisting}[mathescape=true]
let rec normalize t =
  let t' = reduce t in
  if equal t t' then t else normalize t'
\end{lstlisting}

\subsection{Внутрішня мова ДЗК}

Мова $\mathbf{Alonzo}$ є внутрішньою мовою декартово-замкненої категорії (ДЗК).
Вона включає лямбда-абстракції та застосування для замкнутої структури,
пари та проєкції для декартового добутку, а також термінальний об’єкт для
відновлення повної структури ДЗК.


\begin{thebibliography}{9}

\bibitem{church1933}
Alonzo Church.
\textit{A Set of Postulates for the Foundation of Logic}.
1933.

\bibitem{church1941}
Alonzo Church.
\textit{An Unsolvable Problem of Elementary Number Theory}.
1941.

\bibitem{curry1951}
Haskell Curry and Robert Feys.
\textit{Combinatory Logic, Volume I}.
1951.

\bibitem{scott1970}
Dana Scott.
\textit{A Type-Free Theory of Lambda Calculus}.
1970.

\bibitem{reynolds1974}
John C. Reynolds.
\textit{Towards a Theory of Type Structure}.
1974.

\bibitem{barendregt1984}
Henk Barendregt.
\textit{The Lambda Calculus: Its Syntax and Semantics}.
1984.

\bibitem{cousineau1985}
G. Cousineau, P.-L. Curien, and M. Mauny.
\textit{The Categorical Abstract Machine}.
1985.

\end{thebibliography}

\end{document}
