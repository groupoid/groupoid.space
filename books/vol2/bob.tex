\documentclass{article}
\usepackage{indentfirst}
\usepackage{listings}
\usepackage{amsmath}
\usepackage{amssymb}
\usepackage{amsthm}
\usepackage{url}
\usepackage{tikz-cd}
\usepackage[utf8]{inputenc}

\setlength{\parindent}{15pt}

\theoremstyle{definition}
\newtheorem{theorem}{Theorem}
\newtheorem{definition}{Definition}

\lstset{basicstyle=\footnotesize,inputencoding=utf8,
literate=
{𝟎}{{\ensuremath{\mathbf{0}}}}1
{𝟏}{{\ensuremath{\mathbf{1}}}}1
{≔}{{\ensuremath{\mathrm{:=}}}}1
{α}{{\ensuremath{\mathrm{\alpha}}}}1
{ᵂ}{{\ensuremath{^W}}}1
{β}{{\ensuremath{\mathrm{\beta}}}}1
{γ}{{\ensuremath{\mathrm{\gamma}}}}1
{δ}{{\ensuremath{\mathrm{\delta}}}}1
{ε}{{\ensuremath{\mathrm{\varepsilon}}}}1
{ζ}{{\ensuremath{\mathrm{\zeta}}}}1
{η}{{\ensuremath{\mathrm{\eta}}}}1
{θ}{{\ensuremath{\mathrm{\theta}}}}1
{ι}{{\ensuremath{\mathrm{\iota}}}}1
{κ}{{\ensuremath{\mathrm{\kappa}}}}1
{λ}{{\ensuremath{\mathrm{\lambda}}}}1
{μ}{{\ensuremath{\mathrm{\mu}}}}1
{ν}{{\ensuremath{\mathrm{\nu}}}}1
{ξ}{{\ensuremath{\mathrm{\xi}}}}1
{π}{{\ensuremath{\mathrm{\mathnormal{\pi}}}}}1
{ρ}{{\ensuremath{\mathrm{\rho}}}}1
{σ}{{\ensuremath{\mathrm{\sigma}}}}1
{τ}{{\ensuremath{\mathrm{\tau}}}}1
{φ}{{\ensuremath{\mathrm{\varphi}}}}1
{χ}{{\ensuremath{\mathrm{\chi}}}}1
{ψ}{{\ensuremath{\mathrm{\psi}}}}1
{ω}{{\ensuremath{\mathrm{\omega}}}}1
{Π}{{\ensuremath{\mathrm{\Pi}}}}1
{Γ}{{\ensuremath{\mathrm{\Gamma}}}}1
{Δ}{{\ensuremath{\mathrm{\Delta}}}}1
{Θ}{{\ensuremath{\mathrm{\Theta}}}}1
{Λ}{{\ensuremath{\mathrm{\Lambda}}}}1
{Σ}{{\ensuremath{\mathrm{\Sigma}}}}1
{Φ}{{\ensuremath{\mathrm{\Phi}}}}1
{Ξ}{{\ensuremath{\mathrm{\Xi}}}}1
{Ψ}{{\ensuremath{\mathrm{\Psi}}}}1
{Ω}{{\ensuremath{\mathrm{\Omega}}}}1
{ℵ}{{\ensuremath{\aleph}}}1
{≤}{{\ensuremath{\leq}}}1
{≥}{{\ensuremath{\geq}}}1
{≠}{{\ensuremath{\neq}}}1
{≈}{{\ensuremath{\approx}}}1
{≡}{{\ensuremath{\equiv}}}1
{≃}{{\ensuremath{\simeq}}}1
{≤}{{\ensuremath{\leq}}}1
{≥}{{\ensuremath{\geq}}}1
{∂}{{\ensuremath{\partial}}}1
{∆}{{\ensuremath{\triangle}}}1 % or \laplace?
{∫}{{\ensuremath{\int}}}1
{∑}{{\ensuremath{\mathrm{\Sigma}}}}1
{→}{{\ensuremath{\rightarrow}}}1
{⊥}{{\ensuremath{\perp}}}1
{∞}{{\ensuremath{\infty}}}1
{∂}{{\ensuremath{\partial}}}1
{∓}{{\ensuremath{\mp}}}1
{±}{{\ensuremath{\pm}}}1
{×}{{\ensuremath{\times}}}1
{⊕}{{\ensuremath{\oplus}}}1
{⊗}{{\ensuremath{\otimes}}}1
{⊞}{{\ensuremath{\boxplus}}}1
{∇}{{\ensuremath{\nabla}}}1
{√}{{\ensuremath{\sqrt}}}1
{⬝}{{\ensuremath{\cdot}}}1
{•}{{\ensuremath{\cdot}}}1
{∘}{{\ensuremath{\circ}}}1
{⁻}{{\ensuremath{^{-}}}}1
{▸}{{\ensuremath{\blacktriangleright}}}1
{∧}{{\ensuremath{\wedge}}}1
{∨}{{\ensuremath{\vee}}}1
{¬}{{\ensuremath{\neg}}}1
{⊢}{{\ensuremath{\vdash}}}1
{⟨}{{\ensuremath{\langle}}}1
{⟩}{{\ensuremath{\rangle}}}1
{↦}{{\ensuremath{\mapsto}}}1
{→}{{\ensuremath{\rightarrow}}}1
{↔}{{\ensuremath{\leftrightarrow}}}1
{⇒}{{\ensuremath{\Rightarrow}}}1
{⟹}{{\ensuremath{\Longrightarrow}}}1
{⇐}{{\ensuremath{\Leftarrow}}}1
{⟸}{{\ensuremath{\Longleftarrow}}}1
{∩}{{\ensuremath{\cap}}}1
{∪}{{\ensuremath{\cup}}}1
{⊂}{{\ensuremath{\subseteq}}}1
{⊆}{{\ensuremath{\subseteq}}}1
{⊄}{{\ensuremath{\nsubseteq}}}1
{⊈}{{\ensuremath{\nsubseteq}}}1
{⊃}{{\ensuremath{\supseteq}}}1
{⊇}{{\ensuremath{\supseteq}}}1
{⊅}{{\ensuremath{\nsupseteq}}}1
{⊉}{{\ensuremath{\nsupseteq}}}1
{∈}{{\ensuremath{\in}}}1
{∉}{{\ensuremath{\notin}}}1
{∋}{{\ensuremath{\ni}}}1
{∌}{{\ensuremath{\notni}}}1
{∅}{{\ensuremath{\emptyset}}}1
{∖}{{\ensuremath{\setminus}}}1
{†}{{\ensuremath{\dag}}}1
{ℕ}{{\ensuremath{\mathbb{N}}}}1
{ℤ}{{\ensuremath{\mathbb{Z}}}}1
{ℝ}{{\ensuremath{\mathbb{R}}}}1
{ℚ}{{\ensuremath{\mathbb{Q}}}}1
{ℂ}{{\ensuremath{\mathbb{C}}}}1
{⌞}{{\ensuremath{\llcorner}}}1
{⌟}{{\ensuremath{\lrcorner}}}1
{⦃}{{\ensuremath{ \{\!| }}}1
{⦄}{{\ensuremath{ |\!\} }}}1
{₁}{{\ensuremath{_1}}}1
{₂}{{\ensuremath{_2}}}1
{₃}{{\ensuremath{_3}}}1
{₄}{{\ensuremath{_4}}}1
{₅}{{\ensuremath{_5}}}1
{₆}{{\ensuremath{_6}}}1
{₇}{{\ensuremath{_7}}}1
{₈}{{\ensuremath{_8}}}1
{₉}{{\ensuremath{_9}}}1
{₀}{{\ensuremath{_0}}}1
{¹}{{\ensuremath{^1}}}1
{ₙ}{{\ensuremath{_n}}}1
{ₘ}{{\ensuremath{_m}}}1
{↑}{{\ensuremath{\uparrow}}}1
{↓}{{\ensuremath{\downarrow}}}1
{▸}{{\ensuremath{\triangleright}}}1
{∀}{{\ensuremath{\forall}}}1
{∃}{{\ensuremath{\exists}}}1
{λ}{{\ensuremath{\mathrm{\lambda}}}}1
{=}{{\ensuremath{=}}}1
{<}{{\ensuremath{\langle}}}1
{>}{{\ensuremath{\rangle}}}1
{(}{(}1
{(}{(}1
{‖}{‖}1
{+}{{+}}1
{*}{{*}}1,
}

\begin{document}

\title{Issue VIII: Linear Interpreter}
\author{Maksym Sokhatskyi $^1$}
\date{ $^1$ National Technical University of Ukraine \\
       \small Igor Sikorsky Kyiv Polytechnical Institute \\
       \today }

\maketitle

\begin{abstract}

Minimal language for parallel computations in symmetric monoidal categories. \\
{\bf Keywords}: Interaction Networks, Symmetric Monoidal Categories
\end{abstract}

\ifincludeTOC
  \tableofcontents
\fi

\newpage

\subsection{The Bob Language}

Мова програмування $\mathbf{Bob}$ — це внутрішня мова симетричних моноїдальних категорій,
що реалізує паралельні обчислення через взаємодію комбінаторів ($\zeta$, $\delta$, $\epsilon$) з правилами
анігіляції та комутації, придатна для моделювання лінійних і паралельних систем.

\begin{definition}
Терми $\mathbf{Bob}$ складаються зі змінних, комбінаторів (Con, Dup, Era), пар, обміну (Swap),
зв’язування (Let) та одиниці (Unit). Мова підтримує афінну логіку, забороняючи повторне використання змінних.
\begin{lstlisting}[mathescape=true]
I = #identifier
BOB = I | Con BOB  | Dup BOB | Era BOB
        | Pair (BOB, BOB)    | Unit
        | Let (I, BOB, BOB)  | Swap BOB
\end{lstlisting}
\end{definition}

\begin{definition}
Кодування термів у мові OCaml:
\begin{lstlisting}[mathescape=true]
type term =
    | Var of string
    | Con of term
    | Dup of term
    | Era of term
    | Pair of term * term
    | Swap of term
    | Let of string * term * term
    | Unit
\end{lstlisting}
\end{definition}

\begin{theorem}
Правила обчислень у $\mathbf{Bob}$ базуються на анігіляції та комутації комбінаторів:
\begin{lstlisting}[mathescape=true]
Con (Con x) → Pair (x, x)
Dup (Dup x) → Pair (x, x)
Era (Era x) → Unit
Con (Dup x) → Dup (Con x)
Con (Era x) → Pair (Era x, Era x)
Dup (Era x) → Pair (Era x, Era x)
Swap (Pair (t, u)) → Pair (u, t)
Let (x, t, u) → subst x t u
\end{lstlisting}
\end{theorem}

\begin{equation}
\tag{$\zeta$-annihilation}
\dfrac
{\zeta (\zeta (x))}
{(x, x)}
\end{equation}

\begin{equation}
\tag{$\delta$-annihilation}
\dfrac
{\delta (\delta (x))}
{(x, x)}
\end{equation}

\begin{equation}
\tag{$\epsilon$-annihilation}
\dfrac
{\varepsilon (\varepsilon (x))}
{\mathbf{1}}
\end{equation}

\begin{definition}
Підстановка в $\mathbf{Bob}$:
\begin{lstlisting}[mathescape=true]
let rec subst env var term = function
    | Var v ->
        if v = var then
            if is_bound var env then failwith "Affine violation: variable used twice"
            else term
        else Var v
    | Con t -> Con (subst env var term t)
    | Dup t -> Dup (subst env var term t)
    | Era t -> Era (subst env var term t)
    | Pair (t, u) -> Pair (subst env var term t, subst env var term u)
    | Swap t -> Swap (subst env var term t)
    | Let (x, t1, t2) ->
        let t1' = subst env var term t1 in
        if x = var then Let (x, t1', t2)
        else Let (x, t1', subst env var term t2)
    | Unit -> Unit
\end{lstlisting}
\end{definition}

\begin{definition}
Редукція термів у $\mathbf{Bob}$:
\begin{lstlisting}[mathescape=true]
let reduce env term =
    match term with
    | Con (Con x) -> Pair (x, x)
    | Dup (Dup x) -> Pair (x, x)
    | Era (Era x) -> Unit
    | Con (Dup x) -> Dup (Con x)
    | Con (Era x) -> Pair (Era x, Era x)
    | Dup (Era x) -> Pair (Era x, Era x)
    | Swap (Pair (t, u)) -> Pair (u, t)
    | Let (x, t, u) -> subst env x t u
    | _ -> term
\end{lstlisting}
\end{definition}

\begin{definition}
Пошук активних пар для редукції:
\begin{lstlisting}[mathescape=true]
let rec find_redexes env term acc =
    match term with
    | Con (Con x) -> (term, Pair (x, x)) :: acc
    | Dup (Dup x) -> (term, Pair (x, x)) :: acc
    | Era (Era x) -> (term, Unit) :: acc
    | Con (Dup x) -> (term, Dup (Con x)) :: acc
    | Con (Era x) -> (term, Pair (Era x, Era x)) :: acc
    | Dup (Era x) -> (term, Pair (Era x, Era x)) :: acc
    | Swap (Pair (t, u)) -> (term, Pair (u, t)) :: acc
    | Let (x, t, u) -> (term, subst env x t u) :: find_redexes env t (find_redexes env u acc)
    | Con t ->
        (match t with
         | Dup _ | Era _ -> acc
         | Con x -> find_redexes env t ((term, reduce env term) :: acc)
         | _ -> find_redexes env t acc)
    | Dup t -> find_redexes env t acc
    | Era t -> find_redexes env t acc
    | Pair (t, u) ->
        let acc' = find_redexes env t acc in
        find_redexes env u acc'
    | Swap t -> find_redexes env t acc
    | Var _ | Unit -> acc
\end{lstlisting}
\end{definition}

\begin{definition}
Паралельна редукція:
\begin{lstlisting}[mathescape=true]
let eval_parallel pool env term =
    let rec loop term =
        let redexes = find_redexes env term [] in
        if redexes = [] then term
        else
            let new_term = Task.run pool (fun () ->
                List.fold_left
                    (fun acc (old_t, new_t) -> replace_subterm old_t new_t acc)
                    term redexes
            ) in
            loop new_term
    in
    loop term
\end{lstlisting}
\end{definition}

\begin{theorem}
Доведення, що мова $\mathbf{Bob}$ є внутрішньою мовою симетричних моноїдальних категорій:
\[
\begin{cases}
   \mathrm{Let} : A \to C (u \cdot t, t: A \to B, u: B \to C), \\
   \mathrm{Pair} : A \rightarrow B \rightarrow A \otimes B, \\
   \mathrm{Swap} : A \otimes B \rightarrow B \otimes A, \\
   \mathrm{Con} : A \otimes A \rightarrow A, \\
   \mathrm{Dup} : A \rightarrow A \otimes A, \\
   \mathrm{Era} : A \rightarrow \mathbf{1}, \\
   \mathrm{Var} : A, \\
   \mathrm{Unit} : \mathbf{1}.
\end{cases}
\]
\end{theorem}

Конструктори відповідають аксіомам СМК:

- $\mathrm{Pair}$ моделює тензорний добуток $\otimes$.
- $\mathrm{Swap}$ реалізує симетрію $\sigma_{A,B}$ з умовою $\sigma_{B,A} \circ \sigma_{A,B} = \mathrm{id}_{A \otimes B}$.
- $\mathrm{Unit}$ є одиничним об'єктом $I$ для якого $A \otimes I \cong A$.
- $\mathrm{Let}$ моделює композицію морфізмів (асоціативність).
- $\mathrm{Dup}$ та $\mathrm{Era}$ утворюють структуру комоноїда.
- $\mathrm{Con}$ діє як контракція.

Лямбда-функція і аплікація:

\[
\begin{cases}
  \lambda x.t \vdash \mathrm{Con}(\mathrm{Let}(x,\mathrm{Var}(x),t)), \\
  t u \mapsto \mathrm{Con}(\mathrm{Pair}(t,u)).
\end{cases}
\]

\end{document}