\section{Практика}

\subsection{Спочатку йога розуму, потім вже йога тіла}

Рано чи пізно всі, хто займається йогою розуму при належному
успіху, так чи інакше переходять до освоєння йоги тіла, яке
є продовженням розуму. Секрет успішної практики в тому, що
як і йога розуму, йога тіла потребує ще більшої усвідомленості.
Під йогою тіла ми розумітимемо тут такого роду енергії,
які ви можете пережити тільки в режимі ексремального спорту,
там де ревард дуже високий. У спорті це X-спорт, у сексі це
BSDM, підходить усі дисципліни, де є стоп-слово, за кордоном
якого одразу настає термінація істоти.

Чому йога тіла повинна йти обов'язково після укрупнення
на практиці йоги розуму? На це є кілька причин. Вважається,
що основа сталого і дорослого мислення це правильне світогляд,
яке має формуватися істотою під час вивчення філософських
дисциплін, невирішених питань трансгуманізму та інших
базових принципів. Дотримуючись позасектарного іміджу,
самі базові принципи світогляду топового програмісти
були зацементовані в першому випуску.

Поклавши неправильний майндсет у основу Вивчення,
Роздуми та Медитації ви створите пролом, через який
у критичний момент вашого життя при зустрічі з Буддою
ваше бачення світу зруйнується як дитячий замок із піску
після припливу. Для тренування розуму алмазної міцності
та гостроти і призначені практики формування правильно
погляду на об'єкт дослідження свого власного розуму,
за зрадами якого не існують ніякі феномени.

Як і йога розуму, йога тіла передбачає два режими
дослідження: пандіта стайл (вивчення теорії) та йога
стайл (практика). Засвоївшись і зміцнившись у своїй
свідомості, істота, що рухається головним принципом
будда-такості, прагненням до всезнання і повної реалізації,
зерном якого є пізнання феноменів, починає виходити за
рамки ментальних феноменів і починає усвідомлювати себе
і своє тіло як частину мислення, і природним чином
починає експериментувати з тілом, розширюючи свій
фронтир сприйняття.

Головний критерій, який показує, чи можна вам
переходити до спорту, це повний контроль над
дофаміновою та епініфриновою системою. Деякі
занадто вразливі спортсмени використовують THC
для супресії дофаміну та м'якшої йога-сесії.
Зазвичай рекомендується входити в спорт у 40
років, тому що в 25 і кілька років після цього
потрібно присвятити математиці та філософії,
адже кращого часу вже не буде і повернути його
буде непросто! А у 40 уже дофаміновий фон сам
по собі зникне і залишиться лише чистий розум
та террейн. Це друга безжальна причина через
яку спорт краще відкласти до adult віку. Є й
інший бік медалі: рани після 40 гояться гірше,
тому і ставки і гострота і ревард у такому разі
вищі. Ідеально це маючи гострий розум не робити
взагалі серйозних помилок на шляху спорту.
Неідеальні випадки вирішуються імплантацією
титанових пластин.

Непрямий критерій це коли ви досягли рівня
безпосереднього переживання ототожнення мови
простору (випуск Х), карти місцевості (випуск 4)
і свого мислення (випуск 2), у такому разі вихід
з локальної самсари у вигляді темниці розуму
знаменуватиме вихід у реальний світ на планету
Земля. Алегорія яка мені бачиться тут така: мати
відправляють свого сина до університету, передавши
йому всі необхідні знання, які допоможуть йому жити
далі автономно

\subsection{Программування та спорт}

Гуру у спорті знайти так само важко, як і гуру в
програмуванні. У світському житті гуру спорту
працюють олімпійськими чемпіонами або чемпіонами
з дисциплін, які до цього прирівнюються. Саме їхнє
існування вже є вченням. Розбираючи до найдрібніших
деталей покадрове трейси топ-спорсменів на youtube
ви отримуєте алмазні знання віртуоза-ньюскулера
методично відточуючи техніку, маючи зразок для
верифікації. Гуру спорту меншого калібру працюватимуть
інструкторами на найближчому спортивному курорті,
рекламуватимуть газировку та еквіпмент або
пропонуватимуть вам тури національними заповідниками.

Не всі це говорять прямо, але у спорті важливими
є картинки, які проходять через вашу сіточку та
всі органи почуттів, тому цінуються картинки
природного ландшафту Землі, щоб звільнити своє
мислення вже за межами тіла, охопивши своїм
мисленням усю планету та її феномени, головний
з яких гравітація, таким чином ставши воістину
космічною дитиною планетарного масштабу.

Однак у розмовах з локальними спортивними гуру
ви отримаєте максимум репресивні монотонні лекції
про шкоду куріння на Джомолунгмі, жахливі команди
інструкторів-обивателів, юрби туристів на своєму
шляху. Доросла людина, що оволоділа йогою розуму,
сама повинна стати собі гуру і планувати кожну
вилазку на зустріч з гравітацією як проект з
багатьма параметрами-змінними, від опрацювання
якого залежить ваше життя.

Всі ці гуру будуть говорити вам, що тільки вони
розуміють суть речей, пізнали природу в немисленні,
а у вас великий тягар йоги-розуму, який заважає
вам досягати результатів, заморочки, зайва
концептуалізація, надмірна начитаність та інші
смертні гріхи. Тут діє таке ж правило як і в
дитячому садку, школі або університеті "розумних
не люблять", тому майте це на увазі поговорити
по душах за багаттям з черговим спортивним гуру.

Розумна і раціональна людина завжди вибере більш
рідкісну та філігранну йогу розуму замість йоги
тіла, якою володіє значно більша кількість істот.
Адже отримати алмаз розуму набагато складніше,
ніж алмаз тіла, тому партнери зі спорту і навіть
локальні гуру, будуть готові в прямому сенсі
підсвідомо вас убити на схилі, тут теж потрібне око та око.

У буддизмі Тибету аналогом спорту є таємне
посвята каналів, вітрів і сфер структури вашої
алмазної мережі. Всі ці йоги виконуються в парі
з опорою на партнера і дістатися цього рівня буде
важко обивателю. Тим більше, що таких гуру ще
менше, ніж спортивних.

\newpage
\subsection{Баланс}

Головний наркотик для йоги-розуму та йогі-тіла
це цукор. Усі спорсмени як мінімум висять на
кока-колі, ред-булі, гелях, стимуляторах. Баланс
цих речовин і правильне харчування – ключ до
 швидкого відновлення після спортивних сесій,
які так чи інакше потрібно буде заліковувати
перед повноцінними сесіями йоги розуму. Ігнорувати
допоміжні  речовини на шляху спорту марнославно, але
й зловживати не варто. Головний критерій,
безперервність, ви повинні планувати вашу
подорож таким чином, щоб ваша свідомість
не похитнулася від раптової зміни гормонального
фону, дефіциту того чи іншого паливного елемента.

\normalsize
