\section{Мислення}

\subsection{Що таке мислення}

Перед тим як розпочинати процес навчання непогано було
б кілька слів сказати про основний інструмент у процесі
вивчення --- людське мислення. Минаючи фізичні сторони
мислення відразу хочеться поговорити про його когнітивні властивості.

\subsection{Характеристики чистого мислення}

Перша і головна властивість мислення --- це істотність ---
визначальна характеристика істоти. Інтегральна вища форма,
яка керує всіма підсистемами та сприймається істотою,
майндстрімом або аватаром. В одному тілі може жити
кілька майндстрімів, і деякі з них можуть бути програмістами!
Якщо ви думаєте --- ви є істотою.

Друга когнітивна характеристика мислення, яку можна
відчути у медитаціях --- це абсолютна сферична відкритість
у всіх напрямках та її безмежність. Така характеристика
мислення навіває думки про ізоморфізм мислення та простору.
З фізичної точки зору, мислення --- це складна система квантових
полів, які нашаровуються на квантовий, молекулярний рівень,
нервову систему, тому довго доводити не потрібно, що
мислення як квантово-механічна система поширюється на весь простір.

Умовно існує два розділи вищої медитації, перший із яких
називається розділом мислення, а другий розділом простору.
Перший розділ присвячений технікам роботи з феноменами,
аналітичної медитації, роботі з мисленням з погляду
майндстріму, очним вправам, розвитку відчуття перспективи,
роботи з уявою, візуалізаціям. Другий розділ присвячений
технікам роботи з мисленням з погляду простору, де мислення
асоціюється з простором, в якому воно перебуває,
неаналітичної медитації, прагнення до нескінченності,
медитації відпочинку.

Третя когнітивна характеристика мислення, яку можна
сприйняти на досвіді, --- це його необумовленість.
Чим вищий рівень розвитку мислення, тим вища його
воля до свободи і необумовленість, до перевірки,
критичного мислення і переоцінювання. У своїй повній
свободі мислення вільно обирає спрямованість і інтенсивність
потоку, без різких перепадів і гормональних фонів,
рухаючись оптимальною траєкторією дорослішання плоду
мислення на шляху до всезнавства.

Четверта характеристика мислення -- це безперервність.
Будь-які спроби зупинити мислення приводять у місце
самоусвідомлення як несучу частоту відчуття присутності
себе в цьому світі, в медитації. Навіть у процесі сну,
мислення не спить, а перетворюється на інший агрегатний
стан, більш розріджене, часом безформне, нечітке,
мерехтливе. Повний контроль над безперервністю мислення,
від якої не можна відмовитись і яку не можна припинити ---
завдання топового програміста на шляху до звільнення
ресурсів для вивчення програмування. Чим більша точність
дискретизації цього контролю – тим краще. Контроль за
безперервністю мислення називається точністю мислення.

П'ята характеристика мислення – взаємозалежність. Ви
як мислення --- це продукт абсорбції інших фрагментів
мислень чи просто феноменів, тому обумовлені цією спадщиною.
Вирватися за межі цієї традиції та розкопати інсайти на шляху
еволюції свого мислення – справжня коштовність як нагорода за
працю навчання. Коли ви стаєте майстром, обумовленість зникає,
ви реструктуруєте себе наново виходячи вже з особистого досвіду,
побудованого на низці інсайтів, за якими ви стрибаєте на шляху
до майстерності. І навіть їх ви потім зможете видалити і забути
зі свого мислення залишивши тільки пам'ять про те, як потрібно
одразу робити правильно, можливо і не згадайте навіть,
коли вас спитають, як це ви так швидко помудріли, а навіщо.

Ці п'ять характеристик послужать вам підказками у якому
ключі потрібно думати про своє мислення (перша похідна)
як інструмент пізнання, можливо для істот з високими
здібностями це одразу прояснить деякі моменти. Будь-яка
нездатність спостерігати ці характеристики в практичних
медитаціях або роздумах про своє мислення, говорить про
те, що їх потрібно розвивати, або зайнятися йогою, піти
до психолога, розвіятися з друзями, піти в бар, сісти
на таблетки, склянку, все за бажанням --- головне щоб
спрацювало! Чек лист пройшли переходимо до рекомендацій та індикатора

\subsection{Коштовне намисто мислення}

У традиції Тибету існує шість типів мислення або програм, які вважаються,
позитивно можуть вплинути в цілому на процес вивчення, роздуми і медитації.

Щедрість у контексті мислення означає не скупитися в процесі вивчення,
не хапатися за все одразу, мати методологію, з повагою ставиться до
будь-якої обраної теми, раз вона вже спливла в медитації як комлекс,
який все одно доведеться закрити (пізнати). Здатність до реплікації, викладання,
зворотної контрибуції по дорозі поглинання інформації --- це щедрість мислення.

Дисципліна означає, що мислення має дотримуватися якогось спортивного,
бажано олімпійського режиму, надто хаотичні режими мислення не
сприятимуть навчанню, тому приступати до еволюції свого мислення
потрібно, коли гормональне тло може залишатися рівним значний час,
це необхідно для глибоких медитацій, без яких неможливий прогрес.

Терпіння – це здатність переносити проблеми у процесі навчання.
Є матеріал, який може не закриватися роками, але до нього все
одно доведеться повертатися, адже назад дороги немає, обрано
шлях топового програміста. На шляху може бути занадто багато
інсайтів і надто багато наснаги, яке може створювати гормональне
тло, яке не завжди можна контролювати, пересиджувати на бенчі
такі періоди --- це терпіння.

Старанність --- означає з непідробним інтересом вивчати предмети,
тому правильно їх розмістити дуже важливо. Можливо, саме для вас
існує своя послідовність предметів, кожен з яких в окремий момент
часу ви вивчатимете з максимальною старанністю. Із цим доведеться
працювати, кожному індивідуально.

Фокусування --- фокусування, чи концентрація, чи медитація,
чи шаматха --- це основний режим роботи програміста. Ось ви
сіли за комп'ютер, поставили чашку з кавою, протерли дисплей,
всмокталися в пікселі, запустили шелл --- ви сфокусовані на
роботі, це медитація.

Мудрість --- це система накопичених інсайтів, що формує нові
структури мислення, нову його топологію. Ця система може
переписувати старі неефективні та невалідні структури,
з яких ми сміємося подорослішавши. Мислення мудрості --- це
мислення, засноване виключно на таких перевірених рафінованих
структурах, які покладені в фундамент нашої істоти.

\subsection{Отрути мислення}

Три найбільш несприятливі форми мислення з моєї особистої класифікації.

Інертність мислення – це колесо медитації. Будучи вкотре
запущена деяка звичка, йде у автоматичний режим на
підсвідомість --- це інертність. Якби не було інертності
мислення, ми б не змогли вчитися. Хоча це корисна властивість
мислення, іноді буває погано, коли погано --- потрібно
відловлювати. Зрозуміло, що бешкетувати своє мислення,
яке заховано в підсвідомості не вчать у школах, доведеться працювати самому.

Лінощі. Занадто інтенсивне мислення може перерости в затяжну
рекреаційну прокрастинацію, яка зміниться лінощами.
Спостереження за видимим прогресом необхідно, яким би був
охуенный відпочинок треба повертатися за програмування,
оновлювати мотивацію, якщо потрібно щодня.

Байдужість. Корінь усіх отрут, жадібності та іншого. Якщо
вам все раптом стало байдуже, це дуже погано, але не смертельно.
Іноді може перерости в екзистенційну кризу, але ж ми з вами вже
домовилися, що тіло, йогу і таблетки і своє самопочуття ви
берете на себе, з мене тільки рекомендації щодо процесу
навчання. Занедбана байдужість --- це тупість.

Нічого хорошого успішного студента, який виходить за
перерахування чеснот, тут немає. Як і першому випадку
постійно застосовуємо техніку роздуми, вивчення та
медитації до цих видів мислення, як і до основних характеристик
мислення. Постійно валідуємо своє мислення відповідно
до індикаторів.

\subsection{Ядро логіки}

Є тільки три способами якими жива істота може помилятися в рамках
логіки своєї людської мови. Перше --- це побудова неправильних
функцій як елементів певних функціональних алгебріїчних синантур.
Друга --- це недоведене конструктивне існування певного об'єкту,
моделі, системи, тощо, з певними властивостями. І третій --- це
неправильне порівняння, підміна понять, некоректна аналогія,
відсусть доведеного ізоморфізму, який може мати багато вимірів,
особливо в алгебрїчній геометрії. Багатавимірність, або параметричність
кванторів існування та узагальнення моделюється телекскопічність їх
контекстів.

\subsubsection{Квантор узагальнення}



\subsubsection{Квантор існування}

\subsubsection{Багатовимірний ізоморфізм}

\normalsize
