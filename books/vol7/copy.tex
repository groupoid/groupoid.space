\noindent УДК 13\\
УДК 821.161.2

\section*{Топовий програміст \top}

\begin{tabular}{ll}
& Автор: Максим Сохацький (1980)\\
\end{tabular}

\section*{Про Автора}
Намдак Тонпа (Максим Сохацький) --- доктор філософії КПІ (02070921),
буддистський піп-капелан лінії передачі
Лонгчен Нінгтік тибетського буддизму школи Нінгма (38778275),
провідний інженер-програміст ДП «ІНФОТЕХ» (34239034)
у підпорядкуванні МВС України (00032684).
Автор систем «Депозити ПриватБанк» та «МІА: Документообіг»
побудованих на авторських творах ERP.UNO та N2O.DEV.
\\
\\
\\
\\
\\
Постійне посилання твору: https://axiosis.top/top/ \\
Видавець: Інститут формальної математики «Групоїд Інфініті» \\
Сайт інституту: https://groupoid.space/institute/ \\
\\
\\
{\bf ISBN: 978-617-8027-23-0 \hspace{2em}}
\\
\\
\small
\indent Якщо мої передплатники і просять про якусь масштабну контрибуцію,
то це монографію на тему «як стати топовим програмістом».
Хоча таке формулювання інфантильне, воно досить добре
відображає сутність запитуваного: детальний розгляд професії
програміста, стратегію вивчення предмета виходячи з особистого
досвіду, розбавлений автентичною філософією.
\\
\\
\begin{tabular}{ll}
\textcopyright{} 2022 & Максим Сохацький
\end{tabular}

\newpage
\section*{Анотація}
Збірка відкритих листів на тему ультимативного розвитку в
професійному та емоційному аскпектах. Програмування як
спорт і як спосіб досягнення досконалості. У вигляді
гонзо-журналістики, розбавлено східною філософією.

Тема професійного розвитку була атакована багато разів
некваліфікованими педагогами, що намагались денонсувати
педагогіку не тільки вищої школи, але і молодших класів.
Пізніше у корпоративному середовищі, плани розвитку софт
та хард навичок подають вже у більш науковому пакунку,
але до кінця не відкривають двері, які ведуть за межі
професії, після її повної та остаточної реалізації.

Як я чув колись: секрет успіху тільки трохи залежить
від таланту і багато від методики навчання і головним
чином від мотивації досягти успіху, джерелом якого
зазвичай є наші глибокі травми. Східна філософія каже,
що неперервне вдосконалення майстра, може привести
його на вершину реалізації. Поєднання діамантової
мотивації і правильної спортивної техніки розвитку
мислення при гарній долі може відкрити реалізацію
професії програміста не тільки у розрізі повного
циклу виробництва по ISO 9001, але і відкриє осягнення
дотичних професійних реалізацій у сфері дизайну
(продуктовий та промисловий дизайн, машино-будівництво,
промислова та цивільна архітектура), фізики та техніки,
математики та філософії, або машинної лінгвістики.

Оскільки математики, як і програмісти, пенсійним
віком для топ-перформерів вважають вік приблизно
25-30 років, отримати правильне знання як стати
топ-програмістом або топ-математиком особливо важливо
у молодому віці. Ця книга — це скарб двох ключів:
перший ключ — це діамантова мотивація і другий ключ ---
це бажання всевідання, нестримний порив пізнання
феноменів який проявляється на ранніх етапах
розвитку дитини. Такі діти здатні неперервно
тривалий час сфокусовуватися та мають належний
перелік якостей мислення необхідних для курсу
навчання, які теж можна і потрібно розвивати.

Ця книга -- це і мотиваційний спіч, і ретроспектива
власного досвіду, і відоповіді на запитання по техніці
навчання, і звернення до наступних поколінь, і навчальна
методичка для програми <<Інституту формальної
математики>>\footnote{\url{https://groupoid.space/institute/}}.
Бібліографічне забезпечення потрібно шукати в
проекті УДК 51\footnote{\url{https://5ht.co/51.pdf}},
там же ви знайдете курс Дани
Скота\footnote{\url{https://tonpa.guru/stream/2020/2020-04-23 Лестница в HoTT.htm}},
який показує історію теорії обчислень та сучасної
теоретичної інформатики.

\section*{Подяка}

Дякую всім, хто формував своїми питаннями цей твір.

\normalsize
