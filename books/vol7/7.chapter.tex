\section{Посвята всім святим програмістам}

Складаю список святих програмістів. З кого почати? Почну а
Ади Гольдберг та Алана Кея\footnote{автори Smalltalk}. Потім Грінблат та Госпер\footnote{автори TX-0}.
Далі йде Кей Айверсон\footnote{автор APL} і Джон МакКарті\footnote{автор LISP}.
Далі йдуть Лінус Торвальдс\footnote{автор Linux}, Дейв Катлер\footnote{автор Windows NT}, Аветіс Теванян\footnote{автор NeXT}.
Не забуваємо про Джона Бакуса\footnote{автор BNF} та Робіна Мілнера\footnote{автор ML}.
Головним чином пам'ятаємо Ксав'є Лероя\footnote{автор OCaml}, Джо Армстронга\footnote{автор Erlang},
Саймона Пейтона-Джонса\footnote{автор Haskell} і мало-відомого Ральфа Юнга\footnote{автор формальної моделі Rust}.
У часи нюдьги молимося Джону Кармаку\footnote{автор Quake}, Казунорі Ямаучі\footnote{автор Gran Turismo}, Амерікан МакГі\footnote{автор MacGee's Alice}.

\newpage

Виділяємо окремо майстів нідерландської школи: Ерік Мейер\footnote{автор LINQ},
Хенк Барендрегт\footnote{автор лямбда-кубу}, Ніколас де Брейн\footnote{автор фібраційної верифікації}.
Тері Кокан\footnote{автор числення конструкцій}, Жерар Юе\footnote{математик логік} топові математики-програмісти.
Ми дякуємо Вінту Сьорфу\footnote{автор TCP/IP}, Леслі Лампорту\footnote{автор LaTeX та TLA+}, Алану Коксу\footnote{автор Linux SMP},
Тео де Раадту\footnote{атвор OpenBSD}, Кліву Моулеру\footnote{автор LAPACK}, Джефу Діну\footnote{автор leveldb, TensorFlow}.
Не забуваємо про Фабріса Беллара, Тревіса Гейсельбрехта\footnote{автор newos, BeOS, BeFS},
Александра ван дер Грінтена\footnote{автор Managarm}.
Шануємо та поважаємо Ульфа Норела\footnote{автор Agda}, Леонардо де Мура\footnote{aвтор Lean}.
Відмічаємо Бена Фрая\footnote{автор Processing}, Стефана Вольфрама\footnote{автор Mathematica} і Йоахіма Нойбузера\footnote{автор GAP}.
Йохен Лидтке\footnote{автор L4}, Ада Лавлас\footnote{піонер програміст}, Девід Люкхем\footnote{піонер формальної верифікації}.

\normalsize
