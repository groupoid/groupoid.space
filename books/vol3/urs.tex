\documentclass{article}
\usepackage{newunicodechar}
\usepackage{verbatim}
\usepackage[english,ukrainian]{babel}
\usepackage{listings}
\usepackage[utf8]{inputenc}
\usepackage{hyphenat}
\usepackage{hyperref}
\usepackage{adjustbox}
\usepackage{xcolor}
\usepackage{microtype}
\usepackage{amscd}
\usepackage{amsmath}
\usepackage{amssymb}
\usepackage{amsthm}
\usepackage{tikz}
\usepackage{tikz-cd}
\usepackage{url}

\makeatletter
\renewcommand{\lstlisting@font}{\ttfamily\small}
\makeatother

\newunicodechar{λ}{$\lambda$}
\newunicodechar{Π}{$\Pi$}
\newunicodechar{Σ}{$\Sigma$}
\newunicodechar{ε}{$\epsilon$}
\renewcommand{\quotesinglbase}{,}


\addto\captionsukrainian{\renewcommand{\contentsname}{Зміст}}
\addto\captionsukrainian{\renewcommand{\abstractname}{Аннотація}}

\begin{document}

\author { М.Е. Сохацький $^1$ }
\title { Issue XXI: Super Type System }
\date{ \small $^1$ Національний технічний університет України \\
       ім. Ігоря Сікорського \\
       26 листопада 2025 }
\maketitle

\begin{abstract}

\end{abstract}

\ifincludeTOC
  \tableofcontents
\fi

\addtocontents{toc}{\protect\newpage}

\newpage

\section{Introduction to Urs}

\section{Super Type System}

\subsection{Bosonic Modality}

The $\bigcirc$ modality in cohesive type theory projects a type to bosonic parity ($g = 0$).
For a type $A : \mathbf{U}_{i,g}$, $\bigcirc A$ forces the type to be bosonic,
aligning with supergeometry and quantum physics.

In Urs, $\bigcirc$ operates on types in graded universes from $\mathbf{Graded}$,
with applications in bosonic quantum fields $\mathbf{qubit}$ and supergeometry
$\mathbf{SmthSet}$.

\subsection{Bose}

\begin{definition}[Bosonic Modality Formation]\label{def:bosonic-formation}
The $\bigcirc$ modality is a type operator on graded universes, mapping to bosonic parity:
\[
    \bigcirc : \prod_{i : \mathbb{N}} \prod_{g : \mathbf{Grade}}
    \mathbf{U}_{i,g} \to \mathbf{U}_{i,0}.
\]
\begin{lstlisting}[mathescape=true]
def bosonic (i : Nat) (g : Grade) (A : U i g) : U i 0
\end{lstlisting}
\end{definition}

\begin{definition}[Bosonic Modality Introduction]\label{def:bosonic-intro}
Applying $\bigcirc$ to a type $A$ produces $\bigcirc A$ with bosonic parity:
\[
    \Gamma \vdash A : \mathbf{U}_{i,g}
    \quad \to \quad
    \Gamma \vdash \bigcirc A : \mathbf{U}_{i,0}.
\]
\end{definition}

\begin{definition}[Bosonic Modality Elimination]\label{def:bosonic-elim}
The eliminator for $\bigcirc A$ maps bosonic types to properties in $\mathbf{U_0}$:
\[
\mathbf{Ind_{\bigcirc}} :
\prod_{i : \mathbb{N}} \prod_{g : \mathbf{Grade}}
\prod_{A : \mathbf{U}_{i,g}} \prod_{\phi : (\bigcirc A) \to \mathbf{U_0}}
\left( \prod_{a : \bigcirc A} \phi\ a \right) \to \prod_{a : \bigcirc A} \phi\ a.
\]
\begin{lstlisting}[mathescape=true]
def bosonic_ind (i : Nat) (g : Grade) (A : U i g)
    (phi : (bosonic i g A) -> U_0)
    (h : Π (a : bosonic i g A), phi a)
  : Π (a : bosonic i g A), phi a
\end{lstlisting}
\end{definition}

\begin{theorem}[Idempotence of Bosonic]\label{thm:bosonic-idempotence}
The $\bigcirc$ modality is idempotent, as it always projects to bosonic parity:
\[
    \mathbf{\bigcirc\text{-}idem} :
    \prod_{i : \mathbb{N}} \prod_{g : \mathbf{Grade}}
    \prod_{A : \mathbf{U}_{i,g}}
    (\bigcirc (\bigcirc A)) = (\bigcirc A).
\]
\begin{lstlisting}[mathescape=true]
def bosonic_idem (i : Nat) (g : Grade) (A : U i g)
  : (bosonic i 0 (bosonic i g A)) = (bosonic i g A)
\end{lstlisting}
\end{theorem}

\begin{theorem}[Bosonic Qubits]\label{thm:bosonic-qubit}
For $C, H : \mathbf{U_0}$, the type $\bigcirc \mathbf{Qubit}(C, H)$ models bosonic quantum states:
\[
    \mathbf{\bigcirc\text{-}qubit} :
    \prod_{i : \mathbb{N}} \prod_{g : \mathbf{Grade}}
    \prod_{C, H : \mathbf{U_0}}
    (\bigcirc \mathbf{Qubit}(C, H)) : \mathbf{U}_{i,0}.
\]
\begin{lstlisting}[mathescape=true]
def bosonic_qubit (i : Nat) (g : Grade) (C H : U_0) : U i 0
 := bosonic i g (Qubit C H)

\end{lstlisting}
\end{theorem}

\subsection{Braid}

The $\mathbf{Braid}_n(X)$ type models the braid group $B_n(X)$ on $n$ strands over
a smooth set $X : \mathbf{SmthSet}$, the fundamental group of the configuration
space $\mathbf{Conf}^n(X)$, used in knot theory, quantum computing, and smooth geometry.

In Urs, $\mathbf{Braid}_n(X)$ is a type in $\mathbf{U_0}$, parameterized by
$n : \mathbf{Nat}$ and $X : \mathbf{SmthSet}$, supporting braid generators
$\sigma_i$ and relations, with applications to anyonic quantum gates and knot invariants.

\begin{definition}[Braid Formation]
The type $\mathbf{Braid}_n(X)$ is formed for each $n : \mathbf{Nat}$ and $X : \mathbf{SmthSet}$:
\[
    \mathbf{Braid} : \prod_{n : \mathbf{Nat}} \prod_{X : \mathbf{SmthSet}} \mathbf{U_0}.
\]
\begin{lstlisting}[mathescape=true]
def Braid (n : Nat) (X : SmthSet) : U_0
(* Braid group type *)
\end{lstlisting}
\end{definition}

\begin{definition}[Braid Introduction]\label{def:braid-intro}
Terms of type $\mathbf{Braid}_n(X)$ are introduced via the $\mathbf{braid}$ constructor,
representing generators $\sigma_i$ for $i : \mathbf{Fin}\ (n-1)$:
\[
    \mathbf{braid} : \prod_{n : \mathbf{Nat}} \prod_{X : \mathbf{SmthSet}} \prod_{i : \mathbf{Fin}\ (n-1)} \mathbf{Braid}_n(X).
\]
\begin{lstlisting}[mathescape=true]
def braid (n : Nat) (X : SmthSet) (i : Fin (n-1)) : Braid n X
(* Braid generator sigma_i *)
\end{lstlisting}
\end{definition}

\begin{definition}[Braid Elimination]\label{def:braid-elim}
The eliminator for $\mathbf{Braid}_n(X)$ maps braid elements to properties in $\mathbf{U_0}$:
\[
\mathbf{BraidInd} :
\prod_{n : \mathbf{Nat}} \prod_{X : \mathbf{SmthSet}}
\prod_{\beta : \mathbf{Braid}_n(X) \to \mathbf{U_0}}
\left( \prod_{b : \mathbf{Braid}_n(X)} \beta\ b \right) \to
\prod_{b : \mathbf{Braid}_n(X)} \beta\ b.
\]
\begin{lstlisting}[mathescape=true]
def braid_ind (n : Nat) (X : SmthSet)
    (beta : Braid n X -> U_0)
    (h : Π (b : Braid n X), beta b)
  : Π (b : Braid n X), beta b
\end{lstlisting}
\end{definition}

\begin{theorem}[Braid Relations]\label{thm:braid-relations}
For $n : \mathbf{Nat}$, $X : \mathbf{SmthSet}$, $\mathbf{Braid}_n(X)$
satisfies the braid group relations (Commutation and Yang-Baxter):
\[
\prod_{n : \mathbf{Nat}} \prod_{X : \mathbf{SmthSet}}
\prod_{i, j : \mathbf{Fin}\ (n-1),\ |i-j|\geq 2}
\sigma_i \cdot \sigma_j = \sigma_j \cdot \sigma_i,
\]
\[
\prod_{n : \mathbf{Nat}} \prod_{X : \mathbf{SmthSet}}
\prod_{i : \mathbf{Fin}\ (n-2)}
\sigma_i \cdot \sigma_{i+1} \cdot \sigma_i = \sigma_{i+1} \cdot \sigma_i \cdot \sigma_{i+1}.
\]
\begin{lstlisting}[mathescape=true]
def braid_rel_comm (n : Nat) (X : SmthSet) (i j : Fin (n-1))
  : Path (braid i · braid j) (braid j · braid i)

def braid_rel_yang (n : Nat) (X : SmthSet) (i : Fin (n-2))
  : Path (braid i · braid (i+1) · braid i) (braid (i+1) · braid i · braid (i+1))
\end{lstlisting}
\end{theorem}

\begin{theorem}[Configuration Space Link]\label{thm:braid-conf}
For $n : \mathbf{Nat}$, $X : \mathbf{SmthSet}$, $\mathbf{Braid}_n(X)$ is the fundamental groupoid of $\mathbf{Conf}^n(X)$:
\[
\prod_{n : \mathbf{Nat}} \prod_{X : \mathbf{SmthSet}}
\mathbf{Braid}_n(X) \cong \pi_1(\mathbf{Conf}^n(X)).
\]
\begin{lstlisting}[mathescape=true]
def braid_conf (n : Nat) (X : SmthSet)
  : Path (Braid n X) (pi_1 (Conf n X))
\end{lstlisting}
\end{theorem}

\begin{theorem}[Quantum Braiding]\label{thm:braid-qubit}
For $C, H : \mathbf{U_0}$, $\mathbf{Braid}_n(X)$ acts on $\mathbf{Qubit}(C, H)^{\otimes n}$ as braiding operators:
\[
\mathbf{braid\_qubit} :
\prod_{n : \mathbf{Nat}} \prod_{C, H : \mathbf{U_0}} \prod_{X : \mathbf{SmthSet}}
\mathbf{Braid}_n(X) \to
\left( \mathbf{Qubit}(C, H)^{\otimes n} \to \mathbf{Qubit}(C, H)^{\otimes n} \right).
\]
\begin{lstlisting}[mathescape=true]
def braid_qubit (n : Nat) (C H : U_0) (X : SmthSet)
  : Braid n X -> (Qubit C H)^n -> (Qubit C H)^n
\end{lstlisting}
\end{theorem}

\begin{theorem}[Braid Group Delooping]\label{thm:braid-delooping}
For $n : \mathbf{Nat}$, the delooping $\mathbf{BB}_n$ of the braid group $B_n$ is a 1-groupoid:
\[
\mathbf{BB}_n : \mathbf{Grpd}\ 1 \equiv \Im(\mathbf{Conf}^n(\mathbb{R}^2)).
\]
\begin{lstlisting}[mathescape=true]
def BB_n (n : Nat) : Grpd 1 := $\Im$ (Conf n $\mathbb{R}^2$)
\end{lstlisting}
\end{theorem}

\newpage
\subsection{Graded Universes}

\textbf{Graded Universes.}
The $\mathbf{U}_\alpha$ type represents a graded universe indexed by a monoid
$\mathcal{G} = \mathbb{N} \times \mathbb{Z}/2\mathbb{Z}$, where $\alpha \in \mathcal{G}$
encodes a level ($\mathbb{N}$) and parity ($\mathbb{Z}/2\mathbb{Z}$: $0$ = bosonic, $1$ = fermionic).
Graded universes support type hierarchies with cumulativity, graded tensor products,
and coherence rules, used in supergeometry (e.g., bosonic/fermionic types),
quantum systems (e.g., graded qubits), and cohesive type theory.

In Urs, $\mathbf{U}_\alpha$ is a type indexed by $\alpha : \mathcal{G}$, with operations
like lifting, product formation, and graded tensor products, extending standard
universe hierarchies to include parity, building on $\mathbf{Tensor}$.

\begin{definition}[Grading Monoid]
The grading monoid $\mathcal{G}$ is defined as $\mathbb{N} \times \mathbb{Z}/2\mathbb{Z}$,
with operation $\oplus$ and neutral element $\mathbf{0}$, encoding level and parity.
\[
\begin{aligned}
\mathcal{G} & : \mathbf{Type} \equiv \mathbb{N} \times \mathbb{Z}/2\mathbb{Z}, \\
\oplus & : \mathcal{G} \to \mathcal{G} \to \mathcal{G}, \\
(\alpha, \beta) & \mapsto (\text{fst}\ \alpha + \text{fst}\ \beta,\ (\text{snd}\ \alpha + \text{snd}\ \beta) \mod 2), \\
\mathbf{0} & : \mathcal{G} \equiv (0, 0).
\end{aligned}
\]
\begin{lstlisting}[mathescape=true]
def 𝒢 : Type := ℕ × ℤ/2ℤ
def ⊕ (α β : 𝒢) : 𝒢 := (fst α + fst β, (snd α + snd β) mod 2)
def 𝟘 : 𝒢 := (0, 0)
\end{lstlisting}
\end{definition}

\begin{definition}[Graded Universe Formation]
The universe $\mathbf{U}_\alpha$ is a type indexed by $\alpha : \mathcal{G}$,
containing types of grade $\alpha$. A shorthand notation $\mathbf{U}_{i,g}$
is used for $\mathbf{U}\ (i, g)$.
\[
\begin{aligned}
\mathbf{U} & : \mathcal{G} \to \mathbf{Type}, \\
\mathbf{Grade} & : \mathbf{Set} \equiv \{0, 1\}, \\
\mathbf{U}_{i,g} & : \mathbf{Type} \equiv \mathbf{U}(i, g) : \mathbf{U}_{i+1}.
\end{aligned}
\]
\begin{lstlisting}[mathescape=true]
def U (α : 𝒢) : Type := Universe α
def Grade : Set := {0, 1}
def U (i : Nat) (g : Grade) : Type := U (i, g)
def U₀ (g : Grade) : U (1, g) := U (0, g)
def U₀₀ : Type := U (0, 0)
def U₁₀ : Type := U (1, 0)
def U₀₁ : Type := U (0, 1)
\end{lstlisting}
\end{definition}

\begin{definition}[Graded Universe Coherence Rules]
Graded universes support coherence rules for lifting, product formation, and substitution,
ensuring type-theoretic consistency.
\[
\begin{aligned}
\mathbf{lift} & : \prod_{\alpha, \beta : \mathcal{G}} \prod_{\delta : \mathcal{G}}
\mathbf{U}\ \alpha \to (\beta = \alpha \oplus \delta) \to \mathbf{U}\ \beta, \\
\mathbf{univ} & : \prod_{\alpha : \mathcal{G}} \mathbf{U}\ (\alpha \oplus (1, 0)), \\
\mathbf{cumul} & : \prod_{\alpha, \beta : \mathcal{G}} \prod_{A : \mathbf{U}\ \alpha}
\prod_{\delta : \mathcal{G}} (\beta = \alpha \oplus \delta) \to \mathbf{U}\ \beta, \\
\mathbf{prod} & : \prod_{\alpha, \beta : \mathcal{G}} \prod_{A : \mathbf{U}\ \alpha}
\prod_{B : A \to \mathbf{U}\ \beta} \mathbf{U}\ (\alpha \oplus \beta), \\
\mathbf{subst} & : \prod_{\alpha, \beta : \mathcal{G}} \prod_{A : \mathbf{U}\ \alpha}
\prod_{B : A \to \mathbf{U}\ \beta} \prod_{t : A} \mathbf{U}\ \beta, \\
\mathbf{shift} & : \prod_{\alpha, \delta : \mathcal{G}} \prod_{A : \mathbf{U}\ \alpha}
\mathbf{U}\ (\alpha \oplus \delta).
\end{aligned}
\]
\begin{lstlisting}[mathescape=true]
def lift (α β : 𝒢) (δ : 𝒢) (e : U α) : β = α ⊕ δ → U β :=
  λ eq : β = α ⊕ δ, transport (λ x : 𝒢, U x) eq e

def univ-type (α : 𝒢) : U (α ⊕ (1, 0)) :=
  lift α (α ⊕ (1, 0)) (1, 0) (U α) refl

def cumul (α β : 𝒢) (A : U α) (δ : 𝒢) : β = α ⊕ δ → U β :=
  lift α β δ A

def prod-rule (α β : 𝒢) (A : U α) (B : A → U β) : U (α ⊕ β) :=
  Π (x : A), B x

def subst-rule (α β : 𝒢) (A : U α) (B : A → U β) (t : A) : U β :=
  B t

def shift (α δ : 𝒢) (A : U α) : U (α ⊕ δ) :=
  lift α (α ⊕ δ) δ A refl
\end{lstlisting}
\end{definition}

\begin{definition}[Graded Tensor Introduction]
Graded tensor products combine types with matching levels, combining parities.
\[
\begin{aligned}
\mathbf{tensor} & : \prod_{i : \mathbb{N}} \prod_{g_1, g_2 : \mathbf{Grade}}
\mathbf{U}_{i,g_1} \to \mathbf{U}_{i,g_2} \to \mathbf{U}_{i,(g_1 + g_2 \bmod 2)}, \\
\mathbf{pair\text{-}tensor} & : \prod_{i : \mathbb{N}} \prod_{g_1, g_2 : \mathbf{Grade}}
\prod_{A : \mathbf{U}_{i,g_1}} \prod_{B : \mathbf{U}_{i,g_2}} \prod_{a : A} \prod_{b : B}
\mathbf{tensor}(i, g_1, g_2, A, B).
\end{aligned}
\]
\begin{lstlisting}[mathescape=true]
def tensor (i : Nat) (g₁ g₂ : Grade)
    (A : U i g₁) (B : U i g₂) : U i (g₁ + g₂ mod 2)
 := A ⊗ B

def pair-tensor (i : Nat) (g₁ g₂ : Grade) (A : U i g₁)
    (B : U i g₂) (a : A) (b : B) : tensor i g₁ g₂ A B
 := a ⊗ b
\end{lstlisting}
\end{definition}

\begin{definition}[Graded Tensor Eliminators]
Eliminators for graded tensor products project to their components.
\[
\begin{aligned}
\otimes\text{-}\mathbf{prj}_1 & : (A \otimes B) \to A, \\
\otimes\text{-}\mathbf{prj}_2 & : (A \otimes B) \to B.
\end{aligned}
\]
\begin{lstlisting}[mathescape=true]
def pr₁ (i : Nat) (g₁ g₂ : Grade)
    (A : U i g₁) (B : U i g₂) (p : A ⊗ B) : A := p.1

def pr₂ (i : Nat) (g₁ g₂ : Grade)
    (A : U i g₁) (B : U i g₂) (p : A ⊗ B) : B := p.2
\end{lstlisting}
\end{definition}

\begin{theorem}[Monoid Properties]
The grading monoid $\mathcal{G}$ satisfies associativity and identity laws.
\[
\begin{aligned}
\mathbf{assoc} & : ((\alpha \oplus \beta) \oplus \gamma) = (\alpha \oplus (\beta \oplus \gamma)), \\
\mathbf{id\text{-}left} & : (\alpha \oplus \mathbf{0}) = \alpha, \\
\mathbf{id\text{-}right} & : (\mathbf{0} \oplus \alpha) = \alpha.
\end{aligned}
\]
\begin{lstlisting}[mathescape=true]
def assoc (α β γ : 𝒢) : (α ⊕ β) ⊕ γ = α ⊕ (β ⊕ γ) := refl
def ident-left (α : 𝒢) : α ⊕ 𝟘 = α := refl
def ident-right (α : 𝒢) : 𝟘 ⊕ α = α := refl
\end{lstlisting}
\end{theorem}

\end{document}
